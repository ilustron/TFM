\chapter{Resultados}

La siguiente tabla muestra el número de barridos totales realizados para los
diferentes tamaños de la membrana, se realizaron el mismo número de barridos
para todos $\kappa$. No se almacenaban todas las configuraciones sucesivas, en
la última fila de la tabla se muestra el número de configuraciones almacenadas
para cada tamaño. 

\begin{table}[h]
\begin{tabular}{|c|c|c|c|c|c|c|}\hline
Tamaño & $16^2$ & $24^2$ & $32^2$ & $46^2$ & $64^2$ & $128^2$ \\ \hline\hline 
barridos totales & $1.68\cdot 10^8$& $3.02\cdot 10^9$ & $7.41\cdot 10^8$ &
$6.51\cdot 10^8$&$7.212\cdot 10^8$ &$1.01\cdot10^9$\\\hline
n\textdegree configuraciones& $10^4$ & $10^4$ & $1.2\cdot 10^4$ & $7\cdot
10^3$ & $6\cdot 10^4$ & $2.5\cdot 10^3$\\ \hline
\end{tabular}
\end{table}


 % \begin{figure}[h]
 % \centering
 %  \input{test.tex}
 % \caption{Test}
 % \end{figure}

\section{Relaciones de escala Radio de giro}

Como se aprecia en la figura \ref{radio-giro-fig} el radio de giro sufre un
cambio drástico aproximadamente en $\kappa\simeq 0.8$, que separa las dos
fases: plana para $\kappa> 0.8$ y rugosa  para $\kappa< 0.8$. Para estudiar la
relación de escala en ambas fases, tomamos En la fase
rugosa sigue la relación de escala $R_G\sim (\log L)^{1/2}$ (figura
\ref{rg2_rugosa}), y la fase plana $R_G\sim L$

\begin{figure}[h]
  \centering
  \input{Rg2_plot.tex}
  \caption{Radio de giro}\label{radio-giro-fig}
\end{figure}

\begin{figure}[h]
  \centering
  \input{rg2_plano_plot.tex}
  \caption{$R^2_g$ en función de $L$ para $\kappa=2.0$. Resultados del ajuste
    $c_0=0.134\pm 0.025$, $c_1=0.0160\pm 0.0003$, $2\nu=1.9955 \pm
    0.005$. $\chi^2/gl=2.063/2$ , $P(\chi^2>\chi_C^2)=0.356$}\label{Rg2_plana}

  % degrees of freedom    (FIT_NDF)                        : 2
  % rms of residuals      (FIT_STDFIT) = sqrt(WSSR/ndf)    : 1.0157
  % variance of residuals (reduced chisquare) = WSSR/ndf   : 1.03166

  % Final set of parameters            Asymptotic Standard Error
  % =======================            ==========================

  % c0              = 0.134326         +/- 0.02509      (18.68%)
  % c1              = 0.0160099        +/- 0.0003187    (1.991%)
  % nu              = 1.9955           +/- 0.005172     (0.2592%)


  % correlation matrix of the fit parameters:

  % c0     c1     nu     
  % c0              1.000 
  % c1             -0.955  1.000 
  % nu              0.934 -0.997  1.000 

\end{figure}



\begin{figure}[h]
  \centering
  \input{rg2_rugosa_plot.tex}
   \caption{$R_g$ en función de $L$ para $\kappa=0.5$. Resultados del ajuste
     $c_0=0.2929 \pm 0.004$ y $c_1=0.152853\pm 0.001$.$\chi^2/gl=28.94688/3$ ,
     $P(\chi>\chi^2)=2.29\ 10^{-6}$}\label{rg2_rugosa}

% degrees of freedom    (FIT_NDF)                        : 3
% rms of residuals      (FIT_STDFIT) = sqrt(WSSR/ndf)    : 3.10628
% variance of residuals (reduced chisquare) = WSSR/ndf   : 9.64896

% Final set of parameters            Asymptotic Standard Error
% =======================            ==========================

% c0              = 0.292904         +/- 0.01212      (4.138%)
% c1              = 0.152853         +/- 0.003676     (2.405%)


% correlation matrix of the fit parameters:

%                c0     c1     
% c0              1.000 
% c1             -0.990  1.000 

\end{figure}

\begin{figure}[h]
  \centering
  \input{rg2_transicion_plot.tex}
  \caption{$R_g$ en función de $L$ para $\kappa_c(L)$. Resultados del ajuste
    $c_0=-0.535 \pm 0.089$, $c_1=0.047\pm 0.006$ y $2\nu=1.38\pm
    0.03$. $\chi^2/gl=73.6932/2$, $P(\chi^2>\chi_c^2)=9.94 \ 10^{-17}$}\label{Rg2_transicion}

% degrees of freedom    (FIT_NDF)                        : 2
% rms of residuals      (FIT_STDFIT) = sqrt(WSSR/ndf)    : 6.07014
% variance of residuals (reduced chisquare) = WSSR/ndf   : 36.8466

% Final set of parameters            Asymptotic Standard Error
% =======================            ==========================

% c0              = -0.535325        +/- 0.5422       (101.3%)
% c1              = 0.0471556        +/- 0.037        (78.47%)
% nu              = 1.38315          +/- 0.1964       (14.2%)


% correlation matrix of the fit parameters:

%                c0     c1     nu     
% c0              1.000 
% c1             -0.985  1.000 
% nu              0.975 -0.998  1.000 

\end{figure}

\clearpage

\section{Transición de fase}

\subsection{Calor específico}

\begin{figure}[h]
\centering
 \input{Se_plot.tex}
\caption{Energía de curvatura $E_C$}
\end{figure}

\begin{figure}[h]
  \centering
  \input{Cv_plot.tex}
  \caption{Calor específico}
\end{figure}

\begin{figure}[h]
  \centering
  \input{max_Cv_plot.tex}
  \caption{Máximo calor específico}
\end{figure}

\begin{figure}[h]
  \centering
  \input{Cv_L_plot.tex}
  \caption{$C_v$ frente a $L$ en la transición de fase. Resultados del ajuste
    $c_0=0.76\pm 0.36$ $c_1=0.11\pm 0.09$,  $\omega=0.76\pm
    0.17$. $\chi^2/gl=0.349/2$ , $P(\chi>\chi_c^2)=0.83987691$}

% degrees of freedom    (FIT_NDF)                        : 2
% rms of residuals      (FIT_STDFIT) = sqrt(WSSR/ndf)    : 0.417849
% variance of residuals (reduced chisquare) = WSSR/ndf   : 0.174598

% Final set of parameters            Asymptotic Standard Error
% =======================            ==========================

% c0              = 0.767979         +/- 0.1513       (19.7%)
% c1              = 0.110299         +/- 0.03993      (36.21%)
% omega           = 0.759527         +/- 0.07476      (9.843%)


% correlation matrix of the fit parameters:

%                c0     c1     omega  
% c0              1.000 
% c1             -0.993  1.000 
% omega           0.986 -0.999  1.000 

\end{figure}

\begin{figure}[h]
  \centering
  \input{Cv_T_L_plot.tex}
  \caption{$\kappa_c(L)$ frente a $L$. Resultados del ajuste $c_0=0.74\pm
    0.06$ ,$c_1=0.6\pm 0.7 $ $\nu=1.628 \pm 1.508703$. $\chi^2/gl=1.254754 \
    10^{-7}/2$ , $P(\chi^2>\chi_c^2)=0.5339906$ }

% degrees of freedom    (FIT_NDF)                        : 2
% rms of residuals      (FIT_STDFIT) = sqrt(WSSR/ndf)    : 0.792071
% variance of residuals (reduced chisquare) = WSSR/ndf   : 0.627377

% Final set of parameters            Asymptotic Standard Error
% =======================            ==========================

% c0              = 0.739158         +/- 0.05025      (6.798%)
% c1              = 0.647846         +/- 0.5901       (91.09%)
% nu              = 1.62843          +/- 1.195        (73.36%)


% correlation matrix of the fit parameters:

%                c0     c1     nu     
% c0              1.000 
% c1              0.969  1.000 
% nu             -0.991 -0.993  1.000 

\end{figure}
\clearpage

\subsection{Variación del radio de giro}

\begin{figure}[h]
  \centering
  \input{Drg2_plot.tex}
  \caption{Radio de giro conexo}
\end{figure}

\begin{figure}[h]
  \centering
  \input{max_Drg2_plot.tex}
  \caption{Radio de giro}
\end{figure}

\begin{figure}[h]
  \centering
  \input{Drg2_L_plot.tex}
  \caption{$\langle R_g^2 E_c\rangle_{max}$ frente a $L$. Resultados del
    ajuste $c_0=0.001\pm 0.008$ ,$c_1=0.002\pm 0.001 $ $\omega=0.9 \pm
    0.2$. $\chi^2/gl=0.088/2$ , $P(\chi^2>\chi_c^2)=0.96$ }

% degrees of freedom    (FIT_NDF)                        : 2
% rms of residuals      (FIT_STDFIT) = sqrt(WSSR/ndf)    : 0.210417
% variance of residuals (reduced chisquare) = WSSR/ndf   : 0.0442754

% Final set of parameters            Asymptotic Standard Error
% =======================            ==========================

% c0              = 0.000996481      +/- 0.001798     (180.5%)
% c1              = 0.00200217       +/- 0.0003781    (18.89%)
% omega           = 0.906011         +/- 0.04271      (4.714%)


% correlation matrix of the fit parameters:

%                c0     c1     omega  
% c0              1.000 
% c1             -0.991  1.000 
% omega           0.983 -0.998  1.000 
% gnuplot> plot "maximos_Drg2.dat" u 1:4:5 w yerrorbars,g(x)
\end{figure}

\begin{figure}[h]
  \centering
  \input{kappac_plot.tex}
  \caption{$\kappa_c(L)\rightarrow\langle R_g^2 E_c\rangle_{max}$ frente a $L$. Resultados del ajuste $c_0=0.8\pm 0.1$ ,$c_1=0.4\pm 0.8 $ $\omega=1.6 \pm
    1.18$. $\chi^2/gl=1.81924/2$ , $P(\chi^2>\chi_c^2)=0.1774038$. (Debido a falta de convergencia se elimina el L=64)}

%Debido a falta de convergencia se elimina el L=64
% degrees of freedom    (FIT_NDF)                        : 1
% rms of residuals      (FIT_STDFIT) = sqrt(WSSR/ndf)    : 1.34879
% variance of residuals (reduced chisquare) = WSSR/ndf   : 1.81924

% Final set of parameters            Asymptotic Standard Error
% =======================            ==========================

% c0              = 0.779059         +/- 0.146        (18.74%)
% c1              = 0.441665         +/- 1.137        (257.4%)
% nu              = 1.60002          +/- 4.048        (253%)


% correlation matrix of the fit parameters:

%                c0     c1     nu     
% c0              1.000 
% c1              0.980  1.000 
% nu             -0.995 -0.995  1.000 

\end{figure}
\clearpage

%\subsection{Módulo de Poisson}

%%% Local Variables: 
%%% mode: latex
%%% TeX-master: "TFM"
%%% End: 
