\chapter{Resultados}

La siguiente tabla muestra el número de barridos totales realizados para los
diferentes tamaños de la membrana, se realizaron el mismo número de barridos
para todos $\kappa$. No se almacenaban todas las configuraciones sucesivas, en
la última fila de la tabla se muestra el número de configuraciones almacenadas
para cada tamaño. 

\begin{table}[h]
\begin{tabular}{|c|c|c|c|c|c|c|}\hline
Tamaño & $16^2$ & $24^2$ & $32^2$ & $46^2$ & $64^2$ & $128^2$ \\ \hline\hline 
barridos totales & $1.68\cdot 10^8$& $3.02\cdot 10^9$ & $7.41\cdot 10^8$ &
$6.51\cdot 10^8$&$7.212\cdot 10^8$ &$1.01\cdot10^9$\\\hline
n\textdegree configuraciones& $10^4$ & $10^4$ & $1.2\cdot 10^4$ & $7\cdot
10^3$ & $6\cdot 10^4$ & $2.5\cdot 10^3$\\ \hline
\end{tabular}
\end{table}

%  \begin{figure}[h]
% \centering
%   \input{test.tex}
%  \caption{Test}
%  \end{figure}


\section{Transición de fase}

\subsection{Energía de curvatura y calor específico}


Los resultados obtenidos para la energía de curvatura intensiva $\langle
e_c\rangle=\langle E_c\rangle/N$ se muestran en la figura \ref{Ec_fig}, donde
se representan frente a $\kappa$ y en función del tamaño de la red
$N=L^2$. Para todos los valores de $N$ medidos, tiene un comportamiento
asintótico en ambos extremos, para $\kappa\rightarrow 0$ tiende a cero, fase
rugosa con simetría total, y para $\kappa\rightarrow \infty$ tiende al valor
máximo cuyo valor depende del tamaño y viene dado por 
\begin{equation}
e^{max}_c = \frac{(L-1)(3L-5)}{L^2},
\end{equation}
y corresponde a que todas las normales unitarias de la superficies son
paralelas, fase completamente plana. El comportamiento alrededor de ambos
límites es igual para todos los tamaños, con una tendencia de aumento o
disminución de la pendiente según nos acercamos al valor asintótico nulo, fase
rugosa, o al valor máximo, fase plana. Entre ambos, tenemos un valor máximo de
la pendiente, que marca la frontera entre los dos comportamientos. Como se
observa en la figura \ref{Ec_fig}, este valor máximo de la pendiente aumenta
con el tamaño del sistema, lo que conduce a un valor infinito en el límite
termodinámico, por tanto, es una transición de segundo orden. Atendiendo al
espaciado entre los puntos de la gráfica también diferenciamos tres regiones, pues para un
$\kappa$ fijo, en la región plana está mucho más distanciados que en la fase
rugosa, y entre ambos, en transición plana-rugosa, encontramos un
comportamiento intermedio. Estos tres comportamientos vendrán caracterizados por tres
relaciones de escala diferentes propias que no han sido objeto de estudio en
este trabajo.

\begin{figure}[h]
\centering
 \input{Se_plot.tex}
\caption{Energía de curvatura $\langle e_c\rangle$ frente a $\kappa$ y en función del tamaño de la red
$N=L^2$}\label{Ec_fig}
\end{figure}

Los resultados para el calor específico se muestran la figura
\ref{Cv_fig}. Como observamos para cada tamaño tenemos un valor de $\kappa$
para el que es máximo, esto está de acuerdo con los resultados de la energía
de curvatura, puesto que el calor específico es proporcional a la derivada de
la energía de curvatura, respecto a $\kappa$, por tanto, los valores del calor
específico son proporcionales a las pendientes de la gráfica de los resultados
de $\langle e_c\rangle$ (figura \ref{Ec_fig}), las cuales tienen un valor
máximo para cada tamaño. La temperatura $\kappa_c(L)$ y valor de estos máximos
$C_V^{max}(L)$ se calculan mediante el método de la densidad espectral (sección
TAL), los resultados se muestran en la figura \ref{max_Cv_fig}. Como se
observa por el tamaño de los errores, es más fácil encontrar el valor máximo
de $C_V(L)$ que la temperatura a la que ocurren, sobre todo en los menores
tamaños. Esto es debido a que los máximos son más pronunciados a medida que
aumentamos el tamaño del sistema, por tanto, en los tamaños pequeños estarán
bastante aplanados dificultando la estimación de su posición horizontal
(temperatura $\kappa_c(L)$), no así su altura. Realizamos un ajuste
$c_0+c_1L^{c_3}$ sobre $C_V^{max}(L)$, los resultados se muestra en la tabla
\ref{max_Cv_L_tab} y la figura \ref{max_Cv_L_fig}. El valor obtenido para $P(\chi^2>\chi_c^2)$ es de
$82\%=(100\%-18\%)$, por tanto, el ajuste es aceptable tomando como nivel de
confianza el $5\%$  . Podemos estimar los exponentes críticos $\alpha$ y $\nu$,
teniendo en cuenta que el coeficiente $c_3$ es igual al cociente de estos exponentes
críticos
\begin{equation*}
 c_3=\frac{\alpha}{\nu},
\end{equation*}
y teniendo en cuenta la relación de \textit{hyperscaling}
\begin{equation*}
\alpha=2(1-\nu).
\end{equation*}
Con lo que
\begin{align*}
\alpha&=\frac{2c_3}{2+c_3},\\
\nu&=\frac{2}{2+c_3}.
\end{align*}
Obtenemos como resultado $\alpha=0.55(18)$ y $\nu=0.72(4)$, que están de
acuerdo, dentro del margen de error con simulaciones anteriores para $\alpha$: $0.5(1)$
\cite{Bowick_flat_phase}, $0,58(10)$ \cite{Wheater_Critical_exponents},
$0.44(5)$ \cite{Renken_Scaling_behavior}; y para $\nu$: $0.72(2)$
\cite{Renken_Scaling_behavior}, $0.68(10)$ \cite{Wheater_Critical_exponents},
$0.85(14)$ \cite{Espriu:MCRG}

Respecto a las temperaturas críticas aparentes $\kappa_c(L)$, se realiza un
ajuste $c_0+c_1L^{-1/\nu}$, tomando para el valor de $\nu$, el obtenido en el
ajuste de $C_V^{max}(L)$, esto es, $1/\nu=1.38(9)$. Puesto que la estimación de $\nu$ tiene error,
realizamos tres ajustes: uno con el valor central ($1.38$), y los otros dos con los
valores extremos ($1.47$ y $1.29$). Los resultados obtenidos se muestran en la
tabla \ref{kappa_Cv_tab} y la figura \ref{kappa_Cv_fig}. Debido al método
de estimación empleado, las estimaciones de los coeficientes del ajuste
tendrán dos fuentes de error: una debida al propio ajuste 
, tomaremos el máximo error de los tres ajustes; y otra debida a que hemos
tomado como parámetro de la función un valor con error, para este caso, la
máxima diferencia de la estimación correspondiente al ajuste central con los
extremos será la medida que usemos para el error. El valor
de $c_0$ corresponde a una estimación de la temperatura crítica del sistema
infinito $\kappa_c(\infty)$, encontramos un valor de
$\kappa_c(\infty)=0.777(3)(3)$, comparado con trabajos anteriores $0.814(2)$
\cite{Wheater_Critical_exponents} tenemos más de cinco desviaciones estándar
de diferencia. Las discrepancias pueden ser debidas a que hemos subestimado el
error, ya que si nuestro error fuera de $0.01$, tendríamos unas 
tres desviaciones estándar, lo que es aceptable.

\begin{figure}[h]
  \centering
  \input{Cv_plot.tex}
  \caption{Calor específico}\label{Cv_fig}
\end{figure}

\begin{figure}[h]
  \centering
  \input{max_Cv_plot.tex}
  \caption{Máximos correspondientes a cada tamaño $N=L^2$ del calor específico}\label{max_Cv_fig}
\end{figure}
\clearpage
\begin{figure}[h]
  \centering
  \input{Cv_L_plot.tex}
  \caption{Resultados del ajuste para $C_V^{max}(L)$ frente a $\kappa$ y en
    función del tamaño de la red $N=L^2$}\label{max_Cv_L_fig}
\end{figure}

\begin{table}[h]
\centering
\begin{tabular}{|c|c|}\hline
 Coeficientes/Exponentes críticos & Resultado \\\hline
 $c_0$         & $0.7702(36) $ \\ \hline
 $c_1$         & $0.1098(95)$ \\ \hline
 $c_3=\alpha/\nu$  & $0.76(18)$  \\ \hline
$\alpha$      & $0.55(18)$ \\ \hline
$\nu $        & $0.72(4)$ \\ \hline
$\chi_c^2/gl$ &  $0.38/2$   \\ \hline
 $P(\chi^2>\chi_c^2)$&  $0.8257$\\ \hline
\end{tabular}
\caption{Resultados del ajuste para $C_V^{max}(L)$, figura \ref{max_Cv_L_fig}}\label{max_Cv_L_tab}
\end{table}

\begin{figure}[h]
  \centering
  \input{Cv_T_L_plot.tex}
  \caption{Resultados del ajuste para $\kappa_c(L)$ obtenida a partir de los
  máximos de $C_V(L,\kappa)$}\label{kappa_Cv_fig}
\end{figure}

\begin{table}[h]
\centering
\begin{tabular}{|c|c|c|c|}\hline
 Ajuste              & $\frac{1}{\nu}=1.38$ & $\frac{1}{\nu}=1.38-0.09$& $\frac{1}{\nu}=1.38+0.09$ \\ \hline\hline
 $\kappa(\infty)$   & $0.777(2) $          &  $0.774(3)$              &  $0.779(3)$   \\ \hline
 $c_1$              & $4.35(59)$           &  $3.41(46)$              &   $5.55(77)$  \\ \hline
 $\chi_c^2/gl$       &  $3.75/3$            &  $3.27/3$                & $4.29/3$  \\ \hline
 $P(\chi^2>\chi_c^2)$&  $0.29$              &  $0.35$                  &   $0.23$ \\ \hline
\end{tabular}
\caption{Resultados del ajuste para $\kappa_c(L)$, obtenida a partir de los
  máximos de $C_V(L,\kappa)$, figura \ref{kappa_Cv_fig}}\label{kappa_Cv_tab}
\end{table}
\clearpage

\subsection{Variación del radio de giro}

\begin{figure}[h]
  \centering
  \input{Drg2_plot.tex}
  \caption{Radio de giro conexo}
\end{figure}

\begin{figure}[h]
  \centering
  \input{max_Drg2_plot.tex}
  \caption{Radio de giro}
\end{figure}

\begin{figure}[h]
  \centering
  \input{Drg2_L_plot.tex}
  \caption{$\langle R_g^2 E_c\rangle_{max}$ frente a $L$. Resultados del
    ajuste $c_0=0.001\pm 0.008$ ,$c_1=0.002\pm 0.001 $ $\omega=0.9 \pm
    0.2$. $\chi^2/gl=0.088/2$ , $P(\chi^2>\chi_c^2)=0.96$ }

% degrees of freedom    (FIT_NDF)                        : 2
% rms of residuals      (FIT_STDFIT) = sqrt(WSSR/ndf)    : 0.210417
% variance of residuals (reduced chisquare) = WSSR/ndf   : 0.0442754

% Final set of parameters            Asymptotic Standard Error
% =======================            ==========================

% c0              = 0.000996481      +/- 0.001798     (180.5%)
% c1              = 0.00200217       +/- 0.0003781    (18.89%)
% omega           = 0.906011         +/- 0.04271      (4.714%)


% correlation matrix of the fit parameters:

%                c0     c1     omega  
% c0              1.000 
% c1             -0.991  1.000 
% omega           0.983 -0.998  1.000 
% gnuplot> plot "maximos_Drg2.dat" u 1:4:5 w yerrorbars,g(x)
\end{figure}

\begin{figure}[h]
  \centering
  \input{kappac_plot.tex}
  \caption{$\kappa_c(L)\rightarrow\langle R_g^2 E_c\rangle_{max}$ frente a $L$. Resultados del ajuste $c_0=0.8\pm 0.1$ ,$c_1=0.4\pm 0.8 $ $\omega=1.6 \pm
    1.18$. $\chi^2/gl=1.81924/2$ , $P(\chi^2>\chi_c^2)=0.1774038$. (Debido a falta de convergencia se elimina el L=64)}

%Debido a falta de convergencia se elimina el L=64
% degrees of freedom    (FIT_NDF)                        : 1
% rms of residuals      (FIT_STDFIT) = sqrt(WSSR/ndf)    : 1.34879
% variance of residuals (reduced chisquare) = WSSR/ndf   : 1.81924

% Final set of parameters            Asymptotic Standard Error
% =======================            ==========================

% c0              = 0.779059         +/- 0.146        (18.74%)
% c1              = 0.441665         +/- 1.137        (257.4%)
% nu              = 1.60002          +/- 4.048        (253%)


% correlation matrix of the fit parameters:

%                c0     c1     nu     
% c0              1.000 
% c1              0.980  1.000 
% nu             -0.995 -0.995  1.000 

\end{figure}
\clearpage

\section{Estudios de las fases}

\subsection{Relaciones de escala Radio de giro}

Como se aprecia en la figura \ref{radio-giro-fig} el radio de giro sufre un
cambio drástico aproximadamente en $\kappa\simeq 0.8$, que separa las dos
fases: plana para $\kappa> 0.8$ y rugosa  para $\kappa< 0.8$. Para estudiar la
relación de escala en ambas fases tomamos los valores de $\kappa$ que tenemos
más alejados de $0.8$, que son $\kappa=0.5$ para la fase rugosa y $\kappa=2.0$
para la fase plana. En la fase rugosa observamos efectivamente sigue la
relación de escala $R_G\sim (\log L)^{1/2}$ (figura \ref{rg2_rugosa}), y la
fase plana $R_G^2\sim L^{2\nu}$ (figura \ref{Rg2_plana}) con $2\nu=1.9955 \pm
0.005$. Respecto a la transición de fase encontramos un valor de $2\nu=1.38\pm
0.03\rightarrow \nu=0.69\pm 0.02$, el valor teórico predicho es de $\nu=0.76$,
y simulaciones anteriores mediante simulaciones de Monte-Carlo del grupo de
renormalización Monte-Carlo es $\nu=0.76\pm 0.01$,  nuestro valor es
ligeramente menor, y aunque más preciso, no es muy fiable, pues el ajuste no
 es muy bueno como indica el valor alto de $\chi^2$. 

\begin{figure}[h]
  \centering
  \input{Rg2_plot.tex}
  \caption{Radio de giro}\label{radio-giro-fig}
\end{figure}

\begin{figure}[h]
  \centering
  \input{rg2_plano_plot.tex}
  \caption{$R^2_g$ en función de $L$ para $\kappa=2.0$. Resultados del ajuste
    $c_0=0.134\pm 0.025$, $c_1=0.0160\pm 0.0003$, $2\nu=1.9955 \pm
    0.005$. $\chi^2/gl=2.063/2$ , $P(\chi^2>\chi_C^2)=0.356$}\label{Rg2_plana}

  % degrees of freedom    (FIT_NDF)                        : 2
  % rms of residuals      (FIT_STDFIT) = sqrt(WSSR/ndf)    : 1.0157
  % variance of residuals (reduced chisquare) = WSSR/ndf   : 1.03166

  % Final set of parameters            Asymptotic Standard Error
  % =======================            ==========================

  % c0              = 0.134326         +/- 0.02509      (18.68%)
  % c1              = 0.0160099        +/- 0.0003187    (1.991%)
  % nu              = 1.9955           +/- 0.005172     (0.2592%)


  % correlation matrix of the fit parameters:

  % c0     c1     nu     
  % c0              1.000 
  % c1             -0.955  1.000 
  % nu              0.934 -0.997  1.000 

\end{figure}



\begin{figure}[h]
  \centering
  \input{rg2_rugosa_plot.tex}
   \caption{$R_g$ en función de $L$ para $\kappa=0.5$. Resultados del ajuste
     $c_0=0.2929 \pm 0.004$ y $c_1=0.152853\pm 0.001$.$\chi^2/gl=28.94688/3$ ,
     $P(\chi>\chi^2)=2.29\ 10^{-6}$}\label{rg2_rugosa}

% degrees of freedom    (FIT_NDF)                        : 3
% rms of residuals      (FIT_STDFIT) = sqrt(WSSR/ndf)    : 3.10628
% variance of residuals (reduced chisquare) = WSSR/ndf   : 9.64896

% Final set of parameters            Asymptotic Standard Error
% =======================            ==========================

% c0              = 0.292904         +/- 0.01212      (4.138%)
% c1              = 0.152853         +/- 0.003676     (2.405%)


% correlation matrix of the fit parameters:

%                c0     c1     
% c0              1.000 
% c1             -0.990  1.000 

\end{figure}

\begin{figure}[h]
  \centering
  \input{rg2_transicion_plot.tex}
  \caption{$R_g$ en función de $L$ para $\kappa_c(L)$. Resultados del ajuste
    $c_0=-0.535 \pm 0.089$, $c_1=0.047\pm 0.006$ y $2\nu=1.38\pm
    0.03$. $\chi^2/gl=73.6932/2$, $P(\chi^2>\chi_c^2)=9.94 \ 10^{-17}$}\label{Rg2_transicion}

% degrees of freedom    (FIT_NDF)                        : 2
% rms of residuals      (FIT_STDFIT) = sqrt(WSSR/ndf)    : 6.07014
% variance of residuals (reduced chisquare) = WSSR/ndf   : 36.8466

% Final set of parameters            Asymptotic Standard Error
% =======================            ==========================

% c0              = -0.535325        +/- 0.5422       (101.3%)
% c1              = 0.0471556        +/- 0.037        (78.47%)
% nu              = 1.38315          +/- 0.1964       (14.2%)


% correlation matrix of the fit parameters:

%                c0     c1     nu     
% c0              1.000 
% c1             -0.985  1.000 
% nu              0.975 -0.998  1.000 

\end{figure}

\clearpage


\subsection{Fase plana: Módulo de Poisson}

La siguiente figura \ref{poisson-fig} muestra como la tendencia asintótica del
módulo de Poisson 
\begin{figure}[h]
  \centering
  \input{poisson_plot.tex}
  \caption{Módulo de Poisson}\label{poisson-fig}
\end{figure}
%%% Local Variables: 
%%% mode: latex
%%% TeX-master: "TFM"
%%% End: 
