\chapter{Resultados}
\begin{figure}[h]
\centering
 \input{test.tex}
\caption{Test}
\end{figure}
\subsection{Calor específico}
\begin{figure}[h]
\centering
 \input{Se_plot.tex}
\caption{Energía Elástica}
\end{figure}

\begin{figure}[h]
  \centering
  \input{Cv_plot.tex}
  \caption{Calor específico}
\end{figure}

\begin{figure}[h]
  \centering
  \input{max_Cv_plot.tex}
  \caption{Máximo calor específico}
\end{figure}

\begin{figure}[h]
  \centering
  \input{Cv_L_plot.tex}
  \caption{Máximo Calor específico frente a L}
\end{figure}

\begin{figure}[h]
  \centering
  \input{Cv_T_L_plot.tex}
  \caption{$\kappa_c$ (Calor específico) frente a L}
\end{figure}
\clearpage
\subsection{Radio de giro}
\begin{figure}[h]
  \centering
  \input{Rg2_plot.tex}
  \caption{Radio de giro}
\end{figure}

\begin{figure}[h]
  \centering
  \input{Drg2_plot.tex}
  \caption{Radio de giro conexo}
\end{figure}

\begin{figure}[h]
  \centering
  \input{max_Drg2_plot.tex}
  \caption{Radio de giro}
\end{figure}

\begin{figure}[h]
  \centering
  \input{Drg2_L_plot.tex}
  \caption{Radio de giro}
\end{figure}

\begin{figure}[h]
  \centering
  \input{kappac_plot.tex}
  \caption{Temperatura crítica}
\end{figure}
\clearpage
\subsection{Módulo de Poisson}

%%% Local Variables: 
%%% mode: latex
%%% TeX-master: "TFM"
%%% End: 
