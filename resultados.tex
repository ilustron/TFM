\chapter{Resultados}
\begin{figure}[h]
\centering
 \input{test.tex}
\caption{Test}
\end{figure}
\subsection{Calor específico}
\begin{figure}[h]
\centering
 \input{Se_plot.tex}
\caption{Energía de curvatura $E_C$}
\end{figure}

\begin{figure}[h]
  \centering
  \input{Cv_plot.tex}
  \caption{Calor específico}
\end{figure}

\begin{figure}[h]
  \centering
  \input{max_Cv_plot.tex}
  \caption{Máximo calor específico}
\end{figure}

\begin{figure}[h]
  \centering
  \input{Cv_L_plot.tex}
  \caption{$C_v$ frente a $L$ en la transición de fase. Resultados del ajuste
    $c_0=0.8\pm 0.4$ $c_1=0.1\pm 0.1$,  $\omega=0.8\pm 0.2$ }
\end{figure}

\begin{figure}[h]
  \centering
  \input{Cv_T_L_plot.tex}
  \caption{$\kappa_c$ estimada frente a $L$. Resultados del ajuste $c_0=0.773\pm 0.007$ y $\nu=1.04\pm 0.08$ } 
\end{figure}
\clearpage
\subsection{Radio de giro}
\begin{figure}[h]
  \centering
  \input{Rg2_plot.tex}
  \caption{Radio de giro}
\end{figure}

\begin{figure}[h]
  \centering
  \input{rg2_plano_plot.tex}
  \caption{$R^2_g$ en función de $L$ para $\kappa=2.0$. Resultados del ajuste
    $a=0.0177\pm 0.0004$ y $a=1.969\pm 0.002$}
\end{figure}

\begin{figure}[h]
  \centering
  \input{rg2_rugosa_plot.tex}
  \caption{$R^2_g$ en función de $L$ para $\kappa=0.5$. Resultados del ajuste
    $a=0.175\pm 0.005$ y $b=0.247\pm 0.002$}
\end{figure}
\subsection{Radio de giro conexo}
\begin{figure}[h]
  \centering
  \input{Drg2_plot.tex}
  \caption{Radio de giro conexo}
\end{figure}

\begin{figure}[h]
  \centering
  \input{max_Drg2_plot.tex}
  \caption{Radio de giro}
\end{figure}

\begin{figure}[h]
  \centering
  \input{Drg2_L_plot.tex}
  \caption{Resultados del ajuste $a=0.0022\pm 2\cdot 10^{-4}$ y $b=0.88\pm 0.04$ }
\end{figure}

\begin{figure}[h]
  \centering
  \input{kappac_plot.tex}
  \caption{Temperatura crítica, resultados del ajuste $c_0=0.778\pm 0.007$ y $\nu=1.125\pm 0.0599$ }
\end{figure}
\clearpage
\subsection{Módulo de Poisson}

%%% Local Variables: 
%%% mode: latex
%%% TeX-master: "TFM"
%%% End: 
