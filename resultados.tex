\chapter{Resultados}

La siguiente tabla muestra el número de barridos totales realizados para los
diferentes tamaños de la membrana, se realizaron el mismo número de barridos
para todos $\kappa$. No se almacenaban todas las configuraciones sucesivas, en
la última fila de la tabla se muestra el número de configuraciones almacenadas
para cada tamaño. 

\begin{table}[h]
\begin{tabular}{|c|c|c|c|c|c|c|}\hline
Tamaño & $16^2$ & $24^2$ & $32^2$ & $46^2$ & $64^2$ & $128^2$ \\ \hline\hline 
barridos totales & $1.68\cdot 10^8$& $3.02\cdot 10^9$ & $7.41\cdot 10^8$ &
$6.51\cdot 10^8$&$7.212\cdot 10^8$ &$1.01\cdot10^9$\\\hline
n\textdegree configuraciones& $10^4$ & $10^4$ & $1.2\cdot 10^4$ & $7\cdot
10^3$ & $6\cdot 10^4$ & $2.5\cdot 10^3$\\ \hline
\end{tabular}
\end{table}


  \begin{figure}[h]
 \centering
   \input{test.tex}
  \caption{Test}
  \end{figure}

\section{Relaciones de escala Radio de giro}

Como se aprecia en la figura \ref{radio-giro-fig} el radio de giro sufre un
cambio drástico aproximadamente en $\kappa\simeq 0.8$, que separa las dos
fases: plana para $\kappa> 0.8$ y rugosa  para $\kappa< 0.8$. Para estudiar la
relación de escala en ambas fases tomamos los valores de $\kappa$ que tenemos
más alejados de $0.8$, que son $\kappa=0.5$ para la fase rugosa y $\kappa=2.0$
para la fase plana. En la fase rugosa observamos efectivamente sigue la
relación de escala $R_G\sim (\log L)^{1/2}$ (figura \ref{rg2_rugosa}), y la
fase plana $R_G^2\sim L^{2\nu}$ (figura \ref{Rg2_plana}) con $2\nu=1.9955 \pm
0.005$. Respecto a la transición de fase encontramos un valor de $2\nu=1.38\pm
0.03\rightarrow \nu=0.69\pm 0.02$, el valor teórico predicho es de $\nu=0.76$,
y simulaciones anteriores mediante simulaciones de Monte-Carlo del grupo de
renormalización Monte-Carlo es $\nu=0.76\pm 0.01$,  nuestro valor es
ligeramente menor, y aunque más preciso, no es muy fiable, pues el ajuste no
 es muy bueno como indica el valor alto de $\chi^2$. 

\begin{figure}[h]
  \centering
  \input{Rg2_plot.tex}
  \caption{Radio de giro}\label{radio-giro-fig}
\end{figure}

\begin{figure}[h]
  \centering
  \input{rg2_plano_plot.tex}
  \caption{$R^2_g$ en función de $L$ para $\kappa=2.0$. Resultados del ajuste
    $c_0=0.134\pm 0.025$, $c_1=0.0160\pm 0.0003$, $2\nu=1.9955 \pm
    0.005$. $\chi^2/gl=2.063/2$ , $P(\chi^2>\chi_C^2)=0.356$}\label{Rg2_plana}

  % degrees of freedom    (FIT_NDF)                        : 2
  % rms of residuals      (FIT_STDFIT) = sqrt(WSSR/ndf)    : 1.0157
  % variance of residuals (reduced chisquare) = WSSR/ndf   : 1.03166

  % Final set of parameters            Asymptotic Standard Error
  % =======================            ==========================

  % c0              = 0.134326         +/- 0.02509      (18.68%)
  % c1              = 0.0160099        +/- 0.0003187    (1.991%)
  % nu              = 1.9955           +/- 0.005172     (0.2592%)


  % correlation matrix of the fit parameters:

  % c0     c1     nu     
  % c0              1.000 
  % c1             -0.955  1.000 
  % nu              0.934 -0.997  1.000 

\end{figure}



\begin{figure}[h]
  \centering
  \input{rg2_rugosa_plot.tex}
   \caption{$R_g$ en función de $L$ para $\kappa=0.5$. Resultados del ajuste
     $c_0=0.2929 \pm 0.004$ y $c_1=0.152853\pm 0.001$.$\chi^2/gl=28.94688/3$ ,
     $P(\chi>\chi^2)=2.29\ 10^{-6}$}\label{rg2_rugosa}

% degrees of freedom    (FIT_NDF)                        : 3
% rms of residuals      (FIT_STDFIT) = sqrt(WSSR/ndf)    : 3.10628
% variance of residuals (reduced chisquare) = WSSR/ndf   : 9.64896

% Final set of parameters            Asymptotic Standard Error
% =======================            ==========================

% c0              = 0.292904         +/- 0.01212      (4.138%)
% c1              = 0.152853         +/- 0.003676     (2.405%)


% correlation matrix of the fit parameters:

%                c0     c1     
% c0              1.000 
% c1             -0.990  1.000 

\end{figure}

\begin{figure}[h]
  \centering
  \input{rg2_transicion_plot.tex}
  \caption{$R_g$ en función de $L$ para $\kappa_c(L)$. Resultados del ajuste
    $c_0=-0.535 \pm 0.089$, $c_1=0.047\pm 0.006$ y $2\nu=1.38\pm
    0.03$. $\chi^2/gl=73.6932/2$, $P(\chi^2>\chi_c^2)=9.94 \ 10^{-17}$}\label{Rg2_transicion}

% degrees of freedom    (FIT_NDF)                        : 2
% rms of residuals      (FIT_STDFIT) = sqrt(WSSR/ndf)    : 6.07014
% variance of residuals (reduced chisquare) = WSSR/ndf   : 36.8466

% Final set of parameters            Asymptotic Standard Error
% =======================            ==========================

% c0              = -0.535325        +/- 0.5422       (101.3%)
% c1              = 0.0471556        +/- 0.037        (78.47%)
% nu              = 1.38315          +/- 0.1964       (14.2%)


% correlation matrix of the fit parameters:

%                c0     c1     nu     
% c0              1.000 
% c1             -0.985  1.000 
% nu              0.975 -0.998  1.000 

\end{figure}

\clearpage

\section{Transición de fase}

\subsection{Calor específico}

Puesto que la energía de curvatura está acotada, tiene una valor nulo
en la fase completamente desordenada y tiene un valor máximo 
\begin{equation}\label{max_ecurvatura}
e^{max}_c = \frac{(L-1)(3L-5)}{L^2}
\end{equation}
en la fase completamente plana. Definiendo la energía de
curvatura intensiva normalizada $e_c^{norm}$  como el cociente
$e_c/ e_c^{max}$ . Los resultados obtenidos para
esta cantidad en figura \eqref{Ec_fig}, donde se
representan frente a $\kappa$ y en función del tamaño de
la red. En todos los tamaños, $\langle e_c\rangle_{norm}$ tiende a la unidad para
$\kappa\rightarrow \infty$, completamente plana, y a cero para
$\kappa\rightarrow 0$, completamente rugosa. El comportamiento alrededor de ambos
límites es igual para todos los tamaños es similar, con una tendencia de aumento o
disminución de la pendiente según nos acercamos al valor asintótico nulo, fase
rugosa, o a uno, fase plana. Únicamente encontramos
diferencias significativas entre los diferentes tamaños del sistema en la
región intermedia, donde se da el valor 
máximo de la pendiente, que marca la frontera entre los dos
comportamientos. Este valor máximo es mayor cuanto mayor es el tamaño del
sistema, por tanto, para un tamaño infinito del sistema tendremos un valor
infinito, que caracteriza las transiciones de segundo orden. La finitud de
esta pendiente es debido a los efectos de tamaño finito del sistema. 


\begin{table}
\begin{tabular}{|c|c|c|c|}\hline
 Ajuste   & Todos los puntos   & Eliminando $L=64$    & fijando $c_0=0$\\ \hline
 $c_0$    & $0.7702\pm 0.069 $ &  $0.4759\pm 0.0083 $ & $0.0$ \\ \hline
 $c_1$    & $0.1098 \pm 0.018$ &  $0.21471\pm 0.0032$ & $0.4032\pm 0.0219$ \\ \hline
 $\omega$ & $0.7602\pm 0.034$  &  $ 0.618\pm0.003$    & $0.506\pm 0.015$ \\ \hline
 $\chi_c^2/gl$ &  $0.383/2$ & $0.02102/2$ &  $1.597/3$ \\ \hline
 $P(\chi^2>\chi_c^2)$&  $0.8257$& $0.8847$ &  $0.660$ \\ \hline
\end{tabular}
\caption{Resultados del ajuste para $C_V^{max}(L)$, figura \ref{max_Cv_L_fig}}
\end{table}


\begin{table}
\begin{tabular}{|c|c|c|c|}\hline
 Ajuste   & Todos los puntos   \\ \hline
 $\kappa(\infty)$    & $0.7455\pm 0.045 $ \\ \hline
 $c_1$    & $0.745 \pm 0.799$ \\ \hline
 $\nu$ & $1.46\pm 1.137$  & \\ \hline
 $\chi_c^2/gl$ &  $1.7745/2$ & \\ \hline
 $P(\chi^2>\chi_c^2)$&  $0.4127$\\ \hline
\end{tabular}
\caption{Resultados del ajuste para $\kappa_c(L)$ obtenida a partir de los
  máximos de $C_V(L,\kappa)$, figura \ref{kappa_Cv_fig}}
\end{table}

\begin{figure}[h]
\centering
 \input{Se_plot.tex}
\caption{Energía de curvatura $\langle e_c\rangle_{norm}$}\label{Ec_fig}
\end{figure}

\begin{figure}[h]
  \centering
  \input{Cv_plot.tex}
  \caption{Calor específico}\label{Cv_fig}
\end{figure}

\begin{figure}[h]
  \centering
  \input{max_Cv_plot.tex}
  \caption{Máximo calor específico}\label{max_Cv_fig}
\end{figure}

\begin{figure}[h]
  \centering
  \input{Cv_L_plot.tex}
  \caption{Resultados del ajuste para $C_V^{max}(L)$}\label{max_Cv_L_fig}
\end{figure}

\begin{figure}[h]
  \centering
  \input{Cv_T_L_plot.tex}
  \caption{Resultados del ajuste para $\kappa_c(L)$ obtenida a partir de los
  máximos de $C_V(L,\kappa)$}\label{kappa_Cv_fig}
\end{figure}
\clearpage

\subsection{Variación del radio de giro}

\begin{figure}[h]
  \centering
  \input{Drg2_plot.tex}
  \caption{Radio de giro conexo}
\end{figure}

\begin{figure}[h]
  \centering
  \input{max_Drg2_plot.tex}
  \caption{Radio de giro}
\end{figure}

\begin{figure}[h]
  \centering
  \input{Drg2_L_plot.tex}
  \caption{$\langle R_g^2 E_c\rangle_{max}$ frente a $L$. Resultados del
    ajuste $c_0=0.001\pm 0.008$ ,$c_1=0.002\pm 0.001 $ $\omega=0.9 \pm
    0.2$. $\chi^2/gl=0.088/2$ , $P(\chi^2>\chi_c^2)=0.96$ }

% degrees of freedom    (FIT_NDF)                        : 2
% rms of residuals      (FIT_STDFIT) = sqrt(WSSR/ndf)    : 0.210417
% variance of residuals (reduced chisquare) = WSSR/ndf   : 0.0442754

% Final set of parameters            Asymptotic Standard Error
% =======================            ==========================

% c0              = 0.000996481      +/- 0.001798     (180.5%)
% c1              = 0.00200217       +/- 0.0003781    (18.89%)
% omega           = 0.906011         +/- 0.04271      (4.714%)


% correlation matrix of the fit parameters:

%                c0     c1     omega  
% c0              1.000 
% c1             -0.991  1.000 
% omega           0.983 -0.998  1.000 
% gnuplot> plot "maximos_Drg2.dat" u 1:4:5 w yerrorbars,g(x)
\end{figure}

\begin{figure}[h]
  \centering
  \input{kappac_plot.tex}
  \caption{$\kappa_c(L)\rightarrow\langle R_g^2 E_c\rangle_{max}$ frente a $L$. Resultados del ajuste $c_0=0.8\pm 0.1$ ,$c_1=0.4\pm 0.8 $ $\omega=1.6 \pm
    1.18$. $\chi^2/gl=1.81924/2$ , $P(\chi^2>\chi_c^2)=0.1774038$. (Debido a falta de convergencia se elimina el L=64)}

%Debido a falta de convergencia se elimina el L=64
% degrees of freedom    (FIT_NDF)                        : 1
% rms of residuals      (FIT_STDFIT) = sqrt(WSSR/ndf)    : 1.34879
% variance of residuals (reduced chisquare) = WSSR/ndf   : 1.81924

% Final set of parameters            Asymptotic Standard Error
% =======================            ==========================

% c0              = 0.779059         +/- 0.146        (18.74%)
% c1              = 0.441665         +/- 1.137        (257.4%)
% nu              = 1.60002          +/- 4.048        (253%)


% correlation matrix of the fit parameters:

%                c0     c1     nu     
% c0              1.000 
% c1              0.980  1.000 
% nu             -0.995 -0.995  1.000 

\end{figure}
\clearpage

\subsection{Módulo de Poisson}

La siguiente figura \ref{poisson-fig} muestra como la tendencia asintótica del
módulo de Poisson 
\begin{figure}[h]
  \centering
  \input{poisson_plot.tex}
  \caption{Módulo de Poisson}\label{poisson-fig}
\end{figure}
%%% Local Variables: 
%%% mode: latex
%%% TeX-master: "TFM"
%%% End: 
