\chapter{Resultados}

La siguiente tabla \ref{tabla_sweeps} muestra el número de actualizacións totales realizados para los
diferentes tamaños de la membrana, se realizaron el mismo número de actualizaciones
para todos $\kappa$. No se almacenaban todas las configuraciones sucesivas, en
la última fila de la tabla se muestra el número de configuraciones almacenadas
para cada tamaño. 

\begin{table}[h]
\centering
\begin{tabular}{|c|c|c|c|}\hline
Tamaño & $16^2$ & $24^2$ & $32^2$ \\ \hline\hline 
actualizaciones totales & $1.68\times 10^8$& $3.02\times 10^9$ & $7.41\times 10^8$ \\\hline
n\textdegree configuraciones& $10^4$ & $10^4$ & $1.2\times 10^4$ \\ \hline
\end{tabular}\vspace{0.4cm}

\begin{tabular}{|c|c|c|c|}\hline
Tamaño &  $46^2$ & $64^2$ & $128^2$ \\ \hline\hline 
actualizaciones totales & $6.51\times 10^8$&$7.212\times 10^8$ &$1.01\times10^9$\\\hline
n\textdegree configuraciones&  $7\times 10^3$ & $6\times 10^3$ & $2.5\times 10^3$\\ \hline
\end{tabular}
\caption{Número de actualizaciones totales y configuraciones almacenadas para
  las simulaciones numéricas de cada tamaño de la membrana.}\label{tabla_sweeps}
\end{table}

%  \begin{figure}[h]
% \centering
%   \input{test.tex}
%  \caption{Test}
%  \end{figure}


\section{Transición de fase}

\subsection{Energía de curvatura y calor específico}

Los resultados obtenidos para la energía de curvatura intensiva $\langle
e_c\rangle=\langle E_c\rangle/N$ se muestran en la figura \ref{Ec_fig}, donde
se representan frente a $\kappa$ y en función del tamaño de la red
$N=L^2$. Para todos los valores de $N$ medidos, tiene un comportamiento
asintótico en ambos extremos, para $\kappa\rightarrow 0$ tiende a cero, fase
rugosa con simetría total, y para $\kappa\rightarrow \infty$ tiende al valor
máximo cuyo valor depende del tamaño y viene dado por 
\begin{equation}
e^{max}_c = \frac{(L-1)(3L-5)}{L^2},
\end{equation}
y corresponde a que todas las normales unitarias de la superficies son
paralelas, fase completamente plana. El comportamiento alrededor de ambos
límites es igual para todos los tamaños, con una tendencia de aumento o
disminución de la pendiente según nos acercamos al valor asintótico nulo, fase
rugosa, o al valor máximo, fase plana, respectivamente. Entre ambos, tenemos un valor máximo de
la pendiente, que marca la frontera entre los dos comportamientos. Como se
observa en la figura \ref{Ec_fig}, este valor máximo de la pendiente aumenta
con el tamaño del sistema, lo que conduce a un valor infinito en el límite
termodinámico, por tanto, es una transición de segundo orden. Atendiendo al
espaciado entre los puntos de la gráfica también diferenciamos tres regiones, pues para un
$\kappa$ fijo, en la región plana está mucho más distanciados que en la fase
rugosa, y entre ambos, en transición plana-rugosa, encontramos un
comportamiento intermedio. Estos tres comportamientos vendrán caracterizados por tres
relaciones de escala diferentes propias que no han sido objeto de estudio en
este trabajo.

\begin{figure}[h]
\centering
 \input{Se_plot.tex}
\caption{Energía de curvatura $\langle e_c\rangle$ frente a $\kappa$ y en función del tamaño de la red
$N=L^2$.}\label{Ec_fig}
\end{figure}

Los resultados para el calor específico se muestran la figura
\ref{Cv_fig}. Como observamos para cada tamaño tenemos un valor de $\kappa$
para el que es máximo, esto está de acuerdo con los resultados de la energía
de curvatura, puesto que el calor específico es proporcional a la derivada de
la energía de curvatura, respecto a $\kappa$, por tanto, los valores del calor
específico son proporcionales a las pendientes de la gráfica de los resultados
de $\langle e_c\rangle$ (figura \ref{Ec_fig}), las cuales tienen un valor
máximo para cada tamaño. La temperatura $\kappa_c(L)$ y valor de estos máximos
$C_V^{max}(L)$ se calculan mediante el método de la densidad espectral (sección
\ref{densidad_espectral-sec}), los resultados se muestran en la figura \ref{max_Cv_fig}. Como se
observa por el tamaño de los errores, es más fácil encontrar el valor máximo
de $C_V(L)$ que la temperatura a la que ocurren, sobre todo en los menores
tamaños. Esto es debido a que los máximos son más pronunciados a medida que
aumentamos el tamaño del sistema, por tanto, en los tamaños pequeños estarán
bastante aplanados dificultando la estimación de su posición horizontal
(temperatura $\kappa_c(L)$), no así su altura. Realizamos un ajuste
$c_0+c_1L^{c_3}$ sobre $C_V^{max}(L)$, los resultados se muestra en la figura
\ref{max_Cv_L_fig} y en la tabla \ref{max_Cv_L_tab}, donde $\mathrm{ngl}$ son
los grados de libertad del ajuste y $\chi^2_c$ es la varianza de los residuos,
que sigue la distribución $\chi^2$ cuya esperanza es precisamente
$\mathrm{ngl}$. Por tanto, esperamos obtener un valor para $\chi^2_c$ cercano
a $\mathrm{ngl}$, aún así, siempre calculamos la probabilidad
$P(\chi^2>\chi_c^2)$, de manera que si es menor de $0.05$ o mayor de $0.95$,
indica un valor demasiado extraño, lo que implica que el ajuste no es
significativo. En este caso encontramos un valor de $82\%=(100\%-18\%)$, por
tanto, el ajuste es aceptable. Podemos estimar los exponentes críticos $\alpha$ y $\nu$,
teniendo en cuenta que el coeficiente $c_3$ es igual al cociente de estos exponentes
críticos
\begin{equation*}
 c_3=\frac{\alpha}{\nu},
\end{equation*}
y la relación de \textit{hyperscaling} con $D=2$ \cite{Cardy} 
\begin{equation*}
\alpha=2(1-\nu),
\end{equation*}
llegamos a que
\begin{align*}
\alpha&=\frac{2c_3}{2+c_3},\\
\nu&=\frac{2}{2+c_3}.
\end{align*}
Obtenemos como resultado $\alpha=0.55(18)$ y $\nu=0.72(4)$, que están de
acuerdo, dentro del margen de error con simulaciones anteriores para $\alpha$: $0.5(1)$
\cite{Bowick_flat_phase}, $0,58(10)$ \cite{Wheater_Critical_exponents},
$0.44(5)$ \cite{Renken_Scaling_behavior}; y para $\nu$: $0.72(2)$
\cite{Renken_Scaling_behavior}, $0.68(10)$ \cite{Wheater_Critical_exponents},
$0.85(14)$ \cite{Espriu:MCRG}.

Respecto a las temperaturas críticas aparentes $\kappa_c(L)$, se realiza un
ajuste $c_0+c_1L^{-1/\nu}$, tomando para el valor de $\nu$, el obtenido en el
ajuste de $C_V^{max}(L)$, esto es, $1/\nu=1.38(9)$. Puesto que este valor de
$\nu$ está afectado de error, realizamos tres ajustes: uno con el valor
central ($1.38$), y los otros dos con los valores extremos ($1.47$ y
$1.29$). Los resultados obtenidos se muestran 
la figura \ref{kappa_Cv_fig} y en la tabla \ref{kappa_Cv_tab}. Las
estimaciones de los coeficientes del ajuste 
tienen dos fuentes de error: una debida al propio ajuste  
, tomaremos el máximo error de los tres ajustes; y otra debida a que hemos
usado como parámetro de la función de ajuste una cantidad con error, para este caso, la
máxima diferencia de la estimación de los coeficientes, respecto al ajuste central, de los
ajustes extremos será la medida que usemos para el error. El valor
de $c_0$ corresponde a una estimación de la temperatura crítica del sistema
infinito $\kappa_c(\infty)$, encontramos un valor de
$\kappa_c(\infty)=0.777(3)(3)$, comparado con trabajos anteriores $0.814(2)$
\cite{Wheater_Critical_exponents} tenemos más de cinco desviaciones estándar
de diferencia. En general, los resultados citados en la literatura sufren de
análisis inadecuados y los errores seguramente son subestimados, y esta sea
probablemente la causa de las discrepancias.

\begin{figure}[h]
  \centering
  \input{Cv_plot.tex}
  \caption{Calor específico frente a la temperatura $\kappa$ y en función del
    tamaño $N=L^2$.}\label{Cv_fig}
\end{figure}

\begin{figure}[h]
  \centering
  \input{max_Cv_plot.tex}
  \caption{Máximos correspondientes a cada tamaño $N=L^2$ del calor específico.}\label{max_Cv_fig}
\end{figure}
\clearpage
\begin{figure}[h]
  \centering
  \input{Cv_L_plot.tex}
  \caption{Resultados del ajuste para $C_V^{max}(L)$ frente a $\kappa$ y en
    función del tamaño de la red $N=L^2$.}\label{max_Cv_L_fig}
\end{figure}

\begin{table}[h]
\centering
\begin{tabular}{|c|c|}\hline
 Coeficientes/Exponentes críticos & Resultado \\\hline
 $c_0$         & $0.7702(36) $ \\ \hline
 $c_1$         & $0.1098(95)$ \\ \hline
 $c_3=\alpha/\nu$  & $0.76(18)$  \\ \hline
$\alpha$      & $0.55(18)$ \\ \hline
$\nu $        & $0.72(4)$ \\ \hline
$\chi_c^2/\mathrm{ngl}$ &  $0.38/2$   \\ \hline
 $P(\chi^2>\chi_c^2)$&  $0.8257$\\ \hline
\end{tabular}
\caption{Resultados del ajuste para $C_V^{max}(L)$, figura \ref{max_Cv_L_fig}.}\label{max_Cv_L_tab}
\end{table}

\begin{figure}[h]
  \centering
  \input{Cv_T_L_plot.tex}
  \caption{Resultados del ajuste para $\kappa_c(L)$ obtenida a partir de los
  máximos de $C_V(L,\kappa)$.}\label{kappa_Cv_fig}
\end{figure}

\begin{table}[h]
\centering
\begin{tabular}{|c|c|c|c|}\hline
 Ajuste              & $\frac{1}{\nu}=1.38$ & $\frac{1}{\nu}=1.38-0.09$& $\frac{1}{\nu}=1.38+0.09$ \\ \hline\hline
 $c_0=\kappa(\infty)$   & $0.777(2) $          &  $0.774(3)$              &  $0.779(3)$   \\ \hline
 $c_1$              & $4.35(59)$           &  $3.41(46)$              &   $5.55(77)$  \\ \hline
 $\chi_c^2/\mathrm{ngl}$       &  $3.75/3$            &  $3.27/3$                & $4.29/3$  \\ \hline
 $P(\chi^2>\chi_c^2)$&  $0.29$              &  $0.35$                  &   $0.23$ \\ \hline
\end{tabular}
\caption{Resultados del ajuste para $\kappa_c(L)$, obtenida a partir de los
  máximos de $C_V(L,\kappa)$, figura \ref{kappa_Cv_fig}.}\label{kappa_Cv_tab}
\end{table}
\clearpage

\subsection{Variación del radio de giro}

Como se aprecia en la figura \ref{radio-giro-fig} el radio de giro sufre un
cambio drástico aproximadamente en $\kappa\simeq 0.8$, en concordancia con
nuestra estimación de la temperatura crítica anterior $0.777(3)(3)$. Es valor separa las dos
fases: plana para $\kappa> 0.8$ y rugosa  para $\kappa< 0.8$. El
comportamiento del radio de giro es análogo al de la energía de curvatura,
para valores bajos de $\kappa$ tiende hacia un valor nulo y para valores altos
de $\kappa$ tienden hacia un valor asintótico no nulo, que aumenta con el
tamaño del sistema. Frontera entre ambos comportamientos es la temperatura a
la que tiene lugar la pendiente máxima. Esta pendiente viene dada por la
derivada
\begin{equation}
\frac{\partial R_g^2}{\partial \kappa}\equiv\langle R_g^2E_C \rangle_c=L^2 \langle R_g^2e_c \rangle_c.
\end{equation}
En la figura \ref{varradio-giro-fig} se representa $\langle R_g^2e_c 
\rangle_c$ en función de $\kappa$ para cada tamaño de la red $N=L^2$. Como se observa, efectivamente, 
para cada tamaño, $\langle R_g^2e_c \rangle_c$ presenta un máximo en la región
cercana $\simeq 0.8$. Los resultados de la estimación de estos máximos
$\langle R_g^2e_c \rangle^{max}_c$, mediante el método de la densidad espectral se muestran en la figura
\ref{max_varradio-giro-fig}. La relación de escala de  $\langle R_g^2e_c
\rangle^{max}_c$ se puede deducir a partir de \eqref{escala_R2g_conexo}
\begin{equation}
\langle R_g^2E_c\rangle^{max}_c\sim
L^{\frac{4}{d_H}+\frac{1}{\nu}}\Rightarrow \langle R_g^2e_c\rangle^{max}_c\sim
L^{\frac{4}{d_H}+\frac{1}{\nu}-2},
\end{equation}
donde hemos usado $d_H=2/\nu_F$, la dimensión fractal. Realizamos un ajuste a
la función $c_0L^{c_1}$ de los valores obtenidos de 
$\langle R_g^2e_c\rangle^{max}_c$, los resultados se muestran en la tabla
\ref{max_varradio-giro-tab} y la figura \ref{max_varradio-giro-L-fig}. El coeficiente $c_1$ está relacionado con la
dimensión fractal por la igualdad
\begin{equation*}
c_1=\frac{4}{d_H}+\frac{1}{\nu}-2.
\end{equation*}
Tomando como valor de $\nu$ el obtenido en el ajuste del calor específico
$\nu= 0.72(4)$, llegamos a una estimación de $d_H=2.7(1)$ y de
$\nu_F=2/d_H=0.74(6)$, ambos son compatibles con resultados anteriores analíticos
$d_H=2.73\rightarrow \nu_F=0.73$ \cite{Doussal:nu} y simulaciones
$2.77(10)\rightarrow \nu_F=0.71(3)$ \cite{Espriu:MCRG}.


\begin{table}[h]
\centering
\begin{tabular}{|c|c|}\hline
 Coeficientes/Exponentes críticos & Resultado \\\hline
 $c_0$         & $0.002(2) $ \\ \hline
 $c_1$         & $0.86(3)$ \\ \hline
$d_H$      & $2.7(1)$ \\ \hline
$\nu_F $        & $0.74(6)$ \\ \hline
$\chi_c^2/\mathrm{ngl}$ &  $1.41/3$   \\ \hline
 $P(\chi^2>\chi_c^2)$&  $0.70$\\ \hline
\end{tabular}
\caption{Resultados del ajuste a la función $c_0L^{c_1}$ para $\langle R_g^2
  e_c\rangle_{max}$ (figura \ref{max_varradio-giro-fig} ).}\label{max_varradio-giro-tab}
\end{table}

\begin{figure}[h]
  \centering
  \input{Rg2_plot.tex}
  \caption{Radio de giro frente a la temperatura $\kappa$ en función del tamaño $N=L^2$.}\label{radio-giro-fig}
\end{figure}

\begin{figure}[h]
  \centering
  \input{Drg2_plot.tex}
  \caption{$\langle R_g^2e_c\rangle_c$ frente a $\kappa$ y en función
    del tamaño de la membrana $N=L^2$.}\label{varradio-giro-fig}
\end{figure}

\begin{figure}[h]
  \centering
  \input{max_Drg2_plot.tex}
  \caption{$\langle R_g^2e_c\rangle^{max}_c$ para cada tamaño $N=L^2$.}\label{max_varradio-giro-fig}
\end{figure}

\begin{figure}[h]
  \centering
  \input{Drg2_L_plot.tex}
  \caption{$\langle R_g^2 e_c\rangle^{max}_c$ frente a $L$.}\label{max_varradio-giro-L-fig}
\end{figure}

%\begin{figure}[h]
%  \centering
%  \input{kappac_plot.tex}
%  \caption{}
%\end{figure}
\clearpage
\subsection{Función de escala del radio de giro}
El radio de giro cerca la transición de fase sigue la relación:
\begin{equation}
    R^2_G(L)\simeq L^{\frac{4}{d_H}}f_{R_G}\left(L^{1/\nu}(\kappa(L)-\kappa_c(\infty))\right).
\end{equation}
Podemos observar la función de escala $f_{R_G}(x)$ del radio de giro si
representamos $R^2_G(L)/ L^{\frac{4}{d_H}}$ frente a
$L^{1/\nu}(\kappa(L)-\kappa_c(\infty))$. El siguiente gráfico (figura
\ref{funcion_escala_rg2-fig}) muestra esta representación para diferentes los
diferentes tamaños del sistema, con los valores de $d_H=0.73$, $\nu=0.72$ y
$\kappa(\infty)=0.777$, los obtenidos en las secciones anteriores. Puesto que
la función de escala $f_{R_G}(x)$ no depende del tamaño, cerca de $x=0$, donde
conseguimos su mejor estimación, para todos los $L^2$ debemos observar los
mismos valores de la función de escala. Esto es lo que sucede, lo que
constituye una comprobación satisfactoria de nuestros resultados. A medida que
$x$ toma valores más alejados de $0$, deben aparecer discrepancias entre los
diferentes tamaños. Se observa que las discrepancias aumentan a medida que
aumenta $x$, pero para valores negativos parece no haber diferencias.  La
razón de esta asimetría es que realmente también hay discrepancias en $x<0$, pero en una escala mucho
menor, ya que $R_G^2$ escala con el logaritmo en esta región y al cuadrado en
$x>0$. 
\begin{figure}[h]
  \centering
  \input{rg2_funcion_escala_plot.tex}
  \caption{Función de escala del $R^2_g$ para los diferentes tamaños.}\label{funcion_escala_rg2-fig}
\end{figure}

\clearpage


\section{Estudios de las fases}

\subsection{Relaciones de escala del radio de giro}

Para estudiar la relación de escala en ambas fases tomamos los valores de $\kappa$ que tenemos
más alejados de $0.8$, que son $\kappa=0.5$ para la fase rugosa y $\kappa=2.0$
para la fase plana. En la fase rugosa observamos efectivamente sigue la
relación de escala $R_G\sim (\log L)^{1/2}$ (figura \ref{rg2_rugosa-fig}), y la
fase plana $R_G^2\sim L^{2\nu_F}$ (figura \ref{Rg2_plana-fig}) con $2\nu_F=1.9955 \pm
0.005\rightarrow d_H=2.004(5)$. Los resultados de ambos ajustes se muestran en
las tablas \ref{Rg2_plana-tab} y \ref{rg2_rugosa-tab}.

\begin{figure}[h]
  \centering
  \input{rg2_plano_plot.tex}
  \caption{$\langle R^2_g \rangle$ en función de $L$ para $\kappa=2.0$.}\label{Rg2_plana-fig}
\end{figure}

\begin{table}[h]
\centering
\begin{tabular}{|c|c|}\hline
 Ajuste              & Resultados \\ \hline\hline
 $c_0$               & $0.14(2) $  \\ \hline
 $c_1$               & $0.0016(3)$ \\ \hline
 $c_2=2\nu_F$        &  $1.996(5)$ \\ \hline
 $\nu_F$             & $ 0.998(3)$ \\ \hline
$ d_H$               & $ 2.004(5)$ \\ \hline
$\chi_c^2/\mathrm{ngl}$        &  $2.939/2$  \\ \hline
 $P(\chi^2>\chi_c^2)$&  $0.23$    \\ \hline
\end{tabular}
\caption{Resultados del ajuste de la función $c_0+c_1L^{c_2}$ a $\langle R^2_g
  \rangle$ en función de $L$ para $\kappa=2.0$ (figura \ref{Rg2_plana-fig}).}\label{Rg2_plana-tab}
\end{table}
\clearpage

\begin{figure}[h]
  \centering
  \input{rg2_rugosa_plot.tex}
   \caption{$\langle R^2_g \rangle$ en función de $\log L$ para $\kappa=0.5$.}\label{rg2_rugosa-fig}
\end{figure}

\begin{table}[h]
\centering
\begin{tabular}{|c|c|}\hline
 Ajuste              & Resultados \\ \hline\hline
 $c_0$               & $-0.176(5) $  \\ \hline
 $c_1$               & $0.247(1)$ \\ \hline
 $\chi_c^2/\mathrm{ngl}$       &  $1.85/3$  \\ \hline
 $P(\chi^2>\chi_c^2)$&  $0.39$    \\ \hline
\end{tabular}
\caption{Resultados del ajuste de la función $c_0+c_1\log L$ a $\langle R^2_g
  \rangle$ en función de $L$ para $\kappa=0.5$ (figura \ref{rg2_rugosa-fig}).}\label{rg2_rugosa-tab}
\end{table}

\clearpage

\subsection{Fase plana: Módulo de Poisson}

Para la estimación del módulo de Poisson únicamente se tuvieron en cuenta los
nodos de la red que tienen seis primeros y segundos vecinos para eliminar los
efectos de borde. La siguiente figura \ref{poisson-fig} muestra como la tendencia asintótica del
módulo de Poisson hacia un valor cercano a $-1/3$ (valor teórico
\cite{Doussal:nu}). Debido a que los errores son demasiado grandes
no es posible realizar un ajuste satisfactorio, de donde extraer una
estimación de su valor para $\kappa(\infty)$. Estos elevados
márgenes de error son debidos a una falta de isotropía causada por los efectos
de borde. Pues como observamos en la tabla \ref{poisson-tab} $\langle g_{11}^2
\rangle_c$ y $\langle g_{22}^2 \rangle_c$ no son iguales, dentro de los
márgenes de error, en algunos casos, cuando deberían coincidir por isotropía. Además $\langle
g_{22}^2 \rangle_c$ es sistemáticamente mayor que $\langle g_{11}^2\rangle_c$, y
este comportamiento, es más apreciable cuanto menor es el tamaño del sistema,
donde los efectos de borde influyen más.
 
\begin{figure}[h]
  \centering
  \input{poisson_plot.tex}
  \caption{Módulo de Poisson en función del tamaño $N=L^2$}\label{poisson-fig}
\end{figure}

\begin{table}[h]
\centering
\begin{tabular}{|c|c|c|c|}\hline
  & \multicolumn{3}{c|}{Resultados}\\ \hline
 Observable              & $16^2$ & $24^2$ & $32^2$ \\ \hline\hline
 $\langle g_{11}^2 \rangle_c$  & $0.00154(2) $& $0.000577(8)$  & $0.000290(6)$\\ \hline
 $\langle g_{22}^2 \rangle_c$  & $0.00164(2) $&  $0.000603(8)$ & $0.000307(6)$ \\ \hline
$\langle g_{11}g_{22} \rangle_c$& $0.00040(2)$ &  $0.000160(6)$ & $0.000082(4)$\\ \hline
$\sigma$                      & $ -0.244(9)$ &  $-0.264(9)$   & $-0.26(1)$ \\ \hline
\end{tabular}\vspace{0.4cm}
\begin{tabular}{|c|c|c|c|}\hline
   & \multicolumn{3}{c|}{Resultados}\\ \hline
  Observable             & $46^2$ & $64^2$& $128^2$\\ \hline\hline
 $\langle g_{11}^2 \rangle_c$ & $0.000136(2)$& $0.000069(1)$  & $0.000016(1)$\\ \hline
 $\langle g_{22}^2 \rangle_c$ & $0.000140(2)$& $0.000067(1)$  & $0.000016(1)$    \\ \hline
$\langle g_{11}g_{22} \rangle_c$& $0.000039(2)$ & $0.000019(1)$ & $0.000005(1)$\\ \hline
$\sigma$                      & $-0.27(1)$   & $-0.28(1)$     & $-0.31(2)$\\ \hline
\end{tabular}
\caption{Resultados de la estimación del módulo de Poisson (figura \ref{poisson-fig}).}\label{poisson-tab}
\end{table}

%%% Local Variables: 
%%% mode: latex
%%% TeX-master: "TFM"
%%% End: 
