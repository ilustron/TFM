\documentclass[xcolor=dvipsnames]{beamer}
\usepackage[utf8x]{inputenc}
\usepackage[spanish]{babel}
\usepackage{amsmath}
\usepackage{amsfonts}
\usepackage{graphicx}
\usepackage{xcolor,color}
%PARA IMPRIMIR DOS TRANSPARENCIAS POR HOJA
%\usepackage{pgfpages}
%\pgfpagesuselayout{2 on 1}[a4paper,border shrink=5mm]


\usetheme{Singapore}

%\useinnertheme{rectangles}
\definecolor{fblue}{rgb}{0,0.3,0.9}
%\usefonttheme{structuresmallcapsserif}
\usefonttheme{serif}

%\usecolortheme{whale} % outer color
\usecolortheme[RGB={0,50,200}]{structure}


\setbeamertemplate{navigation symbols}{}

\mode<beamer>{\setbeamertemplate{blocks}[rounded][shadow=false]}
\setbeamercolor{block title}{bg=fblue}
\setbeamercolor{block body}{bg=gray!10}
%\logo{\includegraphics[width=28bp]{./Pictures/UEX.jpg}}


\title[Trans. de fase en membranas cristalinas]{Transiciones de fase en
  membranas cristalinas}
\author[P. Monroy Dir.: Juan J.Ruiz Lorenzo]{Pedro Monroy Pérez \\ Director:
  Juan Jesús Ruiz Lorenzo}
\date{19 de Julio de 2013}
\institute[UEx]{Dep. de Física, Facultad de Ciencias, Universidad de Extremadura}

\begin{document}
\decimalpoint
\begin{frame}[plain]{}
\titlepage
\end{frame}


\begin{frame}{Indice}
  \tableofcontents[hideallsubsections]
\end{frame}
\section{Introducción}
\begin{frame}{Introducción}
\begin{itemize}
\item Dos clases de universalidad principales para las membranas flexibles:
  Las membranas fluidas (topología variable) y las cristalinas (conectividad fija). 
\item Las membranas cristalinas presentan una transición de fase plana-rugosa
  (comportamiento crítico).
\item Presentan un módulo de Poisson negativo.
\item Ejemplos: El citoesqueleto de los glóbulos rojos, superficie de óxido de grafito y grafeno. 
\end{itemize}
\end{frame}
\section{Estudio Analítico}

\begin{frame}{Descripción geométrica}
  \begin{columns}
    \begin{column}{5.8cm}
      \centering
      Coordenadas Internas:
      \begin{figure}[h]
        \resizebox{\columnwidth}{!}{\input{coordenadas_internas-fig}}
      \end{figure}
      \begin{equation*}
        \mathbf{x}\equiv (x^1,x^2)\in \mathbb{R}^2
      \end{equation*}
    \end{column}
    \begin{column}{5.8cm}
      \centering
      Coordenadas Externas:
      \begin{figure}[h]
        \resizebox{\columnwidth}{!}{\input{coordenadas_externas-fig}}
      \end{figure}
      \begin{equation*}
        \vec{r}=x\,\vec{i}+y\,\vec{j}+z\,\vec{k}\equiv (x,y,z)\in \mathbb{R}^3
      \end{equation*}
    \end{column}
  \end{columns}
\end{frame}

\begin{frame}
  \begin{columns}[T]
    \begin{column}{5.2cm}
      \centering
      \begin{itemize}
      \item Vectores Tangentes:
      \end{itemize}
      \begin{figure}[h]
        \resizebox{\columnwidth}{!}{\input{vectores_tangentes-fig}}
      \end{figure}
      \begin{equation*}
        \vec{t}_{\alpha}=\frac{\partial \vec{r}}{\partial
          x^{\alpha}}=\partial_{\alpha}\vec{r} \qquad \alpha=1,2;
      \end{equation*}
     
    \end{column}
    \begin{column}{6.2cm}
      \centering
      \begin{itemize}
\item Métrica inducida $g_{\alpha\beta}=\vec{t}_{\alpha}\cdot\vec{t}_{\beta}$:
      \begin{equation*}
      \vec{V}\cdot\vec{W}=g_{\alpha\beta}V^{\alpha}W^{\beta}.
      \end{equation*}
      \item Métrica inversa $g^{\alpha\beta}$:
      \begin{equation*}
       g_{\alpha\gamma}g^{\gamma\beta}= \delta_{\alpha}^{\ \beta}=\begin{cases}
          1&\text{si $\alpha=\beta$}\\
          0&\text{si $\alpha\neq\beta$}.
          \end{cases}
      \end{equation*}
      \item Correspondencia entre vectores covariantes y contravariantes:
      \begin{equation*}
        V_{\alpha}=g_{\alpha\beta}V^{\beta}\rightarrow V^{\alpha}=g^{\alpha\beta}V_{\beta}, 
      \end{equation*}
      \begin{equation*}
        \vec{V}\cdot\vec{W}=V_{\alpha}W^{\alpha}=V^{\alpha}W_{\alpha}.
      \end{equation*}
      \end{itemize}
    \end{column}
  \end{columns}
\end{frame}

\begin{frame}{Curvatura}
  \begin{columns}[T]
    \begin{column}{6.2cm}
      \centering
      Tensor de curvatura extrínseca $K_{\alpha\beta}$:
      \begin{equation*}
        \vec{K}_{\alpha\beta}=K_{\alpha\beta}\vec{n}, \quad\text{con}\quad
        K_{\alpha\beta}=\frac{\partial \vec{t}_{\alpha}}{\partial
          x^{\beta}}\cdot \vec{n}
      \end{equation*}
      \begin{equation*}
        K_{\alpha\beta}\equiv\left(\begin{array}{cc}
            \mathcal{K}_1 & 0\\
            0 & \mathcal{K}_2\\
          \end{array}\right)\, ,
      \end{equation*}
      \begin{equation*}
        \mathcal{R}_{\alpha}=\frac{1}{\mathcal{K}_{\alpha}} \, , \qquad \alpha=1,2.
      \end{equation*}  
    \end{column}
    \begin{column}{6.25cm}
      \centering
      \begin{figure}[h]
        \resizebox{\columnwidth}{!}{\input{curvatura-fig}}
      \end{figure} 
    \end{column}
  \end{columns}   
  \begin{align*}
    \text{Curvatura Media}\longrightarrow H&=\frac{\mathcal{K}_1+\mathcal{K}_2}{2}=\frac{1}{2}K_{\alpha}^{\ \alpha},\\
    \text{Curvatura Gaussiana}\longrightarrow  K&=\mathcal{K}_1\mathcal{K}_2=\frac{1}{2}(K_{\alpha}^{\ \alpha}K_{\beta}^{\ \beta}-K_{\alpha}^{\ \beta}K^{\alpha}_{\ \beta}).
  \end{align*}
\end{frame}

\begin{frame}{Teoría de Landau}
\begin{description}
\item[Parámetro de orden:] Variable que cuantifica el grado de orden de las
  fases, es nulo en la fase de mayor simetría. 
  \begin{itemize}
  \item Microscópico $\rightarrow\eta(\mathbf{x})$, $\mathbf{x}$ denota partículas
    individuales.
  \item Grano grueso $\rightarrow\bar{\eta}(\mathbf{x}_{\Lambda})$,
    $\mathbf{x}_{\Lambda}$ denota bloques de partículas.
\end{itemize}
\item[Energía libre de Landau]$\rightarrow F[\bar{\eta}(\mathbf{x}_{\Lambda}),T]$, potencial
  efectivo resultado de la integración de los grados de libertad microscópicos.
\end{description}
\begin{equation*}
  e^{-F[\bar{\eta}(\mathbf{x}_{\Lambda}),T]}=\int D[\eta(\mathbf{x})]\;
  e^{-\beta
    H[\eta(\mathbf{x}),T]}\;\delta\left[\frac{1}{A_{\Lambda}}\int_{\Lambda}
    d^2\mathbf{x}\; \eta(\mathbf{x})-\bar{\eta}(\mathbf{x}_{\Lambda})\right], 
\end{equation*}
\begin{equation*}
  \mathcal{Z}[T]=\int D[\bar{\eta}(\mathbf{x}_{\Lambda})]\; e^{-F[\bar{\eta}(\mathbf{x}_{\Lambda}),T]}=\int D[\eta(\mathbf{x})]\; e^{-\beta H[\eta(\mathbf{x}),T]}.
\end{equation*}
\end{frame}

\begin{frame}{Parámetro de orden}
\centering
\begin{columns}[T]
    \begin{column}{5.8cm}
      Fase plana (ordenada):
        \centering
        \begin{figure}[h]
        \resizebox{\columnwidth}{!}{\input{fase_plana-fig}}
        \end{figure}
        $$ \langle\vec{t}_{\alpha}\rangle\neq 0$$
    \end{column}
    
    \hspace{0.5cm}
    
    \begin{column}{5.8cm}
      Fase rugosa (desordenada):
        \centering
        \begin{figure}[h]
        \resizebox{\columnwidth}{!}{\input{fase_arrugada-fig}}
        \end{figure}
        $$ \langle\vec{t}_{\alpha}\rangle= 0$$
    \end{column}
  \end{columns}
\vspace{0.8cm}

Parámetro de orden $\eta(\mathbf{x}_{\Lambda})\longrightarrow$ Vectores tangentes $\vec{t}_{\alpha}(\mathbf{x}_{\Lambda})$ 
\end{frame}
\begin{frame}{Energía Libre de Landau}
A partir de las siguientes ligaduras encontramos el potencial la energía Libre
de Landau:
  \begin{description}
  \item[Localidad e invariancia traslacional.]
  \item[Simetría rotacional traslacional en $\mathbb{R}^2$ y $\mathbb{R}^3$.]
  \item[Imposibilidad de autointersección.]
  \end{description}

  \begin{equation*}
    F[\vec{t}_{\alpha}(\mathbf{x}),T]=F_E[\vec{t}_{\alpha}(\mathbf{x}),T]+F_C[\vec{t}_{\alpha}(\mathbf{x}),T]+F_b[\vec{t}_{\alpha}(\mathbf{x}),T],
  \end{equation*}
  donde
  \begin{align*}
    F_E[\vec{t}_{\alpha}(\mathbf{x}),T]&=\int d^2\mathbf{s}\left[
      \frac{t}{2}\!(\vec{t}_{\alpha}\!\cdot\!\vec{t}^{\alpha})+
      u(\vec{t}_{\alpha}\!\cdot\!\vec{t}_{\beta})(\vec{t}^{\alpha}\!\cdot\!\vec{t}^{\beta})+
      v(\vec{t}_{\alpha}\!\cdot\!\vec{t}^{\alpha})(\vec{t}_{\beta}\!\cdot\!\vec{t}^{\beta})\right],\\
    F_C[\vec{t}_{\alpha}(\mathbf{x}),T]&= \frac{\kappa}{2}\int d^2\mathbf{s}\
     \partial_{\alpha}\vec{t}^{\alpha}\cdot\partial^{\beta}\vec{t}_{\beta},\\ 
   F_b[\vec{t}_{\alpha}(\mathbf{x}),T]&=\frac{b}{2}\int d^2\mathbf{x} d^2\mathbf{x'}
\delta^2(\vec{r}(\mathbf{x})-\vec{r}(\mathbf{x'})).
  \end{align*}
\end{frame}
\begin{frame}{Aproximación del campo medio}
  \begin{columns}%[T]
    \begin{column}{6.2cm}
      \centering
      Despreciando las fluctuaciones en la fase plana:
     \begin{align*}
       \vec{r}(\mathbf{x})&\simeq(\zeta x^1,\zeta x^2,0),\\
       \vec{t}_1(\mathbf{x})&\simeq(\zeta ,0,0),\\
       \vec{t}_2(\mathbf{x})&\simeq(0,\zeta,0).
     \end{align*}
     
    \end{column}
    \begin{column}{6.25cm}
      \centering
      \begin{figure}[h]
      \resizebox{\columnwidth}{!}{\input{campo_medio-fig}}
      \end{figure} 
    \end{column}
  \end{columns}
Tenemos una configuración homogénea:
\begin{equation*}
  F(\vec{t}_{\alpha},T)\simeq Af_M(\zeta)\quad \text{con}\quad f_M(\zeta,T)=2\zeta^2\left( \frac{t}{2} +(u+2v)\zeta^2\right)
\end{equation*}
\end{frame}
\begin{frame}
\begin{align*}
 \mathcal{Z}[T]=\int d\zeta\;e^{-Af_M(\zeta,T)}&\simeq\sqrt{\frac{2\pi}{Af"_M(\zeta_0,T)}}\;
 e^{-Af_M(\zeta_0,T)}\\
 \lim_{A\rightarrow \infty}\; \frac{1}{A} \log \mathcal{Z}[T]&=-f_M(\zeta_0,T)=-\beta f_H(T)
\end{align*}
\begin{columns}[T]
    \begin{column}{5.8cm}
      \centering
      Mínimo de $f_M(\zeta,T)$:
      $$\zeta^2_0=\begin{cases}
        0& \text{si $t>0$}\\
        \frac{-t}{4(u+2v)}& \text{si $t<0$}\\
      \end{cases}$$
\end{column}
\begin{column}{5.8cm}
      \centering
      \begin{figure}[h]
      \resizebox{\columnwidth}{!}{\input{energia_libre_CM-fig}}
      \end{figure} 
    \end{column}
  \end{columns}
\end{frame}
\begin{frame}{Exponentes críticos}
\begin{itemize}
  \item La teoría de Landau no tiene en cuenta las fluctuaciones del parámetro de
orden.
  \item  La teoría del grupo de renormalización encuentra que
    estas fluctuaciones contribuyen en una parte no analítica en la energía
    libre:
    \begin{equation*}
      f_s(t)\sim|t|^{2\nu}
    \end{equation*}
  \item Explica el comportamiento divergente en $t\simeq 0$ de los observables
    termodinámicos, (ej. $C_V\sim|t|^{-\alpha}$).
  \item No todos los exponentes críticos son independientes entre sí.
\end{itemize}
\end{frame}
\begin{frame}{Relaciones de escala del radio de giro}
  \begin{equation*}
    R_G^2=\frac{1}{dA_D}\int d^D\mathbf{x}d^D\mathbf{x}' \langle |
    \vec{r}(\mathbf{x})-\vec{r}(\mathbf{x}')|^2\rangle\stackrel{D=2}{-\!\!\!\longrightarrow}\frac{1}{3A}\int d^2\mathbf{x}\, \langle
\vec{R}(\mathbf{x})\cdot\vec{R}(\mathbf{x})\rangle^2,
  \end{equation*}
donde $\vec{R}(\mathbf{x})=\vec{r}(\mathbf{x})-\vec{r}_{CM}$. Podemos aproximar:
  \begin{equation*}
    \vec{t}_{\alpha}=\partial_{\alpha} \vec{r}\simeq \frac{R_G}{L}
    \frac{\vec{t}_{\alpha}}{|\vec{t}_{\alpha}|}\quad \text{y}\quad \int
    d^D\mathbf{x}\simeq L^D,
\end{equation*}
\begin{equation*}
  f(R_G,t)\simeq t R_G^2 L^{D-2}+(u+Dv) R_G^4 L^{D-4}+\kappa R_G^2 L^{D-4}.
\end{equation*}
Minimizando $f(R_G,t)$ respecto a $R_G$ ($D=2$):
\begin{equation*}
  R_G\sim L^{\nu_F}= L^{2/d_H}\rightarrow\begin{cases}
    \text{F. plana }& \nu_F=1\quad d_H=2\\
    \text{Transición }& \nu_F=0.73\quad d_H=2.74\\
    \text{F. rugosa }& \nu_F=0\quad d_H=\infty\ (R_G^2\sim\log L)
  \end{cases}
\end{equation*}
\end{frame}
\begin{frame}{Fase plana}
Coordenadas ortonormales en el plano base ($\zeta=1$) $\Rightarrow\vec{r}(\mathbf{x})=(\mathbf{x}+\mathbf{u(\mathbf{x})},h(\mathbf{x}))$.
  \begin{columns}[T]
    \begin{column}{5.8cm}
      \centering
      \begin{equation*}
        (d\vec{r})^2=(d\vec{r}_0)^2+2u_{\alpha\beta}dx^{\alpha}dx^{\beta},
      \end{equation*}
      con $u_{\alpha\beta}$, tensor de deformaciones:
      \begin{multline*}
        u_{\alpha\beta}=\frac{1}{2}(\vec{t}_{\alpha}\cdot\vec{t}_{\beta}-\delta_{\alpha\beta})\
        \Rightarrow\\\Rightarrow \
        \vec{t}_{\alpha}\cdot\vec{t}_{\beta}=\delta_{\alpha\beta}+2u_{\alpha\beta}.
      \end{multline*}     
    \end{column}
    \begin{column}{5.8cm}
      \centering
      \begin{figure}[h]
        \resizebox{\columnwidth}{!}{\input{deformacion-fig}}
      \end{figure} 
    \end{column}
  \end{columns}
  
  \begin{align*}
    F_E[u_{\alpha\beta},T]&=\int d^2\mathbf{x}
    \left[
      \mu u_{\alpha\beta}u^{\alpha\beta}\! +\!
      \frac{\lambda}{2}u_{\alpha}^{\ \alpha}u^{\beta}_{\ \beta}\right],
    \quad\text{con}\quad -t=\mu+\lambda\\
    F_C[\mathbf{y},T]&\simeq\frac{\kappa}{2}\int d^2\mathbf{y} K_{\alpha}^{\
      \beta}K^{\alpha}_{\ \beta}\simeq \frac{1}{2}\hat{\kappa}\int d^2\mathbf{y}\, H^2 \quad
\text{con} \quad \hat{\kappa}=4\kappa
  \end{align*}
\end{frame}

\begin{frame}{Módulo de Poisson}
\centering
\begin{equation*}
f_E[u_{\alpha\beta},T]=
\mu \left(u_{\alpha\beta}-\frac{1}{2}\delta_{\alpha\beta}u_{\gamma}^{\ \gamma}\right)
    \left(u^{\alpha\beta}-\frac{1}{2}\delta^{\alpha\beta}u_{\nu}^{\ \nu}\right)+
\frac{1}{2}K(u_{\alpha}^{\ \alpha})(u_{\beta}^{\ \beta}),
\end{equation*}
con $K>0\, ,\ \mu>0$.

\begin{columns}[T]
    \begin{column}{5.8cm}
      \centering
      \begin{equation*}
        \sigma=-\frac{\delta x^2 / x^2}{\delta x^1 / x^1}=-\frac{u_{22}}{u_{11}}=\frac{K-\mu}{K+\mu}.
      \end{equation*}
      Membrana $D$-dimensional:
      \begin{equation*}
        \sigma(D)=-\frac{1}{D+1}\ \Rightarrow \ \sigma(2)=-\frac{1}{3}. 
      \end{equation*}
    \end{column}
    \begin{column}{5.8cm}
      \centering
      \begin{figure}[h]
        \resizebox{\columnwidth}{!}{\input{Poisson-fig}}
      \end{figure} 
    \end{column}
  \end{columns}
\end{frame}

\section{Simulaciones Numéricas}
\begin{frame}{Modelo discreto}
\begin{columns}[T]
    \column{.5\textwidth}
    \begin{itemize}
    \item Red triangular bidimensional de $N=L^2$ partículas en un espacio
      tridimensional.
    \item Condiciones de contorno libres (efectos de borde).
    \item  Hamiltoniano efectivo:
    \end{itemize}
    \column{.5\textwidth}
    \begin{figure}[h]
      \resizebox{\columnwidth}{!}{\input{red_triangular-fig}}
    \end{figure} 
\end{columns}
\begin{equation*}
\mathcal{H}=\mathcal{H}_E+\mathcal{H}_C,
\end{equation*}
donde
 $$\mathcal{H}_E=\frac{1}{2}\sum_{\langle
   ij\rangle}|\vec{r}_i-\vec{r}_j|^2 \qquad \mathcal{H}_C=\frac{\kappa}{2}\sum_{\langle ab 
      \rangle}|\vec{n}_a-\vec{n}_b|^2$$
\end{frame}
\begin{frame}{Límite continuo}
\begin{columns}[T]
    \column{.55\textwidth}
    \begin{equation*}
     \mathcal{H}_E=\frac{1}{2}\sum_{\langle
       ij\rangle}(|\vec{r}_i-\vec{r}_j|-a)^2;\quad |\vec{r}_i-\vec{r}_j|\simeq a+\frac{u_{\alpha\beta}}{a}x_q^{\ \alpha}x_q^{\ \beta}. 
    \end{equation*}
    \begin{equation*}
     \mathcal{H}_E\simeq\frac{3a^2}{8}\sum_P 2u_{\alpha\beta}u^{\alpha\beta}+u_{\alpha}^{\ \alpha}u^{\beta}_{\ \beta}
     \end{equation*}
     Límite continuo:
     \begin{equation*}
     \mathcal{F}_E= \frac{\sqrt{3}}{4}\frac{a^2}{\epsilon^2}\int
     d^2\mathbf{x}\,u_{\alpha\beta}u^{\alpha\beta}+\frac{1}{2}u_{\alpha}^{\
       \alpha}u^{\beta}_{\ \beta} \Rightarrow \mu=\lambda= \frac{\sqrt{3}}{4}\frac{a^2}{\epsilon^2}
     \end{equation*}
     Nuestro modelo $a\rightarrow 0$ y $\epsilon$ fijo.
    \column{.53\textwidth}
    \begin{figure}[h]
      \resizebox{\columnwidth}{!}{\input{base-hexagonal-fig}}
    \end{figure} 
\end{columns}
\begin{multline*}
F_C[\mathbf{y},T]\simeq\frac{\kappa}{2}\int d^2\mathbf{y} K_{\alpha}^{\
  \beta}K^{\alpha}_{\ \beta}=\\=\frac{\kappa}{2}\int d^2\mathbf{y}\
\frac{\partial \vec{n_a}}{\partial x_{\alpha}}\cdot\frac{\partial
  \vec{n_a}}{\partial x^{\alpha}} \longrightarrow \mathcal{H}_C=\frac{\kappa}{2}\sum_{\langle ab 
      \rangle}|\vec{n}_a-\vec{n}_b|^2
\end{multline*}
\end{frame}

\begin{frame}{Algoritmo de Metropolis}
Para un valor fijo de $\kappa$:
\begin{itemize}
  \item  Se modifica $\vec{r}_i\rightarrow \vec{r}_i+\vec{\epsilon}$, con
    $\vec{\epsilon}$ elegido aleatoriamente en un cubo de lado $\delta$.
  \item  Si
    $\Delta\mathcal{H}=\mathcal{H}_{NEW}-\mathcal{H}_{OLD}<0$ se
    acepta la nueva posición. 
  \item Si  $\Delta\mathcal{H}>0$, se acepta la nueva posición con
    probabilidad $e^{-\Delta\mathcal{H}}$. 
  \item Se repiten los pasos anteriores para otro punto.
\end{itemize}
\begin{columns}[T]
    \column{.44\textwidth}
    \begin{block}{}
      Una actualización corresponde a aplicar el
      algoritmo a todos los $N$ puntos de la red.
    \end{block}
    \column{.44\textwidth}
    \begin{block}{}
      $\delta$ se ajusta de modo que en cada actualización el número de cambios
      aceptados sea $N/2$.
    \end{block}
\end{columns}

\begin{description}
\centering
\item[Configuraciones resultantes] $\rightarrow$ $P(\vec{r}_i)\propto e^{-\mathcal{H}(\kappa)}$
\end{description}
\end{frame}
\begin{frame}{Tratamiento de los datos}
A partir de las configuraciones resultantes, se estiman los observables teniendo en cuenta:
\begin{columns}
  \column{.55\textwidth}
  \begin{figure}[h]
        \resizebox{\columnwidth}{!}{\input{ejemplo_termal-fig}}
      \end{figure}
  \column{.5\textwidth}
  \begin{itemize}
  \item Tiempo de Termalización.
  \item Periodo de correlación entre medidas.
  \end{itemize}
\end{columns}
\end{frame}
\begin{frame}{Cálculo de errores (Método de Jackknife)}
  \begin{itemize}
  \item Sean $\mathcal{Q}_i,(i=1,\dots N)$ los valores obtenidos de un
    observable $\mathcal{Q}$ en una evolución de Monte-Carlo. 
  \item Dividimos el conjunto de $N$ en $K$ bloques con $n=\frac{N}{K}$
    elementos.
  \item Definimos $\bar{\mathcal{Q}}_k^n(k=1,\dots K)$ el promedio de los
    datos sin el bloque $k$-ésimo:
    $$\bar{\mathcal{Q}}_k^n=\frac{1}{N-n}\left(\sum^{n(k-1)}_{i=1}\mathcal{Q}_i+\sum^{N}_{i=nk+1}\mathcal{Q}_i\right).$$
\item La estimación del error:
$$\delta\bar{\mathcal{Q}}_n=\sqrt{\frac{K-1}{K}\sum^K_{k=1}(\bar{\mathcal{Q}}^n_k-\bar{\mathcal{Q}})},$$
\item ya que
  $$ \langle (\delta\bar{\mathcal{Q}}_n)^2\rangle=\sigma^2(\mathcal{Q})\frac{1}{N}.$$
\end{itemize}
\end{frame}

\begin{frame}
\begin{itemize}
\item Si $n<<\text{Correlación entre medidas}\rightarrow
  \delta\bar{\mathcal{Q}}_n$ aumenta con el tamaño $n$.
\item Si $n\simeq \text{Correlación entre
    medidas}\rightarrow\delta\bar{\mathcal{Q}}_n=const.$ (plateau).
\end{itemize}
\begin{columns}
  \begin{column}{6cm}
    \begin{figure}
      \resizebox{\columnwidth}{!}{\input{ejemplo_logtermal-fig}}
    \end{figure}
  \end{column}
  \begin{column}{6cm}
     \begin{figure}
       \resizebox{\columnwidth}{!}{\input{ejemplo_error-fig}}
    \end{figure}
  \end{column}
\end{columns}
\begin{center}
$\tau_0=120000$
\end{center}
\end{frame}

\begin{frame}{Estimación exponentes críticos}
\begin{columns}
  \begin{column}{5.8cm}
    Observable genérico $\mathcal{O}$:
    \begin{equation*}
      \mathcal{O}_L\sim L^{\frac{x_0}{\nu}}f\left[L (\kappa(L)-\kappa_c(\infty))^{\nu}\right].
    \end{equation*}
  valor máximo del observable respecto a $\kappa$, $\max f(x)=f(x_C)$:
  \begin{align*}
    \mathcal{O}^{max}_L& \sim L^{\frac{x_0}{\nu}},\\
    \kappa_c(L)&=\kappa_c(\infty)+x_CL^{1/\nu}.
  \end{align*}
  \end{column}
  \begin{column}{5.8cm}
    {\color{fblue} Densidad espectral:}

    Tenemos que:
    \begin{equation*}
      \lim _{N_C\rightarrow \infty}\bar{N}(E)=P(E)
    \end{equation*}

    $P(E)\equiv p_{\beta}=W(E)e^{-\beta E+f}$,
    \begin{equation*}
      p_{\beta'}(E)=\frac{p_{\beta}(E)e^{-(\beta'-\beta)E}}{\sum_E e^{-(\beta'-\beta)E}},
    \end{equation*}
    \centering
    \begin{equation*}
      \langle O\rangle_{\beta'}=\frac{\sum_E  O(E)p_{\beta}(E) e^{-(\beta'-\beta)E}}{\sum_E e^{-(\beta'-\beta)E}},
    \end{equation*}
  \end{column}
\end{columns}
\end{frame}

\section{Resultados}
\begin{frame}{Resultados: Calor específico}
\begin{columns}
  \begin{column}{5.8cm}
    \begin{figure}[h]
      \centering
      \resizebox{\columnwidth}{!}{\input{Se_plot.tex}}
    \end{figure}
  \end{column}
  \begin{column}{5.8cm}
    \begin{figure}[h]
      \centering
     \resizebox{\columnwidth}{!}{ \input{Cv_plot.tex}}
    \end{figure}
  \end{column}
\end{columns}
\begin{equation*}
 C_V=\frac{3(N-1)}{2}+\frac{\kappa^2}{N}(\langle E_C^2 \rangle-\langle E_C
\rangle^2),
\end{equation*}
donde
\begin{equation*}
E_C=Ne_C=\sum_{\langle ab \rangle}\vec{n}_a\cdot\vec{n}_b, 
\end{equation*}
es la energía de curvatura.
\end{frame}

\begin{frame}
\begin{columns}[T]
  \begin{column}{5.8cm}
    \begin{figure}[h]
      \centering
      \resizebox{\columnwidth}{!}{\input{max_Cv_plot.tex}}
    \end{figure}
    \begin{equation*}
    C^{\max}_v(L)=C_1+C_2L^{\alpha/\nu}
  \end{equation*}
  \begin{equation*}
    \alpha=2(1-\nu) \quad \text{(hyperscaling)}
  \end{equation*}
  \end{column}
  \begin{column}{5.8cm}
    \begin{figure}[h]
      \centering
     \resizebox{\columnwidth}{!}{ \input{Cv_L_plot.tex}}
    \end{figure}
    \resizebox{\columnwidth}{!}{
      \begin{tabular}{|c|c|}\hline
        Coeficientes/Exponentes críticos & Resultado \\\hline
        $c_0$         & $0.7702(36) $ \\ \hline
        $c_1$         & $0.1098(95)$ \\ \hline
        $c_2=\alpha/\nu$  & $0.76(18)$  \\ \hline
        $\alpha$      & $0.55(18)$ \\ \hline
        $\nu $        & $0.72(4)$ \\ \hline
        $\chi_c^2/\mathrm{ngl}$ &  $0.38/2$   \\ \hline
        $P(\chi^2>\chi_c^2)$&  $0.8257$\\ \hline
      \end{tabular}}
  \end{column}
\end{columns}
\end{frame}

\begin{frame}
\begin{columns}
  \begin{column}{5.8cm}
    \begin{figure}[h]
      \centering
       \begin{center}
         \resizebox{\columnwidth}{!}{
           \begin{tabular}{|c|c|c|c|}\hline
             Ajuste              & $\frac{1}{\nu}=1.38$ & $\frac{1}{\nu}=1.38-0.09$& $\frac{1}{\nu}=1.38+0.09$ \\ \hline\hline
             $c_0=\kappa(\infty)$   & $0.777(2) $          &  $0.774(3)$              &  $0.779(3)$   \\ \hline
             $c_1$              & $4.35(59)$           &  $3.41(46)$              &   $5.55(77)$  \\ \hline
             $\chi_c^2/\mathrm{ngl}$       &  $3.75/3$            &  $3.27/3$                & $4.29/3$  \\ \hline
             $P(\chi^2>\chi_c^2)$&  $0.29$              &  $0.35$                  &   $0.23$ \\ \hline
           \end{tabular}}
       \end{center}
      
    \end{figure}
  \end{column}
  \begin{column}{5.8cm}
    \begin{figure}[h]
      \centering
     \resizebox{\columnwidth}{!}{ \input{Cv_T_L_plot.tex}}
    \end{figure}
  \end{column}
\end{columns}
\begin{equation*}
    \kappa_c(L)=\kappa_c(\infty)+C_0 L^{-1/\nu},\qquad\kappa_c(\infty)=0.777(3)(3)
  \end{equation*}

\end{frame}

\begin{frame}{Resultados: Radio de giro}
\begin{columns}
  \begin{column}{5.8cm}
    \begin{figure}[h]
      \centering
      \resizebox{\columnwidth}{!}{\input{Rg2_plot.tex}}
    \end{figure}
  \end{column}
  \begin{column}{5.8cm}
    \begin{figure}[h]
      \centering
     \resizebox{\columnwidth}{!}{ \input{Drg2_plot.tex}}
    \end{figure}
  \end{column}
\end{columns}
\begin{equation*} 
  R_g^{2}=\frac{1}{3N}\left\langle \sum_{i}
  \vec{R}_i\cdot\vec{R}_i\right\rangle=\frac{1}{3N}(\langle\vec{R}^2 \rangle
-\langle\vec{R} \rangle^2),\quad \text{donde}\quad \vec{R}_i=\vec{r}_{CM}-\vec{r}_i 
\end{equation*}
\begin{equation*}
  \frac{\partial R_g^2}{\partial \kappa}=\langle R_g^2 E_C \rangle-\langle R_G^2\rangle\langle E_C\rangle\equiv\langle R_g^2E_C \rangle_c=N\langle R_g^2e_C \rangle_c,
\end{equation*}
\end{frame}

\begin{frame}
\begin{columns}[T]
  \begin{column}{5.8cm}
    \begin{figure}[h]
      \centering
      \resizebox{\columnwidth}{!}{\input{max_Drg2_plot.tex}}
    \end{figure}
    \begin{equation*}
      \left(\frac{\partial R_g^2(L)}{\partial \kappa}\right)_{max}\sim
      L^{2\nu_F+\frac{1}{\nu}}
    \end{equation*}
    \begin{equation*}
      \langle R_G^2 e_C\rangle_{C,\max}\sim  L^{2\nu_F+\frac{1}{\nu}-2}
    \end{equation*}
  \end{column}
  \begin{column}{5.8cm}
    \begin{figure}[h]
      \centering
     \resizebox{\columnwidth}{!}{ \input{Drg2_L_plot.tex}}
    \end{figure}
    \resizebox{\columnwidth}{!}{
      \begin{tabular}{|c|c|}\hline
        Coeficientes/Exponentes críticos & Resultado \\\hline
        $c_0$         & $0.002(2) $ \\ \hline
        $c_1$         & $0.86(3)$ \\ \hline
        $d_H$      & $2.7(1)$ \\ \hline
        $\nu_F $        & $0.74(6)$ \\ \hline
        $\chi_c^2/\mathrm{ngl}$ &  $1.41/3$   \\ \hline
        $P(\chi^2>\chi_c^2)$&  $0.70$\\ \hline
      \end{tabular}}
  \end{column}
\end{columns} 
\end{frame}

\begin{frame}{Resultados: Radio de giro}
\begin{equation*}
    R^2_G(L)\simeq L^{\frac{4}{d_H}}f_{R_G}\left[L^{1/\nu}(\kappa(L)-\kappa_c(\infty))\right].
\end{equation*} 
\begin{figure}[h]
      \centering
     \resizebox{0.9\columnwidth}{!}{ \input{rg2_funcion_escala_plot.tex}}
    \end{figure}
\end{frame}

\begin{frame}{Resultados: Relaciones de escala del radio de giro}
\begin{columns}[T]
  \begin{column}{5.8cm}
    \begin{figure}[h]
      \centering
     \resizebox{\columnwidth}{!}{ \input{rg2_plano_plot.tex}}
    \end{figure}
    \resizebox{\columnwidth}{!}{\begin{tabular}{|c|c|}\hline
      Coeficientes/Exponentes críticos   & Resultados \\ \hline\hline
        $c_0$               & $0.14(2) $  \\ \hline
        $c_1$               & $0.0016(3)$ \\ \hline
        $c_2=2\nu_F$        &  $1.996(5)$ \\ \hline
        $\nu_F$             & $ 0.998(3)$ \\ \hline
        $ d_H$               & $ 2.004(5)$ \\ \hline
        $\chi_c^2/\mathrm{ngl}$        &  $2.939/2$  \\ \hline
        $P(\chi^2>\chi_c^2)$&  $0.23$    \\ \hline
      \end{tabular}}
  \end{column}
  \begin{column}{5.8cm}
    \begin{figure}[h]
      \centering
     \resizebox{\columnwidth}{!}{ \input{rg2_rugosa_plot.tex}}
    \end{figure}
   \resizebox{\columnwidth}{!}{ \begin{tabular}{|c|c|}\hline
Coeficientes/Exponentes críticos  & Resultados \\ \hline\hline
 $c_0$               & $-0.176(5) $  \\ \hline
 $c_1$               & $0.247(1)$ \\ \hline
 $\chi_c^2/\mathrm{ngl}$       &  $1.85/3$  \\ \hline
 $P(\chi^2>\chi_c^2)$&  $0.39$    \\ \hline
\end{tabular}}
  \end{column}
\end{columns}
\end{frame}
\begin{frame}{Resultados: Módulo de Poisson}
Según el teorema de la respuesta lineal:
\begin{equation*}
\langle u_{\alpha\beta}u_{\gamma\delta} \rangle_c
=\frac{1}{\beta A}\left(\frac{\partial u_{\alpha\beta}}{\partial \sigma^{\gamma\delta}}\right),
\end{equation*}
donde $A$ es área de la membrana. Por otro lado:
\begin{equation*}
\frac{\partial u_{\alpha\beta}}{\partial
  \sigma^{\gamma\delta}}=-\frac{K-\mu}{4\mu K}\delta_{\alpha\beta}\delta_{\gamma\delta}+\frac{1}{4\mu}(\delta_{\alpha\gamma}\delta_{\beta\delta}+\delta_{\alpha\delta}\delta_{\beta\gamma}).
\end{equation*}
Entonces:
\begin{equation*}
\sigma=\frac{K-\mu}{K+\mu}=-\frac{\langle u_{11}u_{22}
  \rangle_c}{\langle u_{22}^2 \rangle_c}=-\frac{\langle g_{11}g_{22}
  \rangle_c}{\langle g_{22}^2 \rangle_c}.
\end{equation*}
{\color{fblue} Cambio de coordenadas}: si $\vec{e}_1,\vec{e}_2,\vec{e}_3$ son
la base natural de vectores de la red triangular, elegimos
$\vec{x}_1$ en la dirección $\vec{e}_1$ y para $\vec{x}_2$ usamos
$(\vec{e}_2+\vec{e}_3)/\sqrt{3}$.

\end{frame}
\begin{frame}{Resultados: Módulo de Poisson}
\begin{columns}
  \begin{column}{5.8cm}

 \resizebox{\columnwidth}{!}{\begin{tabular}{|c|c|c|c|}\hline
  & \multicolumn{3}{c|}{Resultados}\\ \hline
 Observable              & $16^2$ & $24^2$ & $32^2$ \\ \hline\hline
 $\langle g_{11}^2 \rangle_c$  & $0.00154(2) $& $0.000577(8)$  & $0.000290(6)$\\ \hline
 $\langle g_{22}^2 \rangle_c$  & $0.00164(2) $&  $0.000603(8)$ & $0.000307(6)$ \\ \hline
$\langle g_{11}g_{22} \rangle_c$& $0.00040(2)$ &  $0.000160(6)$ & $0.000082(4)$\\ \hline
$\sigma$                      & $ -0.244(9)$ &  $-0.264(9)$   & $-0.26(1)$ \\ \hline
\end{tabular}}\vspace{0.4cm}
\resizebox{\columnwidth}{!}{\begin{tabular}{|c|c|c|c|}\hline
   & \multicolumn{3}{c|}{Resultados}\\ \hline
  Observable             & $46^2$ & $64^2$& $128^2$\\ \hline\hline
 $\langle g_{11}^2 \rangle_c$ & $0.000136(2)$& $0.000069(1)$  & $0.000016(1)$\\ \hline
 $\langle g_{22}^2 \rangle_c$ & $0.000140(2)$& $0.000067(1)$  & $0.000016(1)$    \\ \hline
$\langle g_{11}g_{22} \rangle_c$& $0.000039(2)$ & $0.000019(1)$ & $0.000005(1)$\\ \hline
$\sigma$                      & $-0.27(1)$   & $-0.28(1)$     & $-0.31(2)$\\ \hline
\end{tabular}} 
      
  \end{column}
  \begin{column}{5.8cm}
    \begin{figure}[h]
      \centering
     \resizebox{\columnwidth}{!}{ \input{poisson_plot.tex}}
    \end{figure}
  \end{column}
\end{columns}
\end{frame}

\section{Conclusiones}
\begin{frame}{Conclusiones}
\begin{itemize}
\item Medida de la variación del radio de giro con respecto a la temperatura (novedad).
\item Medida precisa de la temperatura crítica, $\alpha$, $\nu$ en acuerdo con
  trabajos anteriores.
\item Comprobación del escalado el radio de giro, tanto en la fase plana como en la
rugosa.
\item Medida del módulo de Poisson para diferentes tamaños del sistema, aunque los
errores de las medidas no nos permiten estimarlo en el límite a área infinita.
\item Es necesario conseguir más resultados para tamaños mayores del sistema.
\end{itemize}
\end{frame}
\end{document}
