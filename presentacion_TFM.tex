\documentclass[xcolor=dvipsnames]{beamer}
\usepackage[utf8x]{inputenc}
\usepackage[spanish]{babel}
\usepackage{amsmath}
\usepackage{amsfonts}
\usepackage{graphicx}
\usepackage{xcolor,color}
\usepackage{subfigure}

\usetheme{Singapore}
%\useinnertheme{rectangles}
\definecolor{fblue}{rgb}{0.6,0.6,0.9}
%\usefonttheme{structuresmallcapsserif}
\usefonttheme{serif}

%\usecolortheme{whale} % outer color
\usecolortheme[RGB={0,50,200}]{structure}


\setbeamertemplate{navigation symbols}{}

\mode<beamer>{\setbeamertemplate{blocks}[rounded][shadow=false]}
\setbeamercolor{block title}{bg=fblue}
\setbeamercolor{block body}{bg=gray!10}
%\logo{\includegraphics[width=28bp]{./Pictures/UEX.jpg}}


\title[C. crítico de las membranas cristalinas]{Comportamiento crítico de las membranas cristalinas}
\author[P. Monroy Dir.: Juan J.Ruiz Lorenzo]{Pedro Monroy Pérez \\ Director:
  J.J. Ruiz Lorenzo}
\date{\today}
\institute[UEx]{Dep. de Física Teórica, Facultad de Ciencias, Universidad de Extremadura}

\begin{document}

\begin{frame}[plain]{}
\titlepage
\end{frame}


\begin{frame}{Indice}
  \tableofcontents[hideallsubsections]
\end{frame}
\section{Introducción}
\begin{frame}{Introducción}

\end{frame}
\section{Estudio Analítico}

\begin{frame}{Descripción geométrica}
  \begin{columns}
    \begin{column}{5.8cm}
      \centering
      Coordenadas Internas:
      \begin{figure}[h]
        \resizebox{\columnwidth}{!}{\input{coordenadas_internas-fig}}
      \end{figure}
      \begin{equation*}
        \mathbf{x}\equiv (x^1,x^2)\in \mathbb{R}^2
      \end{equation*}
    \end{column}
    \begin{column}{5.8cm}
      \centering
      Coordenadas Externas:
      \begin{figure}[h]
        \resizebox{\columnwidth}{!}{\input{coordenadas_externas-fig}}
      \end{figure}
      \begin{equation*}
        \vec{r}=x\,\vec{i}+y\,\vec{j}+z\,\vec{k}\equiv (x,y,z)\in \mathbb{R}^3
      \end{equation*}
    \end{column}
  \end{columns}
\end{frame}

\begin{frame}{Los Vectores Tangentes y la Métrica inducida}
  \begin{columns}
    \begin{column}{5.8cm}
      \centering
      \begin{figure}[h]
        \resizebox{\columnwidth}{!}{\input{vectores_tangentes-fig}}
      \end{figure}
      \begin{equation*}
        \vec{t}_{\alpha}=\frac{\partial \vec{r}}{\partial
          x^{\alpha}}=\partial_{\alpha}\vec{r} \qquad \alpha=1,2;
      \end{equation*}

    \end{column}
    \begin{column}{5.8cm}
      \centering
      Coordenadas Externas:
      \begin{figure}[h]
        \resizebox{\columnwidth}{!}{\input{coordenadas_externas-fig}}
      \end{figure}
      \begin{equation*}
        \vec{r}=x\,\vec{i}+y\,\vec{j}+z\,\vec{k}\equiv (x,y,z)\in \mathbb{R}^3
      \end{equation*}
    \end{column}
  \end{columns}
\end{frame}

\begin{frame}{Curvatura}

\end{frame}

\begin{frame}{Teoría de Landau}
\end{frame}
\begin{frame}{Parámetro de orden}
\end{frame}
\begin{frame}{Energía Libre de Landau}
\end{frame}
\begin{frame}{Aproximación del campo medio}
\end{frame}
\begin{frame}{Exponentes críticos}
\end{frame}
\begin{frame}{Relaciones de escala del radio de giro}
\end{frame}
\begin{frame}{Fase plana}
\end{frame}
\begin{frame}{Módulo de Poisson}
\end{frame}

\section{Simulaciones Numéricas}
\begin{frame}{Modelo discreto}
\end{frame}
\begin{frame}{límite continuo}
\end{frame}
\begin{frame}{Algoritmo de Metropolis}
\end{frame}
\begin{frame}{Tratamiento de los datos}
\end{frame}
\begin{frame}{Estimación exponentes críticos}
\end{frame}
\begin{frame}{Densidad espectral}
\end{frame}
\section{Resultados}
\begin{frame}{}
\end{frame}
\section{Conclusiones}
\begin{frame}
\end{frame}
\end{document}
