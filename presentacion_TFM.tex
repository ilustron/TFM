\documentclass[xcolor=dvipsnames]{beamer}
\usepackage[utf8x]{inputenc}
\usepackage[spanish]{babel}
\usepackage{amsmath}
\usepackage{amsfonts}
\usepackage{graphicx}
\usepackage{xcolor,color}
\usepackage{subfigure}

\usetheme{Singapore}
%\useinnertheme{rectangles}
\definecolor{fblue}{rgb}{0.6,0.6,0.9}
%\usefonttheme{structuresmallcapsserif}
\usefonttheme{serif}

%\usecolortheme{whale} % outer color
\usecolortheme[RGB={0,50,200}]{structure}


\setbeamertemplate{navigation symbols}{}

\mode<beamer>{\setbeamertemplate{blocks}[rounded][shadow=false]}
\setbeamercolor{block title}{bg=fblue}
\setbeamercolor{block body}{bg=gray!10}
%\logo{\includegraphics[width=28bp]{./Pictures/UEX.jpg}}


\title[C. crítico de las membranas cristalinas]{Comportamiento crítico de las membranas cristalinas}
\author[P. Monroy Dir.: Juan J.Ruiz Lorenzo]{Pedro Monroy Pérez \\ Director:
  J.J. Ruiz Lorenzo}
\date{\today}
\institute[UEx]{Dep. de Física, Facultad de Ciencias, Universidad de Extremadura}

\begin{document}
\decimalpoint
\begin{frame}[plain]{}
\titlepage
\end{frame}


\begin{frame}{Indice}
  \tableofcontents[hideallsubsections]
\end{frame}
\section{Introducción}
\begin{frame}{Introducción}

\end{frame}
\section{Estudio Analítico}

\begin{frame}{Descripción geométrica}
  \begin{columns}
    \begin{column}{5.8cm}
      \centering
      Coordenadas Internas:
      \begin{figure}[h]
        \resizebox{\columnwidth}{!}{\input{coordenadas_internas-fig}}
      \end{figure}
      \begin{equation*}
        \mathbf{x}\equiv (x^1,x^2)\in \mathbb{R}^2
      \end{equation*}
    \end{column}
    \begin{column}{5.8cm}
      \centering
      Coordenadas Externas:
      \begin{figure}[h]
        \resizebox{\columnwidth}{!}{\input{coordenadas_externas-fig}}
      \end{figure}
      \begin{equation*}
        \vec{r}=x\,\vec{i}+y\,\vec{j}+z\,\vec{k}\equiv (x,y,z)\in \mathbb{R}^3
      \end{equation*}
    \end{column}
  \end{columns}
\end{frame}

\begin{frame}
  \begin{columns}[T]
    \begin{column}{5.2cm}
      \centering
      \begin{itemize}
      \item Vectores Tangentes:
      \end{itemize}
      \begin{figure}[h]
        \resizebox{\columnwidth}{!}{\input{vectores_tangentes-fig}}
      \end{figure}
      \begin{equation*}
        \vec{t}_{\alpha}=\frac{\partial \vec{r}}{\partial
          x^{\alpha}}=\partial_{\alpha}\vec{r} \qquad \alpha=1,2;
      \end{equation*}
     
    \end{column}
    \begin{column}{6.2cm}
      \centering
      \begin{itemize}
\item Métrica inducida $g_{\alpha\beta}=\vec{t}_{\alpha}\cdot\vec{t}_{\beta}$:
      \begin{equation*}
      \vec{V}\cdot\vec{W}=g_{\alpha\beta}V^{\alpha}W^{\beta}.
      \end{equation*}
      \item Métrica inversa $g^{\alpha\beta}$:
      \begin{equation*}
       g_{\alpha\gamma}g^{\gamma\beta}= \delta_{\alpha}^{\ \beta}=\begin{cases}
          1&\text{si $\alpha=\beta$}\\
          0&\text{si $\alpha\neq\beta$}.
          \end{cases}
      \end{equation*}
      \item Correspondencia entre vectores covariantes y contravariantes:
      \begin{equation*}
        V_{\alpha}=g_{\alpha\beta}V^{\beta}\rightarrow V^{\alpha}=g^{\alpha\beta}V_{\beta}, 
      \end{equation*}
      \begin{equation*}
        \vec{V}\cdot\vec{W}=V_{\alpha}W^{\alpha}=V^{\alpha}W_{\alpha}.
      \end{equation*}
      \end{itemize}
    \end{column}
  \end{columns}
\end{frame}

\begin{frame}{Curvatura}
  \begin{columns}[T]
    \begin{column}{6.2cm}
      \centering
      Tensor de curvatura extrínseca $K_{\alpha\beta}$:
      \begin{equation*}
        \vec{K}_{\alpha\beta}=K_{\alpha\beta}\vec{n},
      \end{equation*}
      \begin{equation*}
        K_{\alpha\beta}\equiv\left(\begin{array}{cc}
            \mathcal{K}_1 & 0\\
            0 & \mathcal{K}_2\\
          \end{array}\right)\, ,
      \end{equation*}
      \begin{equation*}
        \mathcal{R}_{\alpha}=\frac{1}{\mathcal{K}_{\alpha}} \, , \qquad \alpha=1,2.
      \end{equation*}  
    \end{column}
    \begin{column}{6.25cm}
      \centering
      \begin{figure}[h]
        \resizebox{\columnwidth}{!}{\input{curvatura-fig}}
      \end{figure} 
    \end{column}
  \end{columns}   
  \begin{align*}
    \text{Curvatura Media}\longrightarrow H&=\frac{\mathcal{K}_1+\mathcal{K}_2}{2}=\frac{1}{2}K_{\alpha}^{\ \alpha},\\
    \text{Curvatura Gaussiana}\longrightarrow  K&=\mathcal{K}_1\mathcal{K}_2=\frac{1}{2}(K_{\alpha}^{\ \alpha}K_{\beta}^{\ \beta}-K_{\alpha}^{\ \beta}K^{\alpha}_{\ \beta}).
  \end{align*}
\end{frame}

\begin{frame}{Teoría de Landau}
\begin{description}
\item[Parámetro de orden:] Variable que cuantifica el grado de orden de las
  fases, es nulo en la fase de mayor simetría. 
  \begin{itemize}
  \item Microscópico $\rightarrow\eta(\mathbf{x})$, $\mathbf{x}$ denota partículas
    individuales.
  \item Grano grueso $\rightarrow\bar{\eta}(\mathbf{x}_{\Lambda})$,
    $\mathbf{x}_{\Lambda}$ denota bloques de partículas.
\end{itemize}
\item[Energía libre de Landau]$\rightarrow F[\bar{\eta}(\mathbf{x}_{\Lambda}),T]$, potencial
  efectivo resultado de la integración de los grados de libertad microscópicos.
\end{description}
\begin{equation*}
  e^{-F[\bar{\eta}(\mathbf{x}_{\Lambda}),T]}=\int D[\eta(\mathbf{x})]\;
  e^{-\beta
    H[\eta(\mathbf{x}),T]}\;\delta\left[\frac{1}{A_{\Lambda}}\int_{\Lambda}
    d^2\mathbf{x}\; \eta(\mathbf{x})-\bar{\eta}(\mathbf{x}_{\Lambda})\right], 
\end{equation*}
\begin{equation*}
  \mathcal{Z}[T]=\int D[\bar{\eta}(\mathbf{x}_{\Lambda})]\; e^{-F[\bar{\eta}(\mathbf{x}_{\Lambda}),T]}=\int D[\eta(\mathbf{x})]\; e^{-\beta H[\eta(\mathbf{x}),T]}.
\end{equation*}
\end{frame}

\begin{frame}{Parámetro de orden}
\centering
\begin{columns}[T]
    \begin{column}{5.8cm}
      Fase plana (ordenada):
        \centering
        \begin{figure}[h]
        \resizebox{\columnwidth}{!}{\input{fase_plana-fig}}
        \end{figure}
        $$ \langle\vec{t}_{\alpha}\rangle\neq 0$$
    \end{column}
    
    \hspace{0.5cm}
    
    \begin{column}{5.8cm}
      Fase rugosa (desordenada):
        \centering
        \begin{figure}[h]
        \resizebox{\columnwidth}{!}{\input{fase_arrugada-fig}}
        \end{figure}
        $$ \langle\vec{t}_{\alpha}\rangle= 0$$
    \end{column}
  \end{columns}
\vspace{0.8cm}

Parámetro de orden $\eta(\mathbf{x}_{\Lambda})\longrightarrow$ Vectores tangentes $\vec{t}_{\alpha}(\mathbf{x}_{\Lambda})$ 
\end{frame}
\begin{frame}{Energía Libre de Landau}
A partir de las siguientes ligaduras encontramos el potencial la energía Libre
de Landau:
  \begin{description}
  \item[Localidad e invariancia traslacional.]
  \item[Simetría rotacional traslacional en $\mathbb{R}^2$ y $\mathbb{R}^3$.]
  \item[Imposibilidad de autointersección.]
  \end{description}

  \begin{equation*}
    F[\vec{t}_{\alpha}(\mathbf{x}),T]=F_E[\vec{t}_{\alpha}(\mathbf{x}),T]+F_C[\vec{t}_{\alpha}(\mathbf{x}),T]+F_b[\vec{t}_{\alpha}(\mathbf{x}),T],
  \end{equation*}
  donde
  \begin{align*}
    F_E[\vec{t}_{\alpha}(\mathbf{x}),T]&=\int d^2\mathbf{s}\left[
      \frac{t}{2}\!(\vec{t}_{\alpha}\!\cdot\!\vec{t}^{\alpha})+
      u(\vec{t}_{\alpha}\!\cdot\!\vec{t}_{\beta})(\vec{t}^{\alpha}\!\cdot\!\vec{t}^{\beta})+
      v(\vec{t}_{\alpha}\!\cdot\!\vec{t}^{\alpha})(\vec{t}_{\beta}\!\cdot\!\vec{t}^{\beta})\right],\\
    F_C[\vec{t}_{\alpha}(\mathbf{x}),T]&= \frac{\kappa}{2}\int d^2\mathbf{s}\
     \partial_{\alpha}\vec{t}^{\alpha}\cdot\partial^{\beta}\vec{t}_{\beta},\\ 
   F_b[\vec{t}_{\alpha}(\mathbf{x}),T]&=\frac{b}{2}\int d^2\mathbf{x} d^2\mathbf{x'}
\delta^2(\vec{r}(\mathbf{x})-\vec{r}(\mathbf{x'})).
  \end{align*}
\end{frame}
\begin{frame}{Aproximación del campo medio}
  \begin{columns}%[T]
    \begin{column}{6.2cm}
      \centering
      Despreciando las fluctuaciones en la fase plana:
     \begin{align*}
       \vec{r}(\mathbf{x})&\simeq(\zeta x^1,\zeta x^2,0),\\
       \vec{t}_1(\mathbf{x})&\simeq(\zeta ,0,0),\\
       \vec{t}_2(\mathbf{x})&\simeq(0,\zeta,0).
     \end{align*}
     
    \end{column}
    \begin{column}{6.25cm}
      \centering
      \begin{figure}[h]
      \resizebox{\columnwidth}{!}{\input{campo_medio-fig}}
      \end{figure} 
    \end{column}
  \end{columns}
Tenemos una configuración homogénea:
\begin{equation*}
  F(\vec{t}_{\alpha},T)\simeq Af_M(\zeta)\quad \text{con}\quad f_M(\zeta,T)=2\zeta^2\left( \frac{t}{2} +(u+2v)\zeta^2\right)
\end{equation*}
\end{frame}
\begin{frame}
\begin{align*}
 \mathcal{Z}[T]=\int d\zeta\;e^{-Af_M(\zeta,T)}&\simeq\sqrt{\frac{2\pi}{A\ddot{f}_M(\zeta_0,T)}}\;
 e^{-Af_M(\zeta_0,T)}\\
 \lim_{A\rightarrow \infty}\; \frac{1}{A} \ln \mathcal{Z}[T]&=-f_M(\zeta_0,T)=-\beta f_H(T)
\end{align*}
\begin{columns}[T]
    \begin{column}{5.8cm}
      \centering
      Mínimo de $f_M(\zeta,T)$:
      $$\zeta^2_0=\begin{cases}
        0& \text{si $t>0$}\\
        \frac{-t}{4(u+2v)}& \text{si $t<0$}\\
      \end{cases}$$
\end{column}
\begin{column}{5.8cm}
      \centering
      \begin{figure}[h]
      \resizebox{\columnwidth}{!}{\input{energia_libre_CM-fig}}
      \end{figure} 
    \end{column}
  \end{columns}
\end{frame}
\begin{frame}{Exponentes críticos}
\begin{itemize}
  \item La teoría de Landau no tiene en cuenta las fluctuaciones del parámetro de
orden.
  \item  La teoría del grupo de renormalización encuentra que
    estas fluctuaciones contribuyen en una parte no analítica en la energía
    libre:
    \begin{equation}
      f_s(t)\sim|t|^{2\nu}
    \end{equation}
  \item explica el comportamiento divergente en $t\simeq 0$ de los observables
    termodinámicos, (ej. $C_V\sim|t|^{-\alpha}$).
  \item No todos los exponentes críticos son independientes entre sí. En
    nuestro caso, únicamente tenemos un exponente crítico independiente.
\end{itemize}
\end{frame}
\begin{frame}{Relaciones de escala del radio de giro}
  \begin{equation*}
    R_G^2=\frac{1}{dA_D}\int d^D\mathbf{x}d^D\mathbf{x}' \langle |
    \vec{r}(\mathbf{x})-\vec{r}(\mathbf{x}')|^2\rangle\stackrel{D=2}{-\!\!\!\longrightarrow}\frac{1}{3A}\int d^2\mathbf{x}\, \langle
\vec{R}(\mathbf{x})\cdot\vec{R}(\mathbf{x})\rangle^2,
  \end{equation*}
donde $\vec{R}(\mathbf{x})=\vec{r}(\mathbf{x})-\vec{r}_{CM}$. Podemos aproximar:
  \begin{equation*}
    \vec{t}_{\alpha}=\partial_{\alpha} \vec{r}\simeq \frac{R_G}{L}
    \frac{\vec{t}_{\alpha}}{|\vec{t}_{\alpha}|}\quad \text{y}\quad \int
    d^D\mathbf{x}\simeq L^D,
\end{equation*}
\begin{equation*}
  f(R_G,t)\simeq t R_G^2 L^{D-2}+(u+Dv) R_G^4 L^{D-4}+\kappa R_G^2 L^{D-4}.
\end{equation*}
Minimizando $f(R_G,t)$ respecto a $R_G$ ($D=2$):
\begin{equation*}
  R_G\sim L^{\nu_F}= L^{2/d_H}\rightarrow\begin{cases}
    \text{F. plana }& \nu_F=1\quad d_H=2\\
    \text{Transición }& \nu_F=0.73\quad d_H=2.74\\
    \text{F. rugosa }& \nu_F=0\quad d_H=\infty\ (R_G^2\sim\log L)
  \end{cases}
\end{equation*}
\end{frame}
\begin{frame}{Fase plana}
Coordenadas ortonormales en el plano base ($\zeta=1$) $\Rightarrow\vec{r}(\mathbf{x})=(\mathbf{x}+\mathbf{u(\mathbf{x})},h(\mathbf{x}))$.
  \begin{columns}[T]
    \begin{column}{5.8cm}
      \centering
      \begin{equation*}
        (d\vec{r})^2=(d\vec{r}_0)^2+2u_{\alpha\beta}dx^{\alpha}dx^{\beta},
      \end{equation*}
      con $u_{\alpha\beta}$, tensor de deformaciones:
      \begin{multline*}
        u_{\alpha\beta}=\frac{1}{2}(\vec{t}_{\alpha}\cdot\vec{t}_{\beta}-\delta_{\alpha\beta})\
        \Rightarrow\\\Rightarrow \
        \vec{t}_{\alpha}\cdot\vec{t}_{\beta}=\delta_{\alpha\beta}+2u_{\alpha\beta}.
      \end{multline*}     
    \end{column}
    \begin{column}{5.8cm}
      \centering
      \begin{figure}[h]
        \resizebox{\columnwidth}{!}{\input{deformacion-fig}}
      \end{figure} 
    \end{column}
  \end{columns}
  
  \begin{align*}
    F_E[u_{\alpha\beta},T]&=\int d^2\mathbf{x}
    \left[
      \mu u_{\alpha\beta}u^{\alpha\beta}\! +\!
      \frac{\lambda}{2}u_{\alpha}^{\ \alpha}u^{\beta}_{\ \beta}\right],
    \quad\text{con}\quad -t=\mu+\lambda\\
    F_C[\mathbf{y},T]&\simeq\frac{\kappa}{2}\int d^2\mathbf{y} K_{\alpha}^{\
      \beta}K^{\alpha}_{\ \beta}\simeq \frac{1}{2}\hat{\kappa}\int d^2\mathbf{y}\, H^2 \quad
\text{con} \quad \hat{\kappa}=4\kappa
  \end{align*}
\end{frame}

\begin{frame}{Módulo de Poisson}
\centering
\begin{equation*}
f_E[u_{\alpha\beta},T]=
\mu \left(u_{\alpha\beta}-\frac{1}{2}\delta_{\alpha\beta}u_{\gamma}^{\ \gamma}\right)
    \left(u^{\alpha\beta}-\frac{1}{2}\delta^{\alpha\beta}u_{\nu}^{\ \nu}\right)+
\frac{1}{2}K(u_{\alpha}^{\ \alpha})(u_{\beta}^{\ \beta}),
\end{equation*}
con $K>0\, ,\ \mu>0$.

\begin{columns}[T]
    \begin{column}{5.8cm}
      \centering
      \begin{equation*}
        \sigma=-\frac{\delta x^2 / x^2}{\delta x^1 / x^1}=-\frac{u_{22}}{u_{11}}=\frac{K-\mu}{K+\mu}.
      \end{equation*}
      Membrana $D$-dimensional:
      \begin{equation*}
        \sigma(D)=-\frac{1}{D+1}\ \Rightarrow \ \sigma(2)=-\frac{1}{3}. 
      \end{equation*}
    \end{column}
    \begin{column}{5.8cm}
      \centering
      \begin{figure}[h]
        \resizebox{\columnwidth}{!}{\input{Poisson-fig}}
      \end{figure} 
    \end{column}
  \end{columns}
\end{frame}

\section{Simulaciones Numéricas}
\begin{frame}{Modelo discreto}

\end{frame}
\begin{frame}{límite continuo}
\end{frame}
\begin{frame}{Algoritmo de Metropolis}
Para un valor fijo de $\kappa$:
\begin{itemize}
  \item  Se modifica $\vec{r}_i\rightarrow \vec{r}_i+\vec{\epsilon}$, con
    $\vec{\epsilon}$ elegido aleatoriamente en un cubo de lado $\delta$.
  \item  Si
    $\Delta\mathcal{H}=\mathcal{H}_{NEW}-\mathcal{H}_{OLD}<0$ se
    acepta la nueva posición. 
  \item Si  $\Delta\mathcal{H}>0$, se acepta la nueva posición con
    probabilidad $e^{-\Delta\mathcal{H}}$. 
  \item Se repiten los pasos anteriores para otro punto.
\end{itemize}
\begin{columns}[T]
    \column{.44\textwidth}
    \begin{block}{}
      Una actualización corresponde a aplicar el
      algoritmo a todos los $N$ puntos de la red.
    \end{block}
    \column{.44\textwidth}
    \begin{block}{}
      $\delta$ se ajusta de modo que en cada actualización el número de cambios
      aceptados sea $N/2$.
    \end{block}
\end{columns}

\begin{description}
\centering
\item[Configuraciones resultantes] $\rightarrow$ $P(\vec{r}_i)\propto e^{-\mathcal{H}(\kappa)}$
\end{description}
\end{frame}
\begin{frame}{Tratamiento de los datos}
A partir de las configuraciones resultantes, se estiman los observables teniendo en cuenta:
\begin{columns}
  \column{.55\textwidth}
  \begin{figure}[h]
        \resizebox{\columnwidth}{!}{\input{ejemplo_termal-fig}}
      \end{figure}
  \column{.5\textwidth}
  \begin{itemize}
  \item Tiempo de Termalización.
  \item Periodo de correlación entre medidas.
  \end{itemize}
\end{columns}
\end{frame}
\begin{frame}{Cálculo de errores (Método de Jacknife)}
  \begin{itemize}
  \item Sean $\mathcal{Q}_i,(i=1,\dots N)$ los valores obtenidos de un
    observable $\mathcal{Q}$ en una evolución de Monte-Carlo. 
  \item Dividimos el conjunto de $N$ en $K$ bloques con $n=\frac{N}{K}$
    elementos.
  \item Definimos $\bar{\mathcal{Q}}_k^n(k=1,\dots K)$ el promedio de los
    datos sin el bloque $k$-ésimo:
    $$\bar{\mathcal{Q}}_k^n=\frac{1}{N-n}\left(\sum^{n(k-1)}_{i=1}\mathcal{Q}_i+\sum^{N}_{i=nk+1}\mathcal{Q}_i\right).$$
\item La estimación del error:
$$\delta\bar{\mathcal{Q}}_n=\sqrt{\frac{K-1}{K}\sum^K_{k=1}(\bar{\mathcal{Q}}^n_k-\bar{\mathcal{Q}})},$$
\item ya que
  $$ \langle (\delta\bar{\mathcal{Q}}_n)^2\rangle=\sigma^2(\mathcal{Q})\frac{1}{N}.$$
\end{itemize}
\end{frame}

\begin{frame}
\begin{itemize}
\item Si $n<<\text{Correlación entre medidas}\rightarrow
  \delta\bar{\mathcal{Q}}_n$ aumenta con el tamaño $n$
\item Si $n\simeq \text{Correlación entre
    medidas}\rightarrow\delta\bar{\mathcal{Q}}_n=const.$ (plateau)
\end{itemize}
\begin{columns}
  \begin{column}{6cm}
    \begin{figure}
      \resizebox{\columnwidth}{!}{\input{ejemplo_logtermal-fig}}
    \end{figure}
  \end{column}
  \begin{column}{6cm}
     \begin{figure}
       \resizebox{\columnwidth}{!}{\input{ejemplo_error-fig}}
    \end{figure}
  \end{column}
\end{columns}
\begin{center}
$\tau_0=120000$
\end{center}
\end{frame}

\begin{frame}{Estimación exponentes críticos}
\end{frame}
\begin{frame}{Densidad espectral}
\end{frame}
\section{Resultados}
\begin{frame}{}
\end{frame}
\section{Conclusiones}
\begin{frame}
\end{frame}
\end{document}
