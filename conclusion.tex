\chapter{Conclusiones}

Como novedad respecto a trabajos anteriores hemos conseguido medir la
variación del radio de giro con respecto al módulo de curvatura (temperatura),
apoyándonos en el teorema de la respuesta lineal. Observamos que esta magnitud
tiene una máximo pronunciado para cada tamaño del sistema al igual el calor
específico del sistema, que también fue calculado. El valor de estos máximos, tanto de la
variación del radio de giro como del calor específico, y el módulo de
curvatura al que tiene lugar se han medido satisfactoriamente mediante el
método de la densidad espectral. Con estos resultados, hemos conseguido
estimar $\alpha$, el exponente crítico del calor específico, mediante
hyperscaling; $\nu$ el exponente crítico de la longitud de correlación y
tenemos dos estimaciones de la temperatura crítica del sistema, a área
infinita. En la fase plana, hemos medido el módulo de Poisson para diferentes
tamaños del sistema, aunque los errores de las medidas no nos permiten estimarlo
en el límite a área infinita, se observa una tendencia a asintótica al valor
$-1/3$, lo que están de acuerdo con simulaciones y cálculos anteriores.
También hemos comprobado las relaciones de escala para el radio de giro en la
fase plana, rugosa y en la transición. Los resultados en ambas fases coinciden
con las predicciones teóricas en muy buen acuerdo, y respecto al scaling en la
transición de fase, nos proporciona otra estimación para $\nu$, que tiene un
valor en acuerdo con simulaciones y cálculos anteriores.

Todas las medidas realizadas tienen errores altos, es necesario conseguir
resultados para tamaños mayores del sistema, y en algunos casos, más simulaciones para los
tamaños ya estudiados. Actualmente están en proceso de ejecución simulaciones
del sistema hasta $256^2$ nodos. Con los nuevos datos, podremos estimar mejor
los exponentes $\nu$,$\omega$ y $\alpha$; y extrapolar el valor del módulo de
Poisson a área infinita para dos o tres valores de $\kappa$. También, sería interesante
el estudio de la función de correlación, tanto en la fase plana como en la
transición plana-rugosa, estimando el exponente crítico de dimensión anómala,
los últimos resultados en simulaciones son hasta un tamaño de $128^2$
%%% Local Variables: 
%%% mode: latex
%%% TeX-master: "TFM"
%%% End: 
