\chapter{Conclusiones}

Como novedad respecto a trabajos anteriores hemos conseguido medir la
variación del radio de giro con respecto a la temperatura,
apoyándonos en el teorema de la respuesta lineal. Observamos que esta magnitud
tiene una máximo pronunciado para cada tamaño del sistema, al igual el calor
específico del sistema, que también fue medido. Conseguimos caracterizar estos
máximos, su valor y la temperatura (rigidez de curvatura $\kappa$) a la que
tienen lugar, tanto para la
variación del radio de giro como para el calor específico, usando el
método de la densidad espectral. Con estos máximo, hemos conseguido
estimar el valor de $\alpha$, el exponente crítico del calor específico, mediante
hyperscaling; de $\nu$ el exponente crítico de la longitud de correlación y de
la temperatura crítica del sistema (a área infinita), y compararlos con los
obtenidos en artículos antiguos en los que medían los picos máximos mediante
una discretización no muy fina de la variable $\kappa$, lo que inducía a
errores incontrolables tanto sobre la posición ($\kappa_C(L)$) como en el
valor del observable en el máximo. Respecto al estudio de las fases, hemos
comprobado como escala el radio de giro, tanto en la fase plana como en la
rugosa con resultados precisos y satisfactorios. En la fase plana, hemos
medido el módulo de Poisson para diferentes tamaños del sistema, aunque los
errores de las medidas no nos permiten estimarlo 
en el límite a área infinita, se observa una tendencia a asintótica al valor
$-1/3$, tendencia que está en acuerdo con simulaciones y cálculos anteriores.

Todos los ajustes realizados adolecen de pocos grados de libertad, es necesario conseguir
más resultados para tamaños mayores del sistema, y en algunos casos, más simulaciones para los
tamaños ya estudiados, de esta forma minimizaremos los errores haciendo más
precisas nuestras estimaciones. Actualmente están en proceso de ejecución simulaciones
del sistema hasta $256^2$ nodos (las últimas simulaciones numéricas son hasta
$128^2$ nodos). Con los nuevos datos, podremos estimar mejor
los exponentes $\nu$, $\nu_F$, $d_H$, $\alpha$ e incluso otros exponentes
críticos que no han sido calculados en este trabajo. Con más estadística será
posible extrapolar el valor del módulo de Poisson a área infinita para dos o
tres valores de $\kappa$. También, sería interesante 
el estudio de la función de correlación, tanto en la fase plana como en la
transición plana-rugosa, estimando el exponente crítico de dimensión
anómala. Y si los resultados nos lo permiten, caracterizar las correcciones de
escala dominantes.

%%% Local Variables: 
%%% mode: latex
%%% TeX-master: "TFM"
%%% End: 
