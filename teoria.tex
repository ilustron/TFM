
\chapter{Estudio analítico}

\section{Descripción geométrica de la membranas cristalina}

Consideraremos una membrana como una superficie (grosor nulo) elástica y flexible
bidimensional alojada en un espacio tridimensional Euclídeo. 
Se pueden interpretar las interacciones de los puntos de la superficie como
conexiones, es decir, dos puntos que están interaccionando entre sí han
establecido una conexión entre ambos. Una membrana cristalina corresponde al
tipo de superficie en la que las conexiones entre sus puntos forman una red
regular cuya conectividad permanece constante. Esta última afirmación
significa que si existe una conexión entre dos puntos, está persistirá sea
cual sea la configuración global de la superficie. Podremos entonces,
encontrar estados de la membrana en los que puntos que no interaccionan entre
sí estén mucho menos distanciados que los puntos que sí están interaccionando.

Usaremos una descripción continua de la membrana cristalina en donde la
posición en el espacio tridimensional de sus puntos estará descrita por sus
radio vectores $\vec{r}(\mathbf{x})$, siendo $\mathbf{x}$ la parametrización
de la superficie. Debido al carácter de conectividad constante, podemos
parametrizar la superficie de forma que $\mathbf{x}$ sea un índice continuo
que etiquete cada partícula. Por tanto, para la descripción geométrica de la
membrana cristalina necesitamos dos sistemas de coordenadas:

\begin{figure}[h]
\centering
\subfigure[Coordenadas internas]{
 \resizebox{160bp}{!}{\input{coordenadas_internas-fig}}}
\quad
\subfigure[Coordenadas internas]{
 \resizebox{160bp}{!}{\input{coordenadas_externas-fig}}}
\caption{Coordenadas }
\end{figure}

\begin{description}
\item[Coordenadas internas:] Etiquetan cada partícula, de modo que, permanecen
  constante en cada configuración.
  \begin{equation*}
  \mathbf{x}\equiv (x^1,x^2)\in \mathbb{R}^2
  \end{equation*}
\item[Coordenadas externas:] Describen las posiciones de cada partícula,
  varían para cada configuración. Utilizaremos el sistema de referencia usual
  cartesiano ortonormal.
  \begin{equation*}
    \vec{r}=x\,\vec{i}+y\,\vec{j}+z\,\vec{k}\equiv (x,y,z)\in \mathbb{R}^3
  \end{equation*}
\end{description}

En cada punto $P\equiv\vec{r}(\mathbf{x})$ tenemos un plano tangente $T_PS$ a
la superficie cuya base son los siguientes vectores tangentes:

\begin{equation*}
\vec{t}_{\alpha}=\frac{\partial \vec{r}}{\partial
  x^{\alpha}}=\partial_{\alpha}\vec{r} \qquad \alpha=1,2
\end{equation*}

\begin{figure}[h]
\centering
 \input{vectores_tangentes-fig}
\caption{Vectores tangentes en el punto $\vec{r}(\mathbf{x})$}
\end{figure}

Cualquier vector $\vec{V}$ tangente a la superficie en el punto $P$ se podrá
descomponer como:

\begin{equation*} 
\vec{V}=\sum^2_{\alpha=1}V^{\alpha}\vec{t}_{\alpha}=V^{\alpha}\vec{t}_{\alpha}
\end{equation*}

con el convenio de índices repetidos de Einstein. Las dos cantidades
$V^{\alpha}$ son las componentes del vector $\vec{V}$ en el sistema de
coordenadas internas $\mathbf{x}$ y dependen de la elección del sistema de
coordenadas.
 
La distancia $dl$ entre dos puntos de coordenadas internas $\mathbf{x}$ y
$\mathbf{x}+d\mathbf{x}$ respectivamente es:

\begin{equation*}
dl^2=(\vec{r}(\mathbf{x}+d\mathbf{x})-\vec{r}(\mathbf{x}))^2=g_{\alpha\beta}dx^{\alpha}dx^{\beta}
\end{equation*}

Donde $g_{\alpha\beta}=\vec{r}\cdot\partial_{\beta}\vec{r}$ es el tensor
métrico que nos permite definir el producto escalar entre dos vectores
tangentes $\vec{V}$ y $\vec{W}$ en el punto $P$:
\begin{equation*}
\vec{V}\cdot\vec{W}=g_{\alpha\beta}V^{\alpha}W^{\beta}
\end{equation*}

%Vectores contravariantes y covariantes

\section{Teoría de Landau de la transición de fase}

En una membrana se pueden encontramos dos fases: La fase plana y la fase
rugosa. Ambas se distinguen en sus simetrías y se localizan cada lado
de la temperatura crítica $T_c$, a la cual tiene lugar la transición. 
Sobre la temperatura crítica la membrana es invariante frente a rotaciones,
la membrana se encuentra en la fase rugosa. En cambio, bajo la temperatura
crítica, se encuentra en la fase 
plana donde el promedio del vector normal a la superficie
$\langle \vec{n}\rangle$ define una dirección privilegiada en el espacio,
destruyendo la invariancia frente a rotaciones cuyo eje es perpendicular a
$\langle \vec{n}\rangle$. Los estados de equilibrio que se encuentran en esta
última fase, al tener una menor simetría que la del hamiltoniano del sistema,
se denominan estados con ruptura de simetría.
 
Debido a la reducción de la simetría, es necesario un nuevo parámetro que 
describa la termodinámica de la fase ordenada. Este parámetro
extra, denominado parámetro de orden y que denotaremos por $\eta(\mathbf{x})$, es una
variable que cuantifica el grado de orden de
las fases, por lo que usualmente se define de forma que sea nulo en la fase
de mayor simetría.

\begin{figure}[h]
\centering
\subfigure[Fase plana]{
 \resizebox{160bp}{!}{\input{fase_plana-fig}}}
\quad
\subfigure[Fase rugosa]{
 \resizebox{160bp}{!}{\input{fase_arrugada-fig}}}
\end{figure}

%A partir del potencial efectivo es posible obtener las variables
%termodinámicas mediante la función de partición $\mathcal{Z}$ que viene dada
%por la siguiente integral funcional:

%\begin{equation*}
%\mathcal{Z}[T]=\int D[\vec{r}(\mathbf{x})]\; e^{-\beta 
%F[\vec{r}(\mathbf{x})]}
%\end{equation*}

%En general, evaluar esta integral es muy complicado. Sin embargo podemos
%obtener una aproximación aplicando la aproximación del punto de silla o campo
%medio. 

La teoría de Landau de la transición de fase consiste en una aproximación fenomenológica
que evita tratar con la descripción microscópica del sistema. Para ello utiliza una
descripción de \textit{grano grueso}, que en el caso de una membrana significa
que $\mathbf{x}_{\Lambda}$ continua siendo un índice continuo excepto que etiqueta
\textit{bloques} de partículas y no partículas individuales. El
correspondiente parámetro de orden de grano grueso $\eta(\mathbf{x})$
corresponderá a un promedio sobre todas las partículas que forman el bloque. 
La función de partición del sistema será

\begin{equation*}
\mathcal{Z}[T]=\int D[\eta(\mathbf{x})]\; e^{-F[\eta(\mathbf{x}),T]}
\end{equation*}

donde $F[\eta(\mathbf{x}),T]$ representa un hamiltoniano efectivo dividido por
$k_BT$, es una funcional del parámetro de orden $\eta(\mathbf{x})$:

\begin{equation*}
F[\eta(\mathbf{x}),T]=\int\! d^2\mathbf{s}\ f[\eta(\mathbf{x}),T]
\end{equation*}

Constituye un potencial efectivo se denomina energía libre de Landau y debe tener las mismas
simetrías del sistema. Cerca del punto crítico, donde $\eta(\mathbf{x})$ es
pequeño, se puede desarrollar en potencias de $\eta(\mathbf{x})$ y sus derivadas. 

La simetría traslacional de la membrana implica que su energía libre de Landau
sólo puede depender de las derivadas de $\vec{r}$, estos son, los vectores
tangentes $\vec{t}_{\alpha}$. En la fase
plana, los promedios de los vectores tangentes en un punto son no nulos, en
cambio, en la fase rugosa, su promedio es nulo por isotropía. Por tanto,
podemos utilizar como  parámetro de orden los vectores tangentes
$\vec{t}_{\alpha}(\mathbf{x})$. Como observamos, tenemos un parámetro de orden con
carácter tensorial, con seis grados de libertad en cada punto. La descripción
de grano grueso la obtenemos la obtenemos a partir de los promedios
$\vec{t}_{\alpha}(\mathbf{x}_{\Lambda})$, función de las coordenadas internas
$\mathbf{x}_{\Lambda}$ de las partículas, que corresponden al promedio de los
vectores tangentes de las partículas en la vecindad $\Lambda$ del punto
$\vec{r}(\mathbf{x})$.
 

La justificación de la descripción de grano grueso la encontramos en el
comportamiento de las correlaciones espaciales, éstas crecen a medida que la
temperatura del sistema se acerca a la temperatura crítica, lo que significa
que podemos encontrar regiones de dimensión lineal $\Lambda$ donde los vectores tangentes son
aproximadamente constantes. Es importante puntualizar que,
mientras que tratamos a $\mathbf{x}_{\Lambda}$ como una variable continua, la función
$\vec{t}_{\alpha}(\mathbf{x}_{\Lambda})$ no presenta variaciones a distancias
del orden del espaciado microscópico $a$, o equivalentemente, su transformada
de Fourier tiene vectores de onda con magnitud menor que cierto cut-off
$\Lambda^{-1} \sim 1/a$. En lo que sigue continuaremos con está descripción de
grano grueso y omitiremos el subíndice $\Lambda$ de las variables para no
cargar demasiado la notación.

Una energía libre de Landau $F$ de grano grueso de una membrana cristalina
puede construirse apoyándose en las siguientes propiedades:
\begin{description}
\item[Localidad e invarianza traslacional]: Debe depender de los vectores
  tangentes locales y de las interacciones de corto alcance descritas a través
  del desarrollo en gradientes:
  $$ F=\int\! d^D\mathbf{S}\ f[\vec{t}_{\alpha},\nabla
  \vec{t}_{\alpha},\dots]$$ 
\item[Simetría rotacional en $\mathbb{R}^3$]: Está simetría implica que sólo
  puede depender de productos escalares pares de los vectores tangentes
  locales y sus gradientes. 

\item[Simetría traslacional y rotacional en $\mathbb{R}^2$]: Lo que implica
  que los términos del desarrollo deben ser covariantes, invariantes frente a
  cambios de las coordenadas $\mathbf{x}$. Finalmente, con esta última
  propiedad podemos proponer la siguiente energía libre de Landau $F$:
  \begin{equation*}
    F[\vec{t}_{\alpha}(\mathbf{x})]=\int d^2\mathbf{S}
    \left[
      \frac{t}{2}(\vec{t}_{\alpha})^2+
      u(\vec{t}_{\alpha}\vec{t}_{\beta})^2+
      v(\vec{t}_{\alpha}\vec{t}^{\alpha})^2+
      \frac{\kappa}{2}(\partial_{\alpha}\vec{t}_{\alpha})^2
    \right]
  \end{equation*}
\end{description}

%Explicación coeficientes - dependen de la temperatura

Donde no hemos tenido en cuenta los términos de orden superior, al ser
irrelevantes en el límite de grandes vectores de onda $k\rightarrow \infty$. 
Esta energía libre de Landau se puede interpretar como un potencial efectivo
obtenido mediante la integración de los grados de libertad microscópicos
(grano grueso) sujetos a la condición de que sus promedios sean iguales a los
vectores tangentes $\vec{t}_{\alpha}(\mathbf{x})$. Su forma funcional
únicamente se deduce a partir de las simetrías del sistema, el precio que hay
que pagar por esta aproximación es que en su expresión tenemos los parámetros
fenomenológicos $t$, $u$, $v$ y $\kappa$ cuya dependencia funcional con los
parámetros microscópicos originales no es desconocida, así como con las
ligaduras externas del sistema: Temperatura, fuerzas externas\dots  

En la deducción de $L$ no hemos considerado interacciones entre puntos
distantes de la superficie, lo que permite la membrana puede cruzarse
libremente a sí misma, por este motivo las membranas que son descritas
mediantes esta energía libre son denominadas Membranas fantasma. El siguiente
potencial si incluye interacciones de largo alcance:

\begin{multline}
L(\vec{r})=\int d^D\mathbf{x}
\left[
\frac{t}{2}(\partial_{\alpha}\vec{r})^2+
u(\partial_{\alpha}\vec{r}\partial_{\beta}\vec{r})^2+
v(\partial_{\alpha}\vec{r}\partial^{\alpha}\vec{r})^2+
\frac{\kappa}{2}(\partial^2\vec{r})^2
\right]\\
+\frac{b}{2}\int d^D\mathbf{x} d^D\mathbf{x'}
\delta^{d}(\vec{r}(\mathbf{x})-\vec{r}(\mathbf{x'}))
\end{multline}

Mediante el último término se controla la posibilidad de que la membrana pueda
cruzarse a sí misma, penalizando energéticamente este hecho. Si el factor $b$
es nulo, la superficie puede cruzarse a sí misma sin coste energético y
recuperamos una membrana fantasma.

\section{Aproximación de Campo medio}

\begin{figure}[h]
\centering
 \resizebox{\columnwidth}{!}{\input{campo_medio-fig}}
\caption{Parametrización usada en la aproximación del campo medio}
\end{figure}

La energía libre TAL  describe una membrana isótropa, las interacciones no
dependen de la dirección, lo que nos permite afirmar que en la fase
plana los promedios de las posiciones tridimensionales de las partículas
formarán una red regular plana:  

\begin{equation*}
 \langle\vec{r}(\mathbf{x})\rangle_{p}=(\zeta x^1,\zeta x^2,0)=(\zeta \mathbf{x},0) 
\end{equation*}

El nuevo parámetro introducido $\zeta$ está relacionado con la norma al
cuadrado del promedio de  vectores tangentes

\begin{equation*}
 \left.\begin{array}{c}
\langle\vec{t}_{1}(\mathbf{x})\rangle_{p}=\langle\partial_1\vec{r}\rangle_{p}=(\zeta ,0,0)\\
\langle\vec{t}_{2}(\mathbf{x})\rangle_{p}=\langle\partial_2\vec{r}\rangle_{p}=(0,\zeta ,0)\\
 \end{array}\right\}\Rightarrow
\langle\vec{t}_{\alpha}(\mathbf{x})\rangle_{p}\cdot\langle\vec{t}_{\beta}(\mathbf{x})\rangle_{p}=\zeta^2 \delta_{\alpha\beta}
\end{equation*}

de donde deducimos que

\begin{equation*}
|\langle\vec{t}_1(\mathbf{x})\rangle_{p}|=|\langle\vec{t}_2(\mathbf{x})\rangle_{p}|=\zeta
\end{equation*}

La norma de ambos vectores tangentes coincide, consecuencia directa de la
isotropía de la membrana, y es igual a el parámetro $\zeta$. Puesto que la
longitud de los vectores tangentes están relacionados la distancia relativa
entre los puntos de la superficie, $\zeta$ cuantifica entonces el espaciado el
 entre puntos y da una idea del grado de extensión de la membrana.
 
La aproximación de campo medio asume que se pueden despreciar las
fluctuaciones, y por tanto, podemos aproximar los radio vectores de las
partículas y lo vectores tangentes por sus promedios:

\begin{align}
 \vec{r}(\mathbf{x})&\simeq(\zeta \mathbf{x},0)\\
 \vec{t}_{1}(\mathbf{x})&\simeq(\zeta ,0,0)\\
 \vec{t}_{1}(\mathbf{x})&\simeq(0,\zeta,0)
\end{align}

El parámetro de orden es una variable, todas los estados son homogéneos

El potencial efectivo TAL en esta aproximación será:

\begin{align}
  F(\zeta,T)=&\int\! d^2 \mathbf{S}\ 2\zeta^2\!\left( \frac{t}{2} + (u+2v)\zeta^2\right)\\
  =&Af(\zeta)
\end{align}

donde 

$$A=\int d^2 \mathbf{S}=\zeta^2\int d^2 \mathbf{x}$$

Es el área de la membrana y 

\begin{equation*}
f(\zeta,T)=2\zeta^2\!\left( \frac{t}{2} + (u+2v)\zeta^2\right)
\end{equation*}

Es la densidad de energía libre de Landau. 

\begin{equation*}
\mathcal{Z}[T]=\int D[\vec{t}_{\alpha}(\mathbf{x})]\;
e^{-F[\vec{t}_{\alpha}(\mathbf{x})]}\
\stackrel{\text{C. M.}}{\longrightarrow}\ \mathcal{Z}[T]=\int d\zeta\;e^{-Af_{CM}(\zeta)}
\end{equation*}

Ahora bien, si $A$ es muy grande, únicamente contribuyen los términos de
$f_{CM}$ pequeños, entonces podemos desarrollar en serie alrededor del mínimo
$\zeta_0$:

\begin{equation*}
f_{CM}(\zeta)\simeq f_{CM}(\zeta_0)+\frac{1}{2}\ddot{f}_{CM}(\zeta_0)(\zeta-\zeta_0)^2
\end{equation*}

Encontramos:

\begin{equation*}
 \mathcal{Z}[T]\simeq\int
 d\zeta\;e^{-Af_M(\zeta_0)-\frac{A}{2}\ddot{f}_M(\zeta_0)(\zeta-\zeta_0)^2}=
 \sqrt{\frac{2\pi}{A\ddot{f}_M(\zeta_0)}}\; e^{-Af_M(\zeta_0)}
\end{equation*}

En el límite termodinámico:

\begin{equation*}
\lim_{A\rightarrow 0}\; \frac{1}{A} \ln \mathcal{Z}[T]=-f_M(\zeta_0)
\end{equation*}

Es decir, $f_M(\zeta_0)=\beta \;\text{f}\,(T)$, donde
$\text{f}\,(T)$ es la energía libre de Hemholtz. 

\begin{equation*}
\left(\frac{\partial f_M}{\partial \zeta}\right)_{\!\zeta=\zeta_0}\!=0 \; \Rightarrow \; 4\zeta_0\left(\frac{t}{2}+2(u+vD)\zeta_0^2\right)=0
\end{equation*}

El comportamiento de $f$ depende del signo del parámetro $t$:

\begin{figure}[h]
\centering
 \resizebox{\columnwidth}{!}{\input{energia_libre_CM-fig}}
\caption{Campo medio}
\end{figure} 

\begin{enumerate}
\item Para $t>0$ el mínimo de $F$ ocurre en $\zeta_0=0$, esto indica que la
  simetría es total, la membrana se encuentra en la fase desordenada,
  arrugada.
\item Para $t<0$, la función $f^{MF}$ tiene mínimo doblemente
  degenerado, correspondiente a los valores de $\zeta_0=\mp
  \sqrt{\frac{-t}{4(u+vD)}}$, ocurre una ruptura espontánea de la simetría indicando
  una fase ordenada, la fase plana.
\end{enumerate}

La evaluación del punto mínimo sugiere entonces dos fases diferentes según el
signo del parámetro $t$. Para encontrar el significado del parámetro $t$
podemos desarrollar en serie en función de la temperatura centrada en la
temperatura crítica $T_c$ los parámetros:
\begin{align}
t=&\; t_0+t_1\frac{(T-T_c)}{T_c}+O(T-T_c)^2\\
u=&\; u_0+u_1(T-T_c)+O(T-T_c)^2\\
v=&\; v_0+v_1(T-T_c)+O(T-T_c)^2
\end{align}



Podemos tomar a $u$ y $v$ como constantes, ya que su dependencia en la 
temperatura no contribuye en los primeros órdenes en comportamiento
termodinámico cerca de $T_c$. Ahora bien,
para cualquier valor de $T\!<\!T_c$ sabemos que el parámetro de orden es no
nulo, esto es, $\zeta_0\!\neq\! 0$, lo que implica que $t_0=0$. Por otra
parte, siempre podremos escalar el parámetro de orden 
$\zeta$, de manera que $t_1=1$, con lo que tenemos finalmente que
\begin{equation*}
t=\frac{(T-T_c)}{T_c}+O(T-T_c)^2,
\end{equation*}

el parámetro $t$ es la temperatura reducida.

%Como el parámetro de orden se anula de forma continua en la transición, a este
%tipo de transiciones de fase con ruptura de simetría se les denomina, de forma
%genérica, continuas.

%Ehrenfest introdujo una clasificación de la transiciones de fase basandose en
%la continuidad del potencial termodinámico. Según Ehrenfest una transición dee
%fase es de orden $n$ si la primera derivada del potencial termodinámico que no
%es continua es la derivada $n$-ésima. Esta clasificación no siempre es
%equivalente a la clasificación basada en la ruptura de simetría.
 
\section{Fase plana}

%Explicación estado referencia
Pequeñas deformaciones desde un estado de referencia se pueden parametrizar
como:

\begin{equation*}
\vec{r}(\mathbf{x})=(\zeta \mathbf{x}+\mathbf{u(\mathbf{x})},h(\mathbf{x}))
\end{equation*}

Donde $\mathbf{u(\mathbf{x})}$ son los modos fonocnicos internos y
$h(\mathbf{x})$ son las fluctuaciones fuera del plano.
$\zeta$ no nula describe una membrana plana con pequeñas fluctuaciones.

\subsection{Tensor de Deformaciones}

En una membrana plana es natural introducir el tensor de deformaciones:
\begin{equation*}
u_{\alpha\beta}=\frac{1}{2}(\partial_{\alpha}\vec{r}\cdot\partial_{\beta}\vec{r}-\delta_{\alpha\beta})
\end{equation*}

Los factores del interior del integrando quedan:
\begin{align}
(\partial_{\alpha}\vec{r})^2&=2(1+u_{\alpha}^{\ \alpha})\\
(\partial_{\alpha}\vec{r}\partial^{\alpha}\vec{r})^2&=4(1+2u_{\alpha}^{\
  \alpha}+(u_{\alpha}^{\ \alpha})^2)\\
(\partial_{\alpha}\vec{r}\partial_{\alpha}\vec{r})^2&=2+4u_{\alpha}^{\
  \alpha}+4u_{\alpha\beta}u^{\alpha\beta}
\end{align}

Sustituyendo en la expresión de $F$, y despreciando términos constantes:

\begin{equation*}
F(\vec{r})=\int d^D\mathbf{x}
\left[\tau u_{\alpha}^{\ \alpha}+
\mu u_{\alpha\beta}u^{\alpha\beta} +
\frac{\lambda}{2}(u_{\alpha}^{\ \alpha})^2 +
\frac{\kappa}{2}(\partial^2\vec{r})^2\right]
\end{equation*}

Donde:
\begin{align}
\tau&=t+4u+8v\\
\mu&=4u\\
\frac{\lambda}{2}&=4v
\end{align}

%LANDAU
Consideramos el estado de referencia en ausencia de fuerza externas. Entonces,
para $u_{\alpha\beta}=0$, los esfuerzos internos deben ser nulos también,
$\sigma_{\alpha\beta}=0$. Dado que $\sigma_{\alpha\beta}=\partial F / \partial
u_{\alpha\beta}$, se sigue que no hay término lineal en $u_{\alpha\beta}$ en
el desarrollo de $F$; esto implica que $\tau=0$ y podemos rescribir las
anteriores ecuaciones TAL:

\begin{align}
-t&=\mu+\lambda\\
\mu&=4u\\
\lambda&=8v
\end{align}   

De hecho estamos recuperando la ecuación del campo medio para la temperatura
con $\zeta_0=1$

El término de curvatura:

\subsection{Propiedades elásticas}
%%% Local Variables: 
%%% mode: latex
%%% TeX-master: "TFM"
%%% End: 
