\section{Teoría de Landau de la transición de fase}

Vamos a considerar el caso general de una membrana elástica y flexible
$D$-dimensional alojada en un espacio $d$-dimensional Euclídeo. Los puntos
de la membrana están conectados y formando una red regular, en este
sentido, la membrana es cristalina. La posición en el espacio $d$-dimensional
de un punto de la red viene dada por $\vec{r}(\mathbf{x})$, donde $\mathbf{x}$
es un vector $D$-dimensional interno que etiqueta cada partícula:
$$\mathbf{x}\equiv (x^1,x^2 \dots x^D)\in \mathbb{R}^D \quad \vec{r}\equiv (r^1,r^2 \dots r^d)\in \mathbb{R}^d$$

Este sistema presenta dos fases: una fase desordenada, la fase arrugada con
simetría total y una fase ordenada, fase plana, donde se ha eliminado el
elemento de simetría rotacional. Este tipo de transiciones de fase que
difieren en alguna simetría se denominan transiciones de ruptura de simetría y
pueden, en general, describirse mediante un parámetro de orden, que es una
variable que cuantifica el grado de orden de las fases, por lo que es habitual
definirlo de forma que sea nulo en la fase de mayor simetría.

La idea esencial de la teoría de Landau de las transiciones de fase es
construir un potencial efectivo $L[K_i,\eta(\mathbf{x})]$, la energía libre de
Landau, que depende de las constantes de acoplamiento $K_i$ y de el parámetro
de orden $\eta$. Este potencial efectivo debe tener las mismas simetrías del
sistema y mínimos dan el valor del parámetro de orden cuando el sistema se
encuentra en equilibrio térmico. Aunque puede parecer que este
potencial efectivo es la energía libre del sistema, estrictamente no
lo es pues no es convexo.

La simetría traslacional de la membrana implica que su energía libre de Landau
sólo puede depender de las derivadas de $\vec{r}$, estos son, los vectores
tangentes $\vec{t}_{\alpha}=\partial \vec{r}/\partial x^{\alpha} $. En la fase
plana, los promedios de los vectores tangentes en un punto son no nulos, en
cambio, en la fase arrugada, su promedio es nulo por isotropía. Por tanto,
podemos utilizar como  parámetro de orden los vectores tangentes
$\vec{t}_{\alpha}$.

Las correlaciones espaciales crecen a medida que la temperatura del sistema se
acerca a la temperatura en donde ocurre la transición, la temperatura
crítica $T_c$, lo que significa que podemos encontrar regiones de dimensión
lineal $\Lambda^{-1}\simeq \xi(T) $ donde los vectores tangentes son
aproximadamente constantes. Esto sugiere que podemos dividir la superficie de
la membrana en bloques de tamaño lineal $\Lambda^{-1}$, y definir los vectores
tangentes locales
$\vec{t}_{\alpha}(\mathbf{x})_{\Lambda}=\langle \partial_{\alpha}\vec{r}\rangle_{\Lambda}$
en cada bloque centrados en $\mathbf{x}$. Es importante puntualizar que,
mientras que tratamos a $\mathbf{x}$ como una variable continua, la función
$\vec{t}_{\alpha}(\mathbf{x})_{\Lambda}$ no presenta variaciones a distancias
del orden del espaciado microscópico $a$, o equivalentemente, su transformada
de Fourier tiene vectores de onda con magnitud menor que cierto cut-off
$\Lambda \sim 1/a$. Está descripción se denomina de ``grano grueso'' y en lo
que sigue omitiremos el subíndice $\Lambda$ de las variables.

Una energía libre de Landau $L$ de grano grueso de una membrana cristalina
puede construirse apoyándose en las siguientes propiedades:
\begin{description}
\item[Localidad e invarianza traslacional]: Debe depender de los vectores
  tangentes locales y de las interacciones de corto alcance descritas a través
  del desarrollo en gradientes:
  $$ L=\int\! d^D\mathbf{S}\ \mathcal{L}[\vec{t}_{\alpha},\nabla
  \vec{t}_{\alpha},\dots]$$ 
\item[Simetría rotacional en $\mathbb{R}^d$]: Está simetría implica que sólo
  puede depender de productos escalares pares de los vectores tangentes
  locales y sus gradientes. 

\item[Simetría traslacional y rotacional en $\mathbb{R}^D$]: Lo que implica
  que los términos del desarrollo deben ser covariantes, invariantes frente a
  cambios de las coordenadas $\mathbf{x}$. Finalmente, con esta última
  propiedad podemos proponer la siguiente energía libre de Landau $L$:

  \begin{equation}
    L=\int d^D\mathbf{S}
    \left[
      \frac{t}{2}(\vec{t}_{\alpha})^2+
      u(\vec{t}_{\alpha}\vec{t}_{\beta})^2+
      v(\vec{t}_{\alpha}\vec{t}^{\alpha})^2+
      \frac{\kappa}{2}(\partial_{\alpha}\vec{t}_{\alpha})^2
    \right]
  \end{equation}
\end{description}

%Explicación coeficientes - dependen de la temperatura

Donde no hemos tenido en cuenta los términos de orden superior, al ser
irrelevantes en el límite de grandes vectores de onda $k\rightarrow \infty$. 
Esta energía libre de Landau se puede interpretar como un potencial efectivo
obtenido mediante la integración de los grados de libertad microscópicos
(grano grueso) sujetos a la condición de que sus promedios sean iguales a los
vectores tangentes $\vec{t}_{\alpha}(\mathbf{x})$. Su forma funcional
únicamente se deduce a partir de las simetrías del sistema, el precio que hay
que pagar por esta aproximación es que en su expresión tenemos los parámetros
fenomenológicos $t$, $u$, $v$ y $\kappa$ cuya dependencia funcional con los
parámetros microscópicos originales no es desconocida, así como con las
ligaduras externas del sistema: Temperatura, fuerzas externas\dots  

En la deducción de $L$ no hemos considerado interacciones entre puntos
distantes de la superficie, lo que permite la membrana puede cruzarse
libremente a sí misma, por este motivo las membranas que son descritas
mediantes esta energía libre son denominadas Membranas fantasma. El siguiente
potencial si incluye interacciones de largo alcance:

\begin{multline}
L(\vec{r})=\int d^D\mathbf{x}
\left[
\frac{t}{2}(\partial_{\alpha}\vec{r})^2+
u(\partial_{\alpha}\vec{r}\partial_{\beta}\vec{r})^2+
v(\partial_{\alpha}\vec{r}\partial^{\alpha}\vec{r})^2+
\frac{\kappa}{2}(\partial^2\vec{r})^2
\right]\\
+\frac{b}{2}\int d^D\mathbf{x} d^D\mathbf{x'}
\delta^{d}(\vec{r}(\mathbf{x})-\vec{r}(\mathbf{x'}))
\end{multline}

Mediante el último término se controla la posibilidad de que la membrana pueda
cruzarse a sí misma, penalizando energéticamente este hecho. Si el factor $b$
es nulo, la superficie puede cruzarse a sí misma sin coste energético y
recuperamos una membrana fantasma.

\subsection{Aproximación de Campo medio}

A partir del potencial efectivo es posible obtener las variables
termodinámicas mediante la función de partición $\mathcal{Z}$ que viene dada
por la siguiente integral funcional:

\begin{equation}
\mathcal{Z}[T]=\int D[\vec{r}(\mathbf{x})]\; e^{-\beta L[\vec{r}(\mathbf{x})]}
\end{equation}

En general, evaluar esta integral es muy complicado. Sin embargo podemos
obtener una aproximación aplicando la aproximación del punto de silla o campo
medio. Suponemos que podemos despreciar las fluctuaciones en la fase plana, y
por tanto, es posible entonces expresar las posiciones de los puntos de por
sus posiciones medias $\vec{\mathbf{x}}\simeq \zeta \mathbf{x}$, donde $\zeta$
es una parámetro que nos indica el grado de aumento de la extensión de la
membrana. Los vectores tangentes son iguales en cada punto y sus componentes también
$\vec{t}_{\alpha}(\mathbf{x})\simeq\zeta$. El potencial efectivo en esta
aproximación será:

\begin{align}
  L(\zeta)=&\int d^D \mathbf{S} \left[ \frac{t}{2} D\zeta^2 + uD\zeta^4+vD^2\zeta^4\right]\\
  =&A\mathcal{L}(\zeta)
\end{align}

donde 

$$A=\int d^D \mathbf{S}$$

Es el área de la membrana y 

\begin{equation}
\mathcal{L}(\zeta)=\left[ \frac{t}{2} D\zeta^2 + uD\zeta^4+vD^2\zeta^4\right]
\end{equation}

Es la densidad de energía libre de Landau. Se puede demostrar que:

\begin{equation}
Z[T]= e^{-\beta F[T]}, \quad \beta F[T]\simeq \beta L(\zeta_0)
\end{equation}

Donde F es la energía libre de Helmholtz  y $\zeta_0$ es el valor del
parámetro de orden que minimiza la función. Está aproximación es conocida
como la teoría de campo medio de Landau. El valor mínimo de $f^{MF}(\zeta)$
será la solución a la ecuación:

\begin{equation}
\left(\frac{d\mathcal{L}}{d\zeta}\right)_{\!\zeta=\zeta_0}\!=0 \; \Rightarrow \; \frac{t}{2}+2(u+vD)\zeta_0^2=0
\end{equation}
 El comportamiento de f depende del signo del parámetro $t$:

\begin{enumerate}
\item Para $t>0$ el mínimo de $F$ ocurre en $\zeta_0=0$, esto indica que la
  simetría es total, la membrana se encuentra en la fase desordenada,
  arrugada.
\item Para $t<0$, la función $f^{MF}$ tiene mínimo doblemente
  degenerado, correspondiente a los valores de $\zeta_0=\mp
  \sqrt{\frac{-t}{4(u+vD)}}$, ocurre una ruptura espontánea de la simetría indicando
  una fase ordenada, la fase plana.
\end{enumerate}

La evaluación del punto mínimo sugiere entonces dos fases diferentes según el
signo del parámetro $t$. Para encontrar el significado del parámetro $t$
podemos desarrollar en serie en función de la temperatura centrada en la
temperatura crítica $T_c$ los parámetros:
\begin{align}
t=&\; t_0+t_1\frac{(T-T_c)}{T_c}+O(T-T_c)^2\\
u=&\; u_0+u_1(T-T_c)+O(T-T_c)^2\\
v=&\; v_0+v_1(T-T_c)+O(T-T_c)^2
\end{align}

Podemos tomar a $u$ y $v$ como constantes, ya que su dependencia en la 
temperatura no contribuye en los primeros órdenes en comportamiento
termodinámico cerca de $T_c$. Ahora bien,
para cualquier valor de $t\!<\!0$ tenemos que $\zeta_0\!\neq\! 0$, lo que  
implica que $t_0=0$, para asegurar que esta última desigualdad se cumpla
siempre. Por otra parte, siempre podremos escalar el parámetro de orden
$\zeta$, de manera que $t_1=1$ y tenemos finalmente que
\begin{equation}
t=\frac{(T-T_c)}{T_c}+O(T-T_c)^2,
\end{equation}

el parámetro $t$ es la temperatura reducida.

%Como el parámetro de orden se anula de forma continua en la transición, a este
%tipo de transiciones de fase con ruptura de simetría se les denomina, de forma
%genérica, continuas.

%Ehrenfest introdujo una clasificación de la transiciones de fase basandose en
%la continuidad del potencial termodinámico. Según Ehrenfest una transición dee
%fase es de orden $n$ si la primera derivada del potencial termodinámico que no
%es continua es la derivada $n$-ésima. Esta clasificación no siempre es
%equivalente a la clasificación basada en la ruptura de simetría.
 
\section{Fase plana}

%Explicación estado referencia
Pequeñas deformaciones desde un estado de referencia se pueden parametrizar
como:

\begin{equation}
\vec{r}(\mathbf{x})=(\zeta \mathbf{x}+\mathbf{u(\mathbf{x})},h(\mathbf{x}))
\end{equation}

Donde $\mathbf{u(\mathbf{x})}$ son los modos fonocnicos internos y
$h(\mathbf{x})$ son las fluctuaciones fuera del plano.
$\zeta$ no nula describe una membrana plana con pequeñas fluctuaciones.

\subsection{Tensor de Deformaciones}

En una membrana plana es natural introducir el tensor de deformaciones:
\begin{equation}
u_{\alpha\beta}=\frac{1}{2}(\partial_{\alpha}\vec{r}\cdot\partial_{\beta}\vec{r}-\delta_{\alpha\beta})
\end{equation}

Los factores del interior del integrando quedan:
\begin{align}
(\partial_{\alpha}\vec{r})^2&=2(1+u_{\alpha}^{\ \alpha})\\
(\partial_{\alpha}\vec{r}\partial^{\alpha}\vec{r})^2&=4(1+2u_{\alpha}^{\
  \alpha}+(u_{\alpha}^{\ \alpha})^2)\\
(\partial_{\alpha}\vec{r}\partial_{\alpha}\vec{r})^2&=2+4u_{\alpha}^{\
  \alpha}+4u_{\alpha\beta}u^{\alpha\beta}
\end{align}

Sustituyendo en la expresión de $F$, y despreciando términos constantes:

\begin{equation}
F(\vec{r})=\int d^D\mathbf{x}
\left[\tau u_{\alpha}^{\ \alpha}+
\mu u_{\alpha\beta}u^{\alpha\beta} +
\frac{\lambda}{2}(u_{\alpha}^{\ \alpha})^2 +
\frac{\kappa}{2}(\partial^2\vec{r})^2\right]
\end{equation}

Donde:
\begin{align}
\tau&=t+4u+8v\\
\mu&=4u\\
\frac{\lambda}{2}&=4v
\end{align}

%LANDAU
Consideramos el estado de referencia en ausencia de fuerza externas. Entonces,
para $u_{\alpha\beta}=0$, los esfuerzos internos deben ser nulos también,
$\sigma_{\alpha\beta}=0$. Dado que $\sigma_{\alpha\beta}=\partial F / \partial
u_{\alpha\beta}$, se sigue que no hay término lineal en $u_{\alpha\beta}$ en
el desarrollo de $F$; esto implica que $\tau=0$ y podemos rescribir las
anteriores ecuaciones TAL:

\begin{align}
-t&=\mu+\lambda\\
\mu&=4u\\
\lambda&=8v
\end{align}   

De hecho estamos recuperando la ecuación del campo medio para la temperatura
con $\zeta_0=1$

El término de curvatura:

\subsection{Propiedades elásticas}
%%% Local Variables: 
%%% mode: latex
%%% TeX-master: "TFM"
%%% End: 
