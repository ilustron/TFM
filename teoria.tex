\chapter{Estudio analítico}

\section{Descripción geométrica}

Vamos a exponer la descripción geométrica de las membranas cristalinas,
seguiremos la referencia \cite{David:geometria}.

Consideraremos una membrana como una superficie elástica y flexible 
bidimensional alojada en un espacio tridimensional euclídeo. 
Se pueden interpretar las interacciones de los puntos de la superficie como
conexiones, es decir, dos puntos que están interaccionando entre sí, han
establecido una conexión entre ambos. Una membrana cristalina corresponde al
tipo de superficie en la que las conexiones entre sus puntos forman una red
regular de conectividad fija, lo que significa que si existe una conexión
entre dos puntos, está persistirá sea cual sea la forma tridimensional de la
superficie.

Usaremos una descripción continua de la membrana cristalina en donde la
posición, en el espacio tridimensional, de los nodos de la red vendrá dada por los
radio vectores $\vec{r}(\mathbf{x})$, siendo $\mathbf{x}$ la parametrización
continua de la superficie. Debido al carácter de conectividad fija, podemos
elegir la parametrización de forma que $\mathbf{x}$ sea un índice continuo
que etiquete cada partícula. Por tanto, en la descripción geométrica de la
membrana cristalina necesitamos especificar dos sistemas de coordenadas:

\begin{figure}[h]
\centering
\subfigure[Coordenadas internas.]{
 \resizebox{160bp}{!}{\input{coordenadas_internas-fig}}}
\quad
\subfigure[Coordenadas externas.]{
 \resizebox{160bp}{!}{\input{coordenadas_externas-fig}}}
\caption{Coordenadas utilizadas en la descripción de las membranas cristalinas.}
\end{figure}

\begin{description}
\item[Coordenadas internas:] Es la parametrización, dos índices continuos que
  especifican cada punto de la superficie formada por la membrana. Estos
  índices además etiquetarán los nodos de la red interna, lo que significa que las
  coordenadas internas de un punto, serán los mismos antes y después de una
  deformación tridimensional (externa) de la membrana. 
  \begin{equation*}
  \mathbf{x}\equiv (x^1,x^2)\in \mathbb{R}^2
  \end{equation*}
\item[Coordenadas externas:] Describen las posiciones tridimensional de cada
  nodo. Utilizaremos el sistema de referencia usual cartesiano ortonormal.
  \begin{equation*}
    \vec{r}=x\,\vec{i}+y\,\vec{j}+z\,\vec{k}\equiv (x,y,z)\in \mathbb{R}^3
  \end{equation*}
\end{description}

\subsection{Vectores tangentes}

Si la superficie de formada por la membrana es lo bastante
\textit{suave}\footnote{En la mayoría de los casos será suficiente con que
 $\vec{r}(\mathbf{x})$ sea diferenciable hasta segundo orden con respecto a
 $x^{\alpha}\, \alpha=1,2$.}, en cada punto $P\equiv\vec{r}(\mathbf{x})$ de la
membrana tenemos un plano tangente $T_PS$ a la superficie (figura
\ref{vectores_tang-fig}) cuya base son los siguientes vectores tangentes:
\begin{figure}[h]
\centering
 \input{vectores_tangentes-fig}
\caption{Vectores tangentes en el punto $P$}\label{vectores_tang-fig}
\end{figure}

\begin{equation*}
\vec{t}_{\alpha}=\frac{\partial \vec{r}}{\partial
  x^{\alpha}}=\partial_{\alpha}\vec{r} \qquad \alpha=1,2.
\end{equation*}
Con los que podemos escribir cualquier vector $\vec{V}$ tangente a la superficie en el punto $P$ como:
\begin{equation}\label{ejemplo_vector} 
  \vec{V}=\sum^2_{\alpha=1}V^{\alpha}\vec{t}_{\alpha}=V^{\alpha}\vec{t}_{\alpha},
\end{equation}
con el convenio de índices repetidos de Einstein\footnote{Mientras no se
  indique lo contrario, usaremos el convenio de índices repetidos de
  Einstein y los índices griegos tomarán valores de 1 a 2.}. Las dos cantidades
$V^{\alpha}$ son las componentes del vector $\vec{V}$ en el sistema de
coordenadas internas $\mathbf{x}$ y dependen de la elección del sistema de
coordenadas, esto es, de la parametrización de la superficie. Si consideramos
un nuevo sistema de coordenadas  
\begin{equation*}
 \mathbf{x}'=\{x'^{\alpha}(\mathbf{x});\; \alpha=1,2\},
\end{equation*}
podemos relacionar la base de los vectores tangentes a $\mathbf{x}'$ con la
asociada a $\mathbf{x}$
\begin{equation*}
\vec{t}_{\alpha}^{\ '}=\frac{\partial \vec{r}}{\partial x'^{\alpha}}=
\frac{\partial x^{\beta}}{\partial x'^{\alpha}}\frac{\partial \vec{r}}{\partial x^{\beta}}=\frac{\partial x^{\beta}}{\partial x'^{\alpha}}\vec{t}_{\alpha},
\end{equation*}
También las componentes $V'^{\alpha}$ del vector \eqref{ejemplo_vector}
\begin{equation*} 
  \vec{V}=V'^{\alpha}\vec{t}_{\alpha}^{\
    '}=V^{\alpha}\vec{t}_{\alpha}\Rightarrow V'^{\alpha}=\frac{\partial x'^{\alpha}}{\partial x^{\beta}}V^{\beta}.
\end{equation*}

Tanto $\vec{t}_{\alpha}$ como $V^{\alpha}$ son casos particulares de objetos
más generales, los tensores. Definimos un tensor $T$ de rango $m$
contravariante y rango $n$ covariante como un objeto cuyas $2^{m+n}$ componentes
bajo un cambio de coordenadas $\mathbf{x}\rightarrow\mathbf{x'}$ transforman
como 
\begin{equation*}
T'^{\alpha_1,\dots,\alpha_m}_{\beta_1,\dots,\beta_n}=
\frac{\partial x'^{\alpha_1}}{\partial x^{\mu_1}}\dots \frac{\partial
  x'^{\alpha_m}}{\partial x^{\mu_m}}
\frac{\partial x^{\nu_1}}{\partial x'^{\beta_1}}\dots \frac{\partial
  x^{\nu_n}}{\partial x'^{\beta_n}}T^{\mu_1,\dots,\mu_n}_{\nu_1,\dots,\nu_n} .
\end{equation*}

Entonces $V^{\alpha}$ son las componentes de un tensor de rango $1$
contravariante, que llamaremos vector contravariante. Análogamente,
llamaremos a un tensor de rango $1$ covariante vector covariante.
 
\subsection{Métrica inducida}

La distancia tridimensional $dl$ entre dos puntos $P$ y $Q$ próximos de coordenadas internas
$\mathbf{x}$ y $\mathbf{x}+d\mathbf{x}$ respectivamente es:
\begin{equation}\label{elemento_linea}
dl^2=(\vec{r}(\mathbf{x}+d\mathbf{x})-\vec{r}(\mathbf{x}))^2=g_{\alpha\beta}dx^{\alpha}dx^{\beta},
\end{equation}
siendo $g_{\alpha\beta}=\vec{t}_{\alpha}\cdot\vec{t}_{\beta}$ un tensor, que
denominaremos métrica inducida, que nos permite definir el producto escalar
entre dos vectores tangentes $\vec{V}$ y $\vec{W}$ en el punto $P$:
\begin{equation}\label{producto_escalar}
\vec{V}\cdot\vec{W}=g_{\alpha\beta}V^{\alpha}W^{\beta}.
\end{equation}
Diremos que la base $\{ \vec{t}_{\alpha},\; \alpha=1,2\}$ es ortonormal si
\begin{equation*}
\vec{t}_{\alpha}\cdot\vec{t}_{\beta}=\delta_{\alpha\beta},
\end{equation*}
donde $\delta_{\alpha\beta}$ es la métrica euclídea 
% Encontrar un sistema de coordenadas internas cuya
% base de vectores tangentes sea ortonormal en un punto, equivale a
% diagonalizar $g_{\alpha\beta}$ en dicho punto. Esto siempre es posible gracias a que la
% métrica inducida es simétrica $g_{\alpha\beta}=g_{\beta\alpha}$. Como vemos La
% ortogonalidad depende del punto, es local; únicamente un sistema de cordenadas
% internas generará 
\begin{equation*}
\delta_{\alpha\beta}\equiv\left(\begin{array}{cc}
1&0\\
0&1\\
\end{array}\right).
\end{equation*}
La métrica inducida inversa $g^{\alpha\beta}$ se define como
\begin{equation}\label{expresion_metrica}
g_{\alpha\gamma}g^{\gamma\beta}=\delta_{\alpha}^{\ \beta}=\begin{cases}
1&\text{si $\alpha=\beta$}\\
0&\text{si $\alpha\neq\beta$}.
\end{cases}
\end{equation}

Gracias a esta última expresión \eqref{expresion_metrica} podemos hacer
corresponder los vectores contravariantes con los covariantes, que en la
notación usada equivale a subir o bajar índices
\begin{equation*}
V_{\alpha}=g_{\alpha\beta}V^{\beta}\rightarrow V^{\alpha}=g^{\alpha\beta}V_{\beta}, 
\end{equation*}
y podemos expresar el producto escalar de los vectores
\eqref{producto_escalar} 
\begin{equation*}
\vec{V}\cdot\vec{W}=V_{\alpha}W^{\alpha}=V^{\alpha}W_{\alpha}.
\end{equation*}

El elemento de infinitesimal de área $d^2\mathbf{s}$ se relaciona con el
determinante $g\equiv\det(g_{\alpha\beta})$ de la métrica inducida a través de
\begin{equation}\label{elemento_area}
d^2\mathbf{s}=\sqrt{g}\;d^2\mathbf{x},
\end{equation}
donde $d^2\mathbf{x}\equiv dx^1dx^2$. La demostración de esta expresión
\eqref{elemento_area} es evidente teniendo en cuenta que
\begin{equation*}
d^2\mathbf{s}=|\vec{t}_1\times\vec{t}_2|\;d^2\mathbf{x}
\end{equation*}
y
\begin{multline*}
|\vec{t}_1\times\vec{t}_2|^2=|\vec{t}_1|^2|\vec{t}_2|^2\sen^2
\theta=g_{11}g_{22}(1-\cos^2 \theta)=\\
g_{11}g_{22}-(\vec{t}_1\cdot\vec{t}_2)^2=g_{11}g_{22}-g_{12}g_{21}=g.
\end{multline*}

\subsection{Curvatura}

Para medir la curvatura de la membrana, se generaliza la noción de curvatura
de una curva (unidimensional), definiendo el tensor de curvatura extrínseca:
\begin{equation}\label{tensor_curvatura}
\vec{K}_{\alpha\beta}=D_{\alpha}\vec{t}_{\beta},
\end{equation}
que es un tensor covariante, cuyas componentes son vectores
tridimensionales, mide la variación de los vectores tangentes. Puesto que
estamos interesados en comparar vectores en planos tangentes
diferentes (si compararemos vectores en el mismo plano encontraríamos
curvatura nula), se usa en \eqref{tensor_curvatura} para derivar $D_{\alpha}$,
el operador correspondiente a la derivada covariante, que se define para las
componentes covariantes de un vector $\vec{V}$ como
\begin{equation}\label{derivada_covariante}
D_{\alpha}V_{\beta}=\frac{\partial V_{\beta}}{\partial x^{\alpha}}-\Gamma^{\gamma}_{\alpha\beta}V_{\gamma},
\end{equation}
siendo $\Gamma^{\beta}_{\alpha\nu}$ la conexión afín, que no es un tensor, y
determina qué vectores son paralelos en puntos diferentes usando la noción del transporte
paralelo. Utilizaremos la conexión de Levi Civita, donde dos vectores
tangentes de planos tangentes diferentes serán paralelos, si son tangentes a la
misma curva geodésica: La curva entre los puntos $P$ y $Q$ distantes cuya longitud es
estacionaria. Por tanto, la conexión de Levi Civita está directamente
relacionada con la métrica inducida, de hecho, cumple las siguientes dos condiciones
con las que queda completamente determinada:
\begin{enumerate}
\item Es simétrica $\Rightarrow
  \Gamma^{\gamma}_{\alpha\beta}=\Gamma^{\gamma}_{\beta\alpha}$.
\item El producto escalar de dos vectores es invariante bajo transporte
  paralelo, lo que implica que la derivada covariante de la métrica inducida
  es nula: 
  \begin{equation*}
    D_{\gamma}g_{\alpha\beta}=0.
  \end{equation*}
\end{enumerate}

Apoyándonos en estas dos últimas condiciones de la conexión podemos encontrar que
las componentes del tensor de curvatura \eqref{tensor_curvatura} son normales
a la superficie, pues
\begin{equation*}
\vec{K}_{\alpha\beta}\cdot \vec{t}_{\gamma}=(D_{\alpha}\vec{t}_{\beta})\cdot\vec{t}_{\gamma}=D_{\alpha}g_{\beta\gamma}-\vec{t}_{\beta}\cdot(D_{\alpha}\vec{t}_{\gamma})=-\vec{K}_{\gamma\alpha}\cdot \vec{t}_{\beta},
\end{equation*}
y repitiendo estas operaciones de nuevo dos veces más, encontramos que
\begin{equation*}
\vec{K}_{\alpha\beta}\cdot \vec{t}_{\gamma}=0.
\end{equation*}
Si se sustituye la definición de derivada covariante \eqref{derivada_covariante} en la expresión
\eqref{tensor_curvatura} del tensor de curvatura 
\begin{equation}\label{parcial_t}
\frac{\partial \vec{t}_{\alpha}}{\partial x^{\beta}}=\Gamma^{\gamma}_{\alpha\beta}\vec{t}_{\gamma}+\vec{K}_{\alpha\beta}.
\end{equation}
Teniendo en cuenta que $\partial_{\beta} \vec{t}_{\alpha}=\partial_{\alpha}
\vec{t}_{\beta}$  y la simetría de
$\Gamma^{\gamma}_{\alpha\beta}=\Gamma^{\gamma}_{\beta\alpha}$, deducimos a
partir de está última igualdad \eqref{parcial_t}, que el tensor de curvatura es simétrico 
\begin{equation*}
\vec{K}_{\alpha\beta}=\vec{K}_{\beta\alpha}.
\end{equation*}
También la expresión \eqref{parcial_t} nos muestra que las componentes
tangencial y normal de
$(\partial \vec{t}_{\alpha}/\partial x^{\beta})$ son $\Gamma^{\gamma}_{\alpha\beta}$  y 
$\vec{K}_{\alpha\beta}$, respectivamente. Por tanto, el tensor de curvatura
mide la rotación de los vectores tangentes al ser transportados paralelamente
a la superficie, y será proporcional a la normal $\vec{n}$ a la superficie
\begin{equation*}
 \vec{K}_{\alpha\beta}=K_{\alpha\beta}\vec{n},
\end{equation*}
donde $K_{\alpha\beta}$ es un tensor simétrico que vendrá dado por
\begin{equation}\label{K_alfa_beta}
K_{\alpha\beta}=\frac{\partial \vec{t}_{\alpha}}{\partial x^{\beta}}\cdot\vec{n}. 
\end{equation}

Si derivamos con respecto a $x^{\beta}$ el producto $\vec{t}^{\alpha}\cdot\vec{n}=0$
\begin{equation}\label{weingarten}
\frac{\partial \vec{t}^{\alpha}}{\partial x^{\beta}}\cdot\vec{n}+
\vec{t}^{\alpha}\cdot\frac{\partial \vec{n}}{\partial x^{\beta}}=
K_{\beta}^{\ \alpha}+\vec{t}^{\alpha}\cdot\frac{\partial \vec{n}}{\partial x^{\beta}}=0
\Rightarrow \vec{t}^{\alpha}\cdot\frac{\partial \vec{n}}{\partial x^{\beta}}=-K_{\beta}^{\ \alpha}.
\end{equation}
Por otro lado, derivando de igual forma la expresión $\vec{n}\cdot\vec{n}=1$ obtenemos 
\begin{equation*}
\frac{\partial \vec{n}}{\partial x^{\beta} }\cdot \vec{n}=0,
\end{equation*}
lo que implica que $\frac{\partial \vec{n}}{\partial x^{\beta}}$ es paralelo
al plano tangente y podemos expresarlo en función de su base $\vec{t}_{\alpha}$
\begin{equation*}
\frac{\partial \vec{n}}{\partial x^{\beta}}=A_{\beta}^{\ \alpha}\vec{t}_{\alpha}.
\end{equation*}
Multiplicando en la última relación a ambos lados por $\vec{t}^{\gamma}$ y teniendo en cuenta
\eqref{weingarten}, llegamos finalmente a 
\begin{equation}\label{derivada_n}
\frac{\partial \vec{n}}{\partial x^{\beta}}=-K_{\beta}^{\ \alpha}\vec{t}_{\alpha}.
\end{equation}

\begin{figure}[h]
\centering
 \resizebox{\columnwidth}{!}{\input{curvatura-fig}}
\caption{Radios principales de curvatura}\label{radios_curvatura-fig}
\end{figure}

Al ser $K_{\alpha\beta}$ un tensor simétrico, podemos encontrar para cualquier
punto $P$ dado un sistema de coordenadas $\{\mathbf{x}\}$ donde sea diagonal
\begin{equation*}
K_{\alpha\beta}\equiv\left(\begin{array}{cc}
\mathcal{K}_1 & 0\\
0 & \mathcal{K}_2\\
\end{array}\right)\, ,
\end{equation*} 
donde $\mathcal{K}_1$ y $\mathcal{K}_2$ son los autovalores, llamados
curvaturas principales y están relacionados con los radios principales
de curvatura $\mathcal{R}_{\alpha}$ por
\begin{equation*}
\mathcal{R}_{\alpha}=\frac{1}{\mathcal{K}_{\alpha}} \, , \qquad \alpha=1,2;
\end{equation*}
que corresponden a los radios de la mayor y menor circunferencia osculatriz en
el punto $P$ (figura \ref{radios_curvatura-fig}).
Estas curvaturas principales nos permiten definir dos importantes nuevas magnitudes:
La curvatura media $H$ y la curvatura Gaussiana $K$, definidos como el
promedio y el producto de las curvaturas principales
\begin{align*}
H&=\frac{\mathcal{K}_1+\mathcal{K}_2}{2},\\
K&=\mathcal{K}_1\mathcal{K}_2.
\end{align*}
La curvatura media $H$ es una cantidad geométrica extrínseca \cite{LuisLinares}, en el sentido de
que su valor depende de la forma tridimensional de la superficie. En
contraste, la curvatura de Gauss es una cantidad geométrica intrínseca, es
decir, no depende de como este sumergida la superficie en el espacio euclídeo,
sino que únicamente depende de la geometría propia de la superficie. La
expresión de ambas magnitudes en un sistema de coordenadas cualquiera es
\begin{align}
H&= \frac{1}{2}K_{\alpha}^{\ \alpha},\label{curvatura_media}\\
K&= \det(K_{\alpha\beta})=\frac{1}{2}(K_{\alpha}^{\ \alpha}K_{\beta}^{\ \beta}-K_{\alpha}^{\ \beta}K^{\alpha}_{\ \beta}).\label{curvatura_gaussiana}
\end{align}

\section{Estudio de la transición de fase}

En una membrana se pueden encontrar dos fases: La fase plana y la fase
rugosa, que se distinguen por sus simetrías. A la temperatura crítica $T_c$ se
produce la transición entre estas dos fases. Sobre la temperatura crítica la
membrana es invariante frente a rotaciones, 
la membrana se encuentra en la fase rugosa. En cambio, por debajo de la temperatura
crítica, se encuentra en la fase plana donde el promedio del vector normal a
la superficie $\langle \vec{n}\rangle$ define una dirección privilegiada en el
espacio, destruyendo la invariancia frente a rotaciones cuyo eje es
perpendicular a $\langle \vec{n}\rangle$. Los estados de equilibrio que se
encuentran en esta última fase, al tener una menor simetría que la del
hamiltoniano del sistema, se denominan estados con ruptura espontánea de la
simetría. 
 
Debido a la reducción de la simetría, es necesario un nuevo parámetro que 
describa la termodinámica de la fase ordenada. Este parámetro
extra, denominado parámetro de orden y que denotaremos por $\eta(\mathbf{x})$,
es una 
variable que cuantifica el grado de orden de
las fases, por lo que usualmente se define de forma que sea nulo en la fase
de mayor simetría.

\begin{figure}[h]
\centering
\subfigure[Fase plana]{
 \resizebox{160bp}{!}{\input{fase_plana-fig}}}
\quad
\subfigure[Fase rugosa]{
 \resizebox{160bp}{!}{\input{fase_arrugada-fig}}}
\end{figure}

La teoría de Landau de la transición de fase es una aproximación
fenomenológica que evita tratar con la descripción microscópica del sistema
mediante el uso de una descripción de \textit{grano grueso}
\cite{Goldenfield:Lecture_Notes}. Esto significa, 
en el caso de membranas cristalinas, que usaremos como coordenadas internas
$\mathbf{x}_{\Lambda}$, las cuales seguirán  siendo índices continuos excepto
que etiquetan \textit{bloques} de partículas y no partículas
individuales. Estos bloques tendrán una extensión lineal $\Lambda$ y estarán
centrados en $\mathbf{x}$. El correspondiente parámetro de orden de grano
grueso $\bar{\eta}(\mathbf{x}_{\Lambda})$ corresponderá a un promedio sobre
todas las partículas que forman el bloque.

En una descripción microscópica y continua, la función de partición
$\mathcal{Z}[T]$ de la membrana cristalina vendrá dada por
\begin{equation}\label{Zmicroscopica}
\mathcal{Z}[T]=\int D[\eta(\mathbf{x})]\; e^{-\beta H[\eta(\mathbf{x}),T]},
\end{equation}
donde con $\int D[\eta(\mathbf{x})]$ se denota la integral funcional sobre
todas las configuraciones posibles, $H[\eta(\mathbf{x})]$ es el hamiltoniano
microscópico y $\beta=\frac{1}{k_BT}$ es el factor de Boltzmann. Se define la
energía libre de Landau $F[\bar{\eta}(\mathbf{x}_{\Lambda}),T]$ como 
\begin{equation*}
e^{-F[\bar{\eta}(\mathbf{x}_{\Lambda}),T]}=\int D[\eta(\mathbf{x})]\; e^{-\beta
  H[\eta(\mathbf{x}),T]}
  \;\delta\left[\frac{1}{A_{\Lambda}}\int_{\Lambda} d^2\mathbf{s}\; \eta(\mathbf{x})-\bar{\eta}(\mathbf{x}_{\Lambda})\right],
\end{equation*}
siendo $\delta$ la función delta de Dirac y $A_{\Lambda}$ es el área
formada por las partículas del bloque. Esta energía libre de Landau se puede
interpretar como un potencial efectivo \cite{Heller:Method}, resultado de la integración de los
grados de libertad microscópicos sujetos a la condición de que sus promedios
en los  bloques coincida con el parámetro de orden
$\eta(\mathbf{x}_{\Lambda})$ de grano grueso. La función de partición
\eqref{Zmicroscopica} se puede expresar en función de la energía libre de
Landau como
\begin{equation*}
\mathcal{Z}[T]=\int D[\bar{\eta}(\mathbf{x}_{\Lambda})]\; e^{-F[\bar{\eta}(\mathbf{x}_{\Lambda}),T]}.
\end{equation*}
Esta última relación nos muestra que $F[\bar{\eta}(\mathbf{x}),T]$ es una funcional
que representa un hamiltoniano efectivo dividido por $k_BT$. A su vez, podemos
escribir 
\begin{equation*}
F[\bar{\eta}(\mathbf{x}),T]=\int\! d^2\mathbf{s}\ f[\bar{\eta}(\mathbf{x}),T],
\end{equation*}
siendo $d^2\mathbf{s}$ el elemento de área y $f[\bar{\eta}(\mathbf{x}),T]$ la
densidad de energía libre de Landau. Puesto que cerca de la transición de fase
el parámetro de orden $\bar{\eta}(\mathbf{x})$ toma valores pequeño, podemos
desarrollar $f[\bar{\eta}(\mathbf{x}),T]$ en potencias de
$\bar{\eta}(\mathbf{x})$ y sus derivadas, sin olvidar que los factores que
acompañan a los términos del desarrollo, en principio, dependerán de la
temperatura. Al igual que el hamiltoniano microscópico, la energía libre de Landau
$F[\bar{\eta}(\mathbf{x}_{\Lambda}),T]$ deberá cumplir las simetrías del sistema.

La simetría traslacional de la membrana implica que su energía libre de Landau
sólo puede depender de las derivadas de $\vec{r}(\mathbf{x}_{\Lambda})$, estos
son, los vectores tangentes $\vec{t}_{\alpha}$. En la fase
plana, los promedios de los vectores tangentes en un punto son no nulos, en
cambio, en la fase rugosa, su promedio es nulo por isotropía. Por tanto,
podemos utilizar como  parámetro de orden de grano grueso los vectores
tangentes locales $\vec{t}_{\alpha}(\mathbf{x}_{\Lambda})$, que serán
los promedios de los vectores tangentes de las partículas en la vecindad
$\Lambda$ del punto $\vec{r}(\mathbf{x})$. Entonces, en el caso de
membranas cristalina, tenemos un parámetro de orden con carácter tensorial,
con seis grados de libertad en cada punto.

La justificación de la descripción de grano grueso la encontramos en el
comportamiento de las correlaciones espaciales $\xi$, longitud a la que las
posiciones de las partículas se encuentran correlacionadas, éstas crecen a
medida que la temperatura del sistema se acerca a la temperatura crítica, lo
que significa que podemos encontrar regiones de dimensión lineal
$\Lambda<\xi(T)$ donde los vectores tangentes son aproximadamente
constantes. Además, hay que distinguir que, mientras que tratamos a
$\mathbf{x}_{\Lambda}$ como una variable continua, la función 
$\vec{t}_{\alpha}(\mathbf{x}_{\Lambda})$ no presenta variaciones a distancias
del orden del espaciado microscópico $a$, o equivalentemente, su transformada
de Fourier tiene vectores de onda con magnitud menor que cierto \textit{cut-off}
$\Lambda^{-1} \sim 1/a$. Concluimos entonces que el tamaño del bloque en la
descripción de grano grueso deberá cumplir que $a<<\Lambda<\xi(T)$. En lo que
sigue, siempre se utilizará está descripción de grano grueso, mientras no se
exprese lo contrario, por lo que omitiremos el subíndice $\Lambda$ para no
recargar sin necesidad la notación. 

Vamos a proponer una expresión de energía libre de Landau
$F[\vec{t}_{\alpha}(\mathbf{x}),T]$ para una membrana cristalina
apoyándonos en las siguientes propiedades:
\begin{description}
\item[Localidad e invariancia traslacional]: Dependerá de los vectores
  tangentes locales $\vec{t}_{\alpha}(\mathbf{x})$ y de las
  interacciones de corto alcance descritas a través del desarrollo en gradientes
  $$ F[\vec{t}_{\alpha}(\mathbf{x}),T]=\int\! d^2\mathbf{s}\ f[\vec{t}_{\alpha},\nabla
  \vec{t}_{\alpha},\dots].$$ 
\item[Simetría rotacional en $\mathbb{R}^3$]: Está simetría implica que sólo
  puede depender de productos escalares entre los vectores tangentes
  locales y sus gradientes con un número par de términos. 

\item[Simetría traslacional y rotacional en $\mathbb{R}^2$]: Lo que implica
  que los productos escalares serán con la métrica inducida $g_{\alpha\beta}$,
  de está forma, $F[\vec{t}_{\alpha}(\mathbf{x}),T]$ será invariante frente a
  reparametrizaciones de la superficie, esto es, frente a cambios de 
  las coordenadas $\mathbf{x}$.
\end{description}

 Finalmente, con esta última propiedad podemos proponer la siguiente energía libre de Landau $F$:
  \begin{multline}\label{ELandau}
    F[\vec{t}_{\alpha}(\mathbf{x}),T]=\\
\int d^2\mathbf{s}
    \left[
      \frac{t}{2}(\vec{t}_{\alpha}\cdot\vec{t}^{\alpha})+
      u(\vec{t}_{\alpha}\cdot\vec{t}_{\beta})(\vec{t}^{\alpha}\cdot\vec{t}^{\beta})+
      v(\vec{t}_{\alpha}\cdot\vec{t}^{\alpha})(\vec{t}_{\beta}\cdot\vec{t}^{\beta})+
      \frac{\kappa}{2}(\partial_{\alpha}\vec{t}^{\alpha})\cdot(\partial^{\beta}\vec{t}_{\beta}) \right].
  \end{multline}

El coeficiente $t$ veremos que corresponde a la temperatura reducida, $u$ y $v$ están
relacionados con los módulos elásticos de compresión y cizalladura y $\kappa$
es la rigidez de curvatura. Esta forma funcional únicamente se ha deducido a
partir de las simetrías del sistema, el precio que hay que pagar por esta aproximación es que en su
expresión tenemos los parámetros fenomenológicos $t$, $u$, $v$ y $\kappa$ cuya
dependencia funcional con los parámetros microscópicos originales nos es
desconocida, así como con la temperatura.


La energía libre \eqref{ELandau}, al no haber considerado interacciones de
largo alcance en su deducción, permite que la membrana la membrana pueda 
cruzarse libremente a sí misma. Para evitar que el sistema acceda
configuraciones incompatibles con la estructura cristalina subyacente, se
puede incluir el siguiente término de exclusión de volumen 

\begin{equation*}
\frac{b}{2}\int d^2\mathbf{s} d^2\mathbf{s'}
\delta^2(\vec{r}(\mathbf{x})-\vec{r}(\mathbf{x'})),
\end{equation*}

el parámetro $b$ modula este mecanismo, de forma que si es nulo, la superficie
puede cruzarse a sí misma sin coste energético. Las membranas con $b=0$ se
denominan \textit{membranas fantasmas}, el tratamiento matemático de éstas es
 más sencillo, y serán el objeto de estudio de este presente
trabajo, puesto que únicamente estamos interesados en las propiedades de la
fase plana y la transición de fase, en donde no influye de forma apreciable
este efecto indeseable. 

\subsection{Aproximación de Campo medio}\label{campo_medio}

\begin{figure}[h]
\centering
 \resizebox{\columnwidth}{!}{\input{campo_medio-fig}}
\caption{Parametrización usada en la aproximación del campo medio.}\label{campo-medio-fig}
\end{figure}

La energía libre \eqref{ELandau}, debido a las simetrías que le hemos exigido
que cumpla, describe una membrana isótropa, en la que las interacciones no
dependen de la dirección. En consecuencia, podemos afirmar que en la fase
plana los promedios de las posiciones tridimensionales de las partículas
$\langle\vec{r}(\mathbf{x})\rangle_{p}$ formarán una red regular plana:  
\begin{equation*}
 \langle\vec{r}(\mathbf{x})\rangle_{p}=(\zeta x^1,\zeta x^2,0)=(\zeta \mathbf{x},0), 
\end{equation*}
en donde hemos orientado y situado el sistema de referencia de las coordenadas
externas, de forma que, el plano $xy$ coincida con el de las
posiciones promedio. Las coordenadas internas utilizadas son 
proporcionales a la proyección en el plano $xy$ de las posiciones (figura
\ref{campo-medio-fig}). El valor del factor de proporcionalidad $\zeta$
coincide con la norma del promedio de vectores tangentes de la fase plana
\begin{equation}
 \left.\begin{array}{c}
\langle\vec{t}_{1}(\mathbf{x})\rangle_{p}=\langle\partial_1\vec{r}\rangle_{p}=(\zeta ,0,0)\\
\langle\vec{t}_{2}(\mathbf{x})\rangle_{p}=\langle\partial_2\vec{r}\rangle_{p}=(0,\zeta ,0)\\
 \end{array}\right\}\Rightarrow
\langle\vec{t}_{\alpha}(\mathbf{x})\rangle_{p}\cdot\langle\vec{t}_{\beta}(\mathbf{x})\rangle_{p}=\zeta^2 \delta_{\alpha\beta},
\end{equation}
de donde, en efecto, deducimos que
\begin{equation*}
|\langle\vec{t}_1(\mathbf{x})\rangle_{p}|=|\langle\vec{t}_2(\mathbf{x})\rangle_{p}|=\zeta.
\end{equation*}
La norma de los vectores tangentes es una medida de la distancia relativa
tridimensional entre los nodos de la red, por tanto, $\zeta$ cuantifica el
espaciado promedio entre puntos y nos da una idea del grado de extensión de la
membrana. La igualdad de la norma de los vectores tangentes es una consecuencia
directa de la isotropía de la membrana, de manera que es independiente del
sistema de coordenadas internas elegido.

La aproximación de campo medio asume que se pueden despreciar las
fluctuaciones, y por tanto, podemos aproximar los radio vectores de las
partículas y los vectores tangentes por sus promedios
\begin{align*}
 \vec{r}(\mathbf{x})&\simeq(\zeta \mathbf{x},0),\\
 \vec{t}_1(\mathbf{x})&\simeq(\zeta ,0,0),\\
 \vec{t}_2(\mathbf{x})&\simeq(0,\zeta,0),
\end{align*}
de forma que la dependencia espacial de los vectores tangentes es eliminada. La
aproximación del campo medio sólo considera estados homogéneos, además, que
el carácter tensorial del parámetro de orden, sus seis grados de libertad, se
reducen a un solo parámetro $\zeta$, que será el parámetro de orden en esta
aproximación.

La expresión del potencial efectivo \eqref{ELandau} en este caso será \cite{Bowick:Libro_superficies}:
\begin{align}
  F_M(\zeta,T)=&\int\! d^2 \mathbf{s}\ 2\zeta^2\!\left( \frac{t}{2} + (u+2v)\zeta^2\right)\\
  =&Af_M(\zeta),
\end{align}
donde 
$$A=\int d^2 \mathbf{s}=\zeta^2\int d^2 \mathbf{x} ,$$
es el área total de la membrana y 
\begin{equation}\label{densidadELandau}
f_M(\zeta,T)=2\zeta^2\!\left( \frac{t}{2} + (u+2v)\zeta^2\right),
\end{equation}
es la densidad de energía libre de Landau en la aproximación del campo medio. 
La función de partición corresponde a la siguiente integral
\begin{equation}\label{Zcm}
 \mathcal{Z}[T]=\int d\zeta\;e^{-Af_M(\zeta,T)}.
\end{equation}
Ahora bien, si $A$ es muy grande, únicamente contribuyen los valores menores de
$f_M(\zeta)$, y si además, posee un mínimo con respecto a $\zeta$ en
$\zeta_0$, el integrando de \eqref{Zcm} tendrá un máximo,
tanto más acusado cuanto mayor sea el área $A$. Desarrollando $f_M(\zeta)$ en
serie alrededor del mínimo $\zeta_0$
\begin{equation*}
f_M(\zeta)\simeq f_M(\zeta_0)+\frac{1}{2}\left(\frac{\partial f_M}{\partial \zeta}\right)_{\!\zeta_0}(\zeta-\zeta_0)^2,
\end{equation*}
y sustituyendo en la expresión anterior \eqref{Zcm} de $\mathcal{Z}[T]$ 
\begin{equation}\label{Zcm2}
 \mathcal{Z}[T]\simeq\int
 d\zeta\;e^{-Af_M(\zeta_0)-\frac{A}{2}\ddot{f}_M(\zeta_0)(\zeta-\zeta_0)^2}=
 \sqrt{\frac{2\pi}{A\ddot{f}_M(\zeta_0)}}\; e^{-Af_M(\zeta_0)}.
\end{equation}

La energía libre de Hemholtz intensiva $f_H(T)$ en el límite termodinámico
viene dada por el siguiente límite
\begin{equation*}
\lim_{A\rightarrow \infty}\; \frac{1}{A} \log \mathcal{Z}[T]=-\beta f_H(T).
\end{equation*}
Si tomamos este límite en la expresión \eqref{Zcm2} de $\mathcal{Z}[T]$
encontramos
\begin{equation*}
\lim_{A\rightarrow \infty}\; \frac{1}{A} \log \mathcal{Z}[T]=-f_M(\zeta_0),
\end{equation*}
es decir, $f_M(\zeta_0)=\beta f_H(T)$, lo que significa que los estados de
equilibrio del sistema vienen determinados por el mínimo de la energía libre
de Landau. Hallando el mínimo de \eqref{densidadELandau} 
\begin{equation*}
\left(\frac{\partial f_M}{\partial \zeta}\right)_{\!\zeta=\zeta_0}\!=0 \;
\Rightarrow \; 4\zeta_0\left(\frac{t}{2}+2(u+2v)\zeta_0^2\right)=0,
\end{equation*}
encontramos que el comportamiento de $f_M$ depende del signo del parámetro $t$:
\begin{figure}[h]
\centering
 \resizebox{\columnwidth}{!}{\input{energia_libre_CM-fig}}
\caption{Comportamiento de $f_M(\zeta)$ en función del parámetro $t$.}
\end{figure} 

\begin{enumerate}
\item Para $t>0$ el mínimo de $F_M$ ocurre en $\zeta_0=0$, esto indica que la
  simetría es total, la membrana se encuentra en la fase desordenada,
  arrugada.
\item Para $t<0$, la función $f_M$ tiene mínimo doblemente
  degenerado, correspondiente a los valores
  \begin{equation}\label{zeta_0}
    \zeta_0=\mp \sqrt{\frac{-t}{4(u+2v)}}
  \end{equation}
  que muestra que la membrana se encuentra en un estado con ruptura espontánea
  de la simetría, indicando una fase ordenada, la fase plana. 
\end{enumerate}

La evaluación del punto mínimo sugiere entonces que el
signo del parámetro $t$ determina en cual de las dos fase se encuentra el
sistema. Para encontrar el significado concreto del parámetro $t$
realizamos un desarrollo en serie en función de la temperatura reducida
$(T-T_C)/T_C$, de los parámetros que intervienen en la densidad de energía libre:
\begin{align*}
t=&\; t_0+t_1\frac{(T-T_c)}{T_c}+O(T-T_c)^2,\\
u=&\; u_0+u_1\frac{(T-T_c)}{T_c}+O(T-T_c)^2,\\
v=&\; v_0+v_1\frac{(T-T_c)}{T_c}+O(T-T_c)^2.
\end{align*}
Podemos tomar a $u$ y $v$ como constantes, ya que su dependencia en la 
temperatura no contribuye en los primeros órdenes en el comportamiento
termodinámico cerca de $T_c$. Ahora bien,
para cualquier valor de $T\!<\!T_c$ sabemos que el parámetro de orden del
sistema debe tener un valor no nulo, esto es, $\zeta_0\!\neq\! 0$, lo que
implica que $t_0=0$. Por otra parte, siempre podremos escalar $\zeta$, 
de manera que $t_1=1$, con lo que deducimos finalmente que
\begin{equation*}
t=\frac{(T-T_c)}{T_c}+O(T-T_c)^2,
\end{equation*}
el parámetro $t$ es la temperatura reducida adimensional.

%Como el parámetro de orden se anula de forma continua en la transición, a este
%tipo de transiciones de fase con ruptura de simetría se les denomina, de forma
%genérica, continuas.

%Ehrenfest introdujo una clasificación de la transiciones de fase basandose en
%la continuidad del potencial termodinámico. Según Ehrenfest una transición dee
%fase es de orden $n$ si la primera derivada del potencial termodinámico que no
%es continua es la derivada $n$-ésima. Esta clasificación no siempre es
%equivalente a la clasificación basada en la ruptura de simetría.

\subsection{Exponentes críticos}
\label{exponentes_criticos}
La teoría de Landau no teniendo en cuenta las fluctuaciones del parámetro de
orden (relevantes en la vecindad de la transición de fase) expresa la energía
libre como una función analítica del parámetro de orden. A partir de la teoría
del grupo de renormalización \cite{Cardy} se encuentra que estas fluctuaciones contribuyen
en una parte no analítica en la energía libre\footnote{Cuando escribimos
  $g(t)\sim t^{\gamma}$ estamos indicando que $$\gamma=\lim_{t\rightarrow
  0}\frac{\log g(t)}{\log t}$$.}
\begin{equation}
f_s(t)\sim|t|^{2\nu}
\end{equation}
que explica el comportamiento divergente en la proximidad de la transición de
fase $t\simeq 0$ de los observables termodinámicos, como por ejemplo, el calor específico
\begin{equation*}
C_V\sim|t|^{-\alpha},
\end{equation*}
donde $\alpha$ es el exponente crítico, o la longitud de correlación
normal-normal $\xi$ de la membrana cristalina \cite{David:normal}, $\xi\sim|t|^{-\nu}$,
definida como
\begin{equation*}
\langle \vec{n}(\mathbf{x})\cdot\vec{n}(\mathbf{0})\rangle\propto e^{-x/\xi}.
\end{equation*}
No todos los exponentes críticos son independientes entre sí, existen
relaciones entre ellos.

\subsection{Relaciones de escala del radio de giro}

Para estudiar como escala el tamaño de la membrana vamos a emplear la
aproximación de Flory \cite{Gennes:Scaling}, para ello consideramos una
membrana de $D$ dimensional en un espacio 
$d$ dimensional, el tamaño de esta membrana lo podemos especificar por el
radio de giro $R_G$
\begin{equation*}
R_G^2=\frac{1}{dA_D}\int d^D\mathbf{x}\int d^D\mathbf{x}' \langle |
\vec{r}(\mathbf{x})-\vec{r}(\mathbf{x}')|^2\rangle, 
\end{equation*}
siendo $A_D$ el área $D$ dimensional de la membrana. En dos dimensiones este
observable toma la forma
\begin{equation*}
R_G^2=\frac{1}{3A}\int d^2\mathbf{x}\, (\langle
\vec{r}(\mathbf{x})\cdot\vec{r}(\mathbf{x})\rangle-\langle
\vec{r}(\mathbf{x})\rangle^2)=\frac{1}{3A}\int d^2\mathbf{x}\, \langle
\vec{R}(\mathbf{x})\cdot\vec{R}(\mathbf{x})\rangle^2,
\end{equation*} 
donde $A$ es área de la membrana (hemos eliminado el subíndice $_2$) y
$\vec{R}(\mathbf{x})$ es el vector de posición del punto de coordenadas
internas $\mathbf{x}$ respecto al centro de masas $\vec{r}_{CM}$ de la
membrana
\begin{equation*}
\vec{R}(\mathbf{x})=\vec{r}(\mathbf{x})-\vec{r}_{CM}.
\end{equation*}
Por tanto, $R_G$ corresponde a la distancia cuadrática media respecto al
centro de masas. Volviendo a la membrana $D$ dimensional, si la dimensión
lineal de la membrana es $L$, tiene entonces $L^D$ nodos, podemos
aproximar los términos \cite{David:normal,Gomper:triangulated} de la energía libre \eqref{ELandau}  
\begin{equation*}
\vec{t}_{\alpha}=\partial_{\alpha} \vec{r}\simeq \frac{R_G}{L}
\frac{\vec{t}_{\alpha}}{|\vec{t}_{\alpha}|}\quad \text{y}\quad \int
d^D\mathbf{x}\simeq L^D,
\end{equation*}
con lo que la densidad de energía libre de Landau resulta
\begin{equation*}
f(R_G,t)\simeq t R_G^2 L^{D-2}+(u+Dv) R_G^4 L^{D-4}+\kappa R_G^2 L^{D-4}.
\end{equation*}
Al igual que en la aproximación del campo medio anterior, minimizando esta
energía libre respecto a $R_G$ encontramos el estado de equilibrio del sistema. En la fase
plana, $t<0$ tenemos
\begin{equation*}
\frac{\partial f(R_G,t)}{\partial R_G}\simeq 2t R_G L^{D-2}+4(u+Dv) R_G^3
L^{D-4}+2 \kappa R_G L^{D-2}=0.
\end{equation*}
Despreciando $\kappa\simeq 0$ y para una membrana bidimensional $D=2$
\begin{equation}\label{Rg:escala_plano}
R_G\sim |t|^{1/2}L\sim L^2 \Rightarrow R_G\sim L.
\end{equation}
A este resultado también podíamos haber llegado utilizando los resultados
obtenidos en la aproximación del campo medio, pues el radio de giro debe ser
proporcional al producto de $L$ y $\zeta$, el espaciado entre
nodos. De manera que utilizando el resultado \eqref{zeta_0} para la fase plana
de la distancia entre nodos encontramos la misma relación de escala que \eqref{Rg:escala_plano}
\begin{equation*}
R_G\sim \zeta L \quad \Rightarrow \quad R_G\sim |t|^{1/2}L.
\end{equation*}
Se define $\nu_F$, el exponente de escala de Flory, a partir de la relación de
escala
\begin{equation*}
 R_G\sim L^{\nu_F}. 
\end{equation*}
Este exponente relaciona la extensión espacial de la membrana con su tamaño
interno $L$. Por otro lado, tenemos la dimensión fractal o de Hausdorff $d_H$,
que relaciona el \textit{área interna} $L^D$ con su tamaño espacial
\begin{equation*}
R_G^{d_H}\sim L^D.
\end{equation*} 
Entonces, el exponente de Flory $\nu_F$ está relacionado con la dimensión
fractal por
\begin{equation*}
 \nu_F=\frac{D}{d_H}\ \stackrel{D=2}{-\!\!\!\longrightarrow}\ \nu_F=\frac{2}{d_H}.
\end{equation*}
Y en la fase plana, según \eqref{Rg:escala_plano}, para $D=2$ tenemos que
$\nu_F=1$ y $d_H=2$.


En la fase rugosa $t>0$, el tamaño del sistema disminuye drásticamente, dado que la distancia
entre nodos $\zeta\simeq 0$. Entonces el único término relevante es el correspondiente a $t$
\begin{equation*}
f(R_G,t)\simeq t R_G^2 L^{D-2}, 
\end{equation*}
y minimizando respecto al radio de giro encontramos
\begin{equation*}
R_G^2\sim L^{(2-D)} \quad \text{para } D<2.
\end{equation*}
En $D>2$, las fluctuaciones son demasiado débiles para prevenir el colapso
completo de la red. En $D=2$, el exponente es nulo, lo que significa una
divergencia logarítmica del radio de giro, ya que
\begin{equation*}
R_G^2\sim L^{D-2}=L^{2\nu_F}\Rightarrow \log L=\lim_{2\nu_F\rightarrow
  0}\left[\frac{1-e^{-2\nu_F\log R^2_G}}{2\nu_F}\right]\equiv\lim_{2\nu_F\rightarrow
  0}\left[\frac{1-R^{2\nu_F}_G}{2\nu_F}\right],
\end{equation*}
por tanto, en la fase rugosa para una membrana bidimensional tenemos que
$\nu_F=0\Rightarrow d_H=\infty$ y cumple una relación de escala logarítmica  
\begin{equation*}
R_G\sim(\log L)^{1/2}.
\end{equation*}

Para la transición de fase $t\rightarrow 0$ en una membrana bidimensional,
Le Doussal y Radzihovsky \cite{Doussal:nu} han encontrado, mediante métodos analítico aproximados,
los siguientes valores para el exponente de Flory y la dimensión fractal 
\begin{equation*} 
 \nu_F=0.73\ \Rightarrow\ d_H=2.74 .
\end{equation*}

\section{Estudio de la fase plana}

Hemos visto que en la fase plana las posiciones medias de la partículas forman
un plano. Para el estudio de esta fase vamos a considerar las partículas de la
membrana únicamente sufren pequeños respecto a este plano, que denominaremos
plano base o plano de referencia. Las coordenadas externas
$\vec{r}(\mathbf{x})$ de un punto con coordenadas internas $\mathbf{x}$ serán
\begin{equation}\label{r_fase_plana}
\vec{r}(\mathbf{x})=(\mathbf{x}+\mathbf{u(\mathbf{x})},h(\mathbf{x})),
\end{equation}
donde estamos utilizando las mismas coordenadas que en la sección \ref{campo_medio}, excepto que en
este caso tomamos $\zeta=1$, esto es, las coordenadas internas coinciden con
la proyección en el plano $xy$ de las posiciones promedios. Los términos
$\mathbf{u(\mathbf{x})}$ y $h(\mathbf{x})$ representan las fluctuaciones
paralelas y transversales respecto al plano de
referencia respectivamente. Esta última
expresión \eqref{r_fase_plana} se puede escribir de la siguiente forma
equivalente como
\begin{equation}\label{deformacion}
\vec{r}(\mathbf{x})=\vec{r}_0(\mathbf{x})+\vec{u}(\mathbf{x}),
\end{equation}
siendo $\vec{r}_0(\mathbf{x})=\langle\vec{r}(\mathbf{x}) \rangle_p$ el radio
vector que determina las posiciones en el plano base, es igual a las
posiciones medias, y $\vec{u}(\mathbf{x})$ es el vector desplazamiento, que
mide las diferencias de las posiciones respecto al plano de referencia (figura
\ref{Deformacion-fig}). Puesto
que estamos estudiando una membrana isótropa, el valor medio del
vector desplazamiento es nulo $\langle\vec{u}(\mathbf{x}) \rangle=(0,0,0)$.
Los vectores tangentes en el plano base, que serán los mismos para todos los
puntos de la superficie, coinciden con los vectores unitarios $\hat{i}$ y
$\hat{j}$ del sistema de referencia de las coordenadas externa y los denotaremos por $\vec{e}_\alpha$
\begin{align*}
 \vec{e}_1=&\frac{\partial \vec{r}_0}{\partial x^1}\equiv \hat{i},\\
 \vec{e}_2=&\frac{\partial \vec{r}_0}{\partial x^1}\equiv \hat{j}.
\end{align*}
Por lo tanto, la métrica inducida en
este plano base, es diagonal y unitaria en cualquier punto
$g^{(0)}_{\alpha\beta}=\vec{e}_{\alpha}\cdot\vec{e}_{\beta}
=\delta_{\alpha\beta}$. 

\begin{figure}[h]
\centering
 \input{deformacion-fig}
\caption{Posiciones anterior $\vec{r}_0(\mathbf{x})$ y posterior
  $\vec{r}_0(\mathbf{x})$ a la deformación de un punto de coordenadas internas
$\mathbf{x}$.}\label{Deformacion-fig}
\end{figure}

Atendiendo a las propiedades que caracteriza, la energía libre de Landau
\eqref{ELandau} se puede expresar como la suma de dos términos:

\begin{equation*}
 F[\vec{t}_{\alpha}(\mathbf{x}),T]= F_E[\vec{t}_{\alpha}(\mathbf{x}),T]+F_C[\vec{t}_{\alpha}(\mathbf{x}),T],
\end{equation*}
donde:
\begin{description}
\item[ $F_E\rightarrow $ Energía libre elástica:] Depende del producto escalar
  de los vectores tangentes, o equivalentemente, las distancia relativas entre
  los nodos. Por lo tanto, caracteriza las propiedades elásticas de la membrana.
  \begin{multline}\label{ELandau_elastica}
    F_E[\vec{t}_{\alpha}(\mathbf{x}),T]=\\\int d^2\mathbf{s}
    \left[
      \frac{t}{2}(\vec{t}_{\alpha}\cdot\vec{t}^{\alpha})+
      u(\vec{t}_{\alpha}\cdot\vec{t}_{\beta})(\vec{t}^{\alpha}\cdot\vec{t}^{\beta})+
      v(\vec{t}_{\alpha}\cdot\vec{t}^{\alpha})(\vec{t}_{\beta}\cdot\vec{t}^{\beta})\right].
  \end{multline}
\item[ $F_C\rightarrow $ Energía libre de curvatura:] Depende de la variación
  de los vectores tangentes, caracteriza la curvatura de la membrana. 
  \begin{equation}\label{ELandau_curvatura}
       F_C[\vec{t}_{\alpha}(\mathbf{x}),T]=\int d^2\mathbf{s}\; 
      \frac{\kappa}{2}(\partial_{\alpha}\vec{t}^{\alpha})\cdot(\partial^{\beta}\vec{t}_{\beta})
  \end{equation}
\end{description}

Estudiaremos ambos términos para el caso particular de una membrana cristalina 
en las siguiente secciones.

\subsection{Tensor de deformaciones}

Supongamos que la membrana se encuentra en una determinada configuración
$\vec{r}(\mathbf{x})$ de la fase plana. En vista de la expresión
\eqref{deformacion}, podemos interpretar que esta configuración es el
resultado de una deformación de la membrana a partir de la configuración
$\vec{r}_0(\mathbf{x})$ completamente plana (plano base). Las distancias relativas de los
puntos, en general, cambiarán al producirse la deformación. Si
consideramos dos puntos muy próximos $P$ y $Q$ de coordenadas internas
$\mathbf{x}$ y $\mathbf{x}+d\mathbf{x}$, respectivamente
\cite{Landau_Elasticidad}. El vector tridimensional que los une 
después de la deformación será
\begin{equation}\label{dr}
d\vec{r}= \vec{r}(\mathbf{x}+d\mathbf{x})-\vec{r}(\mathbf{x}).
\end{equation}
De igual forma podemos encontrar el vector correspondiente a la configuración
completamente plana, o equivalentemente, anterior a la deformación
\begin{equation}\label{dr0}
d\vec{r}_0= \vec{r}_0(\mathbf{x}+d\mathbf{x})-\vec{r}_0(\mathbf{x}).
\end{equation}
Una forma de caracterizar la deformación consiste en especificar la variación
de las distancias relativas entre puntos próximos. Estas distancias se
corresponden con la diferencia entre la norma al cuadrado de estos dos vectores
\eqref{dr}\eqref{dr0}. Es razonable suponer que esta diferencia será bilineal
en los diferenciales de $dx^{\alpha}$ de las coordenadas internas 
\begin{equation}\label{drU}
(d\vec{r})^2=(d\vec{r}_0)^2+2u_{\alpha\beta}dx^{\alpha}dx^{\beta}
\end{equation}
donde $u_{\alpha\beta}$ es un tensor, denominado tensor de deformaciones, que justamente mide
la variación de las distancias relativas (al cuadrado) entre puntos, respecto a su valor
inicial en una determinada dirección. En virtud de \eqref{elemento_linea} y
teniendo en cuenta que las coordenadas en la configuración completamente plana
son ortonormales, podemos expresar las normas al cuadrado de los vectores
\eqref{dr}\eqref{dr0} en función de la métricas inducidas como
\begin{align}
(d\vec{r_0})^2&=\delta_{\alpha\beta}dx^{\alpha}dx^{\beta},\label{Meuclidea}\\
(d\vec{r})^2&=g_{\alpha\beta}dx^{\alpha}dx^{\beta},
\end{align}
con las que obtenemos a partir \eqref{drU} la siguiente expresión para el
tensor de deformaciones
\begin{equation}\label{tensor_metrica}
u_{\alpha\beta}=\frac{1}{2}\left(g_{\alpha\beta} - \delta_{\alpha\beta}\right).
\end{equation}
 Supongamos que los $P$ y $Q$ se encuentran
inicialmente sobre la dirección $x^1$, esto es, $(d\vec{r}_0)^2=(dx^1)^2$, y
la distancia entre ambos es $dl_{(x^1)}=dx^1$. A
partir de \eqref{drU} podemos obtener la distancia $dL_{(x^1)}$ entre estos
dos puntos después de la deformación:
\begin{equation*}
dL_{(x^1)}=\sqrt{1+2u_{11}}\;dl_{(x^1)}
\end{equation*}
Y si la deformación es pequeña $u_{11}\simeq 0$
\begin{equation*}
dL_{(x^1)}\simeq (1+u_{11}) dl_{(x^1)} \Rightarrow u_{11}=\frac{dL_{(x^1)}-dl_{(x^1)}}{dl_{(x^1)}}
\end{equation*}
De igual forma podemos obtener una expresión análoga para $u_{22}$. Entonces,
los elementos diagonales de $u_{\alpha\beta}$ determinan las elonganciones
relativas en las direcciones marcadas por los vectores de la base de las
coordenadas internas. Consideremos otro punto $Q$, de forma que la linea $PR$
es perpendicular a $PQ$. El ángulo $\theta$ que forman estos dos vectores
después de la deformación podemos obtenerlo a partir de
\eqref{tensor_metrica}:
\begin{equation*}
u_{12}=u_{21}\simeq \frac{1}{2}\cos \theta \Rightarrow 2u_{12}\simeq\sen \left(\frac{\pi}{2}-\theta\right)\simeq \frac{\pi}{2}-\theta.
\end{equation*}
Por tanto, $u_{\alpha\beta}$ con $\alpha\neq\beta$, los elementos no
diagonales del tensor de deformaciones están directamente relacionados con la
variación del ángulo que forman los vectores tangentes de la base antes y
después de la deformación.

La expresión \eqref{tensor_metrica} muestra claramente que $u_{\alpha\beta}$
es un tensor simétrico, de manera que, siempre 
podremos encontrar para un punto dado un sistema de coordenadas $\mathbf{x}$ en donde
$u_{\alpha\beta}$ sea diagonal, es decir, con las únicas componentes no nulas
$u_{11}$ y $u_{22}$. Estas dos componentes son los valores
principales del tensor de deformaciones, que denotaremos por $u_{(1)}$ y
$u_{(2)}$. Supongamos que un punto dado, encontramos unas coordenadas internas
$\{\mathbf{x}_d\}$, donde $u_{\alpha\beta}$ es diagonal, entonces \eqref{drU},
teniendo en cuenta \eqref{Meuclidea}, es 
\begin{equation*}
(d\vec{r})^2=(1+2u_{(1)})(dx_d^{1})^2+(1+2u_{(2)})(dx_d^{2})^2.
\end{equation*}
Esto significa que la deformación en cualquier elemento de superficie puede
descomponerse como deformaciones independientes en dos direcciones
perpendiculares, los ejes principales del tensor de deformación. Cada una de
estas deformaciones será a una extensión o compresión en la correspondiente
dirección: la longitud $dx^1$ en el primer eje principal será
\begin{equation*}
dx^1=\sqrt{1+2u_{(1)}}\; dx^1_d\simeq (1+u_{(1)})\; dx^1_d,
\end{equation*}
donde hemos considerado pequeñas deformaciones. La expresión para el otro eje
principal se obtiene de forma similar. Consideremos ahora un elemento
de área infinitesimal $d^2\mathbf{s}_d$, en las coordenadas $\{\mathbf{x}_d\}$
corresponde a $dx_d^1dx_d^2$. Después de la deformación, este elemento de área
cambiará a $d^2\mathbf{s}=dx^1dx^2$ y estará relacionado con el volumen inicial por
\begin{equation*}
d^2\mathbf{s}=(1+u_{(1)})(1+u_{(2)})d^2\mathbf{s}_d\simeq (1+u_{(1)}+u_{(2)})d^2\mathbf{s}_d,
\end{equation*}
donde hemos despreciado ordenes superiores. Dado que
$u_{\alpha}^{\ \alpha}=u_{(1)}+u_{(2)}$ es la traza del tensor de deformaciones, la
suma de sus elementos diagonales, cuyo valor es independiente del sistema de
coordenadas, tenemos la siguiente expresión general:
\begin{equation}\label{deformacion_area}
d^2\mathbf{s}=(1+u_{\alpha}^{\ \alpha})d^2\mathbf{s}_0,
\end{equation}
siendo $d^2\mathbf{s}_0$ y $d^2\mathbf{s}$ los elementos de área anterior y
posterior a la deformación respectivamente.

Podemos encontrar otra expresión para el tensor $u_{\alpha\beta}$ si
relacionamos $(d\vec{r})^2$ con $(d\vec{r}_0)^2$ a partir de
la expresión \eqref{deformacion}:
\begin{equation}\label{dl_vectorial}
(d\vec{r})^2=(d\vec{r_0}+d\vec{u})^2=(d\vec{r})^2_0+(d\vec{u})^2+2d\vec{r}_0\cdot d\vec{u}.
\end{equation}
Al comparar está expresión con \eqref{drU}
\begin{equation}\label{2udxdx}
2u_{\alpha\beta}dx^{\alpha}dx^{\beta}=(d\vec{u})^2+2d\vec{r}_0\cdot d\vec{u}.
\end{equation}
Ahora bien, podemos expresar el vector desplazamiento $\vec{u}(\mathbf{x})$ en función de
$\vec{e}_{\alpha}$, los vectores de la base del plano tangente a $\vec{r_0}$:
\begin{equation*}
  \vec{u}(\mathbf{x})=u^{\alpha}(\mathbf{x})\vec{e}_{\alpha}+h(\mathbf{x})\vec{n},
\end{equation*}
donde el vector $\vec{n}$ es el vector normal al plano base
\begin{equation}
\vec{n}=\frac{\vec{e}_1\times \vec{e}_2}{|\vec{e}_1\times \vec{e}_2|}.
\end{equation}
Las componentes del vector desplazamiento dependen las cordenadas internas, por lo que su elemento
diferencial $d\vec{u}$ y su cuadrado $(d\vec{u})^2$ vendrán dados por: 
\begin{equation}\label{du2}
d\vec{u}=\left[\frac{\partial
  u^{\alpha}}{\partial x^{\beta}}\vec{e}_{\alpha}+\frac{\partial
  h}{\partial x^{\beta}} \vec{n}\right]dx^{\beta}\Rightarrow
(d\vec{u})^2=\left[
\frac{\partial u_{\gamma}}{\partial x^{\alpha}}
\frac{\partial u^{\gamma}}{\partial x^{\beta}}+ 
\frac{\partial h}{\partial x^{\alpha}}
\frac{\partial h}{\partial x^{\beta}}\right]
dx^{\alpha}dx^{\beta}.
\end{equation}
Por otro lado tenemos también
\begin{equation}\label{drdu}
d\vec{r}_0\cdot d\vec{u}=\frac{\partial u_{\alpha}}{\partial x^{\beta}}dx^{\alpha}dx^{\beta}=\frac{1}{2}\left[\frac{\partial u_{\beta}}{\partial x^{\alpha}}+ \frac{\partial u_{\alpha}}{\partial x^{\beta}}\right]dx^{\alpha}dx^{\beta},
\end{equation}
donde hemos usado que
 $$\frac{\partial u_{\beta}}{\partial
   x^{\alpha}}dx^{\alpha}dx^{\beta}=\frac{\partial u_{\alpha}}{\partial
   x^{\beta}}dx^{\alpha}dx^{\beta}.$$
Finalmente, sustituyendo \eqref{du2} y \eqref{drdu} en \eqref{2udxdx}
obtenemos otra expresión equivalente para el tensor de deformaciones:
\begin{equation}\label{tensor_deformacion}
u_{\alpha\beta}=\frac{1}{2}\left(
 \frac{\partial u_{\beta}}{\partial x^{\alpha}}+
 \frac{\partial u_{\alpha}}{\partial x^{\beta}}+
 \frac{\partial u_{\gamma}}{\partial x^{\alpha}}
\frac{\partial u^{\gamma}}{\partial x^{\beta}}+
\frac{\partial h}{\partial  x^{\alpha}}\frac{\partial h}{\partial x^{\beta}}\right).
\end{equation}
Si la deformación es pequeña, lo que significa que las distancias entre puntos
próximos, en cualquier dirección, no se modifican demasiado comparadas con 
su valor inicial. Esto implica que, bajo pequeñas deformaciones, podemos
despreciar los términos cuadráticos en las derivadas de $u_{\alpha}$ en
\eqref{tensor_deformacion}; no podemos hacer lo mismo con $h$, dado que no hay
términos en primer orden: 
 \begin{equation}\label{u_aprox}
 u_{\alpha\beta}\simeq\frac{1}{2}\left(
  \frac{\partial u_{\beta}}{\partial x^{\alpha}}+
  \frac{\partial u_{\alpha}}{\partial x^{\beta}}+
 \frac{\partial h}{\partial  x^{\alpha}}\frac{\partial h}{\partial x^{\beta}}\right).
\end{equation}

\subsection{Propiedades elásticas}

Dado que \eqref{ELandau_elastica} depende de los productos escalares de los
vectores tangentes, podemos expresarlo en función del tensor de deformaciones
$u_{\alpha\beta}$, ya que
\begin{equation*}
u_{\alpha\beta}=\frac{1}{2}(\vec{t}_{\alpha}\cdot\vec{t}_{\beta}-\delta_{\alpha\beta})\
\Rightarrow \
\vec{t}_{\alpha}\cdot\vec{t}_{\beta}=\delta_{\alpha\beta}+2u_{\alpha\beta}.
\end{equation*}
Los factores del interior del integrando de \eqref{ELandau_elastica} quedan
\begin{align*}
\vec{t}_{\alpha}\cdot\vec{t}^{\alpha} &=2(1+u_{\alpha}^{\ \alpha}),\\
(\vec{t}_{\alpha}\cdot\vec{t}^{\alpha})(\vec{t}_{\alpha}\cdot\vec{t}^{\alpha})&=4(1+2u_{\alpha}^{\
  \alpha}+u_{\alpha}^{\ \alpha}u^{\beta}_{\ \beta}),\\
(\vec{t}_{\alpha}\cdot\vec{t}_{\beta})(\vec{t}^{\alpha}\cdot\vec{t}^{\beta})&=2+4u_{\alpha}^{\
  \alpha}+4u_{\alpha\beta}u^{\alpha\beta}.
\end{align*}
Sustituyéndolos en la expresión \eqref{ELandau_elastica} de $F_E$, y
despreciando términos constantes, llegamos a que \cite{Bowick:Libro_superficies}
\begin{equation}\label{Elibre_elastica}
F_E[u_{\alpha\beta},T]=\int d^2\mathbf{x}
\left[\tau u_{\alpha}^{\ \alpha}+
\mu u_{\alpha\beta}u^{\alpha\beta} +
\frac{\lambda}{2}u_{\alpha}^{\ \alpha}u^{\beta}_{\ \beta}\right],
\end{equation}
donde las nuevas constantes $\tau$, $\mu$ y $\lambda$ están relacionadas con
las anteriores por
\begin{align}
\tau&=t+4u+8v,\label{tau}\\
\mu&=4u,\\
\frac{\lambda}{2}&=4v.
\end{align}
El tensor de esfuerzos $\sigma_{\alpha\beta}$, representa las tensiones
internas debido a variaciones en la extensión de la membrana. Al igual que
$u_{\alpha\beta}$ es simétrico y podemos obtenerlo a partir de $f_E$, la densidad de
la energía libre \eqref{Elibre_elastica}
\begin{equation*}
f_E[u_{\alpha\beta},T]=\tau u_{\alpha}^{\ \alpha}+
\mu u_{\alpha\beta}u^{\alpha\beta} +
\frac{\lambda}{2}u_{\alpha}^{\ \alpha}u^{\beta}_{\ \beta},
\end{equation*}
ya que corresponde a la variable conjugada de $u_{\alpha\beta}$
\begin{equation}\label{tensor_esfuerzos}
\sigma_{\alpha\beta}=\frac{\partial f_E}{ \partial u_{\alpha\beta}}.
\end{equation}
cuando $u_{\alpha\beta}=0 \ \forall\alpha,\beta$ la membrana se encuentra en
la configuración completamente plana, con todos sus puntos
equidistantes. Ahora bien, en está situación, las tensiones internas deberán
encontrarse equilibradas\cite{Landau_Elasticidad}, lo que implica que también $\sigma_{\alpha\beta}=0 \
\forall\alpha,\beta$. Por lo tanto, no puede haber 
un término lineal en $u_{\alpha\beta}$ en el desarrollo de $f_E$, de forma que,
 $\tau=0$ en la expresión \eqref{Elibre_elastica} y tenemos que \eqref{tau} es ahora
\begin{equation}\label{tau=0}
-t=4(u+2v),
\end{equation}   
expresión que coincide con la ecuación del campo medio para la temperatura reducida
\eqref{zeta_0} para $\zeta_0=1$, en acuerdo con el valor el parámetro $\zeta=1$
que hemos fijado. Finalmente la energía libre elástica de Landau queda como
\begin{equation}
F_E[u_{\alpha\beta},T]=\int d^2\mathbf{x}\, f_E[u_{\alpha\beta},T]\quad \text{con}\quad
f_E[u_{\alpha\beta},T]=
\mu u_{\alpha\beta}u^{\alpha\beta} +
\frac{\lambda}{2}u_{\alpha}^{\ \alpha}u^{\beta}_{\ \beta}\label{densidad_Fe}.
\end{equation} 
Bajo pequeñas deformaciones podemos usar \eqref{u_aprox} 
como expresión de $u_{\alpha\beta}$, lo que equivale a retener términos hasta
$O[(\partial u)^3]$ y $O[(\partial u)^2(\partial h)^2]$. La expresión
\eqref{densidad_Fe} coincide con la referencia \cite{Landau_Elasticidad}, los coeficientes $\mu$ y
$\lambda$ son los coeficientes de Lamé.  

Según la expresión \eqref{deformacion_area} el cambio relativo en el área infinitesimal
viene dada por la suma $u_{\alpha}^{\ \alpha}$ . Si esta suma es nula, el área
de la membrana no se altera tras la deformación. Este tipo de deformaciones se
denominan de cizalladura pura. Por otro lado, tenemos las deformaciones que
mantienen la forma pero producen un cambio en el área, este tipo de denominan
compresiones puras. Cualquier deformación puede descomponerse como una suma de
los dos tipos anteriores. Para ello, usamos la siguiente identidad
\begin{equation}\label{identidad_u}
u_{\alpha\beta}=(u_{\alpha\beta}-\frac{1}{2}\delta_{\alpha\beta}u_{\gamma}^{\
  \gamma})+\frac{1}{2}\delta_{\alpha\beta}u_{\gamma}^{\ \gamma},
\end{equation}
el primer término del miembro de la derecha es una cizalladura pura, dado que
su traza es nula ($\delta_{\alpha}^{\ \alpha}=2$), y el segundo término es una
compresión pura. Sustituyendo esta identidad \eqref{identidad_u} en la expresión de la densidad de
energía elástica \eqref{densidad_Fe}
\begin{equation}\label{densidad_Fe_K}
f_E[u_{\alpha\beta},T]=
\mu \left(u_{\alpha\beta}-\frac{1}{2}\delta_{\alpha\beta}u_{\gamma}^{\ \gamma}\right)
    \left(u^{\alpha\beta}-\frac{1}{2}\delta^{\alpha\beta}u_{\nu}^{\ \nu}\right)+
\frac{1}{2}K(u_{\alpha}^{\ \alpha})(u_{\ \beta}^{\beta}),
\end{equation}
siendo $K$ y $\mu$ el módulo de compresión y de
cizalladura\footnote{El módulo de cizalladura $\mu$ coincide con uno de los
  coeficientes de Lamé, no es un nuevo coeficiente con igual símbolo} respectivamente. $K$
está relacionado con los coeficientes de Lamé por la siguiente igualdad
\begin{equation*}
K=\lambda+\mu.
\end{equation*}
El estado termodinámico de equilibrio del sistema viene dado por el mínimo de
$F_E$. Si no hay fuerzas externas aplicadas a la membrana, entonces $f_E$ como
función de $u_{\alpha\beta}$ deberá tener un mínimo en $u_{\alpha\beta}$. Esto
significa que la forma cuadrática \eqref{densidad_Fe_K} debe ser positiva, y por
tanto, como condición necesaria y suficiente, los coeficientes $K$ y $\mu$ deben ser positivos
\begin{equation*}
K>0\, ,\ \mu>0;
\end{equation*}
en el caso contrario podríamos encontrar que la energía libre disminuiría debido a una
comprensión pura ($K<0$) o una cizalladura pura ($\mu<0$).

El tensor de esfuerzos $\sigma_{\alpha\beta}$, según \eqref{tensor_esfuerzos}
podemos obtenerlo tomando la derivada de $f_E$ \eqref{densidad_Fe_K} respecto
a $u_{\alpha\beta}$, resultando
\begin{equation}\label{tensor_esfuerzos_K}
\sigma_{\alpha\beta}=K\delta_{\alpha\beta}u_{\gamma}^{\ \gamma}+2\mu\left(u_{\alpha\beta}-\frac{1}{2}\delta_{\alpha\beta}u_{\gamma}^{\ \gamma}\right).
\end{equation}
Está última expresión muestra que si la deformación es una compresión pura o una
cizalladura pura, la relación entre $\sigma_{\alpha\beta}$ y $u_{\alpha\beta}$
estará determinada únicamente por el módulo de compresión o por el módulo de
cizalladura respectivamente. Es posible invertir la última igualdad
\eqref{tensor_esfuerzos_K} teniendo en cuenta que
\begin{equation*}
\sigma_{\gamma}^{\ \gamma}=2 K u_{\gamma}^{\ \gamma},
\end{equation*}
con lo que
\begin{equation}\label{hooke}
u_{\alpha\beta}=\frac{1}{4K}\delta_{\alpha\beta}\sigma_{\gamma}^{\
  \gamma}+\frac{1}{2\mu}\left(\sigma_{\alpha\beta}-\frac{1}{2}\delta_{\alpha\beta}\sigma_{\gamma}^{\
    \gamma}\right).
\end{equation}
Está última expresión \eqref{hooke} indica que la deformación es proporcional
a la fuerza aplicada, que es básicamente, la ley de Hooke. 

\begin{figure}[h]
\centering
 \resizebox{\columnwidth}{!}{\input{Poisson-fig}}
\caption{Comportamiento del módulo de Poisson en función de $\mu/K$}\label{esquema_Poisson}
\end{figure} 

El módulo de Poisson $\sigma$ es el cociente entre la deformación transversal
a una tensión aplicada y la deformación longitudinal paralela a la tensión
aplicada \cite{Bowick:Libro_superficies}. Si la tensión, es aplicada por
ejemplo, en la dirección del eje $x^1$ tenemos que la expresión de $\sigma$ es
\begin{equation*}
\sigma=-\frac{\delta x^2 / x^2}{\delta x^1 / x^1}=-\frac{u_{22}}{u_{11}}.
\end{equation*}
Como la tensión es aplicada en la dirección $x^1$, la única componente no nula
del tensor de esfuerzos es $\sigma_{11}$, y las componentes $u_{11}$ y $u_{22}$
se obtienen a partir de \eqref{hooke}
\begin{align*}
u_{11}&=\frac{1}{4K}\sigma_1^{\
  1}+\frac{1}{2\mu}\left(\sigma_{11}-\frac{1}{2}\sigma_1^{\
    1}\right)=\left(\frac{1}{4K}+\frac{1}{4\mu}\right)\sigma_1^{\ 1},\\
u_{22}&=\frac{1}{4K}\sigma_1^{\ 1}-\frac{1}{2\mu}\left(\frac{1}{2}\sigma_1^{\
    1}\right)=\left(\frac{1}{4K}-\frac{1}{4\mu}\right)\sigma_1^{\ 1}.
\end{align*}
Con lo que el módulo de Poisson resulta
\begin{equation*}
\sigma=\frac{K-\mu}{K+\mu}.
\end{equation*}
Teniendo en cuenta los valores que pueden tomar $\mu$ y $K$, para asegurar la estabilidad
termodinámica, el valor módulo de Poisson estará comprendido entre $-1\leq \sigma \leq
1$. Hacia el límite superior ($\sigma\rightarrow 1$) se aproximan los materiales que tienen un módulo
de cizalladura mucho menor que el de compresión y hacia el límite inferior ($\sigma\rightarrow -1$)
aquellos que tiene un módulo de compresión despreciable frente al de
cizalladura (figura \ref{esquema_Poisson}). Claramente el módulo de Poisson
será negativo si $K<\mu$, y corresponderá a membranas que se expanden
transversalmente al ser estiradas. El módulo de Poisson $\sigma$ de una
membrana cristalina fantasma es universal \cite{Doussal:nu} y dentro de la aproximación SCSA
está dado por 
\begin{equation*}
\sigma(D)=-\frac{1}{D+1}\ \Rightarrow \ \sigma(2)=-\frac{1}{3}. 
\end{equation*}
El carácter negativo de $\sigma$ para las membranas cristalinas ha sido
comprobado mediante simulaciones numéricas \cite{Bowick_poisson_ratio} que 
 un valor de $\sigma\simeq -0.32$.
\clearpage


\subsection{Curvatura extrínseca}

El término de curvatura \eqref{ELandau_curvatura}, podemos escribirlo en el sistema de coordenadas
elegido para fase plana como\footnote{No es difícil comprobar que
  efectivamente recuperamos \eqref{ELandau_curvatura} integrando por partes
  dos veces \eqref{ELandau_curvatura_aprox}.}
\begin{equation}\label{ELandau_curvatura_aprox}
F_C[\mathbf{x},T]\simeq\frac{\kappa}{2}\int d^2\mathbf{x} 
\left(\frac{\partial h}{\partial x_{\alpha} \partial x^{\beta}}\right)
\left(\frac{\partial h}{\partial x^{\alpha} \partial x_{\beta}}\right),
\end{equation}
donde hemos supuesto deformaciones pequeñas, pudiendo entonces, despreciar
$\partial_{\alpha}\partial_{\beta} \mathbf{u}$ frente a
$\partial_{\alpha}\partial_{\beta} h$, puesto que en el desarrollo
$\mathbf{u}$ es de un orden menor que $h$. Ahora realizamos un cambio de
coordenadas, vamos a utilizar como coordenadas internas nuevas $\{\mathbf{y}\}$ la proyección
en el plano base de la posición tridimensional de los puntos
\begin{equation*}
\vec{r}(\mathbf{x})=(\mathbf{x}+\mathbf{u(\mathbf{x})},h(\mathbf{x}))\quad
\rightarrow \quad \vec{r}(\mathbf{y})=(\mathbf{y},h(\mathbf{y})),
\end{equation*}
está parametrización es denominada la parametrización de Monge, y es
importante puntualizar que con ella las coordenadas internas de los puntos no
permanecen invariantes después de la deformación, ya que puede suceder que su
proyección en el plano sea diferente antes y después de la deformación. En
estas coordenadas los vectores tangentes son
\begin{align*}
 \vec{t}_1(\mathbf{y})&\simeq\left(1,0,\frac{\partial h}{\partial y^1}\right),\\
 \vec{t}_2(\mathbf{y})&\simeq\left(0,1,\frac{\partial h}{\partial y^2}\right),
\end{align*}
y la métrica inducida
\begin{equation*}
g_{\alpha\beta}=\delta_{\alpha\beta}+\frac{\partial h}{\partial
  y^{\alpha} \partial y^{\beta}}\Rightarrow g=\det
(g_{\alpha\beta})=1+\left(\nabla_{\|} h\right)^2\, ,
\end{equation*}
donde hemos definido el operador gradiente 
\begin{equation*}
\nabla_{\|}\equiv \left(\frac{\partial }{\partial y^1},\frac{\partial }{\partial y^2}\right).
\end{equation*}
Siguiendo en la aproximación de deformaciones pequeñas, los gradientes
$\nabla_{\|} h\simeq 0$ también serán pequeños, y tenemos entonces que
\begin{align*}
g\simeq 1\quad&\Rightarrow\quad d^2\mathbf{x}\simeq d^2\mathbf{y}\\
\left(\frac{\partial h}{\partial x_{\alpha} \partial x^{\beta}}\right)
\left(\frac{\partial h}{\partial x^{\alpha} \partial x_{\beta}}\right)&\simeq \left(\frac{\partial h}{\partial y_{\alpha} \partial y^{\beta}}\right)
\left(\frac{\partial h}{\partial y^{\alpha} \partial y_{\beta}}\right) ,
\end{align*}
y podemos aproximar \eqref{ELandau_curvatura_aprox} como 
\begin{equation*}
F_C[\mathbf{y},T]\simeq\frac{\kappa}{2}\int d^2\mathbf{y} 
\left(\frac{\partial h}{\partial y_{\alpha} \partial y^{\beta}}\right)
\left(\frac{\partial h}{\partial y^{\alpha} \partial y_{\beta}}\right).
\end{equation*}
Resulta que estas coordenadas, dado que la expresión del vector normal
unitario es
\begin{equation*}
\vec{n}=\frac{\vec{t}_1(\mathbf{y})\times\vec{t}_2(\mathbf{y})}{\sqrt{g}}=\frac{1}{\sqrt{1+\left(\nabla_{\|} h\right)^2}} \left( -\frac{\partial h}{\partial y^1},-\frac{\partial h}{\partial y^2},0\right),
\end{equation*}
y usando \eqref{K_alfa_beta}, llegamos a que el tensor de curvatura extrínseca es
\begin{equation*}
K_{\alpha\beta}=\frac{1}{\sqrt{1+\left(\nabla_{\|} h\right)^2}}\frac{\partial h}{\partial
  y^{\alpha} \partial y^{\beta}}\simeq \frac{\partial h}{\partial
  y^{\alpha} \partial y^{\beta}},
\end{equation*}
donde hemos aplicado la aproximación de pequeños gradientes (deformaciones
pequeñas). Por lo tanto, la energía libre de curvatura \cite{Bowick:Membranes_review} se puede escribir como 
\begin{equation}\label{ELandau_curvatura1}
F_C[\mathbf{y},T]\simeq\frac{\kappa}{2}\int d^2\mathbf{y} K_{\alpha}^{\ \beta}K^{\alpha}_{\ \beta}.
\end{equation} 
Ahora bien, sumando y restando la traza al cuadrado $K_{\alpha}^{\
  \alpha}K_{\beta}^{\ \beta}$ en el integrando de la última expresión
\begin{equation*}
K_{\alpha}^{\ \beta}K^{\alpha}_{\ \beta}=K_{\alpha}^{\ \alpha}K_{\beta}^{\
  \beta}+K_{\alpha}^{\ \beta}K^{\alpha}_{\ \beta}-K_{\alpha}^{\ \alpha}K_{\beta}^{\ \beta}.
\end{equation*}
Usando las definiciones de curvatura media \eqref{curvatura_media} y
gaussiana \eqref{curvatura_gaussiana} llegamos a que
\begin{equation*}
K_{\alpha}^{\ \beta}K^{\alpha}_{\ \beta}=4H^2-2K,
\end{equation*} 
donde en esta aproximación
\begin{align*}
H&=\frac{1}{2}\frac{\partial h}{\partial y^{\alpha} \partial y_{\alpha}}+O\left[\left(\nabla_{\|} h\right)^2\right],\\
K&=\det \left(\frac{\partial h}{\partial y^{\alpha} \partial y_{\beta}}\right)+O\left[\left(\nabla_{\|} h\right)^2\right].
\end{align*}

La curvatura Gaussiana se puede expresar como una derivada total \cite{David:normal}
\begin{equation*}
2\det(\partial_{\alpha}\partial^{\beta}h)=
-\epsilon^{\alpha\mu}\epsilon_{\beta\nu}\partial_{\mu}\partial^{\nu}[(\partial_{\alpha}h)(\partial^{\beta}h)],
\end{equation*}
donde hemos usado la notación $\partial_{\alpha}\equiv\frac{\partial}{\partial
  y^{\alpha}}$ y $\epsilon_{\alpha\beta}$ es el símbolo de Levi-Civita
\begin{equation*}
\epsilon_{\alpha\beta}=\left(\begin{array}{cc}
0 & -1\\
1 & 0
\end{array}\right).
\end{equation*}
De hecho, el teorema de Gauss-Bonet\cite{David:geometria} demuestra que la integral de la curvatura
Gaussiana sobre la superficie es invariante bajo cualquier deformación, si la
topología de la superficie es constante. Por tanto, podemos despreciar este
término de contorno constante, con lo que obtenemos finalmente que la energía
libre de curvatura depende únicamente de la curvatura media 
\begin{equation}\label{ELandau_curvatura2}
F_C[\mathbf{y},T]\simeq\frac{1}{2}\hat{\kappa}\int d^2\mathbf{y}\, H^2 \quad
\text{con} \quad \hat{\kappa}=4\kappa.
\end{equation} 

%%% Local Variables: 
%%% mode: latex
%%% TeX-master: "TFM"
%%% End: 
