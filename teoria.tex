\section{Teoría de Landau de la transición de fase}

Algunos estados de equilibrio tienen menor simetría que la del hamiltoniano
del sistema, po rlo que se les denomina estados con ruptura de simetría. Las
diferentes fases pueden por ello clasificarse, por ello, clasificarse en
función de las simetrías rotas. Así, de una fase simétrica completamente
desordenada (uniforma e isótropa), se obtienen fases desordenadas (no
uniformaes y anisótropas), por cada elemento de simetría que se elimina.
Este método de clasificación se puede también aplicar a la transición entre
dos fases y por ello, si las fases difieren en alguna simetría, se dice que la
transición es de ruptura de simetría.

Las transiciones de ruptura de simetría pueden, en general, describirse por
medio de una parámetro de orden, que es una variable que cuantifica el grado
de orden de las fases, por lo que es habitual definirlo de forma que sea nulo
en la fase de mayor simetría.

Vamos a considerar el caso general de una membrana elástica y flexible
$D$-dimensional fluctuando en un espacio $d$-dimensional Euclídeo. Los puntos
de la membrana están interconectados y forman una red regular, en este
sentido, la membrana es cristalina. 
Las posiciones de los puntos de la red vienen dados por $\vec{r}(\mathbf{x})$,
donde $\mathbf{x}$ es un vector $D$-dimensional interno que etiqueta cada
partícula, y $\vec{r}$ es un vector de dimensión $d$ externo, que representa
la posición en el espacio euclídeo de dimensión $d$. 

Una descriptión estadística se desarrolla mediante una variable de grano
grueso, estos serán los vectores $\mathbf{x}$, que serán una variable continua
que etiquetan los bloques de los puntos de la red. 

Las simetrías del Hamiltoniano microscópico delimitan la forma del funcional
de la energía libre para la variable de grano grueso $\vec{r}(\mathbf{x})$. Para una red
uniforme, la invarianza de traslación global implica que esta funcional debe
depender de gradientes de los vectores tagentes.
$\mathbf{t}_{\alpha}=\partial \vec{r}/\partial x^{\alpha} $ , donde
$\alpha=1,2\dots D$. Inavarianza bajo rotaciones en ambos estados implica que
la forma de la energía libre debe ser:

\begin{equation}
F(\vec{r})=\int d^D\mathbf{x}
\left[
\frac{t}{2}(\partial_{\alpha}\vec{r})^2+
u(\partial_{\alpha}\vec{r}\partial_{\beta}\vec{r})^2+
v(\partial_{\alpha}\vec{r}\partial^{\alpha}\vec{r})^2+
\frac{\kappa}{2}(\partial^2\vec{r})^2
\right]
+\frac{b}{2}\int d^D\mathbf{x} d^D\mathbf{x'}
\delta^{d}(\vec{r}(\mathbf{x})-\vec{r}(\mathbf{x'}))
\end{equation}

Donde los términos de orden superior son irrelevantees en el límite de
$\lambda\rightarrow 0$. La física de está ecuación depende de el módulo
elástico $t$, $u$ y $v$, la rígidez de curvatura $\kappa$. El factor $b$
controla la autointersección , de modo que si es nulo, los puntos puede
intersectarse a sí mismo sin coste energético.

Para $b=0$ es posible la autointersección y la membrana es llamada fantasma.

Pequeñas deformaciones desde un estado de referencia se pueden parametrizar
como:

\begin{equation}
\vec{r}(\mathbf{x})=(\zeta x^{\gamma}+u^{\gamma},h)
\end{equation}

Donde $u^{\alpha}$ son los modos fonocnicos internos y h son las fluctuaciones
fuera del plano.

$\zeta$ no nula describe una membrana plana con pequeñas fluctuaciones.

\subsection{Membrana fantasma}

Podemos usar el ansatz en campo medio (no hay fluctuaciones) en donde:

\begin{align}
(\partial_{\alpha} \vec{r})^2&= \xi^2 \delta_{\alpha}^{\
  \gamma}\delta_{\gamma}^{\ \alpha}=D\xi^2 \\
(\partial_{\alpha} \vec{r}\partial_{\beta} \vec{r} )^2&=\xi^4 \delta_{\alpha}^{\
  \gamma}\delta_{\beta}^{\ \rho}\delta_{\gamma}^{\ \alpha}\delta_{\rho}^{\
  \beta}=D^2\xi^4\\
(\partial_{\alpha} \vec{r}\partial^{\alpha} \vec{r} )^2&=D\xi^4 \delta_{\alpha}^{\
  \gamma}\delta^{\ \alpha}_{\gamma}\delta_{\gamma}^{\
  \alpha}\delta_{\alpha}^{\ \gamma}=D^3\xi^4\\
(\partial^2 \vec{r})^2&=0
\end{align}

Sustituyendo:
\begin{align}
F^{MF}(\xi)&=\int d^D \mathbf{x} \left[ \frac{t}{2} D\xi^2 + u D\xi^4+vD^4\xi^4\right]\\
          &=Af^{MF}(\xi)
\end{align}

Donde:
\begin{align}
A&=\int d^D\mathbf{x}\\
f^{MF}(\xi)&=D\xi^2\left[ \frac{t}{2} D\xi^2 + u D^2\xi^4 + v D^2\xi^4\right] \\
\end{align}
 $A$ es la superficie total de la membrana y $f^{MF}$ la densidad de energia
 libre por unidad de superficie.


El valor mínimo de $f^{MF}(\xi_0)$ es:

Como el parámetro de orden se anula de forma continua en la transición, a este
tipo de transiciones de fase con ruptura de simetría se les denomina, de forma
genérica, continuas.

Ehrenfest introdujo una clasificación de la transiciones de fase basandose en
la continuidad del potencial termodinámico. Según Ehrenfest una transición de
fase es de orden $n$ si la primera derivada del potencial termodinámico que no
es continua es la derivada $n$-ésima. Esta clasificación no siempre es
equivalente a la clasificación basada en la ruptura de simetría.
 
%%% Local Variables: 
%%% mode: latex
%%% TeX-master: "TFM"
%%% End: 
