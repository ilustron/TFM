\chapter{Estudio analítico}

\section{Descripción geométrica de la membranas cristalina}

Consideraremos una membrana como una superficie elástica y flexible 
bidimensional alojada en un espacio tridimensional euclídeo. 
Se pueden interpretar las interacciones de los puntos de la superficie como
conexiones, es decir, dos puntos que están interaccionando entre sí han
establecido una conexión entre ambos. Una membrana cristalina corresponde al
tipo de superficie en la que las conexiones entre sus puntos forman una red
regular cuya conectividad permanece constante. Esta última afirmación
significa que si existe una conexión entre dos puntos, está persistirá sea
cual sea la configuración global de la superficie.
% Podremos entonces,encontrar estados de la membrana en los que puntos que no
% interaccionan entre sí estén mucho menos distanciados que los puntos que sí
% están interaccionando. 

Usaremos una descripción continua de la membrana cristalina en donde la
posición en el espacio tridimensional de sus puntos estará descrita por sus
radio vectores $\vec{r}(\mathbf{x})$, siendo $\mathbf{x}$ la parametrización
de la superficie. Debido al carácter de conectividad constante, podemos
parametrizar la superficie de forma que $\mathbf{x}$ sea un índice continuo
que etiquete cada partícula. Por tanto, para la descripción geométrica de la
membrana cristalina necesitamos dos sistemas de coordenadas:

\begin{figure}[h]
\centering
\subfigure[Coordenadas internas]{
 \resizebox{160bp}{!}{\input{coordenadas_internas-fig}}}
\quad
\subfigure[Coordenadas internas]{
 \resizebox{160bp}{!}{\input{coordenadas_externas-fig}}}
\caption{Coordenadas }
\end{figure}

\begin{description}
\item[Coordenadas internas:] Etiquetan cada partícula, de modo que, si la
  membrana sufre una deformación modificándose las posiciones de las
  partículas estas coordenadas permanecen constantes, es decir, las
  coordenadas internas de un punto de la membrana serán las mismas antes y
  después de la deformación. 
  \begin{equation*}
  \mathbf{x}\equiv (x^1,x^2)\in \mathbb{R}^2
  \end{equation*}
\item[Coordenadas externas:] Describen las posiciones de cada partícula,
  varían para cada configuración. Utilizaremos el sistema de referencia usual
  cartesiano ortonormal.
  \begin{equation*}
    \vec{r}=x\,\vec{i}+y\,\vec{j}+z\,\vec{k}\equiv (x,y,z)\in \mathbb{R}^3
  \end{equation*}
\end{description}

\subsection{Vectores tangentes}

En cada punto $P\equiv\vec{r}(\mathbf{x})$ tenemos un plano tangente $T_PS$ a
la superficie cuya base son los siguientes vectores tangentes:

\begin{equation*}
\vec{t}_{\alpha}=\frac{\partial \vec{r}}{\partial
  x^{\alpha}}=\partial_{\alpha}\vec{r} \qquad \alpha=1,2
\end{equation*}

\begin{figure}[h]
\centering
 \input{vectores_tangentes-fig}
\caption{Vectores tangentes en el punto $\vec{r}(\mathbf{x})$}
\end{figure}

Cualquier vector $\vec{V}$ tangente a la superficie en el punto $P$ se podrá
descomponer como:
\begin{equation}\label{ejemplo_vector} 
  \vec{V}=\sum^2_{\alpha=1}V^{\alpha}\vec{t}_{\alpha}=V^{\alpha}\vec{t}_{\alpha}
\end{equation}
con el convenio de índices repetidos de Einstein. Las dos cantidades
$V^{\alpha}$ son las componentes del vector $\vec{V}$ en el sistema de
coordenadas internas $\mathbf{x}$ y dependen de la elección del sistema de
coordenadas, esto es, de la parametrización de la superficie. Si consideramos
un nuevo sistema de coordenadas  
\begin{equation*}
 \mathbf{x}'=\{x'^{\alpha}(\mathbf{x}); \alpha=1,2\}
\end{equation*}
La base de los vectores tangentes en $\mathbf{x}'$ es
\begin{equation*}
\vec{t}_{\alpha}^{\ '}=\frac{\partial \vec{r}}{\partial x'^{\alpha}}=
\frac{\partial x^{\beta}}{\partial x'^{\alpha}}\frac{\partial \vec{r}}{\partial x^{\beta}}=\frac{\partial x^{\beta}}{\partial x'^{\alpha}}\vec{t}_{\alpha}
\end{equation*}
Y las componentes $V'^{\alpha}$ del vector \eqref{ejemplo_vector}
\begin{equation} 
  \vec{V}=V'^{\alpha}\vec{t}_{\alpha}^{\
    '}=V^{\alpha}\vec{t}_{\alpha}\Rightarrow V'^{\alpha}=\frac{\partial x'^{\alpha}}{\partial x^{\beta}}V^{\beta}
\end{equation}

Tanto $\vec{t}_{\alpha}$ como $V^{\alpha}$ son casos particulares de objetos
más generales, los tensores. Definimos un tensor $T$ de rango $m$
contravariante y rango $n$ covariante como un objeto cuyas $2^{m+n}$ componentes
bajo un cambio de coordenadas $\mathbf{x}\rightarrow\mathbf{x'}$ transforman
como 
\begin{equation*}
T'^{\alpha_1,\dots,\alpha_m}_{\beta_1,\dots,\beta_m}=
\frac{\partial x'^{\alpha_1}}{\partial x^{\beta_1}}\dots \frac{\partial
  x'^{\alpha_m}}{\partial x^{\beta_m}}
\frac{\partial x^{\beta_1}}{\partial x'^{\alpha_1}}\dots \frac{\partial
  x^{\beta_n}}{\partial x'^{\alpha_n}}T^{\alpha_1,\dots,\alpha_n}_{\beta_1,\dots,\beta_m}
\end{equation*}

Entonces $V^{\alpha}$ son las componentes de un tensor de rango $1$
contravariante, que llamaremos vector contravariante. Análogamente,
llamaremos a un tensor de rango $1$ covariante vector covariante.
 
\subsection{Métrica inducida}

La distancia euclídea $dl$ entre dos puntos de coordenadas internas
$\mathbf{x}$ y $\mathbf{x}+d\mathbf{x}$ respectivamente es:

\begin{equation}\label{elemento_linea}
dl^2=(\vec{r}(\mathbf{x}+d\mathbf{x})-\vec{r}(\mathbf{x}))^2=g_{\alpha\beta}dx^{\alpha}dx^{\beta}
\end{equation}

Donde $g_{\alpha\beta}=\vec{t}_{\alpha}\cdot\vec{t}_{\beta}$ es un tensor, que
denominaremos métrica inducida, que nos permite definir el producto escalar
entre dos vectores tangentes $\vec{V}$ y $\vec{W}$ en el punto $P$:
\begin{equation}\label{producto_escalar}
\vec{V}\cdot\vec{W}=g_{\alpha\beta}V^{\alpha}W^{\beta}
\end{equation}
Diremos que la base $\{ \vec{t}_{\alpha}, \alpha=1,2\}$ es ortogonal y
normalizada (en coordenadas cartesianas corresponde a una base ortonormal) si
\begin{equation*}
\vec{t}_{\alpha}\cdot\vec{t}_{\beta}=\delta_{\alpha\beta}
\end{equation*}
donde es la métrica euclídea
\begin{equation*}
\delta_{\alpha\beta}\equiv\left(\begin{array}{cc}
1&0\\
0&1\\
\end{array}\right)
\end{equation*}
La métrica inducida inversa $g^{\alpha\beta}$ se define como
\begin{equation}\label{expresion_metrica}
g_{\alpha\gamma}g^{\gamma\beta}=\delta_{\alpha}^{\ \beta}=\begin{cases}
1&\text{si $\alpha=\beta$}\\
0&\text{si $\alpha\neq\beta$}.
\end{cases}
\end{equation}

Gracias a esta última expresión \eqref{expresion_metrica} podemos hacer
corresponder los vectores contravariantes con los covariantes, que en la
notación usada equivale a subir o bajar índices:
\begin{equation*}
V_{\alpha}=g_{\alpha\beta}V^{\beta}\rightarrow V^{\alpha}=g^{\alpha\beta}V_{\beta} 
\end{equation*}
Y podemos expresar el producto escalar de los vectores
\eqref{producto_escalar} 
\begin{equation*}
\vec{V}\cdot\vec{W}=V_{\alpha}W^{\beta}=V^{\alpha}W_{\beta}
\end{equation*}
El elemento de infinitesimal de área $d^2\mathbf{s}$ se relaciona con el
determinante $g$ de la métrica inducida por

\begin{equation}\label{elemento_area}
d^2\mathbf{s}=\sqrt{g}\;d^2\mathbf{x}
\end{equation}
donde $d^2\mathbf{x}=dx^1dx^2$. Se puede demostrar esta expresión
\eqref{elemento_area} teniendo en cuenta que
\begin{equation*}
d^2\mathbf{s}=|\vec{t}_1\times\vec{t}_2|\;d^2\mathbf{x}
\end{equation*}
y
\begin{multline*}
|\vec{t}_1\times\vec{t}_2|^2=|\vec{t}_1|^2|\vec{t}_2|^2\sen^2
\theta=g_{11}g_{22}(1-\cos^2 \theta)=\\
g_{11}g_{22}-(\vec{t}_1\cdot\vec{t}_2)^2=g_{11}g_{22}-g_{12}g_{21}=g
\end{multline*}

\section{Teoría de Landau de la transición de fase}

En una membrana se pueden encontrar dos fases: La fase plana y la fase
rugosa, que se distinguen por sus simetrías. A la temperatura crítica $T_c$ se
produce la transición entre estas dos fases. Sobre la temperatura crítica la
membrana es invariante frente a rotaciones, 
la membrana se encuentra en la fase rugosa. En cambio, bajo la temperatura
crítica, se encuentra en la fase plana donde el promedio del vector normal a
la superficie $\langle \vec{n}\rangle$ define una dirección privilegiada en el
espacio, destruyendo la invariancia frente a rotaciones cuyo eje es
perpendicular a $\langle \vec{n}\rangle$. Los estados de equilibrio que se
encuentran en esta última fase, al tener una menor simetría que la del
hamiltoniano del sistema, se denominan estados con ruptura espontánea de la
simetría. 
 
Debido a la reducción de la simetría, es necesario un nuevo parámetro que 
describa la termodinámica de la fase ordenada. Este parámetro
extra, denominado parámetro de orden y que denotaremos por $\eta(\mathbf{x})$,
es una 
variable que cuantifica el grado de orden de
las fases, por lo que usualmente se define de forma que sea nulo en la fase
de mayor simetría.

\begin{figure}[h]
\centering
\subfigure[Fase plana]{
 \resizebox{160bp}{!}{\input{fase_plana-fig}}}
\quad
\subfigure[Fase rugosa]{
 \resizebox{160bp}{!}{\input{fase_arrugada-fig}}}
\end{figure}

La teoría de Landau de la transición de fase es una aproximación
fenomenológica que evita tratar con la descripción microscópica del sistema
mediante el uso de una descripción de \textit{grano grueso}. Esto significa,
en el caso de membranas cristalinas, que usaremos como coordenadas internas
$\mathbf{x}_{\Lambda}$, las cuales seguirán  siendo índices continuos excepto
que etiquetan \textit{bloques} de partículas y no partículas
individuales. Estos bloques tendrán una extensión lineal $\Lambda$ y estarán
centrados en $\mathbf{x}$. El correspondiente parámetro de orden de grano
grueso $\bar{\eta}(\mathbf{x}_{\Lambda})$ corresponderá a un promedio sobre
todas las partículas que forman el bloque.

En una descripción microscópica y continua, la función de partición
$\mathcal{Z}[T]$ de la membrana cristalina vendrá dada por

\begin{equation}\label{Zmicroscopica}
\mathcal{Z}[T]=\int D[\eta(\mathbf{x})]\; e^{-\beta H[\eta(\mathbf{x})]},
\end{equation}
donde con $\int D[\eta(\mathbf{x})]$ se denota la integral funcional sobre
todas las configuraciones posibles, $H[\eta(\mathbf{x})]$ es el hamiltoniano
microscópico y $\beta=\frac{1}{k_BT}$ es el factor de Boltzmann. Se define la
energía libre de Landau 
$F[\bar{\eta}(\mathbf{x}_{\Lambda}),T]$ como 
\begin{equation*}
e^{-F[\bar{\eta}(\mathbf{x}_{\Lambda}),T]}=\int D[\eta(\mathbf{x})]\; e^{-\beta
  H[\eta(\mathbf{x}),T]}
  \;\delta\left[\frac{1}{A_{\Lambda}}\int_{\Lambda} d^2\mathbf{x}\; \eta(\mathbf{x})-\bar{\eta}(\mathbf{x}_{\Lambda})\right]
\end{equation*}

Siendo $\delta$ la función delta de Dirac y $A_{\Lambda}$ es el área
formada por las partículas del bloque. Esta energía libre de Landau se puede
interpretar como un potencial efectivo, resultado de la integración de los
grados de libertad microscópicos sujetos a la condición de que sus promedios
en los  bloques coincida con el parámetro de orden
$\eta(\mathbf{x}_{\Lambda})$ de grano grueso. La función de partición
\eqref{Zmicroscopica} se puede expresar en función de la energía libre de
Landau

\begin{equation*}
\mathcal{Z}[T]=\int D[\bar{\eta}(\mathbf{x}_{\Lambda})]\; e^{-F[\bar{\eta}(\mathbf{x}_{\Lambda}),T]}
\end{equation*}

Esta expresión nos muestra que $F[\bar{\eta}(\mathbf{x}),T]$ es una funcional
que representa un hamiltoniano efectivo dividido por $k_BT$. A su vez, podemos
escribir 

\begin{equation*}
F[\bar{\eta}(\mathbf{x}),T]=\int\! d^2\mathbf{s}\ f[\bar{\eta}(\mathbf{x}),T]
\end{equation*}

Siendo $d^2\mathbf{x}$ el elemento de área y $f[\bar{\eta}(\mathbf{x}),T]$ la
densidad de energía libre de Landau. Puesto que cerca de la transición de fase
el parámetro de orden $\bar{\eta}(\mathbf{x})$ toma valores pequeño, podemos
desarrollar $f[\bar{\eta}(\mathbf{x}),T]$ en potencias de
$\bar{\eta}(\mathbf{x})$ y sus derivadas, sin olvidar que los factores que
acompañan a los términos del desarrollo, en principio, dependerán de la
temperatura. Al igual que el hamiltoniano microscópico, podemos exigir que
$F[\bar{\eta}(\mathbf{x}_{\Lambda}),T]$ cumpla con las simetrías del sistema.

La simetría traslacional de la membrana implica que su energía libre de Landau
sólo puede depender de las derivadas de $\vec{r}(\mathbf{x}_{\Lambda})$, estos
son, los vectores tangentes $\vec{t}_{\alpha}$. En la fase
plana, los promedios de los vectores tangentes en un punto son no nulos, en
cambio, en la fase rugosa, su promedio es nulo por isotropía. Por tanto,
podemos utilizar como  parámetro de orden de grano grueso los vectores
tangentes locales $\vec{\bar{t}}_{\alpha}(\mathbf{x}_{\Lambda})$, que serán
los promedios de los vectores tangentes de las partículas en la vecindad
$\Lambda$ del punto $\vec{r}(\mathbf{x})$. Entonces, en el caso de
membranas cristalina, tenemos un parámetro de orden con carácter tensorial,
con seis grados de libertad en cada punto.

La justificación de la descripción de grano grueso la encontramos en el
comportamiento de las correlaciones espaciales $\xi$, longitud a la que las
posiciones de las partículas se encuentran correlacionadas, éstas crecen a
medida que la temperatura del sistema se acerca a la temperatura crítica, lo
que significa que podemos encontrar regiones de dimensión lineal
$\Lambda<\xi(T)$ donde los vectores tangentes son aproximadamente
constantes. Además, hay que distinguir que, mientras que tratamos a
$\mathbf{x}_{\Lambda}$ como una variable continua, la función 
$\vec{\bar{t}}_{\alpha}(\mathbf{x}_{\Lambda})$ no presenta variaciones a distancias
del orden del espaciado microscópico $a$, o equivalentemente, su transformada
de Fourier tiene vectores de onda con magnitud menor que cierto \textit{cut-off}
$\Lambda^{-1} \sim 1/a$. Concluimos entonces que el tamaño del bloque en la
descripción de grano grueso deberá cumplir que $a<<\Lambda<\xi(T)$. En lo que
sigue, siempre se utilizará está descripción de grano grueso, mientras no se
exprese lo contrario, por lo que omitiremos el subíndice $\Lambda$ y las
barras de las variables para no recargar sin necesidad la notación.

Vamos a proponer una expresión de energía libre de Landau
$F[\vec{t}_{\alpha}(\mathbf{x}),T]$ para una membrana cristalina
apoyándonos en las siguientes propiedades:
\begin{description}
\item[Localidad e invarianza traslacional]: Dependerá de los vectores
  tangentes locales $\vec{t}_{\alpha}(\mathbf{x})$ y de las
  interacciones de corto alcance descritas a través del desarrollo en gradientes
  $$ F[\vec{t}_{\alpha}(\mathbf{x}),T]=\int\! d^2\mathbf{s}\ f[\vec{t}_{\alpha},\nabla
  \vec{t}_{\alpha},\dots]$$ 
\item[Simetría rotacional en $\mathbb{R}^3$]: Está simetría implica que sólo
  puede depender de productos escalares entre los vectores tangentes
  locales y sus gradientes con un número par de términos. 

\item[Simetría traslacional y rotacional en $\mathbb{R}^2$]: Lo que implica
  que los productos escalares serán con la métrica inducida $g_{\alpha\beta}$,
  de está forma, $F[\vec{t}_{\alpha}(\mathbf{x}),T]$ serà invariante frente a
  reparametrizaciones de la superficie, esto es, frente a cambios de 
  las coordenadas $\mathbf{x}$.
\end{description}

 Finalmente, con esta última propiedad podemos proponer la siguiente energía libre de Landau $F$:
  \begin{multline}\label{ELandau}
    F[\vec{t}_{\alpha}(\mathbf{x}),T]=\\
\int d^2\mathbf{s}
    \left[
      \frac{t}{2}(\vec{t}_{\alpha}\cdot\vec{t}^{\alpha})+
      u(\vec{t}_{\alpha}\cdot\vec{t}_{\beta})(\vec{t}^{\alpha}\cdot\vec{t}^{\beta})+
      v(\vec{t}_{\alpha}\cdot\vec{t}^{\alpha})(\vec{t}_{\alpha}\cdot\vec{t}^{\alpha})+
      \frac{\kappa}{2}(\partial_{\alpha}\vec{t}_{\alpha})\cdot(\partial^{\alpha}\vec{t}^{\alpha}) \right]
  \end{multline}


Esta forma funcional únicamente se ha deducido a partir de las simetrías del
sistema, el precio que hay que pagar por esta aproximación es que en su
expresión tenemos los parámetros fenomenológicos $t$, $u$, $v$ y $\kappa$ cuya
dependencia funcional con los parámetros microscópicos originales no es
desconocida, así como con la temperatura.


La energía libre \eqref{ELandau}, al no haber considerado
interacciones de largo alcance en su deducción, permite la membrana puede
cruzarse libremente a sí misma. Para evitar que el sistema acceda
configuraciones incompatibles con la estructura cristalina subyacente, se
puede incluir el siguiente término de exclusión de volumen

\begin{equation*}
\frac{b}{2}\int d^2\mathbf{x} d^2\mathbf{x'}
\delta^2(\vec{r}(\mathbf{x})-\vec{r}(\mathbf{x'}))
\end{equation*}

El parámetro $b$ modula este mecanismo, de forma que si es nulo, la superficie
puede cruzarse a sí misma sin coste energético. Las membranas con $b=0$ se
denominan \textit{membranas fantasmas}, el tratamiento matemático de éstas es
 más sencillo, y serán el objeto de estudio de este presente
trabajo, puesto que únicamente estamos interesados en las propiedades de la
fase plana y la transición de fase, en donde no influye de forma apreciable
este efecto indeseable. 

\section{Aproximación de Campo medio}

\begin{figure}[h]
\centering
 \resizebox{\columnwidth}{!}{\input{campo_medio-fig}}
\caption{Parametrización usada en la aproximación del campo medio}
\end{figure}

La energía libre \eqref{ELandau}, debido a las simetrías que le hemos exigido
que cumpla, describe una membrana isótropa, en la que las interacciones no
dependen de la dirección. En consecuencia, podemos afirmar que en la fase
plana los promedios de las posiciones tridimensionales de las partículas
$\langle\vec{r}(\mathbf{x})\rangle_{p}$ formarán una red regular plana:  
\begin{equation*}
 \langle\vec{r}(\mathbf{x})\rangle_{p}=(\zeta x^1,\zeta x^2,0)=(\zeta \mathbf{x},0), 
\end{equation*}
en donde hemos orientado y situado el sistema de referencia de las coordenadas
externas, de forma que, el plano $xy$ coincida con el de las
posiciones promedio. Las coordenadas internas utilizadas son 
proporcionales a la proyección en el plano $xy$ de las posiciones. El factor
de proporcionalidad $\zeta$, es un nuevo parámetro, cuya magnitud coincide con
la norma del promedio de  vectores tangentes 
\begin{equation}
 \left.\begin{array}{c}
\langle\vec{t}_{1}(\mathbf{x})\rangle_{p}=\langle\partial_1\vec{r}\rangle_{p}=(\zeta ,0,0)\\
\langle\vec{t}_{2}(\mathbf{x})\rangle_{p}=\langle\partial_2\vec{r}\rangle_{p}=(0,\zeta ,0)\\
 \end{array}\right\}\Rightarrow
\langle\vec{t}_{\alpha}(\mathbf{x})\rangle_{p}\cdot\langle\vec{t}_{\beta}(\mathbf{x})\rangle_{p}=\zeta^2 \delta_{\alpha\beta},
\end{equation}
de donde, en efecto, deducimos que

\begin{equation*}
|\langle\vec{t}_1(\mathbf{x})\rangle_{p}|=|\langle\vec{t}_2(\mathbf{x})\rangle_{p}|=\zeta
\end{equation*}

El que la norma de ambos vectores tangentes sea igual al parámetro $\zeta$ , es
una consecuencia directa de la isotropía de la membrana. La longitud de los
vectores tangentes están relacionados la distancia relativa entre los puntos
de la superficie, $\zeta$ cuantifica entonces el espaciado promedio entre
puntos y nos da una idea del grado de extensión de la membrana.

La aproximación de campo medio asume que se pueden despreciar las
fluctuaciones, y por tanto, podemos aproximar los radio vectores de las
partículas y lo vectores tangentes por sus promedios
\begin{align*}
 \vec{r}(\mathbf{x})&\simeq(\zeta \mathbf{x},0)\\
 \vec{t}_1(\mathbf{x})&\simeq(\zeta ,0,0)\\
 \vec{t}_2(\mathbf{x})&\simeq(0,\zeta,0)
\end{align*}
y dependencia espacial de los vectores tangentes es eliminada. La
aproximación del campo medio sólo considera estados homogéneos, además, que
el carácter tensorial del parámetro de orden, sus seis grados de libertad, se
reducen a un solo parámetro $\zeta$, que será el parámetro de orden en esta
aproximación.

La expresión del potencial efectivo \eqref{ELandau} en este caso será:
\begin{align}
  F_C(\zeta,T)=&\int\! d^2 \mathbf{s}\ 2\zeta^2\!\left( \frac{t}{2} + (u+2v)\zeta^2\right)\\
  =&Af_C(\zeta)
\end{align}
donde 
$$A=\int d^2 \mathbf{s}=\zeta^2\int d^2 \mathbf{x}$$
es el área total de la membrana y 
\begin{equation}\label{densidadELandau}
f_C(\zeta,T)=2\zeta^2\!\left( \frac{t}{2} + (u+2v)\zeta^2\right)
\end{equation}
es la densidad de energía libre de Landau en la aproximación del campo medio. 
La función de partición corresponde a la siguiente integral
\begin{equation}\label{Zcm}
 \mathcal{Z}[T]=\int d\zeta\;e^{-Af_C(\zeta,T)}
\end{equation}
Ahora bien, si $A$ es muy grande, únicamente contribuyen los valores menores de
$f_C(\zeta)$, y si además, posee un mínimo con respecto a $\zeta$ en
$\zeta_0$, la exponencial del integrando de \eqref{Zcm} tendrá un máximo,
tanto más acusado cuanto mayor sea el área $A$. Desarrollando $f_C(\zeta)$ en
serie alrededor del mínimo $\zeta_0$
\begin{equation*}
f_C(\zeta)\simeq f_C(\zeta_0)+\frac{1}{2}\left(\frac{\partial f_C}{\partial \zeta}\right)_{\!\zeta_0}(\zeta-\zeta_0)^2
\end{equation*}
y sustituyendo en la expresión anterior \eqref{Zcm} de $\mathcal{Z}[T]$ 
\begin{equation}\label{Zcm2}
 \mathcal{Z}[T]\simeq\int
 d\zeta\;e^{-Af_C(\zeta_0)-\frac{A}{2}\ddot{f}_C(\zeta_0)(\zeta-\zeta_0)^2}=
 \sqrt{\frac{2\pi}{A\ddot{f}_C(\zeta_0)}}\; e^{-Af_C(\zeta_0)}
\end{equation}

La energía libre de Hemholtz intensiva $f_H(T)$ en el límite termodinámico
viene dada por el siguiente límite
\begin{equation*}
\lim_{A\rightarrow \infty}\; \frac{1}{A} \ln \mathcal{Z}[T]=-\beta f_H(T).
\end{equation*}
Si tomamos este límite en la expresión \eqref{Zcm2} de $\mathcal{Z}[T]$
encontramos
\begin{equation*}
\lim_{A\rightarrow \infty}\; \frac{1}{A} \ln \mathcal{Z}[T]=-f_C(\zeta_0),
\end{equation*}
es decir, $f_C(\zeta_0)=\beta f_H(T)$, lo que significa que los estados de
equilibrio del sistema vienen determinados por el mínimo de la energía libre
de Landau. Hallando el mínimo de \eqref{densidadELandau} 
\begin{equation*}
\left(\frac{\partial f_C}{\partial \zeta}\right)_{\!\zeta=\zeta_0}\!=0 \;
\Rightarrow \; 4\zeta_0\left(\frac{t}{2}+2(u+2v)\zeta_0^2\right)=0
\end{equation*}
encontramos que el comportamiento de $f$ depende del signo del parámetro $t$:
\begin{figure}[h]
\centering
 \resizebox{\columnwidth}{!}{\input{energia_libre_CM-fig}}
\caption{Campo medio}
\end{figure} 

\begin{enumerate}
\item Para $t>0$ el mínimo de $F$ ocurre en $\zeta_0=0$, esto indica que la
  simetría es total, la membrana se encuentra en la fase desordenada,
  arrugada.
\item Para $t<0$, la función $f_C$ tiene mínimo doblemente
  degenerado, correspondiente a los valores
  \begin{equation}\label{zeta_0}
    \zeta_0=\mp \sqrt{\frac{-t}{4(u+2v)}}
  \end{equation}
  que muestra que la membrana se encuentra en un estado con ruptura espontánea
  de la simetría indicando una fase ordenada, la fase plana. 
\end{enumerate}

La evaluación del punto mínimo sugiere entonces que el
signo del parámetro $t$ determina en cual de las dos fase se encuentra el
sistema. Para encontrar el significado concreto del parámetro $t$
realizamos un desarrollo en serie en función de la temperatura, centrado en la
temperatura crítica $T_c$, de los parámetros que intervienen en la densidad
de energía libre:

\begin{align*}
t=&\; t_0+t_1\frac{(T-T_c)}{T_c}+O(T-T_c)^2,\\
u=&\; u_0+u_1\frac{(T-T_c)}{T_c}+O(T-T_c)^2,\\
v=&\; v_0+v_1\frac{(T-T_c)}{T_c}+O(T-T_c)^2.
\end{align*}
Podemos tomar a $u$ y $v$ como constantes, ya que su dependencia en la 
temperatura no contribuye en los primeros órdenes en comportamiento
termodinámico cerca de $T_c$. Ahora bien,
para cualquier valor de $T\!<\!T_c$ sabemos que el parámetro de orden es no
nulo, esto es, $\zeta_0\!\neq\! 0$, lo que implica que $t_0=0$. Por otra
parte, siempre podremos escalar el parámetro de orden 
$\zeta$, de manera que $t_1=1$, con lo que tenemos finalmente que
\begin{equation*}
t=\frac{(T-T_c)}{T_c}+O(T-T_c)^2,
\end{equation*}
el parámetro $t$ es la temperatura reducida adimensional.

%Como el parámetro de orden se anula de forma continua en la transición, a este
%tipo de transiciones de fase con ruptura de simetría se les denomina, de forma
%genérica, continuas.

%Ehrenfest introdujo una clasificación de la transiciones de fase basandose en
%la continuidad del potencial termodinámico. Según Ehrenfest una transición dee
%fase es de orden $n$ si la primera derivada del potencial termodinámico que no
%es continua es la derivada $n$-ésima. Esta clasificación no siempre es
%equivalente a la clasificación basada en la ruptura de simetría.
 
\section{Estudio de la fase plana}

Hemos visto que en la fase plana las posiciones medias de la partículas forman
un plano. Para el estudio de esta fase vamos a considerar pequeñas
deformaciones desde este plano, que utilizaremos como estado de referencia:

\begin{equation}\label{r_fase_plana}
\vec{r}(\mathbf{x})=(\mathbf{x}+\mathbf{u(\mathbf{x})},h(\mathbf{x}))
\end{equation}

Donde estamos utilizando las mismas coordenadas que en TAL , excepto que en
este caso tomamos $\zeta=1$, esto es, las coordenadas internas coinciden con
la proyección en el plano $xy$ de las posiciones promedios, que llamaremos
plano de referencia, de esta forma, conseguimos que la métrica inducida en
este plano sea diagonal y unitaria
$g^{(0)}_{\alpha\beta}=\delta_{\alpha\beta}$. Los términos
$\mathbf{u(\mathbf{x})}$ y $h(\mathbf{x})$ 
representa las fluctuaciones paralelas y transversales respecto al plano de
referencia respectivamente. La expresión \eqref{r_fase_plana} se puede
expresar de la siguiente forma equivalente
\begin{equation}\label{deformacion}
\vec{r}(\mathbf{x})=\vec{r}_0(\mathbf{x})+\vec{u}(\mathbf{x})
\end{equation}

Siendo $\vec{r}_0(\mathbf{x})=\langle\vec{r}(\mathbf{x}) \rangle_p$ el radio
vector que determina las posiciones en el plano de referencia, es igual a las
posiciones medias, y $\vec{u}(\mathbf{x})$ es el vector desplazamiento, que
mide las diferencias de las posiciones respecto al plano de referencia. Puesto
que estamos estudiando una membrana isótropa, tenemos que
$\langle\vec{u}(\mathbf{x}) \rangle=(0,0,0)$, es nulo el valor medio del
vector desplazamiento.

Atendiendo a las propiedades que caracteriza, la energía libre de Landau
\eqref{ELandau} se puede expresar como la suma de dos términos:

\begin{equation}
 F[\vec{t}_{\alpha}(\mathbf{x}),T]= F_E[\vec{t}_{\alpha}(\mathbf{x}),T]+F_C[\vec{t}_{\alpha}(\mathbf{x}),T]
\end{equation}

Donde:
\begin{description}
\item[ $F_E\rightarrow $ Energía libre elástica:] Caracteriza las propiedades elásticas de la membrana.
  \begin{multline}
    F_E[\vec{t}_{\alpha}(\mathbf{x}),T]=\\\int d^2\mathbf{s}
    \left[
      \frac{t}{2}(\vec{t}_{\alpha}\cdot\vec{t}^{\alpha})+
      u(\vec{t}_{\alpha}\cdot\vec{t}_{\beta})(\vec{t}^{\alpha}\cdot\vec{t}^{\beta})+
      v(\vec{t}_{\alpha}\cdot\vec{t}^{\alpha})(\vec{t}_{\alpha}\cdot\vec{t}^{\alpha})\right]
  \end{multline}
\item[ $F_C\rightarrow $ Energía libre de curvatura:] Caracteriza la curvatura
  de la membrana. 
  \begin{equation}
       F_C[\vec{t}_{\alpha}(\mathbf{x}),T]=\int d^2\mathbf{s}\; 
      \frac{\kappa}{2}(\partial_{\alpha}\vec{t}_{\alpha})\cdot(\partial^{\alpha}\vec{t}^{\alpha})
  \end{equation}
\end{description}

Estudiaremos ambos términos para el caso particular de la membrana cristalina
en las siguiente secciones

\subsection{Tensor de deformaciones}

Supongamos que la membrana se encuentra en una determinada configuración
$\vec{r}(\mathbf{x})$ de la fase plana. En vista de la expresión
\eqref{deformacion}, podemos interpretar que esta configuración es el
resultado de una deformación de la membrana a partir de la configuración
$\vec{r}_0(\mathbf{x})$ completamente plana. Las distancias relativas de los
puntos, en general, cambiarán al producirse la deformación. Si
consideramos dos puntos muy próximos $P$ y $Q$ de coordenadas internas
$\mathbf{x}$ y $\mathbf{x}+d\mathbf{x}$, respectivamente. El vector tridimensional que los une
después de la deformación será
\begin{equation}\label{dr}
d\vec{r}= \vec{r}(\mathbf{x}+d\mathbf{x})-\vec{r}(\mathbf{x})
\end{equation}
De igual forma podemos encontrar el vector correspondiente a la configuración
completamente plana, o equivalentemente, anterior a la deformación
\begin{equation}\label{dr0}
d\vec{r}_0= \vec{r}_0(\mathbf{x}+d\mathbf{x})-\vec{r}_0(\mathbf{x})
\end{equation}

Una forma de caracterizar la deformación consiste en especificar la variación
de las distancias relativas entre puntos próximos. Estas distancias se
corresponden con la diferencia entre la norma al cuadrado de estos dos vectores
\eqref{dr}\eqref{dr0}. Es razonable suponer que esta diferencia será bilineal
en los diferenciales de $dx^{\alpha}$ de las coordenadas internas 
\begin{equation}\label{drU}
(d\vec{r})^2=(d\vec{r}_0)^2+2u_{\alpha\beta}dx^{\alpha}dx^{\beta}
\end{equation}
donde $u_{\alpha\beta}$ es un tensor, denominado tensor de deformaciones, que justamente mide
la variación de las distancias relativas (al cuadrado) entre puntos, respecto a su valor
inicial en una determinada dirección. En virtud de \eqref{elemento_linea} y
teniendo en cuenta que las coordenadas en la configuración completamente plana
son ortonormales, podemos expresar las normas al cuadrado de los vectores
\eqref{dr}\eqref{dr0} en función de la métricas inducidas como
\begin{align}
(d\vec{r_0})^2&=\delta_{\alpha\beta}dx^{\alpha}dx^{\beta}\label{Meuclidea}\\
(d\vec{r})^2&=g_{\alpha\beta}dx^{\alpha}dx^{\beta}
\end{align}
con las que obtenemos a partir \eqref{drU} la siguiente expresión para el
tensor de deformaciones:
\begin{equation}\label{tensor_metrica}
u_{\alpha\beta}=\frac{1}{2}\left(g_{\alpha\beta} - \delta_{\alpha\beta}\right)
\end{equation}
 Supongamos que los $P$ y $Q$ se encuentran
inicialmente sobre la dirección $x^1$, esto es, $(d\vec{r}_0)^2=(dx^1)^2$, y
la distancia entre ambos es $dl_{(x^1)}=dx^1$. A
partir de \eqref{drU} podemos obtener la distancia $dL_{(x^1)}$ entre estos
dos puntos después de la deformación:
\begin{equation*}
dL_{(x^1)}=\sqrt{1+2u_{11}}\;dl_{(x^1)}
\end{equation*}
Y si la deformación es pequeña $u_{11}\simeq 0$
\begin{equation*}
dL_{(x^1)}\simeq (1+u_{11}) dl_{(x^1)} \Rightarrow u_{11}=\frac{dL_{(x^1)}-dl_{(x^1)}}{dl_{(x^1)}}
\end{equation*}
De igual forma podemos obtener una expresión análoga para $u_{22}$. Entonces,
los elementos diagonales de $u_{\alpha\beta}$ determinan las elonganciones
relativas en las direcciones marcadas por los vectores de la base de las
coordenadas internas. Consideremos otro punto $Q$, de forma que la linea $PR$
es perpendicular a $PQ$. El ángulo $\theta$ que forman estos dos vectores
después de la deformación podemos obtenerlo a partir de
\eqref{tensor_metrica}:
\begin{equation}
u_{12}=u_{21}\simeq \frac{1}{2}\cos \theta \Rightarrow 2u_{12}\simeq\sen \left(\frac{\pi}{2}-\theta\right)\simeq \frac{\pi}{2}-\theta
\end{equation}
Por tanto, $u_{\alpha\beta}$ con $\alpha\neq\beta$, los elementos no
diagonales del tensor de deformaciones están directamente relacionados con la
variación del ángulo que forman los vectores tangentes de la base antes y
después de la deformación.

La expresión \eqref{tensor_metrica} muestra claramente que $u_{\alpha\beta}$
es un tensor simétrico, de manera que, siempre 
podremos encontrar para un punto dado un sistema de coordenadas $\mathbf{x}$ en donde
$u_{\alpha\beta}$ sea diagonal, es decir, con las únicas componentes no nulas
$u_{11}$ y $u_{22}$. Estas dos componentes son los valores
principales del tensor de deformaciones, que denotaremos por $u_{(1)}$ y
$u_{(2)}$. Supongamos que un punto dado, encontramos unas coordenadas internas
$\{\mathbf{x}_d\}$, donde $u_{\alpha\beta}$ es diagonal, entonces \eqref{drU},
teniendo en cuenta \eqref{Meuclidea}, es 
\begin{equation}
(d\vec{r})^2=(1+2u_{(1)})(dx_d^{1})^2+(1+2u_{(2)})(dx_d^{2})^2
\end{equation}
Esto significa que la deformación en cualquier elemento de superficie puede
descomponerse como deformaciones independientes en dos direcciones
perpendiculares, los ejes principales del tensor de deformación. Cada una de
estas deformaciones será a una extensión o compresión en la correspondiente
dirección: la longitud $dx^1$ en el primer eje principal será
\begin{equation}
dx^1=\sqrt{1+2u_{(1)}}\; dx^1_d\simeq (1+u_{(1)})\; dx^1_d
\end{equation}
donde hemos considerado pequeñas deformaciones. La expresión para el otro eje
principal se obtiene de forma similar. Consideremos ahora un elemento
de área infinitesimal $d^2\mathbf{s}_d$, en las coordenadas $\{\mathbf{x}_d\}$
corresponde a $dx_d^1dx_d^2$. Después de la deformación, este elemento de área
cambiará a $d^2\mathbf{s}=dx^1dx^2$ y estará relacionado con el volumen inicial por
\begin{equation*}
d^2\mathbf{s}=(1+u_{(1)})(1+u_{(2)})d^2\mathbf{s}_d\simeq (1+u_{(1)}+u_{(2)})d^2\mathbf{s}_d
\end{equation*}
donde hemos despreciado ordenes superiores. Dado que
$u_{\alpha}^{\alpha}=u_{(1)}+u_{(2)}$ es la traza del tensor de deformaciones, la
suma de sus elementos diagonales, cuyo valor es independiente del sistema de
coordenadas, tenemos la siguiente expresión general:
\begin{equation}\label{deformacion_area}
d^2\mathbf{s}=(1+u_{\alpha}^{\beta})d^2\mathbf{s}_0
\end{equation}
siendo $d^2\mathbf{s}_0$, $d^2\mathbf{s}$ los elementos de área anterior y
posterior a la deformación respectivamente.

\begin{figure}[h]
\centering
 \input{deformacion-fig}
\caption{Vectores tangentes en el punto $\vec{r}(\mathbf{x})$}
\end{figure}

Podemos encontrar otra expresión para el tensor $u_{\alpha\beta}$ si
relacionamos $(d\vec{r})^2$ con $(d\vec{r}_0)^2$ a partir de
la expresión \eqref{deformacion}:
\begin{equation}\label{dl_vectorial}
(d\vec{r})^2=(d\vec{r_0}+d\vec{u})^2=(d\vec{r})^2_0+(d\vec{u})^2+2d\vec{r}_0\cdot d\vec{u}
\end{equation}
Al comparar está expresión con \eqref{drU}
\begin{equation}\label{2udxdx}
2u_{\alpha\beta}dx^{\alpha}dx^{\beta}=(d\vec{u})^2+2d\vec{r}_0\cdot d\vec{u}
\end{equation}
Ahora bien, podemos expresar el vector desplazamiento $\vec{u}(\mathbf{x})$ en función de
$\vec{t}^{(0)}_{\alpha}$, los vectores de la base del plano tangente a $\vec{r_0}$:
\begin{equation*}
  \vec{u}=u^{\alpha}\vec{t}^{\ (0)}_{\alpha}+h\vec{n}
\end{equation*}
donde el vector $\vec{n}$ es el vector normal a la superficie
\begin{equation}
\vec{n}=\frac{\vec{t}^{\ (0)}_1\times \vec{t}_2^{\ (0)}}{|\vec{t}^{\
    (0)}_1\times \vec{t}_2^{\ (0)}|}
\end{equation}
Las componentes del vector desplazamiento dependen las cordenadas internas, por lo que su elemento
diferencial $d\vec{u}$ y su cuadrado $(d\vec{u})^2$ vendrán dados por: 
\begin{equation}\label{du2}
d\vec{u}=\left[\frac{\partial
  u^{\alpha}}{\partial x^{\beta}}\vec{t}^{\ (0)}_{\alpha}+\frac{\partial
  h}{\partial x^{\beta}} \vec{n}\right]dx^{\beta}\Rightarrow
(d\vec{u})^2=\left[
\frac{\partial u_{\gamma}}{\partial x^{\alpha}}
\frac{\partial u^{\gamma}}{\partial x^{\beta}}+ 
\frac{\partial h}{\partial x^{\alpha}}
\frac{\partial h}{\partial x^{\beta}}\right]
dx^{\alpha}dx^{\beta} 
\end{equation}
Por otro lado tenemos también
\begin{equation}\label{drdu}
d\vec{r}_0\cdot d\vec{u}=\frac{\partial u^{\alpha}}{\partial x^{\beta}}dx^{\alpha}dx^{\beta}=\frac{1}{2}\left[\frac{\partial u_{\beta}}{\partial x^{\alpha}}+ \frac{\partial u_{\alpha}}{\partial x^{\beta}}\right]dx^{\alpha}dx^{\beta} 
\end{equation}
donde hemos usado que
 $$\frac{\partial u_{\beta}}{\partial
   x^{\alpha}}dx^{\alpha}dx^{\beta}=\frac{\partial u_{\alpha}}{\partial
   x^{\beta}}dx^{\alpha}dx^{\beta}$$
Finalmente, sustituyendo \eqref{du2} y \eqref{drdu} en \eqref{2udxdx}
obtenemos otra expresión equivalente para el tensor de deformaciones:
\begin{equation}\label{tensor_deformacion}
u_{\alpha\beta}=\frac{1}{2}\left(
 \frac{\partial u_{\beta}}{\partial x^{\alpha}}+
 \frac{\partial u_{\alpha}}{\partial x^{\beta}}+
 \frac{\partial u_{\gamma}}{\partial x^{\alpha}}
\frac{\partial u^{\gamma}}{\partial x^{\beta}}+
\frac{\partial h}{\partial  x^{\alpha}}\frac{\partial h}{\partial x^{\beta}}\right).
\end{equation}

Si la deformación es pequeña, lo que significa que las distancias
, en cualquier dirección, entre puntos próximos no se modifican demasiado comparadas con
su valor inicial. Esto implica que, bajo pequeñas deformaciones, podemos
despreciar los términos cuadráticos en las derivadas de $u_{\alpha}$ en
\eqref{tensor_deformacion}; no podemos hacer lo mismo con $h$, dado que no hay
términos en primer orden: 

\begin{equation}
u_{\alpha\beta}\simeq\frac{1}{2}\left(
 \frac{\partial u_{\beta}}{\partial x^{\alpha}}+
 \frac{\partial u_{\alpha}}{\partial x^{\beta}}+
\frac{\partial h}{\partial  x^{\alpha}}\frac{\partial h}{\partial x^{\beta}}\right).
\end{equation}

\subsection{Propiedades elásticas}

En una membrana plana es natural introducir el tensor de deformaciones:
\begin{equation*}
u_{\alpha\beta}=\frac{1}{2}(\vec{t}_{\alpha}\cdot\vec{t}_{\beta}-\delta_{\alpha\beta})\
\Rightarrow \
\vec{t}_{\alpha}\cdot\vec{t}_{\beta}=\delta_{\alpha\beta}+2u_{\alpha\beta}
\end{equation*}

Los factores del interior del integrando quedan:
\begin{align}
\vec{t}_{\alpha}\cdot\vec{t}^{\alpha} &=2(1+u_{\alpha}^{\ \alpha})\\
(\vec{t}_{\alpha}\cdot\vec{t}^{\alpha})(\vec{t}_{\alpha}\cdot\vec{t}^{\alpha})&=4(1+2u_{\alpha}^{\
  \alpha}+u_{\alpha}^{\ \alpha}u^{\alpha}_{\ \alpha})\\
(\vec{t}_{\alpha}\cdot\vec{t}_{\beta})(\vec{t}^{\alpha}\cdot\vec{t}^{\beta})&=2+4u_{\alpha}^{\
  \alpha}+4u_{\alpha\beta}u^{\alpha\beta}
\end{align}

Sustituyendo en la expresión de $F_E$, y despreciando términos constantes:

\begin{equation}\label{Elibre_elastica}
F_E[u_{\alpha\beta},T]=\int d^2\mathbf{x}
\left[\tau u_{\alpha}^{\ \alpha}+
\mu u_{\alpha\beta}u^{\alpha\beta} +
\frac{\lambda}{2}u_{\alpha}^{\ \alpha}u^{\alpha}_{\ \alpha}\right]
\end{equation}

Donde:
\begin{align}
\tau&=t+4u+8v\label{tau}\\
\mu&=4u\\
\frac{\lambda}{2}&=4v
\end{align}

%LANDAU
El tensor de esfuerzos $\sigma_{\alpha\beta}$, representa las tensiones
internas debido a variaciones en la extensión de la membrana. Al igual que
$u_{\alpha\beta}$ es simétrico y podemos obtenerlo a partir de $f_E$, la densidad de
la energía libre \eqref{Elibre_elastica}
\begin{equation}
f_E[u_{\alpha\beta},T]=\tau u_{\alpha}^{\ \alpha}+
\mu u_{\alpha\beta}u^{\alpha\beta} +
\frac{\lambda}{2}u_{\alpha}^{\ \alpha}u^{\alpha}_{\ \alpha}
\end{equation}
ya que corresponde a la variable conjugada de $u_{\alpha\beta}$
\begin{equation}\label{tensor_esfuerzos}
\sigma_{\alpha\beta}=\frac{\partial f_E}{ \partial u_{\alpha\beta}}
\end{equation}

Cuando $u_{\alpha\beta}=0 \ \alpha,\beta=1,2$ la membrana se encuentra en
la situación completamente plana con todos sus puntos equidistante. En esta
configuración las tensiones internas se encontraran equilibradas, lo que
implica que también $\sigma_{\alpha\beta}=0 \ \alpha,\beta=1,2$. Por tanto, no puede haber
un término lineal en $u_{\alpha\beta}$ en el desarrollo de $f_E$, de forma que,
 $\tau=0$ y podemos reescribir la expresión \eqref{tau}
\begin{equation}\label{tau=0}
-t=4(u+2v)
\end{equation}   
y la densidad de energía libre elástica
\begin{equation}\label{densidad_Fe}
f_E[u_{\alpha\beta},T]=
\mu u_{\alpha\beta}u^{\alpha\beta} +
\frac{\lambda}{2}u_{\alpha}^{\ \alpha}u^{\alpha}_{\ \alpha}
\end{equation} 
De hecho, con la expresión \eqref{tau=0} estamos recuperando la ecuación del
campo medio para la temperatura reducida \eqref{zeta_0} con $\zeta_0=1$, que
coincide con el valor que hemos tomado en este caso para el parámetro $\zeta=1$.
La expresión \eqref{densidad_Fe} coincide con la referencia REF, los
coeficientes $\mu$ y $\lambda$ son los coeficientes de Lamé. 

Según la expresión \eqref{deformacion_area} el cambio relativo en el área infinitesimal
viene dada por la suma $u_{\alpha}^{\ \alpha}$ . Si esta suma es nula, el área
de la membrana no se altera tras la deformación. Este tipo de deformaciones se
denominan de cizalladura pura. Por otro lado, tenemos las deformaciones que
mantienen la forma pero producen un cambio en el área, este tipo de denominan
compresiones puras. Cualquier deformación puede descomponerse como una suma de
los dos tipos anteriores. Para ello, usamos la siguiente identidad
\begin{equation}\label{identidad_u}
u_{\alpha\beta}=(u_{\alpha\beta}-\frac{1}{2}\delta_{\alpha\beta}u_{\gamma}^{\
  \gamma})+\delta_{\alpha\beta}u_{\gamma}^{\ \gamma}
\end{equation}
El primer término del miembro de la derecha es una cizalladura pura, dado que
su traza es nula ($\delta_{\alpha}^{\ \beta}=0$), y el segundo término es una
compresión pura. Sustituyendo esta identidad \eqref{identidad_u} en la expresión de la densidad de
energía elástica \eqref{densidad_Fe}
\begin{equation}\label{densidad_Fe_K}
f_E[u_{\alpha\beta},T]=
\mu \left(u_{\alpha\beta}-\frac{1}{2}\delta_{\alpha\beta}u_{\gamma}^{\ \gamma}\right)^2+
\frac{1}{2}K(u_{\gamma}^{\ \gamma})^2
\end{equation}
Las constantes $K$ y $\mu$ son el módulo de compresión y de
cizalladura\footnote{El módulo de cizalladura $\mu$ coincide con uno de los
  coeficientes de Lamé, no es un nuevo coeficiente con igual símbolo}. $K$
está relacionado con los coeficientes de Lamé por:
\begin{equation}
K=\lambda+\mu
\end{equation}
El estado termodinámico de equilibrio del sistema viene dado por el mínimo de
$F_E$. Si no hay fuerzas externas aplicadas a la membrana, entonces $f_E$ como
función de $u_{\alpha\beta}$ deberá tener un mínimo en $u_{\alpha\beta}$. Esto
significa que la forma cuadrática \eqref{densidad_Fe_K} debe ser positiva, y por
tanto, como condición necesaria y suficiente, los coeficientes $K$ y $\mu$ deben ser positivos
\begin{equation}
K>0\, ,\ \mu>0
\end{equation}
Si no podríamos encontrar que la energía libre disminuyera debido a una
comprensión pura ($K<0$) o una cizalladura pura ($\mu<0$).

El tensor de esfuerzos $\sigma_{\alpha\beta}$, según \eqref{tensor_esfuerzos}
podemos obtenerlo tomando la derivada de $f_E$ \eqref{densidad_Fe_K} respecto
a $u_{\alpha\beta}$
\begin{equation}\label{tensor_esfuerzos_K}
\sigma_{\alpha\beta}=K\delta_{\alpha\beta}u_{\gamma}^{\ \gamma}+2\mu\left(u_{\alpha\beta}-\frac{1}{2}\delta_{\alpha\beta}u_{\gamma}^{\ \gamma}\right)
\end{equation}
Está expresión muestra que si la deformación es una compresión pura o una
cizalladura pura, la relación entre $\sigma_{\alpha\beta}$ y $u_{\alpha\beta}$
estará determinada únicamente por el módulo de compresión o por el módulo de
cizalladura respectivamente. Es posible invertir la última igualdad
\eqref{tensor_esfuerzos_K} teniendo en cuenta que
\begin{equation}
\sigma_{\gamma}^{\ \gamma}=2 K u_{\gamma}^{\ \gamma}
\end{equation}
con lo resulta
\begin{equation}\label{hooke}
u_{\alpha\beta}=\frac{1}{4K}\delta_{\alpha\beta}\sigma_{\gamma}^{\ \gamma}+\frac{1}{2\mu}\left(\sigma_{\alpha\beta}-\frac{1}{2}\delta_{\alpha\beta}\sigma_{\gamma}^{\gamma}\right)
\end{equation}

%hooke

El módulo de Poisson $\sigma$ es el cociente entre la deformación transversal
a una tensión aplicada y la deformación longitudinal paralela a la tensión
aplicada. Si la tensión, es aplicada por ejemplo, en el eje $x^1$
\begin{equation}
\sigma=-\frac{\delta x^2 / x^2}{\delta x^1 / x^1}=-\frac{u_{22}}{u_{11}}
\end{equation}
como la tensión es aplicada en la dirección $x^1$, la única componente no nula
del tensor de esfuerzo es $\sigma_{11}$, y las componentes $u_{11}$ $u_{22}$
se obtienen a partir de \eqref{hooke}
\begin{align}
u_{11}&=\frac{1}{4K}\sigma_1^1+\frac{1}{2\mu}\left(\sigma_{11}-\frac{1}{2}\sigma_1^1\right)=\left(\frac{1}{4K}+\frac{1}{4\mu}\right)\sigma_1^1\\
u_{11}&=\frac{1}{4K}\sigma_1^1-\frac{1}{2\mu}\left(\frac{1}{2}\sigma_1^1\right)=\left(\frac{1}{4K}-\frac{1}{4\mu}\right)\sigma_1^1
\end{align}
Con lo que el módulo de Poisson resulta
\begin{equation}
\sigma=\frac{K+\mu}{K-\mu}
\end{equation}

%RElacion con la estadística -> correlaciones conexas MODULO DE POISSON


%%% Local Variables: 
%%% mode: latex
%%% TeX-master: "TFM"
%%% End: 
