\section{Teoría de Landau de la transición de fase}

La cantidad $\vec{r}(\boldsymbol{x})$, donde $\boldsymbol{x}$ es un vector
$D$-dimensional interno que etiqueta cada partícula, y $\vec{r}$ es un vector
de $d$-dimensional externo que representa la posición en el espacio donde se
alojan. Una descriptión estadística se desarrolla mediante un grano grueso de
forma que los vectores $\boldsymbol{x}$ son una variable continua que
etiquetan los bloques de los puntos de la red.

Las simetrías del Hamiltoniano microscópico delimitan la forma del funcional
de la energía libre para la variable de grano grueso $\vec{r}$. Para una red
uniforme, la invarianza de traslación global implica que esta funcional debe
depender de gradientes de los vectores tagentes.
$\boldsymbol{t}_{\alpha}=\partial \vec{r}/\partial x^{\alpha} $ , donde
$\alpha=1,2\dots D$. Inavarianza bajo rotaciones en ambos estados implica que
la forma de la energía libre debe ser:

\begin{equation}
F(\vec{r})=\int d^D\boldsymbol{x}
\left[
\frac{t}{2}(\partial_{\alpha}\vec{r})^2+
u(\partial_{\alpha}\vec{r}\partial_{\beta}\vec{r})^2+
v(\partial_{\alpha}\vec{r}\partial^{\alpha}\vec{r})^2+
\frac{\kappa}{2}(\partial^2\vec{r})^2
\right]
+\frac{b}{2}\int d^D\boldsymbol{x} d^D\boldsymbol{x'}
\delta^{d}(\vec{r}(\boldsymbol{x})-\vec{r}(\boldsymbol{x'}))
\end{equation}

Donde los términos de orden superior son irrelevantees en el límite de
$\lambda\rightarrow 0$. La física de está ecuación depende de el módulo
elástico $t$, $u$ y $v$, la rígidez de curvatura $\kappa$. El factor $b$
controla la autointersección , de modo que si es nulo, los puntos puede
intersectarse a sí mismo sin coste energético.

Para $b=0$ es posible la autointersección y la membrana es llamada fantasma.

Pequeñas deformaciones desde un estado de referencia se pueden parametrizar
como:

\begin{equation}
\vec{r}(\boldsymbol{x})=(\zeta x^{\gamma}+u^{\gamma},h)
\end{equation}

Donde $u^{\alpha}$ son los modos fonocnicos internos y h son las fluctuaciones
fuera del plano.

$\zeta$ no nula describe una membrana plana con pequeñas fluctuaciones.

\subsection{Membrana fantasma}



%%% Local Variables: 
%%% mode: latex
%%% TeX-master: "TFM"
%%% End: 
