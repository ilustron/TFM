%\chapter*{Bibliografía}
\begin{thebibliography}{99}

%% \bibitem{label}


 % \bibitem{castro} A. H. Castro Neto, F. Guinea, N.M.R Perez,
 %  K.S. Novoselov y A. K. Geim, \textit{The electronic properties of
 %    graphene},  Rev. Mod. Phys., 109, (2009) 81 

 % \bibitem{vozmediano} M.A.H. Vozmediano, M.I. Katsnelson y  F. Guinea, “Gauge Fields on Graphene”. Physics Reports 496, 109 (2010).

 % \bibitem{rharnish} R. Harnish, J. Wheather, Nuclear Physics B, 350 (1991) 861 


 % \bibitem{bowick} M. Bowick y A. Travesset, \textit{The statistical
 %   mechanics of membranes Physics Reports}    344, (2001) 255.

%review membranas 

\bibitem{Kay:Polimerized_Membranes}C.Domb and J.Lebowitz.
\textit{Phase Transitions and Critical Phenomena}.
vol. 19, Academic Press (2000)

% Artículo: Flat phase
\bibitem{Bowick_flat_phase} Mark J. Bowick, Simon M. Catterall, Marco
  Falcioni, Gudmar Thorleifsson y Konstantinos N. Anagnostopoulos. 
  \textit{The Flat Phase of Crystalline Membranes}. 
  J. Phys. France \textbf{6} (1996) 1321.

%Cálculo analítico de nu
\bibitem{Doussal:nu}P. Le Doussal y L. Radzihovsky.
  \textit{Self-consistent theory of polymerized membranes}
  Phys. Rev. Lett. \textbf{69} (1992) 1209.

 % Libro: Mechanics of the cell
\bibitem{Boal_MCell}
  D. Boal. 
  \textit{Mechanics of the Cell}.  
  Cambridge Univ. Press. (2002).

\bibitem{Espriu:MCRG}D. Espriu y A. Travesset.
  \textit{MCRG study of fixed-connectivity surface}
  Nuc. Phys. \textbf{B468} 514 (1996) 514-540 

\bibitem{Discher:Molecular} D. E. Discher, N. Mohandas y E. A. Evans.
\textit{Maps of red blood cell deformation: Hidden elasticity and in situ connectivity}
  Science \textbf{266} (1994) 1032

\bibitem{Hwa:Conformation}T. Hwa E. Kokufuta y T. Tanaka. 
  \textit{Conformation of graphite oxide membranes in solution}.
  Phys. Rev. \textbf{A 44} (1991) R2235–R2238  

\bibitem{David:geometria}F. David
  \textit{Statistical Mechanics of Membranes and Surfaces, Second edition}.
  D. R. Nelson, T. Piran y S. Weinberg.
  World Scientific (2004) p. 149


\bibitem{LuisLinares}Luis J. Alías
  \textit{El significado geométrico de la curvatura: Superficies de curvatura
    media constante}.
  Fund. Séneca (2004).

\bibitem{Landau_Elasticidad}L. D. Landau y E. M. Lifshitz.
  \textit{Theory of Elasticity}.
  Pergamon Press (1975).

\bibitem{Bowick:Libro_superficies}M. J. Bowick
  \textit{Statistical Mechanics of Membranes and Surfaces, Second edition}.
  D. R. Nelson, T. Piran y S. Weinberg.
  World Scientific (2004) p. 323
  % Libro: Cardy
\bibitem{Cardy} J. L. Cardy.
  \textit{Scaling and Renormalization in Statistical Physics}.
  Cambridge Univ. Press. (1996).

% expresión del calor específico
\bibitem{Harnish:CV}R. Harnish y J. Wheater.
  \textit{The crumpling transition of crystalline random surfaces}
Nucl. Phys. B \textbf{350} (1991) 861-892

\bibitem{Goldenfield:Lecture_Notes}Nigel Goldenfield. 
  \textit{Lectures notes on phase transitions and the renormalization group}.
  Addison-Wesley (1994).

\bibitem{Heller:Method}Urs. M. Heller y Nathan Seiberg.
\textit{Method for numerical simulations of metaestable states}.
Phys. rev. D \textbf{27} 12 (1983) 2980-2989

\bibitem{Binney:critical_phenomema}J. J. Binney, N. J. Dowrick, A. J. Fisher y
  M. E. J. Newman 
  \textit{Theory of critical phenomena (an introductory to the renormalization
    group)}.
  Oxford Univ. Press  (1992).

%correlacion normales
\bibitem{Bowick:Membranes_review}Mark J. Bowick y Alex Travesset.
\textit{The Statistical Mechanics of Membranes}.
Phys.Rept. \textbf{344} (2001) 255-308.

\bibitem{David:normal} David R. Nelson
  \textit{Statistical Mechanics of Membranes and Surfaces, Second edition}.
  D. R. Nelson, T. Piran y S. Weinberg.
  World Scientific (2004) p. 131

 %F radio de giro
\bibitem{Gomper:triangulated}G. Gompper y D.M. Kroll.
  \textit{Statistical Mechanics of Membranes and Surfaces, Second edition}.
  D. R. Nelson, T. Piran y S. Weinberg.
  World Scientific (2004) p. 359

 %aproximación de Flory 
 \bibitem{Gennes:Scaling}P.G. de Gennes.
   \textit{Scaling Concepts in Polymer Physics}
   Cornell University Press (1979)

 %tesis Juan J.
 \bibitem{Juan:tesis}Juan J. Ruiz Lorenzo
    \textit{Phase transitions in gauge theories and spin models}.
    Tesis, Univ. Complutense de Madrid (1993) 

\bibitem{drnelson2004} 
  D. R. Nelson, T. Piran y S. Weinberg. 
  \textit{Statistical Mechanics of Membranes and Surfaces, Second edition}. 
  World Scientific (2004)

 
  % Artículo: Poisson ratio
\bibitem{Bowick_poisson_ratio} M. Falcioni1, M. J. Bowick, E. Guitter y
  G. Thorleifsson.
  \textit{The Poisson ratio of crystalline surfaces}. 
  Europhys. Lett. \textbf{38} (1997) 67

% Artículo dinamica molecular m. cristalina
\bibitem{Zang_Dmolecular} Z. Zhang, H. T. Davis y D. M. Kroll.
\textit{Molecular dynamics simulations of tethered membranes with periodic
  boundary conditions}.
 Phys. Rev. \textbf{E 53} (1996) 2 1422-1428

% Artículo tensor deformación
\bibitem{Parrinello_Crystal} M. Parrinello y A. Rahman.
\textit{Polymorphic transitions in single crystals: A new molecular dynamics
  method}.
J. Appl. Phys. \textbf{52} (1981) 7182-7190  

% Artículo limite discreto
\bibitem{Aranovitz_Fuctuations} J. A. Aranovitz y T. C. Lubensky. 
\textit{Fluctuations of Solids Membranes}.
Phys. Rev. Lett. \textbf{60} 25 (1988) 2634-2637

% Artículo límite discreto sqrt(3)
\bibitem{Kroll_Discretizations}G. Gompper y D. M. Kroll.
\textit{Random Surface Discretizations and Renormalizations of the Bending
  Rigidity}.
J. Phys. I France \textbf{6} (1996) 1305-1320

%Exponentes criticos 
\bibitem{Wheater_Critical_exponents}J. F. Wheater.
\textit{The critical exponents of crystalline random surfaces}
Nuc. Phys. B \textbf{458} (1996) 671-689

%Exponentes criticos 2 
\bibitem{Renken_Scaling_behavior} Ray L. Renken y John B. Kogout.
\textit{Scaling behavior at the crumpling transition}.
Nuc. Phys. B \textbf{342} (1990) 753-763
\end{thebibliography}

%%% Local Variables: 
%%% mode: latex
%%% TeX-master: "TFM"
%%% End: 
