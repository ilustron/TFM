\section{El Tensor de Deformaciones}
La posición tridimensional de un punto de la membrana viene determinado por su
radio vector $\vec{r}(\boldsymbol{\lambda})$, donde
$\boldsymbol{\lambda}$ son las cordenadas internas del
punto. Si la membrana sufre una deformación las posiciones tridimensionales de
sus puntos cambiarán, no así sus cordenadas internas que serán las
mismas. Consideremos un punto particular con cordenadas internas $\boldsymbol{\lambda}=(\lambda^1,\lambda^2)$, y sometamos a una deformación la membrana, el
radio vector del punto antes de la deformación es
$\vec{r_0}(\boldsymbol{\lambda})$, y después de la deformación es
$\vec{r}(\boldsymbol{\lambda})$. El desplazamiento de este punto debido a la
deformación viene dado por el vector desplazamiento $\vec{u}$ que corresponde
a la diferencia de la posición anterior y posterior a la deformación del punto: 

\begin{equation}\label{vector_u}
\vec{u}=\vec{r}-\vec{r_0}
\end{equation}

Las distancias relativas de los puntos de la membrana, en general, cambiarán
al producirse una deformación. Consideremos dos puntos muy próximos de la
membrana, descritos por sus cordenadas internas $\boldsymbol{\lambda}$ y
$\boldsymbol{\lambda}+d\boldsymbol{\lambda}$. Denotamos por $d\vec{r^0}$ y
$d\vec{r}$ a los vectores que unen estos dos puntos antes y después de la
deformación respectivamente. Estos dos vectores al ser infinitesimales
pertenecen al plano tangente a la superficie y por tanto, podemos expresarlos
en función de la base de estos\footnote{En lo que sigue se utilizará el
  convenio de sumación de Einstein, donde las sumas serán en dos dimensiones}:
\begin{align}
d\vec{r_0}&=d\lambda^i \ \vec{t}^{\ (0)}_i\label{dr0_vectorial}\\
d\vec{r}&=d\lambda^i \ \vec{t}_i
\end{align}

Donde
\begin{align*}
\vec{t}^{\ (0)}_i&=\frac{\partial \vec{r_0}}{\partial \lambda^i}\\
\vec{t}_i&=\frac{\partial \vec{r}}{\partial \lambda^i}
\end{align*}

Son los vectores de la base del plano tangente a la superficie anterior y
posterior a la deformación en el punto correspondiente a las cordenadas
internas $(\lambda^1,\lambda^2)$. Las distancias relativas de los dos puntos
considerados son: 
\begin{align}
dl^2_0&=(d\vec{r_0})^2=g^{(0)}_{ij}d\lambda^id\lambda^j\label{dl0}\\
dl^2&=(d\vec{r})^2=g_{ij}d\lambda^id\lambda^j
\end{align}

Donde:
\begin{align*}
g^{(0)}_{ij}&=\vec{t}^{\ (0)}_i\cdot\vec{t}^{\ (0)}_j\\
g_{ij}&=\vec{t}_i\cdot\vec{t}_j
\end{align*}

Son las métricas de cada plano tangente. Por otro lado, podemos relacionar
$dl^2$ con $dl^2_0$ a partir de la expresión 
\eqref{vector_u}:

\begin{equation}\label{dl_vectorial}
dl^2=(d\vec{r_0}+d\vec{u})^2=dl^2_0+(d\vec{u})^2+2d\vec{r_0}\cdot d\vec{u}
\end{equation}


Si expresamos el vector desplazamiento en función de los vectores de la
base del plano tangente a $\vec{r_0}$:

\begin{equation*}
\vec{u}=u^i_0\vec{t}^{\ (0)}_i+h_0\vec{n}
\end{equation*}

El vector $\vec{n}$ es el vector normal a la superficie
$\vec{n}=\frac{\vec{t}^{\ (0)}_1\times \vec{t}_2^{\ (0)}}{|\vec{t}^{\
    (0)}_1\times \vec{t}_2^{\ (0)}|}$. Las componentes del vector
desplazamiento dependen las cordenadas internas, por lo que su elemento
diferencial correspondiente vendrá dado por: 

\begin{equation}\label{du}
d\vec{u}=\left[\frac{\partial
  u_0^i}{\partial\lambda^j}\vec{t}^{\ (0)}_i+\frac{\partial h_0}{\partial \lambda^j} \vec{n}\right]d\lambda^j
\end{equation}

Y cuya norma al cuadrado es:

\begin{equation}\label{norma2_du}
(d\vec{u})^2=\left[g^{(0)}_{kl}
\frac{\partial u_0^k}{\partial\lambda^i}
\frac{\partial u_0^l}{\partial\lambda^j}+ 
\frac{\partial h_0}{\partial \lambda^i}
\frac{\partial h_0}{\partial \lambda^j}\right]
d\lambda^id\lambda^j 
\end{equation}

Sustituyendo en \eqref{dl_vectorial} las expresiones \eqref{dr0_vectorial}, \eqref{dl0}, \eqref{du},
\eqref{norma2_du} y utilizando que

 $$g^{(0)}_{kj}\frac{\partial u^k_0}{\partial
   \lambda^i}d\lambda^id\lambda^j=g^{(0)}_{ki}\frac{\partial u^k_0}{\partial
   \lambda^j}d\lambda^id\lambda^j,$$

se obtiene la expresión

\begin{equation*}
dl^2=dl^2_0+2v_{ij}d\lambda^id\lambda^j,
\end{equation*}

con

\begin{equation*}
v_{ij}=\frac{1}{2}\left(
 g_{0kj}\frac{\partial u^k_0}{\partial \lambda^i}+
  g_{0ki}\frac{\partial u^k_0}{\partial \lambda^j}+
g_{0kl}\frac{\partial u^k_0}{\partial\lambda^i}
\frac{\partial u^l_0}{\partial \lambda^j}+
\frac{\partial h_0}{\partial \lambda^i}\frac{\partial h_0}{\partial \lambda^j}\right).
\end{equation*}

Si ahora hacemos un cambio de cordenadas al sistema ortonormal $x^i=M^i_{\
  j}\lambda^j$, en el cual la métrica $g^{(0)}_{ij}=\delta_{ij}$ es diagonal y
unitaria, la anterior ecuación queda: 

\begin{equation}\label{ecuacion_deformacion}
dl^2=dl^2_0+2u_{ij}dx^idx^j
\end{equation}

Siendo

\begin{equation}\label{tensor_deformacion}
u_{ij}=\frac{1}{2}\left(
\frac{\partial u_j}{\partial x^i}+
\frac{\partial u_i}{\partial x^j}+
\frac{\partial u_k}{\partial x^i}
\frac{\partial u^k}{\partial x^j}+
\frac{\partial h}{\partial x^i}
\frac{\partial h}{\partial x^j}\right)
\end{equation} 

El llamado tensor de deformaciones.

Otra forma útil de expresar el tensor de deformaciones es tener en cuenta que:
\begin{align}
dl^2_0=(\partial_i \vec{r}_0\cdot \partial_j \vec{r}_0)dx^idx^j=\delta_{ij}dx^idx^j\\
dl^2=(\partial_i \vec{r}\cdot \partial_j \vec{r})dx^idx^j\\
\end{align}

Que sustituyendo en la expresión TAL:

\begin{equation}
u_{ij}=\frac{1}{2}\left(\partial_i \vec{r}\cdot \partial_j \vec{r} - \delta_{ij}\right)
\end{equation}

Está expresión es sólo válida en un sistema de cordenadas internas
$\mathbf{x}$ ortonormal en la configuración sin deformar.

%%% Local Variables: 
%%% mode: latex
%%% TeX-master: "TFM"
%%% End: 
