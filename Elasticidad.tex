\section{El Tensor de Deformaciones}
La posición tridimensional de un punto de la membrana viene determinado por su radio vector
$\vec{r}(\sigma^1,\sigma^2)$, donde $\sigma^1$ y $\sigma^2$ son las cordenadas
internas del punto. Si la membrana sufre una deformación las posiciones
tridimensionales de sus puntos cambiarán, no así sus cordenadas
internas que serán las mismas. Consideremos un punto particular con cordenadas internas
$(\sigma^1,\sigma^2)$, y sometamos a una deformación la membrana, el radio
vector del punto antes de la deformación es $\vec{r_0}(\sigma^2,\sigma^2)$, y
después de la deformación es $\vec{r}(\sigma^1,\sigma^2)$. El desplazamiento
de este punto debido a la deformación viene dado por el vector desplazamiento
$\vec{u}$ que corresponde a la diferencia de la posición anterior y posterior
a la deformación del punto: 

\begin{equation}\label{vector_u}
\vec{u}=\vec{r}-\vec{r_0}
\end{equation}

Las distancias relativas de los puntos de la membrana, en general, cambiarán
al producirse una deformación. Consideremos dos puntos muy próximos de la
membrana, descritos por sus cordenadas internas $(\sigma^1,\sigma^2)$ y
$(\sigma^1+d\sigma^1,\sigma^2+d\sigma^2)$. Denotamos por $d\vec{r_0}$ y
$d\vec{r}$ a los vectores que unen estos dos puntos antes y después de la
deformación respectivamente. Estos dos vectores al ser infinitesimales
pertenecen al plano tangente a la superficie y por tanto, podemos expresarlos
en función de la base de estos\footnote{En lo que sigue se utilizará el
  convenio de sumación de Einstein, donde las sumas serán en dos dimensiones}:
\begin{align}
d\vec{r_0}&=d\sigma^i \boldsymbol{\sigma_0}_i\label{dr0_vectorial}\\
d\vec{r}&=d\sigma^i \boldsymbol{\sigma}_i
\end{align}

Donde
\begin{align*}
\boldsymbol{\sigma_0}_i&=\frac{\partial \vec{r_0}}{\partial \sigma^i}\\
\boldsymbol{\sigma}_i&=\frac{\partial \vec{r}}{\partial \sigma^i}
\end{align*}

Son los vectores de la base del plano tangente a la superficie anterior y
posterior a la deformación en el punto correspondiente a las cordenadas
internas $(\sigma^1,\sigma^2)$. Las distancias relativas de los dos puntos
considerados son: 
\begin{align}
dl^2_0&=(d\vec{r_0})^2=g^{(0)}_{ij}d\sigma^id\sigma^j\label{dl0}\\
dl^2&=(d\vec{r})^2=g_{ij}d\sigma^id\sigma^j
\end{align}

Donde:
\begin{align*}
g^{(0)}_{ij}&=\boldsymbol{\sigma_0}_i\cdot\boldsymbol{\sigma_0}_j\\
g_{ij}&=\boldsymbol{\sigma}_i\cdot\boldsymbol{\sigma}_j
\end{align*}

Son las métricas de cada plano tangente. Por otro lado, podemos relacionar
$dl$ con $dl_0$ a partir de la expresión 
\eqref{vector_u}:

\begin{equation}\label{dl_vectorial}
dl^2=(d\vec{r_0}+d\vec{u})^2=dl^2_0+(d\vec{u})^2+2d\vec{r_0}\cdot d\vec{u}
\end{equation}


Si expresamos el vector desplazamiento en función de los vectores de la
base del plano tangente a $\vec{r_0}$:

\begin{equation*}
\vec{u}=u^i_0\boldsymbol{\sigma_0}_i+\xi_{0} \boldsymbol{\sigma_0}_1\times \boldsymbol{\sigma_0}_2
\end{equation*}

Encontramos que: 

\begin{equation}\label{du}
d\vec{u}=\frac{\partial u_0^i}{\partial\sigma^j}d\sigma^j\boldsymbol{\sigma_0}_i+\frac{\partial\xi_0}{\partial \sigma^j}d\sigma^j \boldsymbol{\sigma_0}_1\times \boldsymbol{\sigma_0}_2
\end{equation}

Y cuya norma al cuadrado es\footnote{$g^{(0)}=|\boldsymbol{\sigma_0}_1\times \boldsymbol{\sigma_0}_2|^2=\det(g^{(0)}_{ij})$}

\begin{equation}\label{norma2_du}
(d\vec{u})^2=\left[g_{0kl}
\frac{\partial u_0^k}{\partial\sigma^i}
\frac{\partial u_0^l}{\partial\sigma^j}+ 
g^{(0)}\frac{\partial \xi_0}{\partial \sigma^i}
\frac{\partial \xi_0}{\partial \sigma^j}\right]
d\sigma^id\sigma^j 
\end{equation}

Sustituyendo en \eqref{dl_vectorial} las expresiones \eqref{dr0_vectorial}, \eqref{dl0}, \eqref{du},
\eqref{norma2_du} y utilizando que

 $$g_{0kj}\frac{\partial u^k_0}{\partial
   \sigma^i}d\sigma^id\sigma^j=g_{0ki}\frac{\partial u^k_0}{\partial
   \sigma^j}d\sigma^id\sigma^j,$$

se obtiene la expresión

\begin{equation*}
dl^2=dl^2_0+2v_{ij}d\sigma^id\sigma^j,
\end{equation*}

con

\begin{equation*}
v_{ij}=\frac{1}{2}\left(
 g_{0kj}\frac{\partial u^k_0}{\partial \sigma^i}+
  g_{0ki}\frac{\partial u^k_0}{\partial \sigma^j}+
g_{0kl}\frac{\partial u^k_0}{\partial\sigma^i}
\frac{\partial u^l_0}{\partial \sigma^j}+
g_0\frac{\partial \xi_0}{\partial \sigma^i}\frac{\partial \xi_0}{\partial \sigma^j}\right).
\end{equation*}

Si ahora hacemos un cambio de cordenadas al sistema ortonormal $x^i=M^i_{\
  j}\sigma^j$, en el cual la métrica $g_{0ij}=\delta_{ij}$ es diagonal y
unitaria ($g_0=1$), la anterior ecuación queda: 

\begin{equation}\label{ecuacion_deformacion}
dl=dl_0+2u_{ij}dx^idx^j
\end{equation}

Siendo

\begin{equation}\label{tensor_deformacion}
u_{ij}=\frac{1}{2}\left(
\frac{\partial u_j}{\partial x^i}+
\frac{\partial u_i}{\partial x^j}+
\frac{\partial u_k}{\partial x^i}
\frac{\partial u^k}{\partial x^j}+
\frac{\partial \xi}{\partial x^i}
\frac{\partial \xi}{\partial x^j}\right)
\end{equation} 

Como la métrica en este caso es $\delta_{ij}$ las componentes $u_i$ son:

\begin{equation*}
u_i=\delta_{ij} u^j=(u^1,u^2)
\end{equation*}

Y entonces, con permiso de la SICCCSE\footnote{Sociedad Internacional por el Correcto
  Cumplimiento del Convenio de Sumación de Einstein}, podemos escribir la
ecuación \eqref{tensor_deformacion}:

\begin{equation*}
u_{ij}=\frac{1}{2}\left(
\frac{\partial u^j}{\partial x^i}+
\frac{\partial u^i}{\partial x^j}+
\frac{\partial u^k}{\partial x^i}
\frac{\partial u^k}{\partial x^j}+
\frac{\partial \xi}{\partial x^i}
\frac{\partial \xi}{\partial x^j}\right)
\end{equation*} 

Es el tensor de deformaciones tal y como lo escriben la mayoría de
autores. Otra forma de expresar este tensor es teniendo en cuenta que en el
sistema de cordenadas ortonormal:
\begin{align*}
dl^2_0&=(d\vec{r_0})^2=\delta_{ij}dx^idx^j\\
dl^2&=(d\vec{r})^2=(M^{-1})^k_{\ i}(M^{-1})^l_{\ j}g_{kl}dx^idx^j\label{dl0}
\end{align*}


Sustituyendo en \eqref{ecuacion_deformacion}:

\begin{equation*}
u_{ij}=\frac{1}{2}\left[(M^{-1})^{k}_{\ i}(M^{-1})^l_{\ j}g_{kl}-\delta_{ij}\right]
\end{equation*}

%%% Local Variables: 
%%% mode: latex
%%% TeX-master: "TFM"
%%% End: 
