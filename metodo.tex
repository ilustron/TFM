\chapter{Simulaciones numéricas}

\section{Modelo discreto}

Para la realización de simulaciones numéricas es necesario discretizar la
membrana cristalina. Para ello, vamos a considerar una red 
discreta de $N$ nodos (figura~\ref{red_rombo}), cuyas conexiones o enlaces
forman una red regular triangular. El contorno de esta red es un rombo de lado
$L=\sqrt{N}$, en este sentido podemos calificar la red como cuadrada. 
Las condiciones en la frontera son libres, lo que significa que el índice de
coordinación, el número de primeros vecinos\footnote{Entendemos por primeros
  vecinos de un nodo, aquellos nodos que tienen un sólo enlace con él.} de un
determinado nodo, es menor en el contorno. Esta falta de homogeneidad,
son los llamados efectos de borde, que producen que los valores 
de los observables en el interior son diferentes que en el contorno. Puesto
que la cantidad de nodos interiores de la membrana es proporcional a $L^2$, y
el número de nodos del contorno a $L$, es de esperar que los efectos de borde
disminuyan a medida que aumenta el tamaño del sistema hasta hacerse
despreciables. Aún así, la forma óptima del contorno de una
membrana para minimizar los efectos de borde es el hexágono, donde por
ejemplo, ningún nodo tiene un nodo tiene un índice de coordinación menor que
3, en nuestro caso, con el contorno romboidal el menor número de coordinación
es 2. Aunque el contorno romboidal tenga el inconveniente de mayores efectos
de borde, lo hemos elegido principalmente por poder comparar nuestras medidas
con trabajos anteriores, y si en algún caso influyen
demasiado en las medidas, podremos minimizarlos no teniendo en cuenta
aquellos nodos cercanos al contorno.

\begin{figure}[h]\label{red_rombo}
\centering
\resizebox{275bp}{!}{\input{red_triangular-fig}}
\caption{Conectividad intrínseca de la red}
\end{figure}

Cada nodo es etiquetado mediante dos índices discretos $P=(i,j)\ i,j=1,L$
como se indica en la figura~\ref{red_rombo}. La posición en el espacio
tridimensional vendrá dada por el vector $\vec{r}_P$. Como energía libre discreta del sistema usamos
\begin{equation}
\mathcal{F}=\sum_{\langle PQ \rangle}
|\vec{r}_P-\vec{r}_Q|^2+\kappa\sum_{\langle ab \rangle}
|\vec{n}_a-\vec{n}_b|^2 
\end{equation}
donde con $\langle\rangle$ indicamos que la suma es sobre primeros vecinos,
nodos adyacentes. Estamos considerando únicamente interacciones de corto
alcance,  no inlcuimos ningún término de exclusión, ya que, es irrelevante en
la fase plana. Los  índices en letra minúscula representan las plaquetas, 
los triángulos de la red, y $\vec{n}_{a}$ el vector normal a la plaqueta. La
constante $\kappa$ es el módulo de rigidez y hemos escalado las posiciones de
forma que la constante elástica es unitaria.

%Colapso punto

\subsection{límite continuo}
\begin{figure}[h]
\centering
 \resizebox{\columnwidth}{!}{\input{base-hexagonal-fig}}
\caption{Base hexagono}
\end{figure}   
Consideramos un sistema de coordenadas ortogonal $\{ \vec{e}_1,\vec{e}_2\}$, son los vectores de la base del plano base. La expresión TAL podemos discretizarla
\begin{equation}
(\vec{r}_P-\vec{r}_Q)=a^2+2u_{\alpha\beta}x{\scriptstyle [q]}^{\alpha}x{\scriptstyle [q]}^{\beta}
\end{equation}
donde $u_{\alpha\beta}$ es la versión discretizada del tensor de
deformaciones. En este  sistema de coordenadas la métrica es
\begin{equation}
g_{\alpha\beta}=\vec{e}_{\alpha}\cdot\vec{e}_{\beta}=a^2\delta_{\alpha\beta}
\end{equation}
Podemos aproximar
\begin{equation}
|\vec{r}_P-\vec{r}_Q|=\sqrt{a^2+2u_{\alpha\beta}e^{\alpha}e^{\beta}}\simeq a + \frac{u_{\alpha\beta}}{a}e^{\alpha}e^{\beta}
\end{equation}



Sustituyendo en la parte elástica de la energía libre
\begin{equation}
\mathcal{F}_E\simeq \frac{1}{2}\sum_{P} \sum^6_{j=1} \left(\frac{u_{\alpha\beta}}{a}e^{\alpha}e^{\beta}\right)^2=\frac{1}{2a^2}\sum_{P} \sum^6_{j=1} u_{\alpha\beta}u^{\mu\nu}e^{\alpha}e^{\beta}e_{\mu}e_{\nu}
\end{equation}
Siendo el primer sumatorio sobre todos los nodos y el segundo sobre los
primeros vecinos. Teniendo en cuenta las relaciones
\begin{equation}
1=\frac{\delta^{\alpha\beta}\delta_{\alpha\beta}}{2}=\frac{\delta^{\mu\nu}\delta_{\mu\nu}}{2}
\end{equation}
Podemos escribir el término
\begin{equation}
e^{\alpha}e^{\beta}e_{\mu}e_{\nu}=\frac{1}{4}\delta^{\alpha\beta}\delta_{\mu\nu}\delta_{\alpha\beta}e^{\alpha}e^{\beta}\delta^{\mu\nu}e_{\mu}e_{\nu}=\frac{a^2}{4}\delta^{\alpha\beta}\delta_{\mu\nu}
\end{equation}
Y de forma análoga también podemos obtener
\begin{align}
e^{\alpha}e^{\beta}e_{\mu}e_{\nu}&=\frac{a^2}{4}\delta^{\alpha}_{\ \mu}\delta_{\beta}^{\
  \nu}\\
e^{\alpha}e^{\beta}e_{\mu}e_{\nu}&=\frac{a^2}{4}\delta^{\alpha}_{\ \nu}\delta_{\beta}^{\
  \mu}
\end{align}
Tomando la media de las últimas tres expresiones TAL
\begin{equation}
e^{\alpha}e^{\beta}e_{\mu}e_{\nu}=\frac{a^4}{12}(\delta^{\alpha\beta}\delta_{\mu\nu}+
\delta^{\alpha}_{\ \mu}\delta_{\beta}^{\ \nu}+
\delta^{\alpha}_{\ \nu}\delta_{\beta}^{\ \mu})
\end{equation}
Sustituyendo en TAL
\begin{equation}
\mathcal{F}_E\simeq\frac{a^2}{24}\sum_{P} \sum^6_{j=1}(\delta^{\alpha\beta}\delta_{\mu\nu}+\delta^{\alpha}_{\ \mu}\delta_{\beta}^{\ \nu}+
\delta^{\alpha}_{\ \nu}\delta_{\beta}^{\ \mu})
u_{\alpha\beta}u^{\mu\nu}=\frac{a^2}{4}\sum_{P}u_{\alpha}^{\ \alpha}u^{\alpha}_{\ \alpha}+2u_{\alpha\beta}u^{\alpha\beta}
\end{equation}
Donde en  la última igualdad hemos efectuado la suma sobre todos los vecinos.
Pasamos al límite continuo:
\begin{equation}
\frac{a^2}{4}\sum_{i=1}^L\sum_{j=1}^L\frac{\Delta i\Delta j}{\xi^2\frac{\sqrt{3}}{2}}\xi^2\,\frac{\sqrt{3}}{2}(u_{\alpha}^{\ \alpha}u^{\alpha}_{\ \alpha}+2u_{\alpha\beta}u^{\alpha\beta})
\end{equation}
Tomando el límite $N\rightarrow \infty$
elástica 
\begin{equation}
\mathcal{F}\simeq\int d^2\mathbf{x}\, \frac{\sqrt{3}}{4}\frac{a^2}{\xi^2}\left(\frac{1}{2}u_{\alpha}^{\ \alpha}u^{\alpha}_{\ \alpha}+u_{\alpha\beta}u^{\alpha\beta}\right)
\end{equation}
recuperamos la energía libre elástica con 
\begin{equation}
\lambda=\mu=\frac{\sqrt{3}}{4}\frac{a^2}{\xi^2}
\end{equation}
Nuestro modelo corresponde al límite $a\rightarrow 0$ y $\xi$ fijo.

Respecto al término de curvatura, podemos desarrollar $\vec{n}_{b}$ desde $a$
\begin{equation}
\vec{n}_b\simeq\vec{n}_a+\frac{\partial \vec{n_b}}{\partial x^{\alpha}}(x_b^{\alpha}-x_a^{\alpha})
\end{equation}
Sustituyendo en la energía libre
\begin{equation}
\mathcal{F}_C=\frac{1}{2}\kappa\sum_{\langle ab\rangle}\frac{\partial
  \vec{n_b}}{\partial x^{\alpha}}\cdot\frac{\partial \vec{n_b}}{\partial x^{\beta}} (x_b^{\alpha}-x_a^{\alpha})(x_b^{\beta}-x_a^{\beta})
\end{equation}
Podemos considerar que son triángulos equiláteros
\begin{equation}
g^{\alpha\beta}=(x_b^{\alpha}-x_a^{\alpha})(x_b^{\beta}-x_a^{\beta})\simeq
\sqrt{3} \delta^{\alpha\beta}
\end{equation}
Entonces
\begin{equation}
\mathcal{F}_C=\frac{\sqrt{3}}{2}\kappa\sum_{\langle ab\rangle}\frac{\partial
  \vec{n_b}}{\partial x_{\alpha}}\cdot\frac{\partial \vec{n_b}}{\partial x^{\alpha}}
\end{equation}
En el límite continuo, para $N\rightarrow \infty$
\begin{equation}
\mathcal{F}_C=\frac{1}{2}\hat{\kappa}\int d^2\mathbf{x}\ \frac{\partial
  \vec{n_b}}{\partial x_{\alpha}}\cdot\frac{\partial \vec{n_b}}{\partial x^{\alpha}}
\end{equation}
A partir de la expresión TAL
\begin{equation}
\frac{\partial \vec{n_b}}{\partial x_{\alpha}}\cdot\frac{\partial
  \vec{n_b}}{\partial x^{\alpha}}=(K_{\alpha\beta}\vec{t}^{\beta})\cdot(K_{\nu}^{\
  \alpha}\vec{t}^{\nu})=K_{\alpha\beta}g^{\beta\nu}K_{\nu}^{\ \alpha}=K_{\alpha\beta}K^{\alpha\beta}
\end{equation}
Entonces 
\begin{equation}
\mathcal{F}_C\simeq\frac{1}{2}\hat{\kappa}\int d^2\mathbf{x}\ K_{\alpha\beta}K^{\alpha\beta}
\end{equation}
Y recuperamos la ecuación TAL de la energía libre de Landua.

\section{Algoritmo de Metropolis}
Para especificar una determinada configuración $\mathcal{R}$ de la membrana
cristalina discreta corresponde a una lista de $N$ posiciones
tridimensionales:
\begin{equation}
\mathcal{R}=\{ \vec{r}_1,\vec{r}_2,\dots \vec{r}_N\}\,.
\end{equation}
Si tenemos $M$ configuraciones independientes que siguen una distribución
proporcional al factor $e^{-\mathcal{F}(\kappa)}$, A partir de las medidas
$A_1,A_2,\dots A_N$ en cada configuración de un observable $A$, podemos
estimar su valor medio para ese $\kappa$
\begin{equation}
\langle A \rangle_{\kappa}\simeq \frac{1}{M}\sum^M_{i=1} A_i\,.
\end{equation}
Para encontrar estas configuraciones empleamos un método de Monte Carlo, que
consiste en un camino aleatorio en el espacio de configuraciones, en donde, en
cada uno de los pasos al menos se modifica la posición de un nodo. Este camino
aleatorio no necesita estar relacionado con ningún proceso físico, la eficacia
en la estimación de $\langle A \rangle_{\kappa}$ es la única premisa. Este
camino aleatorio debe, a partir de una configuración inicial y tras un número
de pasos (termzalización), alcanzar la región de interés donde
$P(\mathcal{R})\propto e^{-\mathcal{F}(\kappa)}$. En esta región, se debe
recoger una muestra representativa de $M$ configuraciones, el problema es que
las configuraciones sucesivas no son independientes, pues no se
diferencian lo suficiente entre sí, esto implica que el error de $\langle A
\rangle_{\kappa}$ no será del orden de $O(\sqrt{M})$ sino de $O(\sqrt{M_I})$,
donde $M_I$ es el número de configuraciones independientes. 

% El cociente
% $\tau=M/M_I$ corresponderá entonces, al número pasos necesarios para obtener una
% configuración independiente a una de partida, se denomina tiempo de
% decorrelación. 

El camino aleatorio será una cadena de Markov, esto es, la configuración futura
depende solamente de la configuración actual, o dicho de otro modo, nuestro
\textit{caminante} decide a donde ir en base a su posición actual. Entonces,
el proceso está completamente descrito mediante un operador 
de transición $W(\mathcal{R},\mathcal{S})$, que determina la probabilidad
alcanzar una configuración $\mathcal{S}$ en el tiempo $t_0+1$ desde una
configuración $\mathcal{R}$ en el tiempo $t_0$. Algunas propiedades de este
operador son 
\begin{itemize}
\item $W(\mathcal{R},\mathcal{S})\geq 0$ para todas las configuraciones
  $\mathcal{R}$ y $\mathcal{S}$. Además $1=\int
  d\mathcal{S}\,W(\mathcal{R},\mathcal{S}) $, ya que es una probabilidad.
\item La probabilidad de alcanzar la configuración $\mathcal{U}$ desde
  $\mathcal{R}$ en $l$ pasos es:
  \begin{multline}
    P(\mathcal{U}_{l+s}|\mathcal{R}_s)=\int
  d\mathcal{R}_{l+s-1}\dots d\mathcal{R}_{s+2}
  d\mathcal{R}_{s+1}\\W(\mathcal{U},\mathcal{R}_{l+s-1})\dots
  W(\mathcal{R}_{s+2},\mathcal{R}_{s+1}) W(\mathcal{R}_{s+1},\mathcal{R}_{s})=W_l(\mathcal{U},\mathcal{R})
  \end{multline}
  también $W_l(\mathcal{U},\mathcal{R}) \geq 0$ y $1=\int
  d\mathcal{S}\, W_l(\mathcal{U},\mathcal{R})$.
\item Si en el tiempo $s$ la probabilidad de para la posición de nuestros
  caminantes es $\rho_s(\mathcal{R})$, en el tiempo $t+s$ será
\begin{equation}
\rho_{s+t}(\mathcal{S})=\int
  d\mathcal{R}\, W_l(\mathcal{R},\mathcal{S})\rho_s(\mathcal{R})
\end{equation}
\end{itemize}
Estas propiedades son generales para cualquier cadena de Markov. Todavía
tenemos dos propiedades cruciales para construir un método válido de Monte
Carlo:
\begin{itemize}
\item La condición de balance:
  \begin{equation}
    \frac{e^{-\mathcal{H}(R,\kappa)}}{\mathcal{Z}}=\int
    dS\, W(R,S)\frac{e^{-\mathcal{H}(S,\kappa)}}{\mathcal{Z}}
  \end{equation}
  Si la posición de un conjunto de caminantes está distribuida, en un tiempo
  $t_0$, de acuerdo a $\frac{e^{-\mathcal{H}(S,\kappa)}}{\mathcal{Z}}$,en el
  tiempo $t_0+1$, las posiciones cambiarán pero seguirán distribuyéndose de
  acuerdo al factor de Boltzmann.
\item Existe un entero $m_t$ tal que $W_l(R,S)>0$ para toda configuración $R$ y
  $S$ y todo $t>m_t$. 
\end{itemize}

El teorema que apoya el método de Monte Carlo dinámico es:

Para un operador que satisfaga la ecuación de balance TAL, tenemos que
\begin{equation}
\lim_{t\rightarrow \infty}W_t(R,S)=\frac{e^{-\mathcal{H}(R,\kappa)}}{\mathcal{Z}}
\end{equation}
Este teorema significa que no importa donde comience nuestro camino aleatorio,
siempre se alcanzará la región donde las configuraciones siguen una
distribución  de acuerdo al factor de Boltzamann. 

\subsubsection{construcción del camino}
\subsection{Método de Jacknife}

\section{Observables}

\subsection{Densidad esectral}
\subsection{Calor Específico}
\subsection{Radio de giro}
\subsection{Módulo de Poisson}

%%% Local Variables: 
%%% mode: latex
%%% TeX-master: "TFM"
%%% End: 
