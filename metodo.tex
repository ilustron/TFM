\chapter{Simulaciones numéricas}

\section{Modelo discreto}

Para la realización de simulaciones numéricas es necesario discretizar la
membrana cristalina. Para ello, vamos a considerar una red 
discreta de $N$ nodos (figura~\ref{red_rombo}), cuyas conexiones o enlaces
forman una red regular triangular. El contorno de esta red es un rombo de lado
$L=\sqrt{N}$, en este sentido podemos calificar la red como cuadrada. 
Las condiciones en la frontera son libres, lo que significa que el índice de
coordinación, el número de primeros vecinos\footnote{Entendemos por primeros
  vecinos de un nodo, aquellos nodos que tienen un sólo enlace con él.} de un
determinado nodo, es menor en el contorno. Esta falta de homogeneidad,
son los llamados efectos de borde, que producen que los valores 
de los observables en el interior son diferentes que en el contorno. Puesto
que la cantidad de nodos interiores de la membrana es proporcional a $L^2$, y
el número de nodos del contorno a $L$, es de esperar que los efectos de borde
disminuyan a medida que aumenta el tamaño del sistema hasta hacerse
despreciables. Aún así, la forma óptima del contorno de una
membrana para minimizar los efectos de borde es el hexágono, donde por
ejemplo, ningún nodo tiene un nodo tiene un índice de coordinación menor que
3, en nuestro caso, con el contorno romboidal el menor número de coordinación
es 2. Aunque el contorno romboidal tenga el inconveniente de mayores efectos
de borde, lo hemos elegido principalmente por poder comparar nuestras medidas
con trabajos anteriores, y si en algún caso influyen
demasiado en las medidas, podremos minimizarlos no teniendo en cuenta
aquellos nodos cercanos al contorno.

\begin{figure}[h]\label{red_rombo}
\centering
\resizebox{275bp}{!}{\input{red_triangular-fig}}
\caption{Conectividad intrínseca de la red}
\end{figure}

Cada nodo es etiquetado mediante dos índices discretos $P=(i,j)\ i,j=1,L$
como se indica en la figura~\ref{red_rombo}. La posición en el espacio
tridimensional vendrá dada por el vector $\vec{r}_P$. Como energía libre discreta del sistema usamos
\begin{equation}
\mathcal{H}=\sum_{\langle PQ \rangle}
|\vec{r}_P-\vec{r}_Q|^2+\kappa\sum_{\langle ab \rangle}
|\vec{n}_a-\vec{n}_b|^2 
\end{equation}
donde con $\langle\rangle$ indicamos que la suma es sobre primeros vecinos,
nodos adyacentes. Estamos considerando únicamente interacciones de corto
alcance,  no inlcuimos ningún término de exclusión, ya que, es irrelevante en
la fase plana. Los  índices en letra minúscula representan las plaquetas, 
los triángulos de la red, y $\vec{n}_{a}$ el vector normal a la plaqueta. La
constante $\kappa$ es el módulo de curvatura y hemos escalado las posiciones de
forma que la constante elástica es unitaria.

%Colapso punto

\subsection{Límite continuo}
\begin{figure}[h]
\centering
 \resizebox{\columnwidth}{!}{\input{base-hexagonal-fig}}
\caption{Base hexagono}
\end{figure}   
Consideramos un sistema de coordenadas ortogonal $\{ \vec{e}_1,\vec{e}_2\}$, son los vectores de la base del plano base. La expresión \eqref{drU} podemos discretizarla
\begin{equation}
(\vec{r}_P-\vec{r}_Q)^2=a^2+2u_{\alpha\beta}x{\scriptstyle [q]}^{\alpha}x{\scriptstyle [q]}^{\beta}
\end{equation}
donde $u_{\alpha\beta}$ es la versión discretizada del tensor de
deformaciones. En este  sistema de coordenadas la métrica es
\begin{equation}
g_{\alpha\beta}=\vec{e}_{\alpha}\cdot\vec{e}_{\beta}=a^2\delta_{\alpha\beta}
\end{equation}
Podemos aproximar
\begin{equation}
|\vec{r}_P-\vec{r}_Q|=\sqrt{a^2+2u_{\alpha\beta}e^{\alpha}e^{\beta}}\simeq a + \frac{u_{\alpha\beta}}{a}e^{\alpha}e^{\beta}
\end{equation}

Sustituyendo en la parte elástica de la energía libre
\begin{equation}\label{He}
\mathcal{H}_E\simeq \frac{1}{2}\sum_{P} \sum^6_{j=1} \left(\frac{u_{\alpha\beta}}{a}e^{\alpha}e^{\beta}\right)^2=\frac{1}{2a^2}\sum_{P} \sum^6_{j=1} u_{\alpha\beta}u^{\mu\nu}e^{\alpha}e^{\beta}e_{\mu}e_{\nu}
\end{equation}
Siendo el primer sumatorio sobre todos los nodos y el segundo sobre los
primeros vecinos. Teniendo en cuenta las relaciones
\begin{equation}
1=\frac{\delta^{\alpha\beta}\delta_{\alpha\beta}}{2}=\frac{\delta^{\mu\nu}\delta_{\mu\nu}}{2}
\end{equation}
Podemos escribir el término
\begin{equation}
e^{\alpha}e^{\beta}e_{\mu}e_{\nu}=\frac{1}{4}\delta^{\alpha\beta}\delta_{\mu\nu}\delta_{\alpha\beta}e^{\alpha}e^{\beta}\delta^{\mu\nu}e_{\mu}e_{\nu}=\frac{a^2}{4}\delta^{\alpha\beta}\delta_{\mu\nu}
\end{equation}
Y de forma análoga también podemos obtener
\begin{align}
e^{\alpha}e^{\beta}e_{\mu}e_{\nu}&=\frac{a^2}{4}\delta^{\alpha}_{\ \mu}\delta_{\beta}^{\
  \nu}\\
e^{\alpha}e^{\beta}e_{\mu}e_{\nu}&=\frac{a^2}{4}\delta^{\alpha}_{\ \nu}\delta_{\beta}^{\
  \mu}
\end{align}
Tomando la media de las últimas tres expresiones
\begin{equation}
e^{\alpha}e^{\beta}e_{\mu}e_{\nu}=\frac{a^4}{12}(\delta^{\alpha\beta}\delta_{\mu\nu}+
\delta^{\alpha}_{\ \mu}\delta_{\beta}^{\ \nu}+
\delta^{\alpha}_{\ \nu}\delta_{\beta}^{\ \mu})
\end{equation}
Sustituyendo en \eqref{He}
\begin{equation}
\mathcal{H}_E\simeq\frac{a^2}{24}\sum_{P} \sum^6_{j=1}(\delta^{\alpha\beta}\delta_{\mu\nu}+\delta^{\alpha}_{\ \mu}\delta_{\beta}^{\ \nu}+
\delta^{\alpha}_{\ \nu}\delta_{\beta}^{\ \mu})
u_{\alpha\beta}u^{\mu\nu}=\frac{a^2}{4}\sum_{P}u_{\alpha}^{\ \alpha}u^{\alpha}_{\ \alpha}+2u_{\alpha\beta}u^{\alpha\beta}
\end{equation}
Donde en  la última igualdad hemos efectuado la suma sobre todos los vecinos.
Pasamos al límite continuo:
\begin{equation}
\frac{a^2}{4}\sum_{i=1}^L\sum_{j=1}^L\frac{\Delta i\Delta j}{\xi^2\frac{\sqrt{3}}{2}}\xi^2\,\frac{\sqrt{3}}{2}(u_{\alpha}^{\ \alpha}u^{\alpha}_{\ \alpha}+2u_{\alpha\beta}u^{\alpha\beta})
\end{equation}
Tomando el límite $N\rightarrow \infty$
elástica 
\begin{equation}
\mathcal{H}\simeq\int d^2\mathbf{x}\, \frac{\sqrt{3}}{4}\frac{a^2}{\xi^2}\left(\frac{1}{2}u_{\alpha}^{\ \alpha}u^{\alpha}_{\ \alpha}+u_{\alpha\beta}u^{\alpha\beta}\right)
\end{equation}
recuperamos la energía libre elástica con 
\begin{equation}
\lambda=\mu=\frac{\sqrt{3}}{4}\frac{a^2}{\xi^2}
\end{equation}
Nuestro modelo corresponde al límite $a\rightarrow 0$ y $\xi$ fijo.

Respecto al término de curvatura, podemos desarrollar $\vec{n}_{b}$ desde $a$
\begin{equation}
\vec{n}_b\simeq\vec{n}_a+\frac{\partial \vec{n_b}}{\partial x^{\alpha}}(x_b^{\alpha}-x_a^{\alpha})
\end{equation}
Sustituyendo en la energía libre
\begin{equation}
\mathcal{H}_C=\frac{1}{2}\kappa\sum_{\langle ab\rangle}\frac{\partial
  \vec{n_b}}{\partial x^{\alpha}}\cdot\frac{\partial \vec{n_b}}{\partial x^{\beta}} (x_b^{\alpha}-x_a^{\alpha})(x_b^{\beta}-x_a^{\beta})
\end{equation}
Podemos considerar que son triángulos equiláteros
\begin{equation}
g^{\alpha\beta}=(x_b^{\alpha}-x_a^{\alpha})(x_b^{\beta}-x_a^{\beta})\simeq
\sqrt{3} \delta^{\alpha\beta}
\end{equation}
Entonces
\begin{equation}
\mathcal{H}_C=\frac{\sqrt{3}}{2}\kappa\sum_{\langle ab\rangle}\frac{\partial
  \vec{n_b}}{\partial x_{\alpha}}\cdot\frac{\partial \vec{n_b}}{\partial x^{\alpha}}
\end{equation}
En el límite continuo, para $N\rightarrow \infty$
\begin{equation}
\mathcal{H}_C=\frac{1}{2}\hat{\kappa}\int d^2\mathbf{x}\ \frac{\partial
  \vec{n_b}}{\partial x_{\alpha}}\cdot\frac{\partial \vec{n_b}}{\partial x^{\alpha}}
\end{equation}
A partir de la expresión \eqref{derivada_n}
\begin{equation}
\frac{\partial \vec{n_b}}{\partial x_{\alpha}}\cdot\frac{\partial
  \vec{n_b}}{\partial x^{\alpha}}=(K_{\alpha}^{\ \beta}\vec{t}_ {\beta})\cdot(K^{\nu
  \alpha}\vec{t}_{\nu})=K_{\alpha}^{\ \beta}g_ {\beta\nu}K^{\nu \alpha}=K_{\alpha\beta}K^{\alpha\beta}
\end{equation}
Entonces 
\begin{equation}
\mathcal{H}_C\simeq\frac{1}{2}\hat{\kappa}\int d^2\mathbf{x}\ K_{\alpha\beta}K^{\alpha\beta}
\end{equation}
Y recuperamos la ecuación \eqref{ELandau_curvatura1} de la energía libre de Landau.

\section{Algoritmo de Metropolis}
Una determinada configuración $R$ de la membrana cristalina discreta 
corresponde a una lista de $N$ posiciones tridimensionales:
\begin{equation}
R=\{ \vec{r}_1,\vec{r}_2,\dots \vec{r}_N\}\,.
\end{equation}
Supongamos que hemos generado $M$ configuraciones independientes que siguen una distribución
proporcional al factor $e^{-\mathcal{H}(\kappa)}$. A partir de las medidas
$A_1,A_2,\dots A_N$ en cada configuración de un observable $A$, podemos
estimar su valor medio para ese $\kappa$
\begin{equation}
\langle A \rangle_{\kappa}\simeq \frac{1}{M}\sum^M_{i=1} A_i\,.
\end{equation}
Para generar estas configuraciones empleamos un método de Monte Carlo, que
consiste en un camino aleatorio en el espacio de configuraciones, en donde, en
cada uno de los pasos al menos se modifica en la posición de un nodo. Este camino
aleatorio no necesita estar relacionado con ningún proceso físico, la eficacia
en la estimación de $\langle A \rangle_{\kappa}$ es la única premisa. Este
camino aleatorio debe, a partir de una configuración inicial y tras un número
de pasos (termalización), alcanzar la región de interés donde la probabilidad
de encontrar una configuración dada es proporcional al factor de Boltzmann
$e^{-\mathcal{H}(\kappa)}$. El inconveniente de este método es que $M$ 
configuraciones sucesivas en esta región de interés no son independientes, pues no se
diferencian lo suficiente entre sí, están correlacionadas, lo que implica que el
error de la estimación de $\langle A \rangle_{\kappa}$ con estas $M$
configuraciones no será del orden de $O(\sqrt{M})$ sino de $O(\sqrt{M_I})$,
donde $M_I$ es el número de configuraciones independientes.  

El camino aleatorio será una cadena de Markov: La configuración futura
depende solamente de la configuración actual, o dicho de otro modo, nuestro
\textit{caminante} decide a donde ir en base a su posición actual. Entonces,
el proceso está completamente descrito mediante un operador 
de transición $W(R,S)$, que determina la probabilidad
alcanzar una configuración $S$ en el tiempo $t_0+1$ desde una
configuración $R$ en el tiempo $t_0$. Algunas propiedades de este
operador son 
\begin{itemize}
\item $W(R,S)\geq 0$ para todas las configuraciones
  $R$ y $S$. Además $1=\int
  dS\,W(R,S) $, ya que es una probabilidad.
\item La probabilidad de alcanzar la configuración $U$ desde
  $R$ en $l$ pasos es el producto de los operadores intermedios para cada paso 
  \begin{multline}
    P(U_{l+m}|R_m)=\int
  dR_{l+m-1}\dots dR_{m+2}
  dR_{m+1}\\\times W(U,R_{l+m-1})\dots
  W(R_{m+2},R_{m+1}) W(R_{m+1},R_{m})=W_l(\mathcal{U},R)
  \end{multline}
  también $W_l(\mathcal{U},R) \geq 0$ y $1=\int
  dS\, W_l(\mathcal{U},R)$.
\item Si en el tiempo $t_0$ la probabilidad densidad de probabilidad de
  encontrar a nuestro caminante en una región $dR$ es $\rho_{t_0}(R)dR$, en el tiempo $t_0+t$ será
\begin{equation}
\rho_{t_0+t}(S)=\int
  dR\, W_l(R,S)\rho_s(R)
\end{equation}
\end{itemize}
Estas propiedades son generales para cualquier cadena de Markov. Las
siguientes propiedades son necesarias para construir nuestro método de Monte
Carlo:
\begin{itemize}
\item La condición de balance:
  \begin{equation}\label{ecuacion_balance}
    \frac{e^{-\mathcal{H}(R,\kappa)}}{\mathcal{Z}}=\int
    dS\, W(R,S)\frac{e^{-\mathcal{H}(S,\kappa)}}{\mathcal{Z}}
  \end{equation}
  Si la posición de un conjunto de caminantes está distribuida, en un tiempo
  $t_0$, de acuerdo a $\frac{e^{-\mathcal{H}(S,\kappa)}}{\mathcal{Z}}$,en el
  tiempo $t_0+1$, las posiciones cambiarán, resultando una nueva configuración
  $S$, pero seguirán distribuyéndose de acuerdo al factor de Boltzmann $e^{-\mathcal{H}(R,\kappa)}$.
\item Todas las configuraciones son accesibles: Existe un entero $m_t$ tal que
  $W_{m_t}(R,S)>0$ para toda configuración $R$ y $S$ y todo $t>m_t$. 
\end{itemize}

Es posible demostrar que un operador de transición que satisfaga las dos
últimas propiedades cumplirá que
\begin{equation}
\lim_{t\rightarrow \infty}W_t(R,S)=\frac{e^{-\mathcal{H}(R,\kappa)}}{\mathcal{Z}}
\end{equation}
Este teorema nos asegura que nuestro caminante siempre alcanzará la región de
interés, no importa donde comience, siempre se alcanzará la región donde las
configuraciones siguen una distribución de acuerdo al factor de Boltzmann. 

Podemos descomponer el operador de transición $W(R,S)$ en operadores
individuales $w^i(R,S)$, en los que sólo varía la posición $\vec{r}_i$. El
operador de transición total corresponderá a aplicar sucesivamente los $N$
operadores individuales. Si el orden de aplicación es siempre el mismo tenemos
\begin{equation*}
w(R,S)=\prod^{N}_{i=1}w^i(R,S)
\end{equation*}
Si los operadores de transición cumplen la condición de
balance \label{ecuacion_balance}, el operador total también la cumplirá, dado
que es un producto de operadores. Esto lo podemos conseguir exigiendo que los operadores
$w^i(R,S)$ cumplan
\begin{equation}\label{balance_detallado}
w^i(R,S)\frac{e^{-\mathcal{H}(R,\kappa)}}{\mathcal{Z}}=w^i(S,R)\frac{e^{-\mathcal{H}(S,\kappa)}}{\mathcal{Z}}\Rightarrow \frac{w^i(R,S)}{w^i(S,R)}=e^{-[\mathcal{H}(S,\kappa)+\mathcal{H}(R,\kappa)]}
\end{equation}
dado que $1=\int dS w^i(R,S)$. Una forma de construir estos operadores de
transición individuales es el algoritmo de Metropolis, que consiste en
descomponer el operador de transición individual en 
 \begin{equation*}
 w^i(R,S)=A_{RS}\omega^i(R,S)
\end{equation*}
donde $A_{RS}$ es un factor simétrico, la probabilidad de alcanzar la
configuración $S$ desde $R$, es la misma que alcanzar $R$ desde $S$; y
$\omega^i(R,S)$ es un peso elegido como función del cociente entre los
facotres de Boltzmann
\begin{equation*}
\omega^i(R,S)=g(e^{-[\mathcal{H}(S,\kappa)+\mathcal{H}(R,\kappa)]}).
\end{equation*}
La condición \label{balance_detallado} será satisfecha si la función $g$
cumple
\begin{equation}
\frac{g(z)}{g(1/z)}=z.
\end{equation}
El algoritmo de Metropolis utiliza $g_M(z)=min\{1,z\}$ y lo hemos implementado de
 la siguiente forma:
\begin{itemize}
\item A partir de la configuración $R$ de la membrana proponemos como
  siguiente configuración $S_i$, que únicamente se diferencia de $R$ en la
  posición de un punto que ha sido modificada en un vector $\vec{\epsilon}$,
  elegido aleatoriamente y uniformemente en un cubo de volumen 
  $(2\varepsilon)^3$. Estamos proponiendo un cambio de forma simétrica, a
  partir de $S$ también es posible proponer $R$ como siguiente configuración
  con la misma probabilidad.
\item Calculamos la diferencia de energía entre las dos configuraciones
  $\Delta\mathcal{H}=\mathcal{H}_S-\mathcal{H}_R$:
  \begin{itemize}
    \item Si es negativa aceptamos la configuración $S$.
    \item Si es positiva, aceptamos la configuración con una probabilidad $e^{-\Delta\mathcal{H}}$.
  \end{itemize}
\end{itemize}

Un barrido de corresponde a aplicar los pasos anteriores de forma secuencial a
todos los $N$ puntos de la membrana. El valor de $\varepsilon$ se ajusta de
modo que en cada barrido el número de cambios aceptados sea la mitad del total
de puntos que componen la superficie. 
 
\subsection{Tratamiento de los datos}

Puesto que las sucesivas configuraciones de una evolución de Monte Carlo están
correlacionadas, en las simulaciones numéricas se guardaban las configuraciones cada $\tau_0$
barridos y partir de un cierto valor $t_{termal}$, ambos son propuestas
iniciales del valor del periodo de correlación entre medidas y de la
termalización. Estos valores iniciales se estiman a partir de los resultados de las
simulaciones realizadas en el trabajo anterior \cite{Bowick_flat_phase}, ya 
que utilizaron el mismo algoritmo que nosotros. Posteriormente, para comprobar
si efectivamente las medidas de un observable efectivamente han termalizado, se
gráfica el valor que toma el observable en las
configuraciones sucesivas en escala logarítmica, el gráfico histórico (figura
\ref{grafico_historico}). También se representa el promedio del observable,
resultado de eliminar configuraciones iniciales, en función del inverso del
número de configuraciones conservadas (figura \ref{grafico_logtermal}). De
esta forma, en ambos gráficas, se comprueba si hay una tendencia inicial en
los valores medidos. Como se observa, la primera gráfica constituye un método
para encontrar el valor de la termalización más fino que la segunda,
en la que incluso se pueden dar resultados falsos debido a
fluctuaciones. Posteriormente veremos que la utilidad de esta última gráfica
es doble. 

\begin{figure}[h]
\centering
 \resizebox{\columnwidth}{!}{\input{ejemplo_termal-fig}}\label{grafico_historico}
\caption{Gráfico histórico de medidas para $\langle R_G^2\rangle$}\label{grafico_historico}
\end{figure}    

\begin{figure}[h]
\centering
 \resizebox{\columnwidth}{!}{\input{ejemplo_logtermal-fig}}
\caption{Gráfico del promedio del radio de giro, resultado de eliminar
  configuraciones iniciales, en función del inverso del número de configuraciones conservadas}\label{grafico_logtermal}
\end{figure}

Tanto para estimar el periodo de correlación de las medidas, como para el
cálculo de los errores, en las estimaciones de los observables, 
utilizamos el método del jacknife \cite{Juan:tesis}, que tiene en cuenta la
falta de independencia de las medidas. Consideremos los valores
$\{Q_i\,;\, i=1,\dots M\}$ que un observable $Q$ toma en una evolución de Monte
Carlo. La estimación del valor medio será
\begin{equation}
\bar{Q}=\frac{1}{M}\sum^M_{i=1}Q_i.
\end{equation}
Para estimar el error se divide el conjunto de $M$ valores en $K$ bloques
iguales, tendrán $m=M/K$ datos cada uno. Definimos $\bar{Q}^m_k \, \{k=1,\dots
M\}$ como el promedio de los datos después de eliminar el bloque $k$-ésimo
\begin{equation}
\bar{Q}_k^m=\frac{1}{M-m}\left(\sum^{m(k-1)}_{i=1}Q_i+\sum^{M}_{i=mk+1}Q_i\right).
\end{equation}
La estimación del error para el tamaño de bloque $m$ es
\begin{equation}
\delta\bar{Q}_m=\sqrt{\frac{K-1}{K}\sum^K_{k=1}(\bar{Q}^m_k-\bar{Q})}.
\end{equation}
Esta última expresión da una estimación no sesgada del error, para comprobarlo
basta con tomar la esperanza $\langle\dots \rangle$ a ambos lados de la ecuación
\begin{equation}
\langle (\delta\bar{Q}_n)^2\rangle=\sigma^2(Q)\frac{1}{N} 
\end{equation}

\begin{figure}[h]
\centering
 \resizebox{\columnwidth}{!}{\input{ejemplo_error-fig}}
\caption{Estimación del error para $\langle R_G^2\rangle$ en función del tamaño del bloque}\label{figura_ejemplo_error}
\end{figure} 

Si el tamaño del bloque, $m$, es muy pequeño comparada con la longitud de
correlación entre medidas, el estimador $\delta\bar{Q}_m$ aumentará en un
factor $\sqrt{2}$ cada vez que doblamos el tamaño del bloque (figura
\ref{figura_ejemplo_error}). Este crecimiento cesará cuando el tamaño del
bloque sea mayor que la correlación entre medidas. A partir de aquí, el valor
del error se mantiene constante hasta que el número de bloques es demasiado
pequeño (2 o 3), que vuelve a crecer con un factor $\sqrt{2}$. En la gráfica
mostrada en la figura \ref{figura_ejemplo_error} la estimación del error es
$\simeq 0.015$. Para comprobar esta estimación se utiliza la gráfica de la
figura \ref{grafico_logtermal}, según la cual la
dispersión máxima de la media del radio de giro (la diferencia del valor máximo y el
mínimo) es $0.04$, que se corresponde aproximadamente con el doble de la
estimación del error, por tanto, la estimación es correcta.
\clearpage
\section{Observables}

Para estudio de la transición de fase se miden los siguientes observables en
un rango de valores $\kappa$ centrado en $0.8$, ya que anteriores estudios \cite{Bowick_flat_phase}
estiman que la transición de fase ocurre en $\simeq 0.79$:
\begin{description}
\item[Calor específico:] Utilizamos la expresión \cite{Harnish:CV}
\begin{equation}\label{CV_discreto}
 C_V=\frac{3(N-1)}{2}+\frac{\kappa^2}{N}(\langle E_C^2 \rangle-\langle E_C
\rangle^2),
\end{equation}
donde
\begin{equation}
E_C=\sum_{\langle ab \rangle}\vec{n}_a\cdot\vec{n}_b, 
\end{equation}
es la energía de curvatura.
\item[Radio de giro:] Cuya expresión discreta es 
\begin{equation} R_g^{2}=\frac{1}{3N}\left\langle \sum_{i}
  \vec{R}_i\cdot\vec{R}_i\right\rangle=\frac{1}{3N}(\langle\vec{R}^2 \rangle
-\langle\vec{R} \rangle^2),
\end{equation}
donde $\vec{R}_i=\vec{r}_{CM}-\vec{r}_i $ el es vector de posición del nodo
$i$ relativo al centro de masas $\vec{r}_{CM}$ de la membrana.

\item[Variación del radio de giro:] A partir del teorema de la respuesta
  lineal \cite{Binney:critical_phenomema}, podemos calcular la derivada del
  radio de giro respecto a $\kappa$ (temperatura) 
\begin{equation}
  \frac{\partial R_g^2}{\partial \kappa}=\langle R_g^2 E_C \rangle-\langle R_G^2\rangle\langle E_C\rangle\equiv\langle R_g^2E_C \rangle_c,
\end{equation}
donde $\langle R_g^2E_C \rangle_c$ es la correlación conexa del radio de giro
y la energía de curvatura.
\end{description}

\subsection{Estimación de los exponentes críticos}

Puesto que la transición de fase plana-rugosa de la membrana cristalina es de
segundo orden, tanto $C_V$ como $\frac{\partial R_g^2}{\partial \kappa}$  con
$L\rightarrow \infty$ tendrán un comportamiento divergente en las proximidades
de la transición. Puesto en las simulaciones numéricas, nuestro sistema
discreto es finito, es imposible obtener magnitudes divergentes, todas las
magnitudes que podemos obtener serán funciones analíticas de los parámetros de
la teoría (sumas de funciones analíticas). Aunque, podemos inferir sobre el
comportamiento de los parámetros en el límite termodinámico observando como
crecen los observables del sistema finito en función del tamaño $L$ \cite{Juan:tesis}. 
Para ello, consideramos que nuestro sistema presenta una transición de fase de
segundo orden a una temperatura, cuyo valor en el límite termodinámico es
$\kappa_c\!\equiv\!\kappa_c{\scriptstyle
  (L=\infty)}\!=\!\kappa_c(\infty)$. Según nos vamos acercando a $\kappa_c$ la
longitud de correlación $\xi$ va creciendo hasta que satura el tamaño del
sistema y comienzan a aparecer los efectos de tamaño finito. Asumiendo la
anterior afirmación podemos deducir como escala $\kappa_c(L)$ con $L$ ya que
\begin{align}
\xi&\propto L\\ 
\xi&\sim (\kappa(L)-\kappa_c(\infty))^{-\nu}
\end{align}
entonces
\begin{equation}
\kappa_c(L)-\kappa(L)\sim L^{-\frac{1}{\nu}}
\end{equation}
En las proximidades de la transición de fase, donde la longitud de correlación
es del orden del tamaño del sistema, es razonable suponer que la dinámica está
gobernada por una sola variable $L/\xi\sim L \kappa_L^{\nu}$, donde $\kappa_L$
es la \textit{temperatura reducida}
\begin{equation}
\kappa_L\equiv \kappa(L)-\kappa_c(\infty).
\end{equation}
Esto implica que un comportamiento, para un observable genérico $\mathcal{O}$, del tipo
\begin{equation}
\mathcal{O}_L\sim L^{\frac{x_0}{\nu}}f(L \kappa_L^{\nu}),
\end{equation}
donde $f$ es una función ``universal'' en el sentido que es independiente de
$L$ pero dependiente del observable $\mathcal{O}$. Ahora bien, dado que
debemos recuperar el comportamiento descrito en \ref{exponentes_criticos}
cuando $L\rightarrow \infty$ 
\begin{equation}
\mathcal{O}_L\sim \kappa_L^{-x_0},
\end{equation}
tenemos que la función de escala $f(x)$ debe cumplir que
\begin{equation}
f(x)\sim x^{-\frac{x_0}{\nu}}\quad x\rightarrow \infty.
\end{equation}
Analicemos el valor máximo del observable respecto a $\kappa$, para ello
igualamos la derivada a cero
\begin{equation}
\frac{d\mathcal{O}_L}{\kappa_L}=L^{\frac{x_0+1}{\nu}}\tilde{f}(L^{1/\nu} \kappa_L)=0,
\end{equation}
donde hemos definido $\tilde{f}(L^{1/\nu}\kappa_L)=f(L\kappa_L^{\nu})$. Suponiendo que
$\tilde{f}(x)$ tiene un máximo en $x_C$ (este máximo no depende del tamaño del
sistema), podemos encontrar la relación de escala
temperatura $\kappa_c(L)$ a la que tiene lugar el máximo del observable
\begin{equation}
x_C=L^{1/\nu}(\kappa_c(L)-\kappa_c(\infty))\Rightarrow \kappa_c(L)=\kappa_c(\infty)+x_CL^{1/\nu}
\end{equation}
$\kappa_c(L)$ es la temperatura crítica aparente, es la temperatura a la cual
la longitud de correlación satura con el tamaño del sistema. El valor que toma
el observable en este máximo escalará con el tamaño del sistema como
\begin{equation}
\mathcal{O}^{max}_L\sim L^{\frac{x_0}{\nu}}.
\end{equation}

Resumiendo, tenemos las siguientes relaciones para los observables de nuestro sistema:
\begin{itemize}
\item Tanto el máximo del calor específico $C_V$ como de la variación del
  radio de giro $\frac{\partial R_g^2}{\partial \kappa}$, ocurren a una temperatura crítica
  aparente $\kappa_c(L)$, la cual está relacionada con la temperatura crítica
  en el límite termodinámico $\kappa_c(\infty)$, por
  \begin{equation}
    \kappa_c(L)=\kappa_c(\infty)+C_0 L^{-1/\nu},
  \end{equation}
  donde la constate $C_0$ depende del observable, no será la misma para $C_V$
  que para $\frac{\partial R_g^2}{\partial \kappa}$.
\item El máximo $C_V$ del sistema finito está dado por la siguiente relación 
  \begin{equation}
    C^{\max}_v(L)=C_1+C_2L^{\alpha/\nu},
  \end{equation}
  donde se ha añadido la constante $C_1$ debido a correcciones analíticas
  \cite{Cardy}. 
\item Para hallar la relación de escala que sigue la variación del radio de
  giro, debemos tener en cuenta que el radio de giro escala con el tamaño del
  sistema como
  \begin{equation}
    R^2_G(L)\simeq L^{2\nu_F}f_{R_G}\left(L^{1/\nu}(\kappa(L)-\kappa_c(\infty))\right),
  \end{equation}
  donde $f_{R_G}(x)$ es la función de escala del radio de giro. Derivando esta
  última expresión respecto $\kappa(L)$
  \begin{equation}\label{escala_R2g_conexo}
    \frac{\partial R_g^2(L)}{\partial \kappa}\simeq L^{2\nu_F+\frac{1}{\nu}}f'_{R_G}\left(L^{1/\nu}(\kappa(L)-\kappa_c(\infty))\right.)
  \end{equation}
  Entonces el máximo de $\frac{\partial R_g^2(L)}{\partial \kappa}$ se
  comportará como
  \begin{equation}
    \left( \frac{\partial R_g^2(L)}{\partial \kappa}\right)_{max}=C_3  L^{2\nu_F+\frac{1}{\nu}},
  \end{equation}
  siendo $C_3$ una constante.
\end{itemize}

Para encontrar el valor máximo de un observable, a partir de los resultados de
las simulaciones numéricas, utilizamos el método de la densidad espectral que
se describe a continuación:

\subsubsection{Método de la densidad espectral}\label{densidad_espectral-sec}

Consideremos una evolución de Monte Carlo, cada una de las configuraciones
generadas tiene una energía $E$, y denotamos por $\bar{N}(E)$ a la frecuencia
relativa de la energía $E$, el cociente del número de configuraciones con
energía $E$ entre el número total de configuraciones. Cuando el número de
configuraciones generadas tiende a infinito, la frecuencia relativa de la
energía $E$ tiende a la probabilidad $P(E)$ de que el sistema tenga una
energía $E$. De la mecánica estadística clásica sabemos que esta probabilidad
viene dada por
\begin{equation}
P(E)=\frac{1}{\mathcal{Z}}W(E)e^{-\beta E},  
\end{equation}
donde $W(E)$ es la densidad de estados de energía $E$. Si definimos $f\equiv
\beta F$, donde $F$ es la energía libre extensiva tenemos
\begin{equation}
f=-\log \mathcal{Z}.
\end{equation}
Y en el límite en el cual el número de configuraciones es infinito encontramos
que
\begin{equation}
P(E)\equiv p_{\beta}=W(E)e^{-\beta E+f} ,
\end{equation}
lo que significa que podemos estimar numéricamente la densidad de estados
salvo el factor $e^{-f}$ \cite{Juan:tesis}. Con esta estimación ya podemos calcular la
probabilidad de la energía a otra $\beta$ y relacionarla con la ya calculada
\begin{equation}
p_{\beta'}(E)=\frac{p_{\beta}(E)e^{-(\beta'-\beta)E}}{\sum_E e^{-(\beta'-\beta)E}},
\end{equation}
también podemos estimar el valor esperado de un observable $\langle
O\rangle_{\beta'}$  para una $\beta'$ partir del valor $\langle
O\rangle_{\beta}$ a otra beta
\begin{equation}
\langle O\rangle_{\beta'}=\frac{\langle
O\rangle_{\beta} e^{-(\beta'-\beta)E}}{\sum_E e^{-(\beta'-\beta)E}},
\end{equation}

Las ecuaciones son exactas cuando el número de configuraciones es infinito,
mientras que para una simulación de Monte Carlo serán aproximadas. Dada una
simulación en $\beta$ podremos calcular $P_{\beta'}(E)$ siempre y cuando la
diferencia $\beta'-\beta$ sea pequeña. Podemos cuantificar lo anterior de la
siguiente manera: Todas las energías mayores o menores que $E\pm A \sigma(E)$,
con $A\sim 1$, deben estar suprimidas en la nueva $P_{\beta'}$ ya que la
estimación de $P_{\beta}$ apenas nos da información sobre estas energías al
estar a más de una desviación típica de la media. Podremos exigir, por ejemplo
que  
\begin{equation}\label{beta_sigma}
|\beta-\beta'|\sigma(E)=A.
\end{equation}
Con $A=3$ la nueva probabilidad tendrá un factor $e^{-3}$ respecto a la
probabilidad máxima en esos puntos, es decir, prácticamente no contribuirá.
Además es interesante realizar la simulación lo más cerca posible de la
transición. Aunque cerca de la transición, las fluctuaciones serán máximas y ,según
 la ecuación anterior \eqref{beta_sigma}, nos podremos mover muy poco de la
 temperatura de la simulación. A pesar de este inconveniente, el método de la
 densidad espectral se ha revelado muy útil para localizar la transición.

\subsection{Módulo de Poisson}

A partir del teorema de la respuesta lineal \cite{Binney:critical_phenomema}
tenemos que
\begin{equation}
\langle u_{\alpha\beta}u_{\gamma\delta} \rangle_c
=\frac{1}{\beta A}\left(\frac{\partial u_{\alpha\beta}}{\partial \sigma^{\gamma\delta}}\right),
\end{equation}
donde $A$ es área de la membrana. Por otro lado, derivando la expresión
\eqref{hooke} respecto a $\sigma^{\gamma\delta}$
\begin{equation}\frac{\partial u_{\alpha\beta}}{\partial
  \sigma^{\gamma\delta}}=-\frac{K-\mu}{4\mu K}\delta_{\alpha\beta}\delta_{\gamma\delta}+\frac{1}{4\mu}(\delta_{\alpha\gamma}\delta_{\beta\delta}+\delta_{\alpha\delta}\delta_{\beta\gamma})
\end{equation}
Usando estas dos últimas expresiones encontramos que 
\begin{align}
\langle u_{11}u_{22} \rangle_c&=-\frac{1}{\beta A}\frac{K-\mu}{4\mu K}\\
\langle u_{12}u_{12} \rangle_c&=\frac{1}{\beta A}\frac{1}{4\mu}\\
\langle u_{11}^2 \rangle_c=\langle u_{22}^2 \rangle_c&=\frac{1}{\beta A}\frac{K+\mu}{4\mu K}\\
\end{align}
A partir de estas expresiones podemos encontrar una definición alternativa
para el módulo de Poisson \cite{Zang_Dmolecular,Parrinello_Crystal} en
función de las correlaciones conexas del tensor 
de deformaciones
\begin{equation}
\sigma=\frac{K-\mu}{K+\mu}=-\frac{\langle u_{11}u_{22}
  \rangle_c}{\langle u_{22}^2 \rangle_c}
\end{equation}
Teniendo en cuenta \eqref{tensor_metrica} podemos expresar la anterior
expresión en función de la métrica inducida 
\begin{equation}
\sigma=\frac{K-\mu}{K+\mu}=-\frac{\langle g_{11}g_{22}
  \rangle_c}{\langle g_{22}^2 \rangle_c}
\end{equation}
Para nuestra superficie discreta podemos estimar la métrica inducida
aproximando los vectores tangentes por enlaces entre nodos en la diferentes
direcciones. Ahora bien, al ser nuestra red triangular únicamente tenemos acceso a
$\partial_i \vec{r}$, con $i=1,2,3$ las direcciones naturales de la red
triangular, las cuales no son ortogonales y en la definición de $\sigma$ se
asume implícitamente que lo son. Por tanto, debemos hacer un cambio de coordenadas, si
$\vec{e}_1,\vec{e}_2,\vec{e}_3$ son la base natural de vectores de la red
triangular elegimos la dirección $\vec{x}_1$ en la dirección $\vec{e}_1$ y
para $\vec{x}_2$ usamos $(\vec{e}_2+\vec{e}_3)/\sqrt{3}$, obteniendo así una
base ortonormal con la que poder calcular $\sigma$ \cite{Bowick_poisson_ratio}.

%%% Local Variables: 
%%% mode: latex
%%% TeX-master: "TFM"
%%% End: 
