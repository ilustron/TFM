\chapter{Introducción}

Las membranas se pueden considerar como la generalización bidimensional (2D)
de los polímeros, cadenas unidimensionales de moléculas (monómeros). En este
sentido, la generalización inmediata es una membrana cristalina, una
red regular 2D cuyos nodos se encuentran en un espacio tridimensional. Las conexiones
entre los nodos, los enlaces, permanecen inalteradas
 bajo cualquier deformación, por tanto, el número coordinación fijo en cada
 nodo. Esta característica es el motivo de otras denominaciones para este tipo
 de membranas como membranas "conectadas" (\textit{conected,thetered}) o "polimerizadas".  

Si la membrana es isótropa, su estructura interna no depende de la dirección,
y existen dos modos de deformación elástica: Compresión, preserva
los ángulos bajo un escalamiento uniforme de su dimensión lineal, y de
cizalladura, que no altera el área total. La resistencia de la membrana a
estos dos tipos de deformación está descrita por dos parámetros, el módulo
elástico de compresión y de cizalladura; de manera que, si una membrana
presenta valores altos de estos módulos elásticos, no sufre grandes
deformaciones comparada con otra que tiene unos menores módulo elásticos para
una misma tensión aplicada. El efecto de una tensión aplicada en una
dirección dada, será una deformación paralela y transversal en la dirección de
la tensión. La razón entre estas deformaciones está caracterizada por el
módulo de Poisson. Las membranas cristalinas presentan un valor
negativo para el módulo de Poisson, comportamiento contrario al de la mayoría
de materiales, pues se expanden transversalmente al ser estiradas. 
Si la distancia entre nodos es constante durante la deformación, la membrana
únicamente puede curvarse, este tipo de deformación, al igual que las
deformaciones elásticas, tiene asociado un parámetro que caracteriza la
resistencia de la membrana a ser deformada, la resistencia o rigidez de
curvatura. Para caracterizar la extensión de la membrana, se puede utilizar el
radio de giro $R^2_g$, la desviación cuadrática media de los puntos de la
membrana respecto al centro de masas, que representa el tamaño de una membrana cuadrada
de $L^2$ puntos. Si la membrana está completamente plana, correspondiente a
$T\rightarrow 0$, el radio de giro sigue la ley de escala $R^2_g\simeq L^2$. En cambio,
para $T>0$, debido a las fluctuaciones térmicas, tenemos que para una \textit{membrana
fantasma} (puede cruzarse a sí misma sin coste energético) el radio de giro
obedece $R^2_g\simeq \ln L$. Este cambio drástico en la extensión significa
que entre ambos comportamientos hay una transición de fase, entre la fase
plana y la rugosa. Simulaciones numéricas muestran que está transición de fase
es de segundo orden.

Como ejemplo de membranas cristalinas encontramos el citoesqueleto de los
eritrocitos (glóbulos rojos), una red de proteínas que se encuentra entre
la bicapa lipídica de la membrana celular. Gracias a las propiedades de está
membrana cristalina los erotrocitos son capaces de recorrer delgados capilares
y después recuperar su forma original. Como ejemplos inorgánico es el grafeno,
con prometedoras en el campo de la electrónica, que consiste en una red
bidimensional hexagonal de átomos de carbono.

En el presente trabajo ...



  
%%% Local Variables: 
%%% mode: latex
%%% TeX-master: "TFM"
%%% End: 
