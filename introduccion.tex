\chapter{Introducción}

Una de las más importantes ideas en la física moderna es el concepto de
universalidad: Ciertas propiedades no dependen de los
detalles microscópicos, y además son equivalentes para sistemas
físicos que aparentemente no tienen relación alguna (clase de universalidad). Son un ejemplo de universalidad los sistemas que sufren una transición de fase debido a una ruptura
espontánea de la simetría. El comportamiento crítico de estos sistemas, cerca
de la temperatura crítica a la que sucede la transición, está descrito por una
serie de exponentes que están determinados por la dimensión y simetrías del
sistema. La universalidad está asegurada dado que los grados de libertad
microscópicos están promediados y no afectan a las fluctuaciones de larga
escala \cite{Kay:Polimerized_Membranes}. 

En las membranas flexibles \cite{Bowick_flat_phase}, superficies bidimensionales que fluctúan en un
espacio de tres dimensiones, se pueden distinguir principalmente dos clases de
universalidad: Las membranas fluidas y cristalinas. El primer tipo posee, además de poder
deformarse en el espacio tridimensional, la cualidad de variar su topología o
su geometría interna. Como ejemplo biológico de este tipo encontramos las
membranas lipídicas. Por otro lado, las clase cristalina, cuya estructura
interna es una red regular sin defectos topológicos, dislocaciones ni disclinaciones.  
. La principal característica de este tipo de membranas es que la conectividad
de su red interna es fija, de ahí que también reciban el nombre de supeficies
conectadas o de conectividad fija (\textit{fixed-connectivity/thetered surface}).  

Las membranas cristalinas se pueden considerar como la
generalización bidimensional de los polímeros, cadenas unidimensionales
de moléculas (monómeros), de ahí que también reciban el nombre de
\textit{membranas polimerizadas}. Los polímeros en un solvente forman
estructuras de dimensión fractal $5/3$. En cambio, las membranas tienen un
comportamiento más rico, en particular, tienen una fase ordenada a baja
temperatura. Está fase ordenada, o plana, está caracterizada por interacciones
de largo alcance en las orientaciones de las normales. A alta temperatura,
encontramos una fase desordenada o rugosa, con alta simetría. Separando ambas fases debe haber
una transición de ruptura espontánea de la simetría. Evidencias de esta
transición de fase, las encontramos en el comportamiento del radio de giro
$R^2_g$, la desviación cuadrática media de los puntos de la membrana respecto
al centro de masas, que es una medida del tamaño de la membrana. En la fase
plana, el radio de giro sigue la ley de escala $R^2_g\simeq L^2$. En cambio,
en la fase rugosa, debido a las fluctuaciones térmicas, tenemos que para una \textit{membrana
fantasma} (puede cruzarse a sí misma sin coste energético) el radio de giro
obedece $R^2_g\simeq \ln L$. Simulaciones numéricas \cite{Bowick_flat_phase}\cite{Espriu:MCRG}
y predicciones teóricas \cite{Doussal:nu} muestran que está transición de fase es de segundo orden.

Otra cualidad destacada de las membranas cristalinas es que el valor de su módulo Poisson,
la razón entre la deformación paralela y transversal a la dirección de 
la fuerza aplicad, es negativo \cite{Bowick_poisson_ratio}, comportamiento
contrario al de la mayoría de materiales, pues se expanden transversalmente al
ser estiradas. La explicación de este comportamiento es que al aplicar la
tensión se suprimen las fluctuaciones transversales al plano, forzando
entrópicamente a que se expanda en las direcciones del plano. 

Como ejemplos de experimentos realizados más destacados encontramos:
\begin{itemize}
\item El citoesqueleto de los glóbulos rojos forma una membrana cristalina
  natural \cite{Boal_MCell}, una red de proteínas que se encuentra entre la
  bicapa  lipídica de la membrana celular. Todavía no se han
  realizado experimentos que muestren una transición de fase, probablemente
  debido a que su módulo de curvatura es demasiado alto. La mejor estimación experimental de su
  módulo de Poisson encuentra un valor de $\simeq 1/3$ \cite{Discher:Molecular}, a partir de las
  medidas del módulo de cizalladura y de compresión.
\item Óxido de grafito, es un material formado por capas, y solamente existen
  fuerzas débiles (Van der Waals) entre las diferentes capas. Es posible
  obtener capas individuales de grafito con átomos de oxígeno en sus
  contornos. El módulo de curvatura de esta capa será pequeño. Experimentos
  realizados \cite{Hwa:Conformation} muestran una transición de fase con una dimensión fractal de
  $2.5$, que separa la fase plana y la rugosa.
\end{itemize}

El presente trabajo tiene como objetivo estudio del comportamiento crítico de
las membranas cristalinas mediante simulaciones numéricas utilizando el
algoritmo de Metropolis. Las únicas simulaciones numéricas datan de 1996 (casi
dos décadas) \cite{Bowick_flat_phase} y la red más grande que se simuló fue
128x128. Nuestra meta es mejorar sustancialmente estas simulaciones usando
ordenadores muchísimo más potentes, así como el uso de técnicas 
numéricas más avanzadas. Pretendemos caracterizar con mucha más precisión la
transición de fase plana-rugosa y caracterizar de una manera más precisa la
fase plana. 

El segundo capítulo está dedicado a un estudio analítico general, basado en
aproximaciones de campo medio, de las membranas
cristalinas. En el tercero se describen las técnicas de simulación
empleadas. Los capítulos cuarto y quinto se muestran los resultados obtenidos
y las conclusiones respectivamente.




  
%%% Local Variables: 
%%% mode: latex
%%% TeX-master: "TFM"
%%% End: 
